\documentclass{protokoll_en}
\newcommand{\assistent}{U. Wiedemann}
\newcommand{\versuch}{Optical Frequency Doubling}
\newcommand{\nummer}{A245}

\begin{document}

\section{Preface}
The object of this experiment is the production of frequency doubled visible (blue) light by coupling infrared laser light into a nonlinear $KNbO_3$-crystal. We are supposed to analyse the dependency of the output power on several parameters and the wavelength of the second harmonic wave.

\section{Theoretical Background}
\subsection{Diode laser}
A LASER (\emph{Light Amplification by Stimulated Emission of Radiation}) amplifies as the name implies light by means of a resonator via stimulated emission. Principle constituents of all types of lasers are therefore an active medium which is converted into inversion thru pumping and a resonator. Nevertheless there are several diversified types of lasers. The typical construction of a diode laser is shown in figure \ref{fig:laser}. Due to its inexpensive production and its high efficiency (about $\unit[50]{\%}$) diode lasers are widely-used. Furthermore it operates within a long range of wavelengths ($\unit[550]{nm}$ to $\unit[60]{\mu m}$).
\begin{figure}[H]
	\centering
		\includegraphics[width=0.5\textwidth]{graphics/laser}
	\caption{Setup of a typical diode laser~\cite{demtroed}}
	\label{fig:laser}
\end{figure}
The active medium is represented by a biased p-n-junction. If a voltage is applied (see figure \ref{fig:pn} right), the electrons are able to reach the boundary layer and recombine with the holes. Thus a photon of an energy correlated with the gap energy is emitted.
\begin{figure}[H]
	\centering
		\includegraphics[width=0.9\textwidth]{graphics/pn}
	\caption{p-n-junction of a diode laser without (left) and with (right) applied voltage~\cite{demtroed}}
	\label{fig:pn}
\end{figure}
As the semiconductor crystal has a high refraction index (i.\ e.\ $n_{\mathrm{GaAs}} = 3.6$) in comparison to air ($n_{\mathrm{air}} \approx 1$), the reflectivity of the boundary surface of the crystal amounts to $R=[(n_{\mathrm{GaAs}}-1)/(n_{\mathrm{GaAs}}+1)]^2 = 0.32$. This can be optimized by polishing those surfaces. Therefore the orthogonal boundary layers of the crystal can indeed be used as a resonator.

Of course it is neccessary to adjust the power of the laser. For this the diode current has to be varied. But one has to consider that a threshold current $I_{\mathrm{thr}}$ exists, because the photons can be absorbed or scattered in the resonator which results in an energyloss. Only above this threshold the laser enters the linear working regime and is pumped efficently.

\subsection{Gaussian Beams}
Because laser beams have a finite beam size, it is neccessary to go beyond the model of plane waves and introduce so-called \textsc{gaussian} beams (see figure \ref{fig:gaussianbeams}). In this theoretical approach the beam has a \textsc{gaussian} intensity distribution perpendicular to the wave vector $\vec{k}$. Furthermore it is characterized by its beam radius $w(z)$ and the \textsc{Rayleigh}-length $z_0 = \pi w^2_0/\lambda$. For $|z| \leq z_0$ the beam is focussed and has its smallest radius $w_0$ which is called waist at $z=0$. For $z \ll z_0$ nearly plane waves propagate and for large $z$ one observes spherical wavefronts. Furthermore the angle of divergence can be derived to $\theta_{\mathrm{div}} = w_0/z_0 = \lambda/\pi w_0$.
\begin{figure}[H]
	\centering
		\includegraphics[width=0.8\textwidth]{graphics/gaussianbeams}
	\caption{Sketch of a \textsc{gaussian} beam~\cite{meschi}}
	\label{fig:gaussianbeams}
\end{figure}

\subsection{Birefringence}
\label{cha:bi}
If light propagates through an optical medium, the refraction index $n$ of this medium has to be considered. In addition crystals like $KNbO_3$ with anisotropic refraction indices because of anisotropic linkage forces exist. The symmetry axis (if there is one) of the crystal is called optical axis. 
\begin{figure}[H]
	\centering
		\includegraphics[width=0.7\textwidth]{graphics/brech}
	\caption{Walk off of the extraordinary beam~\cite{meschi}}
	\label{fig:brech}
\end{figure}
Due to this the polarisation $\vec{P}$ induced by the electromagnetic wave is not parallel to $\vec{E}$ anymore, which is explained thru the dielectric constant which turns into a three dimensional tensor. Therefore different polarized beams take different paths. The one which follows \textsc{Snellius'} law is called ordinary beam, while the other one experiencing a walk off (see also figure \ref{fig:brech}) is called extraordinary beam. Those beams have orthogonal polarisations. This effect of splitting an incoming beam is called birefringence and is used for example in $\lambda/2$-plates.


\subsection{Nonlinear Optics and Frequency Doubling}
The electrons of a crystal being exposed to electromagnetic waves are displaced from their position of rest. If these shifts are small, the restoring forces will depend linearly on the displacementamplitude, so indeed the \textsc{Hooke} law will apply. As a result the polarisation $\vec{P}$ is proportional to the present electric field $\vec{E}$. 

At high intensities of (coherent) light (i.\ e.\ laser light) though this approximation is not applicable anymore. Thereby higher orders of the \textsc{Taylor}-expansion of the polarisation also have to be considered. Assuming an isotropic medium $P_\mathrm{x} = \varepsilon_0 \left( \chi_1 E_\mathrm{x} + \chi_2 E^2_\mathrm{x} + \dots \right)$ holds for the x-component of the polarisation. Now one can insert the electric field of a monochromatic light wave and yields:
\begin{align}
P_\mathrm{x} &\approx \varepsilon_0 \left( \chi_1 E_0 \cos{\omega t} + \chi_2 E^2_0 \cos^2{\omega t} \right) \\
             &= \varepsilon_0 \left( \frac{1}{2} \chi_2 E^2_0 + \chi_1 E_0 \cos{\omega t} + \frac{1}{2} \chi_2 E^2_0 \cos{2 \omega t} \right) \ \ldotp
\end{align}
So indeed one gets an oscillation term at twice of the original frequency and a constant polarisation term in addition to the one for the linear approximation. With correct phase relation those second harmonic oscillations can add up to a macroscopic wave. But in general the phases do not match due to dispersion at different frequencies. To obtain phase matching do the following considerations.

Assuming plane waves and constant intensity $I_\mathrm{f}$ of the fundamental wave within a crystal of length $L$ the output intensity of the second harmonic $I_\mathrm{sh}$ after the crystal is given by:
\begin{align}
I_{\mathrm{sh}}\propto \, L^2 I^2_{\mathrm{f}} \left( \frac{\sin{\Delta k \frac{L}{2}}}{\Delta k \frac{L}{2}} \right)^2 \ ,
\end{align}
whereas $\Delta k = k(2\omega) - 2 k(\omega) = \frac{2\omega}{v_\mathrm{sh}} - 2 \frac{\omega}{v_\mathrm{f}} = \frac{2\omega}{c} \left( n(2\omega) - n(\omega) \right)$ is the phase mismatch. At minimal mismatch the output intensity of the second harmonic reaches its maximum due to maximal constructive interference, therefore the phase matching condition reads:
\begin{align}
\Delta k = 0 \hspace{1cm} \Leftrightarrow \hspace{1cm} n(2\omega) = n(\omega) \ \ldotp
\end{align}
On account of that nonlinear crystals have to be used, because such a phase matching is not possible with normally dispersive media. Indeed there $n(2\omega) > n(\omega)$ always holds. In anisotropic crystals though the refractive index of the second harmonic can be lowered for example by using the second harmonic as the extraordinary beam.

For this the angle between the incident wave and the optical axis is adjusted such that the above condition holds. This method is called \emph{angle tuning}. But there is a problem with this approach. Of course as written in section \ref{cha:bi} the extraordinary beam experiences a walk off for $\theta \neq 0, \pi/2$. This is problematic because the beams should overlap behind the crystal to obtain an interference pattern. Therefore one uses \emph{temperature tuning}. This furthermore also allows a high conversion efficiency into the second harmonic, which improves the intensity.


\subsection{Coherence}
Coherence is a measure of the phase correlation of wavefronts. Two waves are called coherent, if roughly speaking their phase difference is constant within a relevant range. Indeed coherence is neccessary for interference. One distinguishes between time coherence and spatial coherence, but for means of brevity we will not discuss this further.

\subsection{Optical Grating}
A grating is used to create an interference pattern on a screen via illuminating it with coherent light. The maxima obtain the relation $\sin{\alpha} = n \lambda/g$, whereas $\alpha$ is the angle of diffraction of the maximum of $n^{\mathrm{th}}$ order, $\lambda$ is the wavelength of the light and $g$ the grating constant.

The spectral resolution of such a grating amounts to $\lambda/\Delta\lambda = N n$, where $N$ is the number of illuminated slits, so indeed one can improve the resolution by illuminating the grating properly.

\subsection{Michelson Interferometer}
The \textsc{Michelson} interferometer is a very common configuration for optical interferometry and was invented by Albert Abraham Michelson. 
\begin{figure}[H]
	\centering
		\includegraphics[width=0.6\textwidth]{graphics/michelson}
	\caption{\textsc{Michelson} interferometer~\cite{demtroed2}}
	\label{fig:michelson}
\end{figure}
As depicted above the incoming beam hits a beam splitter, so one reflects off, propagates to the top mirror and then reflects back, goes through the semi-transparent mirror again and reaches the detector. The other beam first passes the beam splitter and hits the mirror on the right. It is then reflected back and reflects off the beam splitter into the detector. So both beams travel indeed along different paths. As the right mirror can be displaced on an optical bank one can change the distance $s_2$ and therefore change the relative path difference of the two beams. This results in a phase difference $\Delta \phi = 2\pi (s_1-s_2)/\lambda$.



\section{Experimentation and Analysis}
\subsection{Diode Laser and power measurements}\label{ana_laserpower}
To get familiar with the laser we first examine the laser power depencence on the input current. Therefore we put a gauged photodiode in front of the laser and record the laser power for different values of the input current. Do avoid the photodiode to leave its 'linear range' we put an attenuator in front of it. The measured values are displayed in table \ref{tab:ana_laserpower_att} in the appendix. In order to gauge the attenuator we also recorded some values in the range between threshold current and an output of $\unit[50]{\mu W}$ without using the attenuator. These values are displayed in table \ref{tab:ana_laserpower_woatt} in the appendix.

Figure \ref{fig:ana_laser_watt} and \ref{fig:ana_laser_woatt} show plots of laser power with and without the attenuator as well as linear fits to the data taken above the threshold current. The fit results read:
\begin{align}
P_\textrm{att}(I) &= b_\textrm{att} + m_\textrm{att} \,\cdotp\, I = \unit[(-33.8 \pm 0.14)]{\mu W} + \unit[(0.574 \pm 0.002
)]{\frac{\mu W}{mA}}\,\cdotp\, I\\
P_\textrm{real}(I) &= b_\textrm{real} + m_\textrm{real} \,\cdotp\, I =\unit[(-36 \pm 1.5)]{mW} + \unit[(0.60 \pm 0.02
)]{\frac{mW}{mA}}\,\cdotp\, I
\end{align}
The ratio of the gradients yields the attenuation:
\begin{align}
\alpha = \unit[(1049 \pm 36)]{}
\end{align}
where the error is calculated via
\begin{align*}
\Delta \alpha = \sqrt{\left(\frac{\Delta m_\textrm{att}}{m_\textrm{real}}\right)^2 + \left(\frac{m_\textrm{att}}{m^2_\textrm{real}}\Delta m_\textrm{real}\right)^2}
\end{align*}
\begin{figure}[H]
\begin{floatrow}
\ffigbox[0.5\textwidth]{}
{
\resizebox{0.5\textwidth}{!}{
%	\begin{tikzpicture}[gnuplot]
%% generated with GNUPLOT 4.4p0 (Lua 5.1.4; terminal rev. 97, script rev. 96a)
%% 29.04.2010 21:37:31
\gpcolor{gp lt color border}
\gpsetlinetype{gp lt border}
\gpsetlinewidth{1.00}
\draw[gp path] (1.688,1.337)--(1.868,1.337);
\draw[gp path] (12.039,1.337)--(11.859,1.337);
\node[gp node right] at (1.504,1.337) { 0};
\draw[gp path] (1.688,3.098)--(1.868,3.098);
\draw[gp path] (12.039,3.098)--(11.859,3.098);
\node[gp node right] at (1.504,3.098) { 50};
\draw[gp path] (1.688,4.859)--(1.868,4.859);
\draw[gp path] (12.039,4.859)--(11.859,4.859);
\node[gp node right] at (1.504,4.859) { 100};
\draw[gp path] (1.688,6.620)--(1.868,6.620);
\draw[gp path] (12.039,6.620)--(11.859,6.620);
\node[gp node right] at (1.504,6.620) { 150};
\draw[gp path] (1.688,8.381)--(1.868,8.381);
\draw[gp path] (12.039,8.381)--(11.859,8.381);
\node[gp node right] at (1.504,8.381) { 200};
\draw[gp path] (1.688,0.985)--(1.688,1.165);
\draw[gp path] (1.688,8.381)--(1.688,8.201);
\node[gp node center] at (1.688,0.677) { 0};
\draw[gp path] (3.413,0.985)--(3.413,1.165);
\draw[gp path] (3.413,8.381)--(3.413,8.201);
\node[gp node center] at (3.413,0.677) { 50};
\draw[gp path] (5.138,0.985)--(5.138,1.165);
\draw[gp path] (5.138,8.381)--(5.138,8.201);
\node[gp node center] at (5.138,0.677) { 100};
\draw[gp path] (6.863,0.985)--(6.863,1.165);
\draw[gp path] (6.863,8.381)--(6.863,8.201);
\node[gp node center] at (6.863,0.677) { 150};
\draw[gp path] (8.589,0.985)--(8.589,1.165);
\draw[gp path] (8.589,8.381)--(8.589,8.201);
\node[gp node center] at (8.589,0.677) { 200};
\draw[gp path] (10.314,0.985)--(10.314,1.165);
\draw[gp path] (10.314,8.381)--(10.314,8.201);
\node[gp node center] at (10.314,0.677) { 250};
\draw[gp path] (12.039,0.985)--(12.039,1.165);
\draw[gp path] (12.039,8.381)--(12.039,8.201);
\node[gp node center] at (12.039,0.677) { 300};
\draw[gp path] (1.688,8.381)--(1.688,0.985)--(12.039,0.985)--(12.039,8.381)--cycle;
\node[gp node center,rotate=-270] at (0.430,4.683) {Laser power [$\mu$W]};
\node[gp node center] at (6.863,0.215) {Input current [mA]};
\node[gp node right] at (10.571,8.047) {Measured Data};
\gpcolor{gp lt color 0}
\gpsetlinetype{gp lt plot 0}
\draw[gp path] (10.755,8.047)--(11.671,8.047);
\draw[gp path] (10.755,8.137)--(10.755,7.957);
\draw[gp path] (11.671,8.137)--(11.671,7.957);
\draw[gp path] (1.633,1.337)--(1.813,1.337);
\draw[gp path] (1.633,1.337)--(1.813,1.337);
\draw[gp path] (1.978,1.337)--(2.158,1.337);
\draw[gp path] (1.978,1.337)--(2.158,1.337);
\draw[gp path] (2.288,1.337)--(2.468,1.337);
\draw[gp path] (2.288,1.337)--(2.468,1.337);
\draw[gp path] (2.633,1.337)--(2.813,1.337);
\draw[gp path] (2.633,1.337)--(2.813,1.337);
\draw[gp path] (2.978,1.337)--(3.158,1.337);
\draw[gp path] (2.978,1.337)--(3.158,1.337);
\draw[gp path] (3.151,1.337)--(3.331,1.337);
\draw[gp path] (3.151,1.337)--(3.331,1.337);
\draw[gp path] (3.323,1.337)--(3.503,1.337);
\draw[gp path] (3.323,1.337)--(3.503,1.337);
\draw[gp path] (3.496,1.337)--(3.676,1.337);
\draw[gp path] (3.496,1.337)--(3.676,1.337);
\draw[gp path] (3.668,1.359)--(3.848,1.359);
\draw[gp path] (3.668,1.359)--(3.848,1.359);
\draw[gp path] (3.931,1.454)--(3.931,1.455);
\draw[gp path] (3.841,1.454)--(4.021,1.454);
\draw[gp path] (3.841,1.455)--(4.021,1.455);
\draw[gp path] (4.103,1.560)--(4.103,1.561);
\draw[gp path] (4.013,1.560)--(4.193,1.560);
\draw[gp path] (4.013,1.561)--(4.193,1.561);
\draw[gp path] (4.276,1.674)--(4.276,1.675);
\draw[gp path] (4.186,1.674)--(4.366,1.674);
\draw[gp path] (4.186,1.675)--(4.366,1.675);
\draw[gp path] (4.448,1.788)--(4.448,1.795);
\draw[gp path] (4.358,1.788)--(4.538,1.788);
\draw[gp path] (4.358,1.795)--(4.538,1.795);
\draw[gp path] (4.793,1.915)--(4.793,1.922);
\draw[gp path] (4.703,1.915)--(4.883,1.915);
\draw[gp path] (4.703,1.922)--(4.883,1.922);
\draw[gp path] (5.138,2.211)--(5.138,2.218);
\draw[gp path] (5.048,2.211)--(5.228,2.211);
\draw[gp path] (5.048,2.218)--(5.228,2.218);
\draw[gp path] (5.483,2.334)--(5.483,2.341);
\draw[gp path] (5.393,2.334)--(5.573,2.334);
\draw[gp path] (5.393,2.341)--(5.573,2.341);
\draw[gp path] (5.828,2.580)--(5.828,2.587);
\draw[gp path] (5.738,2.580)--(5.918,2.580);
\draw[gp path] (5.738,2.587)--(5.918,2.587);
\draw[gp path] (6.173,2.778)--(6.173,2.785);
\draw[gp path] (6.083,2.778)--(6.263,2.778);
\draw[gp path] (6.083,2.785)--(6.263,2.785);
\draw[gp path] (6.518,2.954)--(6.518,2.961);
\draw[gp path] (6.428,2.954)--(6.608,2.954);
\draw[gp path] (6.428,2.961)--(6.608,2.961);
\draw[gp path] (6.863,3.165)--(6.863,3.172);
\draw[gp path] (6.773,3.165)--(6.953,3.165);
\draw[gp path] (6.773,3.172)--(6.953,3.172);
\draw[gp path] (7.209,3.394)--(7.209,3.401);
\draw[gp path] (7.119,3.394)--(7.299,3.394);
\draw[gp path] (7.119,3.401)--(7.299,3.401);
\draw[gp path] (7.554,3.588)--(7.554,3.595);
\draw[gp path] (7.464,3.588)--(7.644,3.588);
\draw[gp path] (7.464,3.595)--(7.644,3.595);
\draw[gp path] (7.899,3.792)--(7.899,3.799);
\draw[gp path] (7.809,3.792)--(7.989,3.792);
\draw[gp path] (7.809,3.799)--(7.989,3.799);
\draw[gp path] (8.244,4.000)--(8.244,4.007);
\draw[gp path] (8.154,4.000)--(8.334,4.000);
\draw[gp path] (8.154,4.007)--(8.334,4.007);
\draw[gp path] (8.589,4.148)--(8.589,4.155);
\draw[gp path] (8.499,4.148)--(8.679,4.148);
\draw[gp path] (8.499,4.155)--(8.679,4.155);
\draw[gp path] (8.934,4.363)--(8.934,4.370);
\draw[gp path] (8.844,4.363)--(9.024,4.363);
\draw[gp path] (8.844,4.370)--(9.024,4.370);
\draw[gp path] (9.279,4.514)--(9.279,4.521);
\draw[gp path] (9.189,4.514)--(9.369,4.514);
\draw[gp path] (9.189,4.521)--(9.369,4.521);
\draw[gp path] (9.624,4.785)--(9.624,4.792);
\draw[gp path] (9.534,4.785)--(9.714,4.785);
\draw[gp path] (9.534,4.792)--(9.714,4.792);
\draw[gp path] (9.969,4.965)--(9.969,5.035);
\draw[gp path] (9.879,4.965)--(10.059,4.965);
\draw[gp path] (9.879,5.035)--(10.059,5.035);
\draw[gp path] (10.314,5.247)--(10.314,5.317);
\draw[gp path] (10.224,5.247)--(10.404,5.247);
\draw[gp path] (10.224,5.317)--(10.404,5.317);
\draw[gp path] (10.659,5.317)--(10.659,5.387);
\draw[gp path] (10.569,5.317)--(10.749,5.317);
\draw[gp path] (10.569,5.387)--(10.749,5.387);
\draw[gp path] (11.004,5.669)--(11.004,5.740);
\draw[gp path] (10.914,5.669)--(11.094,5.669);
\draw[gp path] (10.914,5.740)--(11.094,5.740);
\draw[gp path] (11.349,5.740)--(11.349,5.810);
\draw[gp path] (11.259,5.740)--(11.439,5.740);
\draw[gp path] (11.259,5.810)--(11.439,5.810);
\draw[gp path] (1.705,1.337)--(1.740,1.337);
\draw[gp path] (1.705,1.247)--(1.705,1.427);
\draw[gp path] (1.740,1.247)--(1.740,1.427);
\draw[gp path] (2.050,1.337)--(2.085,1.337);
\draw[gp path] (2.050,1.247)--(2.050,1.427);
\draw[gp path] (2.085,1.247)--(2.085,1.427);
\draw[gp path] (2.361,1.337)--(2.395,1.337);
\draw[gp path] (2.361,1.247)--(2.361,1.427);
\draw[gp path] (2.395,1.247)--(2.395,1.427);
\draw[gp path] (2.706,1.337)--(2.740,1.337);
\draw[gp path] (2.706,1.247)--(2.706,1.427);
\draw[gp path] (2.740,1.247)--(2.740,1.427);
\draw[gp path] (3.051,1.337)--(3.085,1.337);
\draw[gp path] (3.051,1.247)--(3.051,1.427);
\draw[gp path] (3.085,1.247)--(3.085,1.427);
\draw[gp path] (3.223,1.337)--(3.258,1.337);
\draw[gp path] (3.223,1.247)--(3.223,1.427);
\draw[gp path] (3.258,1.247)--(3.258,1.427);
\draw[gp path] (3.396,1.337)--(3.430,1.337);
\draw[gp path] (3.396,1.247)--(3.396,1.427);
\draw[gp path] (3.430,1.247)--(3.430,1.427);
\draw[gp path] (3.568,1.337)--(3.603,1.337);
\draw[gp path] (3.568,1.247)--(3.568,1.427);
\draw[gp path] (3.603,1.247)--(3.603,1.427);
\draw[gp path] (3.741,1.359)--(3.775,1.359);
\draw[gp path] (3.741,1.269)--(3.741,1.449);
\draw[gp path] (3.775,1.269)--(3.775,1.449);
\draw[gp path] (3.913,1.454)--(3.948,1.454);
\draw[gp path] (3.913,1.364)--(3.913,1.544);
\draw[gp path] (3.948,1.364)--(3.948,1.544);
\draw[gp path] (4.086,1.560)--(4.120,1.560);
\draw[gp path] (4.086,1.470)--(4.086,1.650);
\draw[gp path] (4.120,1.470)--(4.120,1.650);
\draw[gp path] (4.258,1.675)--(4.293,1.675);
\draw[gp path] (4.258,1.585)--(4.258,1.765);
\draw[gp path] (4.293,1.585)--(4.293,1.765);
\draw[gp path] (4.431,1.792)--(4.466,1.792);
\draw[gp path] (4.431,1.702)--(4.431,1.882);
\draw[gp path] (4.466,1.702)--(4.466,1.882);
\draw[gp path] (4.776,1.918)--(4.811,1.918);
\draw[gp path] (4.776,1.828)--(4.776,2.008);
\draw[gp path] (4.811,1.828)--(4.811,2.008);
\draw[gp path] (5.121,2.214)--(5.156,2.214);
\draw[gp path] (5.121,2.124)--(5.121,2.304);
\draw[gp path] (5.156,2.124)--(5.156,2.304);
\draw[gp path] (5.466,2.337)--(5.501,2.337);
\draw[gp path] (5.466,2.247)--(5.466,2.427);
\draw[gp path] (5.501,2.247)--(5.501,2.427);
\draw[gp path] (5.811,2.584)--(5.846,2.584);
\draw[gp path] (5.811,2.494)--(5.811,2.674);
\draw[gp path] (5.846,2.494)--(5.846,2.674);
\draw[gp path] (6.156,2.781)--(6.191,2.781);
\draw[gp path] (6.156,2.691)--(6.156,2.871);
\draw[gp path] (6.191,2.691)--(6.191,2.871);
\draw[gp path] (6.501,2.957)--(6.536,2.957);
\draw[gp path] (6.501,2.867)--(6.501,3.047);
\draw[gp path] (6.536,2.867)--(6.536,3.047);
\draw[gp path] (6.846,3.169)--(6.881,3.169);
\draw[gp path] (6.846,3.079)--(6.846,3.259);
\draw[gp path] (6.881,3.079)--(6.881,3.259);
\draw[gp path] (7.191,3.398)--(7.226,3.398);
\draw[gp path] (7.191,3.308)--(7.191,3.488);
\draw[gp path] (7.226,3.308)--(7.226,3.488);
\draw[gp path] (7.536,3.591)--(7.571,3.591);
\draw[gp path] (7.536,3.501)--(7.536,3.681);
\draw[gp path] (7.571,3.501)--(7.571,3.681);
\draw[gp path] (7.881,3.795)--(7.916,3.795);
\draw[gp path] (7.881,3.705)--(7.881,3.885);
\draw[gp path] (7.916,3.705)--(7.916,3.885);
\draw[gp path] (8.226,4.003)--(8.261,4.003);
\draw[gp path] (8.226,3.913)--(8.226,4.093);
\draw[gp path] (8.261,3.913)--(8.261,4.093);
\draw[gp path] (8.571,4.151)--(8.606,4.151);
\draw[gp path] (8.571,4.061)--(8.571,4.241);
\draw[gp path] (8.606,4.061)--(8.606,4.241);
\draw[gp path] (8.916,4.366)--(8.951,4.366);
\draw[gp path] (8.916,4.276)--(8.916,4.456);
\draw[gp path] (8.951,4.276)--(8.951,4.456);
\draw[gp path] (9.261,4.517)--(9.296,4.517);
\draw[gp path] (9.261,4.427)--(9.261,4.607);
\draw[gp path] (9.296,4.427)--(9.296,4.607);
\draw[gp path] (9.607,4.789)--(9.641,4.789);
\draw[gp path] (9.607,4.699)--(9.607,4.879);
\draw[gp path] (9.641,4.699)--(9.641,4.879);
\draw[gp path] (9.952,5.000)--(9.986,5.000);
\draw[gp path] (9.952,4.910)--(9.952,5.090);
\draw[gp path] (9.986,4.910)--(9.986,5.090);
\draw[gp path] (10.297,5.282)--(10.331,5.282);
\draw[gp path] (10.297,5.192)--(10.297,5.372);
\draw[gp path] (10.331,5.192)--(10.331,5.372);
\draw[gp path] (10.642,5.352)--(10.676,5.352);
\draw[gp path] (10.642,5.262)--(10.642,5.442);
\draw[gp path] (10.676,5.262)--(10.676,5.442);
\draw[gp path] (10.987,5.704)--(11.021,5.704);
\draw[gp path] (10.987,5.614)--(10.987,5.794);
\draw[gp path] (11.021,5.614)--(11.021,5.794);
\draw[gp path] (11.332,5.775)--(11.366,5.775);
\draw[gp path] (11.332,5.685)--(11.332,5.865);
\draw[gp path] (11.366,5.685)--(11.366,5.865);
\gpsetpointsize{4.00}
\gppoint{gp mark 1}{(1.723,1.337)}
\gppoint{gp mark 1}{(2.068,1.337)}
\gppoint{gp mark 1}{(2.378,1.337)}
\gppoint{gp mark 1}{(2.723,1.337)}
\gppoint{gp mark 1}{(3.068,1.337)}
\gppoint{gp mark 1}{(3.241,1.337)}
\gppoint{gp mark 1}{(3.413,1.337)}
\gppoint{gp mark 1}{(3.586,1.337)}
\gppoint{gp mark 1}{(3.758,1.359)}
\gppoint{gp mark 1}{(3.931,1.454)}
\gppoint{gp mark 1}{(4.103,1.560)}
\gppoint{gp mark 1}{(4.276,1.675)}
\gppoint{gp mark 1}{(4.448,1.792)}
\gppoint{gp mark 1}{(4.793,1.918)}
\gppoint{gp mark 1}{(5.138,2.214)}
\gppoint{gp mark 1}{(5.483,2.337)}
\gppoint{gp mark 1}{(5.828,2.584)}
\gppoint{gp mark 1}{(6.173,2.781)}
\gppoint{gp mark 1}{(6.518,2.957)}
\gppoint{gp mark 1}{(6.863,3.169)}
\gppoint{gp mark 1}{(7.209,3.398)}
\gppoint{gp mark 1}{(7.554,3.591)}
\gppoint{gp mark 1}{(7.899,3.795)}
\gppoint{gp mark 1}{(8.244,4.003)}
\gppoint{gp mark 1}{(8.589,4.151)}
\gppoint{gp mark 1}{(8.934,4.366)}
\gppoint{gp mark 1}{(9.279,4.517)}
\gppoint{gp mark 1}{(9.624,4.789)}
\gppoint{gp mark 1}{(9.969,5.000)}
\gppoint{gp mark 1}{(10.314,5.282)}
\gppoint{gp mark 1}{(10.659,5.352)}
\gppoint{gp mark 1}{(11.004,5.704)}
\gppoint{gp mark 1}{(11.349,5.775)}
\gppoint{gp mark 1}{(11.213,8.047)}
\gpcolor{gp lt color border}
\node[gp node right] at (10.571,7.739) {Fit};
\gpsetlinetype{gp lt border}
\draw[gp path] (10.755,7.739)--(11.671,7.739);
\draw[gp path] (3.120,0.985)--(3.152,1.004)--(3.256,1.065)--(3.361,1.126)--(3.465,1.187)%
  --(3.570,1.249)--(3.675,1.310)--(3.779,1.371)--(3.884,1.432)--(3.988,1.493)--(4.093,1.555)%
  --(4.197,1.616)--(4.302,1.677)--(4.406,1.738)--(4.511,1.800)--(4.616,1.861)--(4.720,1.922)%
  --(4.825,1.983)--(4.929,2.045)--(5.034,2.106)--(5.138,2.167)--(5.243,2.228)--(5.347,2.289)%
  --(5.452,2.351)--(5.557,2.412)--(5.661,2.473)--(5.766,2.534)--(5.870,2.596)--(5.975,2.657)%
  --(6.079,2.718)--(6.184,2.779)--(6.288,2.841)--(6.393,2.902)--(6.498,2.963)--(6.602,3.024)%
  --(6.707,3.086)--(6.811,3.147)--(6.916,3.208)--(7.020,3.269)--(7.125,3.330)--(7.229,3.392)%
  --(7.334,3.453)--(7.439,3.514)--(7.543,3.575)--(7.648,3.637)--(7.752,3.698)--(7.857,3.759)%
  --(7.961,3.820)--(8.066,3.882)--(8.170,3.943)--(8.275,4.004)--(8.380,4.065)--(8.484,4.127)%
  --(8.589,4.188)--(8.693,4.249)--(8.798,4.310)--(8.902,4.371)--(9.007,4.433)--(9.111,4.494)%
  --(9.216,4.555)--(9.321,4.616)--(9.425,4.678)--(9.530,4.739)--(9.634,4.800)--(9.739,4.861)%
  --(9.843,4.923)--(9.948,4.984)--(10.052,5.045)--(10.157,5.106)--(10.262,5.167)--(10.366,5.229)%
  --(10.471,5.290)--(10.575,5.351)--(10.680,5.412)--(10.784,5.474)--(10.889,5.535)--(10.993,5.596)%
  --(11.098,5.657)--(11.203,5.719)--(11.307,5.780)--(11.412,5.841)--(11.516,5.902)--(11.621,5.964)%
  --(11.725,6.025)--(11.830,6.086)--(11.934,6.147)--(12.039,6.208);
\draw[gp path] (1.688,8.381)--(1.688,0.985)--(12.039,0.985)--(12.039,8.381)--cycle;
%% coordinates of the plot area
\gpdefrectangularnode{gp plot 1}{\pgfpoint{1.688cm}{0.985cm}}{\pgfpoint{12.039cm}{8.381cm}}
\end{tikzpicture}
%% gnuplot variables

}
	\caption{Attenuated laser output power versus input current and corresponding linear fit}
	\label{fig:ana_laser_watt}
}
\ffigbox[0.5\textwidth]{}
{
\resizebox{0.5\textwidth}{!}{
	%\begin{tikzpicture}[gnuplot]
%% generated with GNUPLOT 4.4p0 (Lua 5.1.4; terminal rev. 97, script rev. 96a)
%% 29.04.2010 21:37:32
\gpcolor{gp lt color border}
\gpsetlinetype{gp lt border}
\gpsetlinewidth{1.00}
\draw[gp path] (1.504,0.985)--(1.684,0.985);
\draw[gp path] (12.039,0.985)--(11.859,0.985);
\node[gp node right] at (1.320,0.985) { 0};
\draw[gp path] (1.504,2.218)--(1.684,2.218);
\draw[gp path] (12.039,2.218)--(11.859,2.218);
\node[gp node right] at (1.320,2.218) { 5};
\draw[gp path] (1.504,3.450)--(1.684,3.450);
\draw[gp path] (12.039,3.450)--(11.859,3.450);
\node[gp node right] at (1.320,3.450) { 10};
\draw[gp path] (1.504,4.683)--(1.684,4.683);
\draw[gp path] (12.039,4.683)--(11.859,4.683);
\node[gp node right] at (1.320,4.683) { 15};
\draw[gp path] (1.504,5.916)--(1.684,5.916);
\draw[gp path] (12.039,5.916)--(11.859,5.916);
\node[gp node right] at (1.320,5.916) { 20};
\draw[gp path] (1.504,7.148)--(1.684,7.148);
\draw[gp path] (12.039,7.148)--(11.859,7.148);
\node[gp node right] at (1.320,7.148) { 25};
\draw[gp path] (1.504,8.381)--(1.684,8.381);
\draw[gp path] (12.039,8.381)--(11.859,8.381);
\node[gp node right] at (1.320,8.381) { 30};
\draw[gp path] (2.557,0.985)--(2.557,1.165);
\draw[gp path] (2.557,8.381)--(2.557,8.201);
\node[gp node center] at (2.557,0.677) { 60};
\draw[gp path] (4.664,0.985)--(4.664,1.165);
\draw[gp path] (4.664,8.381)--(4.664,8.201);
\node[gp node center] at (4.664,0.677) { 70};
\draw[gp path] (6.771,0.985)--(6.771,1.165);
\draw[gp path] (6.771,8.381)--(6.771,8.201);
\node[gp node center] at (6.771,0.677) { 80};
\draw[gp path] (8.878,0.985)--(8.878,1.165);
\draw[gp path] (8.878,8.381)--(8.878,8.201);
\node[gp node center] at (8.878,0.677) { 90};
\draw[gp path] (10.985,0.985)--(10.985,1.165);
\draw[gp path] (10.985,8.381)--(10.985,8.201);
\node[gp node center] at (10.985,0.677) { 100};
\draw[gp path] (1.504,8.381)--(1.504,0.985)--(12.039,0.985)--(12.039,8.381)--cycle;
\node[gp node center,rotate=-270] at (0.430,4.683) {Laser power [mW]};
\node[gp node center] at (6.771,0.215) {Input current [mA]};
\node[gp node right] at (10.571,8.047) {Measured Data};
\gpcolor{gp lt color 0}
\gpsetlinetype{gp lt plot 0}
\draw[gp path] (10.755,8.047)--(11.671,8.047);
\draw[gp path] (10.755,8.137)--(10.755,7.957);
\draw[gp path] (11.671,8.137)--(11.671,7.957);
\draw[gp path] (2.557,1.142)--(2.557,1.146);
\draw[gp path] (2.467,1.142)--(2.647,1.142);
\draw[gp path] (2.467,1.146)--(2.647,1.146);
\draw[gp path] (3.611,1.739)--(3.611,1.744);
\draw[gp path] (3.521,1.739)--(3.701,1.739);
\draw[gp path] (3.521,1.744)--(3.701,1.744);
\draw[gp path] (4.664,2.353)--(4.664,2.358);
\draw[gp path] (4.574,2.353)--(4.754,2.353);
\draw[gp path] (4.574,2.358)--(4.754,2.358);
\draw[gp path] (5.718,3.221)--(5.718,3.226);
\draw[gp path] (5.628,3.221)--(5.808,3.221);
\draw[gp path] (5.628,3.226)--(5.808,3.226);
\draw[gp path] (6.771,3.968)--(6.771,4.017);
\draw[gp path] (6.681,3.968)--(6.861,3.968);
\draw[gp path] (6.681,4.017)--(6.861,4.017);
\draw[gp path] (7.825,4.535)--(7.825,4.584);
\draw[gp path] (7.735,4.535)--(7.915,4.535);
\draw[gp path] (7.735,4.584)--(7.915,4.584);
\draw[gp path] (8.878,5.250)--(8.878,5.299);
\draw[gp path] (8.788,5.250)--(8.968,5.250);
\draw[gp path] (8.788,5.299)--(8.968,5.299);
\draw[gp path] (9.932,6.236)--(9.932,6.285);
\draw[gp path] (9.842,6.236)--(10.022,6.236);
\draw[gp path] (9.842,6.285)--(10.022,6.285);
\draw[gp path] (10.985,6.926)--(10.985,6.976);
\draw[gp path] (10.895,6.926)--(11.075,6.926);
\draw[gp path] (10.895,6.976)--(11.075,6.976);
\draw[gp path] (2.452,1.144)--(2.663,1.144);
\draw[gp path] (2.452,1.054)--(2.452,1.234);
\draw[gp path] (2.663,1.054)--(2.663,1.234);
\draw[gp path] (3.506,1.742)--(3.716,1.742);
\draw[gp path] (3.506,1.652)--(3.506,1.832);
\draw[gp path] (3.716,1.652)--(3.716,1.832);
\draw[gp path] (4.559,2.356)--(4.770,2.356);
\draw[gp path] (4.559,2.266)--(4.559,2.446);
\draw[gp path] (4.770,2.266)--(4.770,2.446);
\draw[gp path] (5.613,3.224)--(5.823,3.224);
\draw[gp path] (5.613,3.134)--(5.613,3.314);
\draw[gp path] (5.823,3.134)--(5.823,3.314);
\draw[gp path] (6.666,3.993)--(6.877,3.993);
\draw[gp path] (6.666,3.903)--(6.666,4.083);
\draw[gp path] (6.877,3.903)--(6.877,4.083);
\draw[gp path] (7.720,4.560)--(7.930,4.560);
\draw[gp path] (7.720,4.470)--(7.720,4.650);
\draw[gp path] (7.930,4.470)--(7.930,4.650);
\draw[gp path] (8.773,5.275)--(8.984,5.275);
\draw[gp path] (8.773,5.185)--(8.773,5.365);
\draw[gp path] (8.984,5.185)--(8.984,5.365);
\draw[gp path] (9.827,6.261)--(10.037,6.261);
\draw[gp path] (9.827,6.171)--(9.827,6.351);
\draw[gp path] (10.037,6.171)--(10.037,6.351);
\draw[gp path] (10.880,6.951)--(11.091,6.951);
\draw[gp path] (10.880,6.861)--(10.880,7.041);
\draw[gp path] (11.091,6.861)--(11.091,7.041);
\gpsetpointsize{4.00}
\gppoint{gp mark 1}{(2.557,1.144)}
\gppoint{gp mark 1}{(3.611,1.742)}
\gppoint{gp mark 1}{(4.664,2.356)}
\gppoint{gp mark 1}{(5.718,3.224)}
\gppoint{gp mark 1}{(6.771,3.993)}
\gppoint{gp mark 1}{(7.825,4.560)}
\gppoint{gp mark 1}{(8.878,5.275)}
\gppoint{gp mark 1}{(9.932,6.261)}
\gppoint{gp mark 1}{(10.985,6.951)}
\gppoint{gp mark 1}{(11.213,8.047)}
\gpcolor{gp lt color border}
\node[gp node right] at (10.571,7.739) {Fit};
\gpsetlinetype{gp lt border}
\draw[gp path] (10.755,7.739)--(11.671,7.739);
\draw[gp path] (2.598,0.985)--(2.675,1.039)--(2.781,1.114)--(2.887,1.189)--(2.994,1.264)%
  --(3.100,1.339)--(3.207,1.414)--(3.313,1.489)--(3.419,1.564)--(3.526,1.639)--(3.632,1.714)%
  --(3.739,1.788)--(3.845,1.863)--(3.952,1.938)--(4.058,2.013)--(4.164,2.088)--(4.271,2.163)%
  --(4.377,2.238)--(4.484,2.313)--(4.590,2.388)--(4.696,2.463)--(4.803,2.538)--(4.909,2.613)%
  --(5.016,2.688)--(5.122,2.763)--(5.228,2.838)--(5.335,2.912)--(5.441,2.987)--(5.548,3.062)%
  --(5.654,3.137)--(5.761,3.212)--(5.867,3.287)--(5.973,3.362)--(6.080,3.437)--(6.186,3.512)%
  --(6.293,3.587)--(6.399,3.662)--(6.505,3.737)--(6.612,3.812)--(6.718,3.887)--(6.825,3.962)%
  --(6.931,4.036)--(7.038,4.111)--(7.144,4.186)--(7.250,4.261)--(7.357,4.336)--(7.463,4.411)%
  --(7.570,4.486)--(7.676,4.561)--(7.782,4.636)--(7.889,4.711)--(7.995,4.786)--(8.102,4.861)%
  --(8.208,4.936)--(8.315,5.011)--(8.421,5.086)--(8.527,5.160)--(8.634,5.235)--(8.740,5.310)%
  --(8.847,5.385)--(8.953,5.460)--(9.059,5.535)--(9.166,5.610)--(9.272,5.685)--(9.379,5.760)%
  --(9.485,5.835)--(9.591,5.910)--(9.698,5.985)--(9.804,6.060)--(9.911,6.135)--(10.017,6.210)%
  --(10.124,6.284)--(10.230,6.359)--(10.336,6.434)--(10.443,6.509)--(10.549,6.584)--(10.656,6.659)%
  --(10.762,6.734)--(10.868,6.809)--(10.975,6.884)--(11.081,6.959)--(11.188,7.034)--(11.294,7.109)%
  --(11.401,7.184)--(11.507,7.259)--(11.613,7.334)--(11.720,7.408)--(11.826,7.483)--(11.933,7.558)%
  --(12.039,7.633);
\draw[gp path] (1.504,8.381)--(1.504,0.985)--(12.039,0.985)--(12.039,8.381)--cycle;
%% coordinates of the plot area
\gpdefrectangularnode{gp plot 1}{\pgfpoint{1.504cm}{0.985cm}}{\pgfpoint{12.039cm}{8.381cm}}
\end{tikzpicture}
%% gnuplot variables

}
	\caption{Real laser output power versus input current and corresponding linear fit}
	\label{fig:ana_laser_woatt}
}
\end{floatrow}
\end{figure}
With this value we can determine the dependency of the laser power on the input current:
\begin{align}
P(I) &= b + m\,\cdotp\, I=\unit[(-3.6 \pm 0.1)\cdotp 10^{4}]{\mu W} + \unit[(602 \pm 21)]{\frac{\mu W}{mA}}\,\cdotp\, I
\end{align}
and we can deduce the threshold current:
\begin{align}
I_\textrm{thr} = -\frac{b}{m} = \unit[(59 \pm 3)]{mA}
\end{align}
where the error is given by:
\begin{align}
\Delta I_\textrm{thr} = \sqrt{\left(\frac{\Delta b}{m}\right)^2+\left(\frac{b}{m^2}\Delta m\right)^2}
\end{align}
The quantum efficiency $\eta$, which is defined as the number of emitted photons per electron injected in the laser diode, can be deduced using:
\begin{align}
P = \frac{h\nu N_\gamma}{t}\hspace{1cm} I = \frac{eN_\textrm{e}}{t}
\end{align}
where $N_\gamma$ and $N_\textrm{e}$ are the number of photons emitted and the number of electrons injected in the time intervall $t$ respectively. With the wavelength of the laser $\lambda = \unit[987]{nm}$ we get:
\begin{align}
\eta = \frac{N_\gamma}{N_\textrm{e}} = \frac{eP}{Ih\nu} \approx \frac{\partial P}{\partial I}\frac{e\lambda}{hc} = m\frac{e\lambda}{hc} = \unit[(0,48 \pm 0,02)]{}
\end{align}
with the error
\begin{align}
\Delta \eta = \frac{e\lambda}{hc}\Delta m
\end{align}
\subsection{Calibration of the variable attenuator}\label{subsec:var_att_calib}
To realize a variable attenuator, we place a rotatable $\lambda /2$ plate and a polarization filter between the laser and the photodiode. The scale of the $\lambda /2$ plate faces towards the laser and the angle of the analyzer is set to 90�. The polarization filter is then used as an analyzer whose transmitted intensity is given by the Malus law:
\begin{align*}
I(\alpha)=I_0\cdot\cos^2(\alpha)
\end{align*}
where $I_0$ is the intensity of the laser and $\alpha$ is given by the angle between the polarization axis of the beam and the polarizer. Because the polarizer is aligned colinear to the polarization axis of the beam and a $\lambda /2$ plate rotates the polarization axis of the beam by $2\Phi$, with $\Phi$ being the angle between the optical axis of the $\lambda /2$ plate and the beam polarization axis, we expect the following intensity behind the polarizer:
\begin{align*}
I(\Phi)=I_0\cdot\cos^2(2\Phi)
\end{align*}
Our data is presented in table \ref{tab:ana_var_att} in the appendix. Figure \ref{fig:ana_var_att} shows the data with the corresponding fit. The graph shows only the attenuated power because these value's error is smaller since it does not depend on the error containing attenuation of the fixed attenuator. The fit parameters read:
\begin{align*}
P_\textrm{attenuated} = \unit[(117.9 \pm 0.6)]{\mu W}\,\cdotp\,\cos^2\left(\unit[(2.0029 \pm 0.0009)]{\frac{1}{�}} \,\cdotp\, \Phi\right)\textrm{.}
\end{align*}
Therefore the transmitted power of the variable attenuator is:
\begin{align}
P = \unit[(124 \pm 4)]{mW}\,\cdotp\,\cos^2\left(\unit[(2.0029 \pm 0.0009)]{\frac{1}{�}} \,\cdotp\, \Phi\right)\textrm{,}
\end{align}
where the error of the amplitude is given by
\begin{align}
\Delta P_0 = \sqrt{\left(P_0^\textrm{attenuated}\Delta \alpha\right)^2+\left(\alpha\Delta P_0^\textrm{attenuated}\right)^2}\textrm{.}
\end{align}
\begin{figure}[H]
  \resizebox{0.8\textwidth}{!}{
   % \begin{tikzpicture}[gnuplot]
%% generated with GNUPLOT 4.4p0 (Lua 5.1.4; terminal rev. 97, script rev. 96a)
%% 25.04.2010 14:04:11
\gpcolor{gp lt color border}
\gpsetlinetype{gp lt border}
\gpsetlinewidth{1.00}
\draw[gp path] (1.688,0.985)--(1.868,0.985);
\draw[gp path] (12.039,0.985)--(11.859,0.985);
\node[gp node right] at (1.504,0.985) { 0};
\draw[gp path] (1.688,2.218)--(1.868,2.218);
\draw[gp path] (12.039,2.218)--(11.859,2.218);
\node[gp node right] at (1.504,2.218) { 20};
\draw[gp path] (1.688,3.450)--(1.868,3.450);
\draw[gp path] (12.039,3.450)--(11.859,3.450);
\node[gp node right] at (1.504,3.450) { 40};
\draw[gp path] (1.688,4.683)--(1.868,4.683);
\draw[gp path] (12.039,4.683)--(11.859,4.683);
\node[gp node right] at (1.504,4.683) { 60};
\draw[gp path] (1.688,5.916)--(1.868,5.916);
\draw[gp path] (12.039,5.916)--(11.859,5.916);
\node[gp node right] at (1.504,5.916) { 80};
\draw[gp path] (1.688,7.148)--(1.868,7.148);
\draw[gp path] (12.039,7.148)--(11.859,7.148);
\node[gp node right] at (1.504,7.148) { 100};
\draw[gp path] (1.688,8.381)--(1.868,8.381);
\draw[gp path] (12.039,8.381)--(11.859,8.381);
\node[gp node right] at (1.504,8.381) { 120};
\draw[gp path] (1.688,0.985)--(1.688,1.165);
\draw[gp path] (1.688,8.381)--(1.688,8.201);
\node[gp node center] at (1.688,0.677) { 0};
\draw[gp path] (3.758,0.985)--(3.758,1.165);
\draw[gp path] (3.758,8.381)--(3.758,8.201);
\node[gp node center] at (3.758,0.677) { 50};
\draw[gp path] (5.828,0.985)--(5.828,1.165);
\draw[gp path] (5.828,8.381)--(5.828,8.201);
\node[gp node center] at (5.828,0.677) { 100};
\draw[gp path] (7.899,0.985)--(7.899,1.165);
\draw[gp path] (7.899,8.381)--(7.899,8.201);
\node[gp node center] at (7.899,0.677) { 150};
\draw[gp path] (9.969,0.985)--(9.969,1.165);
\draw[gp path] (9.969,8.381)--(9.969,8.201);
\node[gp node center] at (9.969,0.677) { 200};
\draw[gp path] (12.039,0.985)--(12.039,1.165);
\draw[gp path] (12.039,8.381)--(12.039,8.201);
\node[gp node center] at (12.039,0.677) { 250};
\draw[gp path] (1.688,8.381)--(1.688,0.985)--(12.039,0.985)--(12.039,8.381)--cycle;
\node[gp node center,rotate=-270] at (0.430,4.683) {P [$\mu$W]};
\node[gp node center] at (6.863,0.215) {$\Phi$ [�]};
\node[gp node right] at (10.571,8.047) {Measured Data};
\gpcolor{gp lt color 0}
\gpsetlinetype{gp lt plot 0}
\draw[gp path] (10.755,8.047)--(11.671,8.047);
\draw[gp path] (10.755,8.137)--(10.755,7.957);
\draw[gp path] (11.671,8.137)--(11.671,7.957);
\draw[gp path] (1.688,7.718)--(1.688,8.206);
\draw[gp path] (1.598,7.718)--(1.778,7.718);
\draw[gp path] (1.598,8.206)--(1.778,8.206);
\draw[gp path] (1.792,7.546)--(1.792,8.023);
\draw[gp path] (1.702,7.546)--(1.882,7.546);
\draw[gp path] (1.702,8.023)--(1.882,8.023);
\draw[gp path] (1.895,7.432)--(1.895,7.901);
\draw[gp path] (1.805,7.432)--(1.985,7.432);
\draw[gp path] (1.805,7.901)--(1.985,7.901);
\draw[gp path] (1.999,7.089)--(1.999,7.534);
\draw[gp path] (1.909,7.089)--(2.089,7.089);
\draw[gp path] (1.909,7.534)--(2.089,7.534);
\draw[gp path] (2.102,6.745)--(2.102,7.168);
\draw[gp path] (2.012,6.745)--(2.192,6.745);
\draw[gp path] (2.012,7.168)--(2.192,7.168);
\draw[gp path] (2.206,6.345)--(2.206,6.741);
\draw[gp path] (2.116,6.345)--(2.296,6.345);
\draw[gp path] (2.116,6.741)--(2.296,6.741);
\draw[gp path] (2.309,5.830)--(2.309,6.191);
\draw[gp path] (2.219,5.830)--(2.399,5.830);
\draw[gp path] (2.219,6.191)--(2.399,6.191);
\draw[gp path] (2.413,5.258)--(2.413,5.581);
\draw[gp path] (2.323,5.258)--(2.503,5.258);
\draw[gp path] (2.323,5.581)--(2.503,5.581);
\draw[gp path] (2.516,4.685)--(2.516,4.972);
\draw[gp path] (2.426,4.685)--(2.606,4.685);
\draw[gp path] (2.426,4.972)--(2.606,4.972);
\draw[gp path] (2.620,4.112)--(2.620,4.362);
\draw[gp path] (2.530,4.112)--(2.710,4.112);
\draw[gp path] (2.530,4.362)--(2.710,4.362);
\draw[gp path] (2.723,3.595)--(2.723,3.815);
\draw[gp path] (2.633,3.595)--(2.813,3.595);
\draw[gp path] (2.633,3.815)--(2.813,3.815);
\draw[gp path] (2.827,3.020)--(2.827,3.207);
\draw[gp path] (2.737,3.020)--(2.917,3.020);
\draw[gp path] (2.737,3.207)--(2.917,3.207);
\draw[gp path] (2.930,2.443)--(2.930,2.601);
\draw[gp path] (2.840,2.443)--(3.020,2.443);
\draw[gp path] (2.840,2.601)--(3.020,2.601);
\draw[gp path] (3.034,2.096)--(3.034,2.239);
\draw[gp path] (2.944,2.096)--(3.124,2.096);
\draw[gp path] (2.944,2.239)--(3.124,2.239);
\draw[gp path] (3.137,1.689)--(3.137,1.818);
\draw[gp path] (3.047,1.689)--(3.227,1.689);
\draw[gp path] (3.047,1.818)--(3.227,1.818);
\draw[gp path] (3.241,1.338)--(3.241,1.460);
\draw[gp path] (3.151,1.338)--(3.331,1.338);
\draw[gp path] (3.151,1.460)--(3.331,1.460);
\draw[gp path] (3.344,1.103)--(3.344,1.222);
\draw[gp path] (3.254,1.103)--(3.434,1.103);
\draw[gp path] (3.254,1.222)--(3.434,1.222);
\draw[gp path] (3.448,0.985)--(3.448,1.091);
\draw[gp path] (3.358,0.985)--(3.538,0.985);
\draw[gp path] (3.358,1.091)--(3.538,1.091);
\draw[gp path] (3.551,0.985)--(3.551,1.064);
\draw[gp path] (3.461,0.985)--(3.641,0.985);
\draw[gp path] (3.461,1.064)--(3.641,1.064);
\draw[gp path] (3.655,1.014)--(3.655,1.133);
\draw[gp path] (3.565,1.014)--(3.745,1.014);
\draw[gp path] (3.565,1.133)--(3.745,1.133);
\draw[gp path] (3.758,1.209)--(3.758,1.329);
\draw[gp path] (3.668,1.209)--(3.848,1.209);
\draw[gp path] (3.668,1.329)--(3.848,1.329);
\draw[gp path] (3.862,1.444)--(3.862,1.567);
\draw[gp path] (3.772,1.444)--(3.952,1.444);
\draw[gp path] (3.772,1.567)--(3.952,1.567);
\draw[gp path] (3.965,1.864)--(3.965,1.998);
\draw[gp path] (3.875,1.864)--(4.055,1.864);
\draw[gp path] (3.875,1.998)--(4.055,1.998);
\draw[gp path] (4.069,2.270)--(4.069,2.420);
\draw[gp path] (3.979,2.270)--(4.159,2.270);
\draw[gp path] (3.979,2.420)--(4.159,2.420);
\draw[gp path] (4.172,2.790)--(4.172,2.964);
\draw[gp path] (4.082,2.790)--(4.262,2.790);
\draw[gp path] (4.082,2.964)--(4.262,2.964);
\draw[gp path] (4.276,3.308)--(4.276,3.511);
\draw[gp path] (4.186,3.308)--(4.366,3.308);
\draw[gp path] (4.186,3.511)--(4.366,3.511);
\draw[gp path] (4.379,3.882)--(4.379,4.119);
\draw[gp path] (4.289,3.882)--(4.469,3.882);
\draw[gp path] (4.289,4.119)--(4.469,4.119);
\draw[gp path] (4.483,4.456)--(4.483,4.728);
\draw[gp path] (4.393,4.456)--(4.573,4.456);
\draw[gp path] (4.393,4.728)--(4.573,4.728);
\draw[gp path] (4.586,5.086)--(4.586,5.398);
\draw[gp path] (4.496,5.086)--(4.676,5.086);
\draw[gp path] (4.496,5.398)--(4.676,5.398);
\draw[gp path] (4.690,5.544)--(4.690,5.886);
\draw[gp path] (4.600,5.544)--(4.780,5.544);
\draw[gp path] (4.600,5.886)--(4.780,5.886);
\draw[gp path] (4.793,6.116)--(4.793,6.497);
\draw[gp path] (4.703,6.116)--(4.883,6.116);
\draw[gp path] (4.703,6.497)--(4.883,6.497);
\draw[gp path] (4.897,6.517)--(4.897,6.924);
\draw[gp path] (4.807,6.517)--(4.987,6.517);
\draw[gp path] (4.807,6.924)--(4.987,6.924);
\draw[gp path] (5.000,6.974)--(5.000,7.412);
\draw[gp path] (4.910,6.974)--(5.090,6.974);
\draw[gp path] (4.910,7.412)--(5.090,7.412);
\draw[gp path] (5.104,7.260)--(5.104,7.718);
\draw[gp path] (5.014,7.260)--(5.194,7.260);
\draw[gp path] (5.014,7.718)--(5.194,7.718);
\draw[gp path] (5.207,7.546)--(5.207,8.023);
\draw[gp path] (5.117,7.546)--(5.297,7.546);
\draw[gp path] (5.117,8.023)--(5.297,8.023);
\draw[gp path] (5.311,7.660)--(5.311,8.145);
\draw[gp path] (5.221,7.660)--(5.401,7.660);
\draw[gp path] (5.221,8.145)--(5.401,8.145);
\draw[gp path] (5.414,7.660)--(5.414,8.145);
\draw[gp path] (5.324,7.660)--(5.504,7.660);
\draw[gp path] (5.324,8.145)--(5.504,8.145);
\draw[gp path] (5.518,7.603)--(5.518,8.084);
\draw[gp path] (5.428,7.603)--(5.608,7.603);
\draw[gp path] (5.428,8.084)--(5.608,8.084);
\draw[gp path] (5.621,7.432)--(5.621,7.901);
\draw[gp path] (5.531,7.432)--(5.711,7.432);
\draw[gp path] (5.531,7.901)--(5.711,7.901);
\draw[gp path] (5.725,7.203)--(5.725,7.657);
\draw[gp path] (5.635,7.203)--(5.815,7.203);
\draw[gp path] (5.635,7.657)--(5.815,7.657);
\draw[gp path] (5.828,6.803)--(5.828,7.229);
\draw[gp path] (5.738,6.803)--(5.918,6.803);
\draw[gp path] (5.738,7.229)--(5.918,7.229);
\draw[gp path] (5.932,6.402)--(5.932,6.802);
\draw[gp path] (5.842,6.402)--(6.022,6.402);
\draw[gp path] (5.842,6.802)--(6.022,6.802);
\draw[gp path] (6.035,5.945)--(6.035,6.314);
\draw[gp path] (5.945,5.945)--(6.125,5.945);
\draw[gp path] (5.945,6.314)--(6.125,6.314);
\draw[gp path] (6.139,5.429)--(6.139,5.764);
\draw[gp path] (6.049,5.429)--(6.229,5.429);
\draw[gp path] (6.049,5.764)--(6.229,5.764);
\draw[gp path] (6.242,4.799)--(6.242,5.093);
\draw[gp path] (6.152,4.799)--(6.332,4.799);
\draw[gp path] (6.152,5.093)--(6.332,5.093);
\draw[gp path] (6.346,4.284)--(6.346,4.545);
\draw[gp path] (6.256,4.284)--(6.436,4.284);
\draw[gp path] (6.256,4.545)--(6.436,4.545);
\draw[gp path] (6.449,3.595)--(6.449,3.815);
\draw[gp path] (6.359,3.595)--(6.539,3.595);
\draw[gp path] (6.359,3.815)--(6.539,3.815);
\draw[gp path] (6.553,3.193)--(6.553,3.389);
\draw[gp path] (6.463,3.193)--(6.643,3.193);
\draw[gp path] (6.463,3.389)--(6.643,3.389);
\draw[gp path] (6.656,2.559)--(6.656,2.722);
\draw[gp path] (6.566,2.559)--(6.746,2.559);
\draw[gp path] (6.566,2.722)--(6.746,2.722);
\draw[gp path] (6.760,2.154)--(6.760,2.299);
\draw[gp path] (6.670,2.154)--(6.850,2.154);
\draw[gp path] (6.670,2.299)--(6.850,2.299);
\draw[gp path] (6.864,1.689)--(6.864,1.818);
\draw[gp path] (6.774,1.689)--(6.954,1.689);
\draw[gp path] (6.774,1.818)--(6.954,1.818);
\draw[gp path] (6.967,1.397)--(6.967,1.519);
\draw[gp path] (6.877,1.397)--(7.057,1.397);
\draw[gp path] (6.877,1.519)--(7.057,1.519);
\draw[gp path] (7.071,1.162)--(7.071,1.281);
\draw[gp path] (6.981,1.162)--(7.161,1.162);
\draw[gp path] (6.981,1.281)--(7.161,1.281);
\draw[gp path] (7.174,1.044)--(7.174,1.163);
\draw[gp path] (7.084,1.044)--(7.264,1.044);
\draw[gp path] (7.084,1.163)--(7.264,1.163);
\draw[gp path] (7.278,0.985)--(7.278,1.091);
\draw[gp path] (7.188,0.985)--(7.368,0.985);
\draw[gp path] (7.188,1.091)--(7.368,1.091);
\draw[gp path] (7.381,1.038)--(7.381,1.157);
\draw[gp path] (7.291,1.038)--(7.471,1.038);
\draw[gp path] (7.291,1.157)--(7.471,1.157);
\draw[gp path] (7.485,1.221)--(7.485,1.341);
\draw[gp path] (7.395,1.221)--(7.575,1.221);
\draw[gp path] (7.395,1.341)--(7.575,1.341);
\draw[gp path] (7.588,1.455)--(7.588,1.579);
\draw[gp path] (7.498,1.455)--(7.678,1.455);
\draw[gp path] (7.498,1.579)--(7.678,1.579);
\draw[gp path] (7.692,1.864)--(7.692,1.998);
\draw[gp path] (7.602,1.864)--(7.782,1.864);
\draw[gp path] (7.602,1.998)--(7.782,1.998);
\draw[gp path] (7.795,2.212)--(7.795,2.360);
\draw[gp path] (7.705,2.212)--(7.885,2.212);
\draw[gp path] (7.705,2.360)--(7.885,2.360);
\draw[gp path] (7.899,2.732)--(7.899,2.904);
\draw[gp path] (7.809,2.732)--(7.989,2.732);
\draw[gp path] (7.809,2.904)--(7.989,2.904);
\draw[gp path] (8.002,3.250)--(8.002,3.450);
\draw[gp path] (7.912,3.250)--(8.092,3.250);
\draw[gp path] (7.912,3.450)--(8.092,3.450);
\draw[gp path] (8.106,3.825)--(8.106,4.058);
\draw[gp path] (8.016,3.825)--(8.196,3.825);
\draw[gp path] (8.016,4.058)--(8.196,4.058);
\draw[gp path] (8.209,4.398)--(8.209,4.667);
\draw[gp path] (8.119,4.398)--(8.299,4.398);
\draw[gp path] (8.119,4.667)--(8.299,4.667);
\draw[gp path] (8.313,5.029)--(8.313,5.337);
\draw[gp path] (8.223,5.029)--(8.403,5.029);
\draw[gp path] (8.223,5.337)--(8.403,5.337);
\draw[gp path] (8.416,5.544)--(8.416,5.886);
\draw[gp path] (8.326,5.544)--(8.506,5.544);
\draw[gp path] (8.326,5.886)--(8.506,5.886);
\draw[gp path] (8.520,6.116)--(8.520,6.497);
\draw[gp path] (8.430,6.116)--(8.610,6.116);
\draw[gp path] (8.430,6.497)--(8.610,6.497);
\draw[gp path] (8.623,6.517)--(8.623,6.924);
\draw[gp path] (8.533,6.517)--(8.713,6.517);
\draw[gp path] (8.533,6.924)--(8.713,6.924);
\draw[gp path] (8.727,7.031)--(8.727,7.473);
\draw[gp path] (8.637,7.031)--(8.817,7.031);
\draw[gp path] (8.637,7.473)--(8.817,7.473);
\draw[gp path] (8.830,7.317)--(8.830,7.779);
\draw[gp path] (8.740,7.317)--(8.920,7.317);
\draw[gp path] (8.740,7.779)--(8.920,7.779);
\draw[gp path] (8.934,7.546)--(8.934,8.023);
\draw[gp path] (8.844,7.546)--(9.024,7.546);
\draw[gp path] (8.844,8.023)--(9.024,8.023);
\draw[gp path] (9.037,7.660)--(9.037,8.145);
\draw[gp path] (8.947,7.660)--(9.127,7.660);
\draw[gp path] (8.947,8.145)--(9.127,8.145);
\draw[gp path] (9.141,7.718)--(9.141,8.206);
\draw[gp path] (9.051,7.718)--(9.231,7.718);
\draw[gp path] (9.051,8.206)--(9.231,8.206);
\draw[gp path] (9.244,7.660)--(9.244,8.145);
\draw[gp path] (9.154,7.660)--(9.334,7.660);
\draw[gp path] (9.154,8.145)--(9.334,8.145);
\draw[gp path] (9.348,7.432)--(9.348,7.901);
\draw[gp path] (9.258,7.432)--(9.438,7.432);
\draw[gp path] (9.258,7.901)--(9.438,7.901);
\draw[gp path] (9.451,7.260)--(9.451,7.718);
\draw[gp path] (9.361,7.260)--(9.541,7.260);
\draw[gp path] (9.361,7.718)--(9.541,7.718);
\draw[gp path] (9.555,6.803)--(9.555,7.229);
\draw[gp path] (9.465,6.803)--(9.645,6.803);
\draw[gp path] (9.465,7.229)--(9.645,7.229);
\draw[gp path] (9.658,6.459)--(9.658,6.863);
\draw[gp path] (9.568,6.459)--(9.748,6.459);
\draw[gp path] (9.568,6.863)--(9.748,6.863);
\draw[gp path] (9.762,5.887)--(9.762,6.252);
\draw[gp path] (9.672,5.887)--(9.852,5.887);
\draw[gp path] (9.672,6.252)--(9.852,6.252);
\draw[gp path] (9.865,5.429)--(9.865,5.764);
\draw[gp path] (9.775,5.429)--(9.955,5.429);
\draw[gp path] (9.775,5.764)--(9.955,5.764);
\draw[gp path] (9.969,4.799)--(9.969,5.093);
\draw[gp path] (9.879,4.799)--(10.059,4.799);
\draw[gp path] (9.879,5.093)--(10.059,5.093);
\draw[gp path] (10.072,4.284)--(10.072,4.545);
\draw[gp path] (9.982,4.284)--(10.162,4.284);
\draw[gp path] (9.982,4.545)--(10.162,4.545);
\draw[gp path] (10.176,3.595)--(10.176,3.815);
\draw[gp path] (10.086,3.595)--(10.266,3.595);
\draw[gp path] (10.086,3.815)--(10.266,3.815);
\draw[gp path] (10.279,3.193)--(10.279,3.389);
\draw[gp path] (10.189,3.193)--(10.369,3.193);
\draw[gp path] (10.189,3.389)--(10.369,3.389);
\draw[gp path] (10.383,2.559)--(10.383,2.722);
\draw[gp path] (10.293,2.559)--(10.473,2.559);
\draw[gp path] (10.293,2.722)--(10.473,2.722);
\draw[gp path] (10.486,2.096)--(10.486,2.239);
\draw[gp path] (10.396,2.096)--(10.576,2.096);
\draw[gp path] (10.396,2.239)--(10.576,2.239);
\draw[gp path] (10.590,1.689)--(10.590,1.818);
\draw[gp path] (10.500,1.689)--(10.680,1.689);
\draw[gp path] (10.500,1.818)--(10.680,1.818);
\draw[gp path] (10.693,1.397)--(10.693,1.519);
\draw[gp path] (10.603,1.397)--(10.783,1.397);
\draw[gp path] (10.603,1.519)--(10.783,1.519);
\draw[gp path] (10.797,1.162)--(10.797,1.281);
\draw[gp path] (10.707,1.162)--(10.887,1.162);
\draw[gp path] (10.707,1.281)--(10.887,1.281);
\draw[gp path] (10.900,0.985)--(10.900,1.103);
\draw[gp path] (10.810,0.985)--(10.990,0.985);
\draw[gp path] (10.810,1.103)--(10.990,1.103);
\draw[gp path] (11.004,0.985)--(11.004,1.068);
\draw[gp path] (10.914,0.985)--(11.094,0.985);
\draw[gp path] (10.914,1.068)--(11.094,1.068);
\draw[gp path] (1.688,7.962)--(1.729,7.962);
\draw[gp path] (1.688,7.872)--(1.688,8.052);
\draw[gp path] (1.729,7.872)--(1.729,8.052);
\draw[gp path] (1.750,7.785)--(1.833,7.785);
\draw[gp path] (1.750,7.695)--(1.750,7.875);
\draw[gp path] (1.833,7.695)--(1.833,7.875);
\draw[gp path] (1.854,7.666)--(1.936,7.666);
\draw[gp path] (1.854,7.576)--(1.854,7.756);
\draw[gp path] (1.936,7.576)--(1.936,7.756);
\draw[gp path] (1.957,7.312)--(2.040,7.312);
\draw[gp path] (1.957,7.222)--(1.957,7.402);
\draw[gp path] (2.040,7.222)--(2.040,7.402);
\draw[gp path] (2.061,6.957)--(2.143,6.957);
\draw[gp path] (2.061,6.867)--(2.061,7.047);
\draw[gp path] (2.143,6.867)--(2.143,7.047);
\draw[gp path] (2.164,6.543)--(2.247,6.543);
\draw[gp path] (2.164,6.453)--(2.164,6.633);
\draw[gp path] (2.247,6.453)--(2.247,6.633);
\draw[gp path] (2.268,6.011)--(2.350,6.011);
\draw[gp path] (2.268,5.921)--(2.268,6.101);
\draw[gp path] (2.350,5.921)--(2.350,6.101);
\draw[gp path] (2.371,5.420)--(2.454,5.420);
\draw[gp path] (2.371,5.330)--(2.371,5.510);
\draw[gp path] (2.454,5.330)--(2.454,5.510);
\draw[gp path] (2.475,4.828)--(2.557,4.828);
\draw[gp path] (2.475,4.738)--(2.475,4.918);
\draw[gp path] (2.557,4.738)--(2.557,4.918);
\draw[gp path] (2.578,4.237)--(2.661,4.237);
\draw[gp path] (2.578,4.147)--(2.578,4.327);
\draw[gp path] (2.661,4.147)--(2.661,4.327);
\draw[gp path] (2.682,3.705)--(2.765,3.705);
\draw[gp path] (2.682,3.615)--(2.682,3.795);
\draw[gp path] (2.765,3.615)--(2.765,3.795);
\draw[gp path] (2.785,3.114)--(2.868,3.114);
\draw[gp path] (2.785,3.024)--(2.785,3.204);
\draw[gp path] (2.868,3.024)--(2.868,3.204);
\draw[gp path] (2.889,2.522)--(2.972,2.522);
\draw[gp path] (2.889,2.432)--(2.889,2.612);
\draw[gp path] (2.972,2.432)--(2.972,2.612);
\draw[gp path] (2.992,2.168)--(3.075,2.168);
\draw[gp path] (2.992,2.078)--(2.992,2.258);
\draw[gp path] (3.075,2.078)--(3.075,2.258);
\draw[gp path] (3.096,1.754)--(3.179,1.754);
\draw[gp path] (3.096,1.664)--(3.096,1.844);
\draw[gp path] (3.179,1.664)--(3.179,1.844);
\draw[gp path] (3.199,1.399)--(3.282,1.399);
\draw[gp path] (3.199,1.309)--(3.199,1.489);
\draw[gp path] (3.282,1.309)--(3.282,1.489);
\draw[gp path] (3.303,1.162)--(3.386,1.162);
\draw[gp path] (3.303,1.072)--(3.303,1.252);
\draw[gp path] (3.386,1.072)--(3.386,1.252);
\draw[gp path] (3.406,1.032)--(3.489,1.032);
\draw[gp path] (3.406,0.942)--(3.406,1.122);
\draw[gp path] (3.489,0.942)--(3.489,1.122);
\draw[gp path] (3.510,1.005)--(3.593,1.005);
\draw[gp path] (3.510,0.915)--(3.510,1.095);
\draw[gp path] (3.593,0.915)--(3.593,1.095);
\draw[gp path] (3.613,1.074)--(3.696,1.074);
\draw[gp path] (3.613,0.984)--(3.613,1.164);
\draw[gp path] (3.696,0.984)--(3.696,1.164);
\draw[gp path] (3.717,1.269)--(3.800,1.269);
\draw[gp path] (3.717,1.179)--(3.717,1.359);
\draw[gp path] (3.800,1.179)--(3.800,1.359);
\draw[gp path] (3.820,1.505)--(3.903,1.505);
\draw[gp path] (3.820,1.415)--(3.820,1.595);
\draw[gp path] (3.903,1.415)--(3.903,1.595);
\draw[gp path] (3.924,1.931)--(4.007,1.931);
\draw[gp path] (3.924,1.841)--(3.924,2.021);
\draw[gp path] (4.007,1.841)--(4.007,2.021);
\draw[gp path] (4.027,2.345)--(4.110,2.345);
\draw[gp path] (4.027,2.255)--(4.027,2.435);
\draw[gp path] (4.110,2.255)--(4.110,2.435);
\draw[gp path] (4.131,2.877)--(4.214,2.877);
\draw[gp path] (4.131,2.787)--(4.131,2.967);
\draw[gp path] (4.214,2.787)--(4.214,2.967);
\draw[gp path] (4.234,3.409)--(4.317,3.409);
\draw[gp path] (4.234,3.319)--(4.234,3.499);
\draw[gp path] (4.317,3.319)--(4.317,3.499);
\draw[gp path] (4.338,4.000)--(4.421,4.000);
\draw[gp path] (4.338,3.910)--(4.338,4.090);
\draw[gp path] (4.421,3.910)--(4.421,4.090);
\draw[gp path] (4.441,4.592)--(4.524,4.592);
\draw[gp path] (4.441,4.502)--(4.441,4.682);
\draw[gp path] (4.524,4.502)--(4.524,4.682);
\draw[gp path] (4.545,5.242)--(4.628,5.242);
\draw[gp path] (4.545,5.152)--(4.545,5.332);
\draw[gp path] (4.628,5.152)--(4.628,5.332);
\draw[gp path] (4.648,5.715)--(4.731,5.715);
\draw[gp path] (4.648,5.625)--(4.648,5.805);
\draw[gp path] (4.731,5.625)--(4.731,5.805);
\draw[gp path] (4.752,6.306)--(4.835,6.306);
\draw[gp path] (4.752,6.216)--(4.752,6.396);
\draw[gp path] (4.835,6.216)--(4.835,6.396);
\draw[gp path] (4.855,6.720)--(4.938,6.720);
\draw[gp path] (4.855,6.630)--(4.855,6.810);
\draw[gp path] (4.938,6.630)--(4.938,6.810);
\draw[gp path] (4.959,7.193)--(5.042,7.193);
\draw[gp path] (4.959,7.103)--(4.959,7.283);
\draw[gp path] (5.042,7.103)--(5.042,7.283);
\draw[gp path] (5.062,7.489)--(5.145,7.489);
\draw[gp path] (5.062,7.399)--(5.062,7.579);
\draw[gp path] (5.145,7.399)--(5.145,7.579);
\draw[gp path] (5.166,7.785)--(5.249,7.785);
\draw[gp path] (5.166,7.695)--(5.166,7.875);
\draw[gp path] (5.249,7.695)--(5.249,7.875);
\draw[gp path] (5.269,7.903)--(5.352,7.903);
\draw[gp path] (5.269,7.813)--(5.269,7.993);
\draw[gp path] (5.352,7.813)--(5.352,7.993);
\draw[gp path] (5.373,7.903)--(5.456,7.903);
\draw[gp path] (5.373,7.813)--(5.373,7.993);
\draw[gp path] (5.456,7.813)--(5.456,7.993);
\draw[gp path] (5.476,7.844)--(5.559,7.844);
\draw[gp path] (5.476,7.754)--(5.476,7.934);
\draw[gp path] (5.559,7.754)--(5.559,7.934);
\draw[gp path] (5.580,7.666)--(5.663,7.666);
\draw[gp path] (5.580,7.576)--(5.580,7.756);
\draw[gp path] (5.663,7.576)--(5.663,7.756);
\draw[gp path] (5.683,7.430)--(5.766,7.430);
\draw[gp path] (5.683,7.340)--(5.683,7.520);
\draw[gp path] (5.766,7.340)--(5.766,7.520);
\draw[gp path] (5.787,7.016)--(5.870,7.016);
\draw[gp path] (5.787,6.926)--(5.787,7.106);
\draw[gp path] (5.870,6.926)--(5.870,7.106);
\draw[gp path] (5.891,6.602)--(5.973,6.602);
\draw[gp path] (5.891,6.512)--(5.891,6.692);
\draw[gp path] (5.973,6.512)--(5.973,6.692);
\draw[gp path] (5.994,6.129)--(6.077,6.129);
\draw[gp path] (5.994,6.039)--(5.994,6.219);
\draw[gp path] (6.077,6.039)--(6.077,6.219);
\draw[gp path] (6.098,5.597)--(6.180,5.597);
\draw[gp path] (6.098,5.507)--(6.098,5.687);
\draw[gp path] (6.180,5.507)--(6.180,5.687);
\draw[gp path] (6.201,4.946)--(6.284,4.946);
\draw[gp path] (6.201,4.856)--(6.201,5.036);
\draw[gp path] (6.284,4.856)--(6.284,5.036);
\draw[gp path] (6.305,4.414)--(6.387,4.414);
\draw[gp path] (6.305,4.324)--(6.305,4.504);
\draw[gp path] (6.387,4.324)--(6.387,4.504);
\draw[gp path] (6.408,3.705)--(6.491,3.705);
\draw[gp path] (6.408,3.615)--(6.408,3.795);
\draw[gp path] (6.491,3.615)--(6.491,3.795);
\draw[gp path] (6.512,3.291)--(6.594,3.291);
\draw[gp path] (6.512,3.201)--(6.512,3.381);
\draw[gp path] (6.594,3.201)--(6.594,3.381);
\draw[gp path] (6.615,2.641)--(6.698,2.641);
\draw[gp path] (6.615,2.551)--(6.615,2.731);
\draw[gp path] (6.698,2.551)--(6.698,2.731);
\draw[gp path] (6.719,2.227)--(6.801,2.227);
\draw[gp path] (6.719,2.137)--(6.719,2.317);
\draw[gp path] (6.801,2.137)--(6.801,2.317);
\draw[gp path] (6.822,1.754)--(6.905,1.754);
\draw[gp path] (6.822,1.664)--(6.822,1.844);
\draw[gp path] (6.905,1.664)--(6.905,1.844);
\draw[gp path] (6.926,1.458)--(7.008,1.458);
\draw[gp path] (6.926,1.368)--(6.926,1.548);
\draw[gp path] (7.008,1.368)--(7.008,1.548);
\draw[gp path] (7.029,1.222)--(7.112,1.222);
\draw[gp path] (7.029,1.132)--(7.029,1.312);
\draw[gp path] (7.112,1.132)--(7.112,1.312);
\draw[gp path] (7.133,1.103)--(7.215,1.103);
\draw[gp path] (7.133,1.013)--(7.133,1.193);
\draw[gp path] (7.215,1.013)--(7.215,1.193);
\draw[gp path] (7.236,1.032)--(7.319,1.032);
\draw[gp path] (7.236,0.942)--(7.236,1.122);
\draw[gp path] (7.319,0.942)--(7.319,1.122);
\draw[gp path] (7.340,1.097)--(7.422,1.097);
\draw[gp path] (7.340,1.007)--(7.340,1.187);
\draw[gp path] (7.422,1.007)--(7.422,1.187);
\draw[gp path] (7.443,1.281)--(7.526,1.281);
\draw[gp path] (7.443,1.191)--(7.443,1.371);
\draw[gp path] (7.526,1.191)--(7.526,1.371);
\draw[gp path] (7.547,1.517)--(7.629,1.517);
\draw[gp path] (7.547,1.427)--(7.547,1.607);
\draw[gp path] (7.629,1.427)--(7.629,1.607);
\draw[gp path] (7.650,1.931)--(7.733,1.931);
\draw[gp path] (7.650,1.841)--(7.650,2.021);
\draw[gp path] (7.733,1.841)--(7.733,2.021);
\draw[gp path] (7.754,2.286)--(7.836,2.286);
\draw[gp path] (7.754,2.196)--(7.754,2.376);
\draw[gp path] (7.836,2.196)--(7.836,2.376);
\draw[gp path] (7.857,2.818)--(7.940,2.818);
\draw[gp path] (7.857,2.728)--(7.857,2.908);
\draw[gp path] (7.940,2.728)--(7.940,2.908);
\draw[gp path] (7.961,3.350)--(8.044,3.350);
\draw[gp path] (7.961,3.260)--(7.961,3.440);
\draw[gp path] (8.044,3.260)--(8.044,3.440);
\draw[gp path] (8.064,3.941)--(8.147,3.941);
\draw[gp path] (8.064,3.851)--(8.064,4.031);
\draw[gp path] (8.147,3.851)--(8.147,4.031);
\draw[gp path] (8.168,4.533)--(8.251,4.533);
\draw[gp path] (8.168,4.443)--(8.168,4.623);
\draw[gp path] (8.251,4.443)--(8.251,4.623);
\draw[gp path] (8.271,5.183)--(8.354,5.183);
\draw[gp path] (8.271,5.093)--(8.271,5.273);
\draw[gp path] (8.354,5.093)--(8.354,5.273);
\draw[gp path] (8.375,5.715)--(8.458,5.715);
\draw[gp path] (8.375,5.625)--(8.375,5.805);
\draw[gp path] (8.458,5.625)--(8.458,5.805);
\draw[gp path] (8.478,6.306)--(8.561,6.306);
\draw[gp path] (8.478,6.216)--(8.478,6.396);
\draw[gp path] (8.561,6.216)--(8.561,6.396);
\draw[gp path] (8.582,6.720)--(8.665,6.720);
\draw[gp path] (8.582,6.630)--(8.582,6.810);
\draw[gp path] (8.665,6.630)--(8.665,6.810);
\draw[gp path] (8.685,7.252)--(8.768,7.252);
\draw[gp path] (8.685,7.162)--(8.685,7.342);
\draw[gp path] (8.768,7.162)--(8.768,7.342);
\draw[gp path] (8.789,7.548)--(8.872,7.548);
\draw[gp path] (8.789,7.458)--(8.789,7.638);
\draw[gp path] (8.872,7.458)--(8.872,7.638);
\draw[gp path] (8.892,7.785)--(8.975,7.785);
\draw[gp path] (8.892,7.695)--(8.892,7.875);
\draw[gp path] (8.975,7.695)--(8.975,7.875);
\draw[gp path] (8.996,7.903)--(9.079,7.903);
\draw[gp path] (8.996,7.813)--(8.996,7.993);
\draw[gp path] (9.079,7.813)--(9.079,7.993);
\draw[gp path] (9.099,7.962)--(9.182,7.962);
\draw[gp path] (9.099,7.872)--(9.099,8.052);
\draw[gp path] (9.182,7.872)--(9.182,8.052);
\draw[gp path] (9.203,7.903)--(9.286,7.903);
\draw[gp path] (9.203,7.813)--(9.203,7.993);
\draw[gp path] (9.286,7.813)--(9.286,7.993);
\draw[gp path] (9.306,7.666)--(9.389,7.666);
\draw[gp path] (9.306,7.576)--(9.306,7.756);
\draw[gp path] (9.389,7.576)--(9.389,7.756);
\draw[gp path] (9.410,7.489)--(9.493,7.489);
\draw[gp path] (9.410,7.399)--(9.410,7.579);
\draw[gp path] (9.493,7.399)--(9.493,7.579);
\draw[gp path] (9.513,7.016)--(9.596,7.016);
\draw[gp path] (9.513,6.926)--(9.513,7.106);
\draw[gp path] (9.596,6.926)--(9.596,7.106);
\draw[gp path] (9.617,6.661)--(9.700,6.661);
\draw[gp path] (9.617,6.571)--(9.617,6.751);
\draw[gp path] (9.700,6.571)--(9.700,6.751);
\draw[gp path] (9.720,6.070)--(9.803,6.070);
\draw[gp path] (9.720,5.980)--(9.720,6.160);
\draw[gp path] (9.803,5.980)--(9.803,6.160);
\draw[gp path] (9.824,5.597)--(9.907,5.597);
\draw[gp path] (9.824,5.507)--(9.824,5.687);
\draw[gp path] (9.907,5.507)--(9.907,5.687);
\draw[gp path] (9.927,4.946)--(10.010,4.946);
\draw[gp path] (9.927,4.856)--(9.927,5.036);
\draw[gp path] (10.010,4.856)--(10.010,5.036);
\draw[gp path] (10.031,4.414)--(10.114,4.414);
\draw[gp path] (10.031,4.324)--(10.031,4.504);
\draw[gp path] (10.114,4.324)--(10.114,4.504);
\draw[gp path] (10.134,3.705)--(10.217,3.705);
\draw[gp path] (10.134,3.615)--(10.134,3.795);
\draw[gp path] (10.217,3.615)--(10.217,3.795);
\draw[gp path] (10.238,3.291)--(10.321,3.291);
\draw[gp path] (10.238,3.201)--(10.238,3.381);
\draw[gp path] (10.321,3.201)--(10.321,3.381);
\draw[gp path] (10.341,2.641)--(10.424,2.641);
\draw[gp path] (10.341,2.551)--(10.341,2.731);
\draw[gp path] (10.424,2.551)--(10.424,2.731);
\draw[gp path] (10.445,2.168)--(10.528,2.168);
\draw[gp path] (10.445,2.078)--(10.445,2.258);
\draw[gp path] (10.528,2.078)--(10.528,2.258);
\draw[gp path] (10.548,1.754)--(10.631,1.754);
\draw[gp path] (10.548,1.664)--(10.548,1.844);
\draw[gp path] (10.631,1.664)--(10.631,1.844);
\draw[gp path] (10.652,1.458)--(10.735,1.458);
\draw[gp path] (10.652,1.368)--(10.652,1.548);
\draw[gp path] (10.735,1.368)--(10.735,1.548);
\draw[gp path] (10.755,1.222)--(10.838,1.222);
\draw[gp path] (10.755,1.132)--(10.755,1.312);
\draw[gp path] (10.838,1.132)--(10.838,1.312);
\draw[gp path] (10.859,1.044)--(10.942,1.044);
\draw[gp path] (10.859,0.954)--(10.859,1.134);
\draw[gp path] (10.942,0.954)--(10.942,1.134);
\draw[gp path] (10.962,1.009)--(11.045,1.009);
\draw[gp path] (10.962,0.919)--(10.962,1.099);
\draw[gp path] (11.045,0.919)--(11.045,1.099);
\gpsetpointsize{4.00}
\gppoint{gp mark 1}{(1.688,7.962)}
\gppoint{gp mark 1}{(1.792,7.785)}
\gppoint{gp mark 1}{(1.895,7.666)}
\gppoint{gp mark 1}{(1.999,7.312)}
\gppoint{gp mark 1}{(2.102,6.957)}
\gppoint{gp mark 1}{(2.206,6.543)}
\gppoint{gp mark 1}{(2.309,6.011)}
\gppoint{gp mark 1}{(2.413,5.420)}
\gppoint{gp mark 1}{(2.516,4.828)}
\gppoint{gp mark 1}{(2.620,4.237)}
\gppoint{gp mark 1}{(2.723,3.705)}
\gppoint{gp mark 1}{(2.827,3.114)}
\gppoint{gp mark 1}{(2.930,2.522)}
\gppoint{gp mark 1}{(3.034,2.168)}
\gppoint{gp mark 1}{(3.137,1.754)}
\gppoint{gp mark 1}{(3.241,1.399)}
\gppoint{gp mark 1}{(3.344,1.162)}
\gppoint{gp mark 1}{(3.448,1.032)}
\gppoint{gp mark 1}{(3.551,1.005)}
\gppoint{gp mark 1}{(3.655,1.074)}
\gppoint{gp mark 1}{(3.758,1.269)}
\gppoint{gp mark 1}{(3.862,1.505)}
\gppoint{gp mark 1}{(3.965,1.931)}
\gppoint{gp mark 1}{(4.069,2.345)}
\gppoint{gp mark 1}{(4.172,2.877)}
\gppoint{gp mark 1}{(4.276,3.409)}
\gppoint{gp mark 1}{(4.379,4.000)}
\gppoint{gp mark 1}{(4.483,4.592)}
\gppoint{gp mark 1}{(4.586,5.242)}
\gppoint{gp mark 1}{(4.690,5.715)}
\gppoint{gp mark 1}{(4.793,6.306)}
\gppoint{gp mark 1}{(4.897,6.720)}
\gppoint{gp mark 1}{(5.000,7.193)}
\gppoint{gp mark 1}{(5.104,7.489)}
\gppoint{gp mark 1}{(5.207,7.785)}
\gppoint{gp mark 1}{(5.311,7.903)}
\gppoint{gp mark 1}{(5.414,7.903)}
\gppoint{gp mark 1}{(5.518,7.844)}
\gppoint{gp mark 1}{(5.621,7.666)}
\gppoint{gp mark 1}{(5.725,7.430)}
\gppoint{gp mark 1}{(5.828,7.016)}
\gppoint{gp mark 1}{(5.932,6.602)}
\gppoint{gp mark 1}{(6.035,6.129)}
\gppoint{gp mark 1}{(6.139,5.597)}
\gppoint{gp mark 1}{(6.242,4.946)}
\gppoint{gp mark 1}{(6.346,4.414)}
\gppoint{gp mark 1}{(6.449,3.705)}
\gppoint{gp mark 1}{(6.553,3.291)}
\gppoint{gp mark 1}{(6.656,2.641)}
\gppoint{gp mark 1}{(6.760,2.227)}
\gppoint{gp mark 1}{(6.864,1.754)}
\gppoint{gp mark 1}{(6.967,1.458)}
\gppoint{gp mark 1}{(7.071,1.222)}
\gppoint{gp mark 1}{(7.174,1.103)}
\gppoint{gp mark 1}{(7.278,1.032)}
\gppoint{gp mark 1}{(7.381,1.097)}
\gppoint{gp mark 1}{(7.485,1.281)}
\gppoint{gp mark 1}{(7.588,1.517)}
\gppoint{gp mark 1}{(7.692,1.931)}
\gppoint{gp mark 1}{(7.795,2.286)}
\gppoint{gp mark 1}{(7.899,2.818)}
\gppoint{gp mark 1}{(8.002,3.350)}
\gppoint{gp mark 1}{(8.106,3.941)}
\gppoint{gp mark 1}{(8.209,4.533)}
\gppoint{gp mark 1}{(8.313,5.183)}
\gppoint{gp mark 1}{(8.416,5.715)}
\gppoint{gp mark 1}{(8.520,6.306)}
\gppoint{gp mark 1}{(8.623,6.720)}
\gppoint{gp mark 1}{(8.727,7.252)}
\gppoint{gp mark 1}{(8.830,7.548)}
\gppoint{gp mark 1}{(8.934,7.785)}
\gppoint{gp mark 1}{(9.037,7.903)}
\gppoint{gp mark 1}{(9.141,7.962)}
\gppoint{gp mark 1}{(9.244,7.903)}
\gppoint{gp mark 1}{(9.348,7.666)}
\gppoint{gp mark 1}{(9.451,7.489)}
\gppoint{gp mark 1}{(9.555,7.016)}
\gppoint{gp mark 1}{(9.658,6.661)}
\gppoint{gp mark 1}{(9.762,6.070)}
\gppoint{gp mark 1}{(9.865,5.597)}
\gppoint{gp mark 1}{(9.969,4.946)}
\gppoint{gp mark 1}{(10.072,4.414)}
\gppoint{gp mark 1}{(10.176,3.705)}
\gppoint{gp mark 1}{(10.279,3.291)}
\gppoint{gp mark 1}{(10.383,2.641)}
\gppoint{gp mark 1}{(10.486,2.168)}
\gppoint{gp mark 1}{(10.590,1.754)}
\gppoint{gp mark 1}{(10.693,1.458)}
\gppoint{gp mark 1}{(10.797,1.222)}
\gppoint{gp mark 1}{(10.900,1.044)}
\gppoint{gp mark 1}{(11.004,1.009)}
\gppoint{gp mark 1}{(11.213,8.047)}
\gpcolor{gp lt color border}
\node[gp node right] at (10.571,7.739) {Fit};
\gpsetlinetype{gp lt border}
\draw[gp path] (10.755,7.739)--(11.671,7.739);
\draw[gp path] (1.688,7.922)--(1.691,7.922)--(1.693,7.922)--(1.696,7.922)--(1.698,7.921)%
  --(1.701,7.921)--(1.704,7.921)--(1.706,7.920)--(1.709,7.920)--(1.711,7.919)--(1.714,7.919)%
  --(1.716,7.918)--(1.719,7.917)--(1.722,7.916)--(1.724,7.915)--(1.727,7.915)--(1.729,7.914)%
  --(1.732,7.912)--(1.735,7.911)--(1.737,7.910)--(1.740,7.909)--(1.742,7.907)--(1.745,7.906)%
  --(1.748,7.904)--(1.750,7.903)--(1.753,7.901)--(1.755,7.900)--(1.758,7.898)--(1.760,7.896)%
  --(1.763,7.894)--(1.766,7.892)--(1.768,7.890)--(1.771,7.888)--(1.773,7.886)--(1.776,7.884)%
  --(1.779,7.881)--(1.781,7.879)--(1.784,7.877)--(1.786,7.874)--(1.789,7.872)--(1.792,7.869)%
  --(1.794,7.866)--(1.797,7.864)--(1.799,7.861)--(1.802,7.858)--(1.804,7.855)--(1.807,7.852)%
  --(1.810,7.849)--(1.812,7.846)--(1.815,7.843)--(1.817,7.839)--(1.820,7.836)--(1.823,7.833)%
  --(1.825,7.829)--(1.828,7.826)--(1.830,7.822)--(1.833,7.818)--(1.836,7.815)--(1.838,7.811)%
  --(1.841,7.807)--(1.843,7.803)--(1.846,7.799)--(1.848,7.795)--(1.851,7.791)--(1.854,7.787)%
  --(1.856,7.783)--(1.859,7.778)--(1.861,7.774)--(1.864,7.770)--(1.867,7.765)--(1.869,7.761)%
  --(1.872,7.756)--(1.874,7.751)--(1.877,7.747)--(1.880,7.742)--(1.882,7.737)--(1.885,7.732)%
  --(1.887,7.727)--(1.890,7.722)--(1.892,7.717)--(1.895,7.712)--(1.898,7.706)--(1.900,7.701)%
  --(1.903,7.696)--(1.905,7.690)--(1.908,7.685)--(1.911,7.679)--(1.913,7.674)--(1.916,7.668)%
  --(1.918,7.662)--(1.921,7.657)--(1.924,7.651)--(1.926,7.645)--(1.929,7.639)--(1.931,7.633)%
  --(1.934,7.627)--(1.936,7.621)--(1.939,7.614)--(1.942,7.608)--(1.944,7.602)--(1.947,7.595)%
  --(1.949,7.589)--(1.952,7.582)--(1.955,7.576)--(1.957,7.569)--(1.960,7.562)--(1.962,7.556)%
  --(1.965,7.549)--(1.968,7.542)--(1.970,7.535)--(1.973,7.528)--(1.975,7.521)--(1.978,7.514)%
  --(1.980,7.507)--(1.983,7.500)--(1.986,7.492)--(1.988,7.485)--(1.991,7.477)--(1.993,7.470)%
  --(1.996,7.462)--(1.999,7.455)--(2.001,7.447)--(2.004,7.440)--(2.006,7.432)--(2.009,7.424)%
  --(2.012,7.416)--(2.014,7.408)--(2.017,7.400)--(2.019,7.392)--(2.022,7.384)--(2.024,7.376)%
  --(2.027,7.368)--(2.030,7.360)--(2.032,7.351)--(2.035,7.343)--(2.037,7.334)--(2.040,7.326)%
  --(2.043,7.317)--(2.045,7.309)--(2.048,7.300)--(2.050,7.292)--(2.053,7.283)--(2.056,7.274)%
  --(2.058,7.265)--(2.061,7.256)--(2.063,7.247)--(2.066,7.238)--(2.068,7.229)--(2.071,7.220)%
  --(2.074,7.211)--(2.076,7.202)--(2.079,7.192)--(2.081,7.183)--(2.084,7.174)--(2.087,7.164)%
  --(2.089,7.155)--(2.092,7.145)--(2.094,7.136)--(2.097,7.126)--(2.100,7.116)--(2.102,7.106)%
  --(2.105,7.097)--(2.107,7.087)--(2.110,7.077)--(2.112,7.067)--(2.115,7.057)--(2.118,7.047)%
  --(2.120,7.037)--(2.123,7.027)--(2.125,7.016)--(2.128,7.006)--(2.131,6.996)--(2.133,6.986)%
  --(2.136,6.975)--(2.138,6.965)--(2.141,6.954)--(2.144,6.944)--(2.146,6.933)--(2.149,6.922)%
  --(2.151,6.912)--(2.154,6.901)--(2.156,6.890)--(2.159,6.879)--(2.162,6.869)--(2.164,6.858)%
  --(2.167,6.847)--(2.169,6.836)--(2.172,6.825)--(2.175,6.814)--(2.177,6.802)--(2.180,6.791)%
  --(2.182,6.780)--(2.185,6.769)--(2.188,6.757)--(2.190,6.746)--(2.193,6.735)--(2.195,6.723)%
  --(2.198,6.712)--(2.201,6.700)--(2.203,6.689)--(2.206,6.677)--(2.208,6.665)--(2.211,6.654)%
  --(2.213,6.642)--(2.216,6.630)--(2.219,6.618)--(2.221,6.606)--(2.224,6.594)--(2.226,6.582)%
  --(2.229,6.570)--(2.232,6.558)--(2.234,6.546)--(2.237,6.534)--(2.239,6.522)--(2.242,6.510)%
  --(2.245,6.498)--(2.247,6.485)--(2.250,6.473)--(2.252,6.461)--(2.255,6.448)--(2.257,6.436)%
  --(2.260,6.423)--(2.263,6.411)--(2.265,6.398)--(2.268,6.386)--(2.270,6.373)--(2.273,6.360)%
  --(2.276,6.348)--(2.278,6.335)--(2.281,6.322)--(2.283,6.309)--(2.286,6.297)--(2.289,6.284)%
  --(2.291,6.271)--(2.294,6.258)--(2.296,6.245)--(2.299,6.232)--(2.301,6.219)--(2.304,6.206)%
  --(2.307,6.193)--(2.309,6.179)--(2.312,6.166)--(2.314,6.153)--(2.317,6.140)--(2.320,6.127)%
  --(2.322,6.113)--(2.325,6.100)--(2.327,6.087)--(2.330,6.073)--(2.333,6.060)--(2.335,6.046)%
  --(2.338,6.033)--(2.340,6.019)--(2.343,6.006)--(2.345,5.992)--(2.348,5.978)--(2.351,5.965)%
  --(2.353,5.951)--(2.356,5.937)--(2.358,5.924)--(2.361,5.910)--(2.364,5.896)--(2.366,5.882)%
  --(2.369,5.869)--(2.371,5.855)--(2.374,5.841)--(2.377,5.827)--(2.379,5.813)--(2.382,5.799)%
  --(2.384,5.785)--(2.387,5.771)--(2.389,5.757)--(2.392,5.743)--(2.395,5.729)--(2.397,5.715)%
  --(2.400,5.700)--(2.402,5.686)--(2.405,5.672)--(2.408,5.658)--(2.410,5.644)--(2.413,5.629)%
  --(2.415,5.615)--(2.418,5.601)--(2.421,5.586)--(2.423,5.572)--(2.426,5.558)--(2.428,5.543)%
  --(2.431,5.529)--(2.433,5.514)--(2.436,5.500)--(2.439,5.486)--(2.441,5.471)--(2.444,5.457)%
  --(2.446,5.442)--(2.449,5.427)--(2.452,5.413)--(2.454,5.398)--(2.457,5.384)--(2.459,5.369)%
  --(2.462,5.354)--(2.465,5.340)--(2.467,5.325)--(2.470,5.310)--(2.472,5.296)--(2.475,5.281)%
  --(2.477,5.266)--(2.480,5.251)--(2.483,5.237)--(2.485,5.222)--(2.488,5.207)--(2.490,5.192)%
  --(2.493,5.177)--(2.496,5.163)--(2.498,5.148)--(2.501,5.133)--(2.503,5.118)--(2.506,5.103)%
  --(2.509,5.088)--(2.511,5.073)--(2.514,5.058)--(2.516,5.043)--(2.519,5.028)--(2.521,5.013)%
  --(2.524,4.998)--(2.527,4.983)--(2.529,4.968)--(2.532,4.953)--(2.534,4.938)--(2.537,4.923)%
  --(2.540,4.908)--(2.542,4.893)--(2.545,4.878)--(2.547,4.863)--(2.550,4.848)--(2.553,4.833)%
  --(2.555,4.818)--(2.558,4.803)--(2.560,4.788)--(2.563,4.773)--(2.565,4.757)--(2.568,4.742)%
  --(2.571,4.727)--(2.573,4.712)--(2.576,4.697)--(2.578,4.682)--(2.581,4.667)--(2.584,4.652)%
  --(2.586,4.636)--(2.589,4.621)--(2.591,4.606)--(2.594,4.591)--(2.597,4.576)--(2.599,4.561)%
  --(2.602,4.545)--(2.604,4.530)--(2.607,4.515)--(2.609,4.500)--(2.612,4.485)--(2.615,4.470)%
  --(2.617,4.454)--(2.620,4.439)--(2.622,4.424)--(2.625,4.409)--(2.628,4.394)--(2.630,4.379)%
  --(2.633,4.363)--(2.635,4.348)--(2.638,4.333)--(2.641,4.318)--(2.643,4.303)--(2.646,4.288)%
  --(2.648,4.272)--(2.651,4.257)--(2.653,4.242)--(2.656,4.227)--(2.659,4.212)--(2.661,4.197)%
  --(2.664,4.182)--(2.666,4.166)--(2.669,4.151)--(2.672,4.136)--(2.674,4.121)--(2.677,4.106)%
  --(2.679,4.091)--(2.682,4.076)--(2.685,4.061)--(2.687,4.046)--(2.690,4.031)--(2.692,4.016)%
  --(2.695,4.000)--(2.697,3.985)--(2.700,3.970)--(2.703,3.955)--(2.705,3.940)--(2.708,3.925)%
  --(2.710,3.910)--(2.713,3.895)--(2.716,3.880)--(2.718,3.865)--(2.721,3.851)--(2.723,3.836)%
  --(2.726,3.821)--(2.729,3.806)--(2.731,3.791)--(2.734,3.776)--(2.736,3.761)--(2.739,3.746)%
  --(2.741,3.731)--(2.744,3.717)--(2.747,3.702)--(2.749,3.687)--(2.752,3.672)--(2.754,3.657)%
  --(2.757,3.643)--(2.760,3.628)--(2.762,3.613)--(2.765,3.598)--(2.767,3.584)--(2.770,3.569)%
  --(2.773,3.554)--(2.775,3.540)--(2.778,3.525)--(2.780,3.510)--(2.783,3.496)--(2.785,3.481)%
  --(2.788,3.467)--(2.791,3.452)--(2.793,3.438)--(2.796,3.423)--(2.798,3.409)--(2.801,3.394)%
  --(2.804,3.380)--(2.806,3.365)--(2.809,3.351)--(2.811,3.337)--(2.814,3.322)--(2.817,3.308)%
  --(2.819,3.294)--(2.822,3.279)--(2.824,3.265)--(2.827,3.251)--(2.829,3.237)--(2.832,3.222)%
  --(2.835,3.208)--(2.837,3.194)--(2.840,3.180)--(2.842,3.166)--(2.845,3.152)--(2.848,3.138)%
  --(2.850,3.124)--(2.853,3.110)--(2.855,3.096)--(2.858,3.082)--(2.861,3.068)--(2.863,3.054)%
  --(2.866,3.040)--(2.868,3.026)--(2.871,3.012)--(2.873,2.999)--(2.876,2.985)--(2.879,2.971)%
  --(2.881,2.957)--(2.884,2.944)--(2.886,2.930)--(2.889,2.916)--(2.892,2.903)--(2.894,2.889)%
  --(2.897,2.876)--(2.899,2.862)--(2.902,2.849)--(2.905,2.835)--(2.907,2.822)--(2.910,2.809)%
  --(2.912,2.795)--(2.915,2.782)--(2.917,2.769)--(2.920,2.755)--(2.923,2.742)--(2.925,2.729)%
  --(2.928,2.716)--(2.930,2.703)--(2.933,2.690)--(2.936,2.677)--(2.938,2.664)--(2.941,2.651)%
  --(2.943,2.638)--(2.946,2.625)--(2.949,2.612)--(2.951,2.599)--(2.954,2.586)--(2.956,2.574)%
  --(2.959,2.561)--(2.961,2.548)--(2.964,2.535)--(2.967,2.523)--(2.969,2.510)--(2.972,2.498)%
  --(2.974,2.485)--(2.977,2.473)--(2.980,2.460)--(2.982,2.448)--(2.985,2.435)--(2.987,2.423)%
  --(2.990,2.411)--(2.993,2.399)--(2.995,2.386)--(2.998,2.374)--(3.000,2.362)--(3.003,2.350)%
  --(3.005,2.338)--(3.008,2.326)--(3.011,2.314)--(3.013,2.302)--(3.016,2.290)--(3.018,2.278)%
  --(3.021,2.267)--(3.024,2.255)--(3.026,2.243)--(3.029,2.231)--(3.031,2.220)--(3.034,2.208)%
  --(3.037,2.197)--(3.039,2.185)--(3.042,2.174)--(3.044,2.162)--(3.047,2.151)--(3.049,2.140)%
  --(3.052,2.128)--(3.055,2.117)--(3.057,2.106)--(3.060,2.095)--(3.062,2.084)--(3.065,2.073)%
  --(3.068,2.062)--(3.070,2.051)--(3.073,2.040)--(3.075,2.029)--(3.078,2.018)--(3.081,2.007)%
  --(3.083,1.996)--(3.086,1.986)--(3.088,1.975)--(3.091,1.965)--(3.093,1.954)--(3.096,1.944)%
  --(3.099,1.933)--(3.101,1.923)--(3.104,1.912)--(3.106,1.902)--(3.109,1.892)--(3.112,1.882)%
  --(3.114,1.871)--(3.117,1.861)--(3.119,1.851)--(3.122,1.841)--(3.125,1.831)--(3.127,1.821)%
  --(3.130,1.812)--(3.132,1.802)--(3.135,1.792)--(3.138,1.782)--(3.140,1.773)--(3.143,1.763)%
  --(3.145,1.753)--(3.148,1.744)--(3.150,1.735)--(3.153,1.725)--(3.156,1.716)--(3.158,1.706)%
  --(3.161,1.697)--(3.163,1.688)--(3.166,1.679)--(3.169,1.670)--(3.171,1.661)--(3.174,1.652)%
  --(3.176,1.643)--(3.179,1.634)--(3.182,1.625)--(3.184,1.616)--(3.187,1.608)--(3.189,1.599)%
  --(3.192,1.591)--(3.194,1.582)--(3.197,1.574)--(3.200,1.565)--(3.202,1.557)--(3.205,1.548)%
  --(3.207,1.540)--(3.210,1.532)--(3.213,1.524)--(3.215,1.516)--(3.218,1.508)--(3.220,1.500)%
  --(3.223,1.492)--(3.226,1.484)--(3.228,1.476)--(3.231,1.468)--(3.233,1.461)--(3.236,1.453)%
  --(3.238,1.445)--(3.241,1.438)--(3.244,1.430)--(3.246,1.423)--(3.249,1.416)--(3.251,1.408)%
  --(3.254,1.401)--(3.257,1.394)--(3.259,1.387)--(3.262,1.380)--(3.264,1.373)--(3.267,1.366)%
  --(3.270,1.359)--(3.272,1.352)--(3.275,1.345)--(3.277,1.339)--(3.280,1.332)--(3.282,1.325)%
  --(3.285,1.319)--(3.288,1.312)--(3.290,1.306)--(3.293,1.300)--(3.295,1.293)--(3.298,1.287)%
  --(3.301,1.281)--(3.303,1.275)--(3.306,1.269)--(3.308,1.263)--(3.311,1.257)--(3.314,1.251)%
  --(3.316,1.245)--(3.319,1.240)--(3.321,1.234)--(3.324,1.228)--(3.326,1.223)--(3.329,1.217)%
  --(3.332,1.212)--(3.334,1.206)--(3.337,1.201)--(3.339,1.196)--(3.342,1.191)--(3.345,1.186)%
  --(3.347,1.181)--(3.350,1.176)--(3.352,1.171)--(3.355,1.166)--(3.358,1.161)--(3.360,1.156)%
  --(3.363,1.152)--(3.365,1.147)--(3.368,1.142)--(3.370,1.138)--(3.373,1.133)--(3.376,1.129)%
  --(3.378,1.125)--(3.381,1.121)--(3.383,1.116)--(3.386,1.112)--(3.389,1.108)--(3.391,1.104)%
  --(3.394,1.100)--(3.396,1.097)--(3.399,1.093)--(3.402,1.089)--(3.404,1.085)--(3.407,1.082)%
  --(3.409,1.078)--(3.412,1.075)--(3.414,1.071)--(3.417,1.068)--(3.420,1.065)--(3.422,1.062)%
  --(3.425,1.058)--(3.427,1.055)--(3.430,1.052)--(3.433,1.049)--(3.435,1.047)--(3.438,1.044)%
  --(3.440,1.041)--(3.443,1.038)--(3.446,1.036)--(3.448,1.033)--(3.451,1.031)--(3.453,1.028)%
  --(3.456,1.026)--(3.458,1.024)--(3.461,1.021)--(3.464,1.019)--(3.466,1.017)--(3.469,1.015)%
  --(3.471,1.013)--(3.474,1.011)--(3.477,1.009)--(3.479,1.008)--(3.482,1.006)--(3.484,1.004)%
  --(3.487,1.003)--(3.490,1.001)--(3.492,1.000)--(3.495,0.998)--(3.497,0.997)--(3.500,0.996)%
  --(3.502,0.995)--(3.505,0.994)--(3.508,0.993)--(3.510,0.992)--(3.513,0.991)--(3.515,0.990)%
  --(3.518,0.989)--(3.521,0.988)--(3.523,0.988)--(3.526,0.987)--(3.528,0.987)--(3.531,0.986)%
  --(3.534,0.986)--(3.536,0.986)--(3.539,0.985)--(3.541,0.985)--(3.544,0.985)--(3.546,0.985)%
  --(3.549,0.985)--(3.552,0.985)--(3.554,0.985)--(3.557,0.986)--(3.559,0.986)--(3.562,0.986)%
  --(3.565,0.987)--(3.567,0.987)--(3.570,0.988)--(3.572,0.988)--(3.575,0.989)--(3.578,0.990)%
  --(3.580,0.991)--(3.583,0.991)--(3.585,0.992)--(3.588,0.993)--(3.590,0.994)--(3.593,0.996)%
  --(3.596,0.997)--(3.598,0.998)--(3.601,0.999)--(3.603,1.001)--(3.606,1.002)--(3.609,1.004)%
  --(3.611,1.006)--(3.614,1.007)--(3.616,1.009)--(3.619,1.011)--(3.622,1.013)--(3.624,1.015)%
  --(3.627,1.017)--(3.629,1.019)--(3.632,1.021)--(3.634,1.023)--(3.637,1.025)--(3.640,1.028)%
  --(3.642,1.030)--(3.645,1.033)--(3.647,1.035)--(3.650,1.038)--(3.653,1.040)--(3.655,1.043)%
  --(3.658,1.046)--(3.660,1.049)--(3.663,1.052)--(3.666,1.055)--(3.668,1.058)--(3.671,1.061)%
  --(3.673,1.064)--(3.676,1.067)--(3.678,1.071)--(3.681,1.074)--(3.684,1.077)--(3.686,1.081)%
  --(3.689,1.084)--(3.691,1.088)--(3.694,1.092)--(3.697,1.096)--(3.699,1.099)--(3.702,1.103)%
  --(3.704,1.107)--(3.707,1.111)--(3.710,1.115)--(3.712,1.120)--(3.715,1.124)--(3.717,1.128)%
  --(3.720,1.132)--(3.722,1.137)--(3.725,1.141)--(3.728,1.146)--(3.730,1.150)--(3.733,1.155)%
  --(3.735,1.160)--(3.738,1.165)--(3.741,1.169)--(3.743,1.174)--(3.746,1.179)--(3.748,1.184)%
  --(3.751,1.189)--(3.754,1.195)--(3.756,1.200)--(3.759,1.205)--(3.761,1.211)--(3.764,1.216)%
  --(3.766,1.221)--(3.769,1.227)--(3.772,1.233)--(3.774,1.238)--(3.777,1.244)--(3.779,1.250)%
  --(3.782,1.256)--(3.785,1.261)--(3.787,1.267)--(3.790,1.273)--(3.792,1.280)--(3.795,1.286)%
  --(3.798,1.292)--(3.800,1.298)--(3.803,1.304)--(3.805,1.311)--(3.808,1.317)--(3.810,1.324)%
  --(3.813,1.330)--(3.816,1.337)--(3.818,1.344)--(3.821,1.350)--(3.823,1.357)--(3.826,1.364)%
  --(3.829,1.371)--(3.831,1.378)--(3.834,1.385)--(3.836,1.392)--(3.839,1.399)--(3.842,1.407)%
  --(3.844,1.414)--(3.847,1.421)--(3.849,1.429)--(3.852,1.436)--(3.854,1.444)--(3.857,1.451)%
  --(3.860,1.459)--(3.862,1.466)--(3.865,1.474)--(3.867,1.482)--(3.870,1.490)--(3.873,1.498)%
  --(3.875,1.506)--(3.878,1.514)--(3.880,1.522)--(3.883,1.530)--(3.886,1.538)--(3.888,1.546)%
  --(3.891,1.555)--(3.893,1.563)--(3.896,1.572)--(3.898,1.580)--(3.901,1.589)--(3.904,1.597)%
  --(3.906,1.606)--(3.909,1.614)--(3.911,1.623)--(3.914,1.632)--(3.917,1.641)--(3.919,1.650)%
  --(3.922,1.659)--(3.924,1.668)--(3.927,1.677)--(3.930,1.686)--(3.932,1.695)--(3.935,1.704)%
  --(3.937,1.714)--(3.940,1.723)--(3.942,1.732)--(3.945,1.742)--(3.948,1.751)--(3.950,1.761)%
  --(3.953,1.770)--(3.955,1.780)--(3.958,1.790)--(3.961,1.799)--(3.963,1.809)--(3.966,1.819)%
  --(3.968,1.829)--(3.971,1.839)--(3.974,1.849)--(3.976,1.859)--(3.979,1.869)--(3.981,1.879)%
  --(3.984,1.889)--(3.986,1.900)--(3.989,1.910)--(3.992,1.920)--(3.994,1.931)--(3.997,1.941)%
  --(3.999,1.952)--(4.002,1.962)--(4.005,1.973)--(4.007,1.983)--(4.010,1.994)--(4.012,2.005)%
  --(4.015,2.015)--(4.018,2.026)--(4.020,2.037)--(4.023,2.048)--(4.025,2.059)--(4.028,2.070)%
  --(4.030,2.081)--(4.033,2.092)--(4.036,2.103)--(4.038,2.114)--(4.041,2.126)--(4.043,2.137)%
  --(4.046,2.148)--(4.049,2.160)--(4.051,2.171)--(4.054,2.182)--(4.056,2.194)--(4.059,2.205)%
  --(4.062,2.217)--(4.064,2.229)--(4.067,2.240)--(4.069,2.252)--(4.072,2.264)--(4.075,2.276)%
  --(4.077,2.287)--(4.080,2.299)--(4.082,2.311)--(4.085,2.323)--(4.087,2.335)--(4.090,2.347)%
  --(4.093,2.359)--(4.095,2.371)--(4.098,2.384)--(4.100,2.396)--(4.103,2.408)--(4.106,2.420)%
  --(4.108,2.433)--(4.111,2.445)--(4.113,2.457)--(4.116,2.470)--(4.119,2.482)--(4.121,2.495)%
  --(4.124,2.507)--(4.126,2.520)--(4.129,2.532)--(4.131,2.545)--(4.134,2.558)--(4.137,2.570)%
  --(4.139,2.583)--(4.142,2.596)--(4.144,2.609)--(4.147,2.622)--(4.150,2.635)--(4.152,2.648)%
  --(4.155,2.661)--(4.157,2.674)--(4.160,2.687)--(4.163,2.700)--(4.165,2.713)--(4.168,2.726)%
  --(4.170,2.739)--(4.173,2.752)--(4.175,2.766)--(4.178,2.779)--(4.181,2.792)--(4.183,2.805)%
  --(4.186,2.819)--(4.188,2.832)--(4.191,2.846)--(4.194,2.859)--(4.196,2.873)--(4.199,2.886)%
  --(4.201,2.900)--(4.204,2.913)--(4.207,2.927)--(4.209,2.941)--(4.212,2.954)--(4.214,2.968)%
  --(4.217,2.982)--(4.219,2.995)--(4.222,3.009)--(4.225,3.023)--(4.227,3.037)--(4.230,3.051)%
  --(4.232,3.065)--(4.235,3.078)--(4.238,3.092)--(4.240,3.106)--(4.243,3.120)--(4.245,3.134)%
  --(4.248,3.148)--(4.251,3.163)--(4.253,3.177)--(4.256,3.191)--(4.258,3.205)--(4.261,3.219)%
  --(4.263,3.233)--(4.266,3.247)--(4.269,3.262)--(4.271,3.276)--(4.274,3.290)--(4.276,3.305)%
  --(4.279,3.319)--(4.282,3.333)--(4.284,3.348)--(4.287,3.362)--(4.289,3.376)--(4.292,3.391)%
  --(4.295,3.405)--(4.297,3.420)--(4.300,3.434)--(4.302,3.449)--(4.305,3.463)--(4.307,3.478)%
  --(4.310,3.492)--(4.313,3.507)--(4.315,3.522)--(4.318,3.536)--(4.320,3.551)--(4.323,3.566)%
  --(4.326,3.580)--(4.328,3.595)--(4.331,3.610)--(4.333,3.624)--(4.336,3.639)--(4.339,3.654)%
  --(4.341,3.669)--(4.344,3.683)--(4.346,3.698)--(4.349,3.713)--(4.351,3.728)--(4.354,3.743)%
  --(4.357,3.758)--(4.359,3.772)--(4.362,3.787)--(4.364,3.802)--(4.367,3.817)--(4.370,3.832)%
  --(4.372,3.847)--(4.375,3.862)--(4.377,3.877)--(4.380,3.892)--(4.383,3.907)--(4.385,3.922)%
  --(4.388,3.937)--(4.390,3.952)--(4.393,3.967)--(4.395,3.982)--(4.398,3.997)--(4.401,4.012)%
  --(4.403,4.027)--(4.406,4.042)--(4.408,4.057)--(4.411,4.072)--(4.414,4.087)--(4.416,4.102)%
  --(4.419,4.118)--(4.421,4.133)--(4.424,4.148)--(4.427,4.163)--(4.429,4.178)--(4.432,4.193)%
  --(4.434,4.208)--(4.437,4.223)--(4.439,4.239)--(4.442,4.254)--(4.445,4.269)--(4.447,4.284)%
  --(4.450,4.299)--(4.452,4.314)--(4.455,4.329)--(4.458,4.345)--(4.460,4.360)--(4.463,4.375)%
  --(4.465,4.390)--(4.468,4.405)--(4.471,4.420)--(4.473,4.436)--(4.476,4.451)--(4.478,4.466)%
  --(4.481,4.481)--(4.483,4.496)--(4.486,4.511)--(4.489,4.527)--(4.491,4.542)--(4.494,4.557)%
  --(4.496,4.572)--(4.499,4.587)--(4.502,4.602)--(4.504,4.618)--(4.507,4.633)--(4.509,4.648)%
  --(4.512,4.663)--(4.515,4.678)--(4.517,4.693)--(4.520,4.709)--(4.522,4.724)--(4.525,4.739)%
  --(4.527,4.754)--(4.530,4.769)--(4.533,4.784)--(4.535,4.799)--(4.538,4.814)--(4.540,4.829)%
  --(4.543,4.844)--(4.546,4.860)--(4.548,4.875)--(4.551,4.890)--(4.553,4.905)--(4.556,4.920)%
  --(4.559,4.935)--(4.561,4.950)--(4.564,4.965)--(4.566,4.980)--(4.569,4.995)--(4.571,5.010)%
  --(4.574,5.025)--(4.577,5.040)--(4.579,5.055)--(4.582,5.070)--(4.584,5.085)--(4.587,5.099)%
  --(4.590,5.114)--(4.592,5.129)--(4.595,5.144)--(4.597,5.159)--(4.600,5.174)--(4.603,5.189)%
  --(4.605,5.204)--(4.608,5.218)--(4.610,5.233)--(4.613,5.248)--(4.615,5.263)--(4.618,5.277)%
  --(4.621,5.292)--(4.623,5.307)--(4.626,5.322)--(4.628,5.336)--(4.631,5.351)--(4.634,5.366)%
  --(4.636,5.380)--(4.639,5.395)--(4.641,5.409)--(4.644,5.424)--(4.647,5.439)--(4.649,5.453)%
  --(4.652,5.468)--(4.654,5.482)--(4.657,5.497)--(4.659,5.511)--(4.662,5.525)--(4.665,5.540)%
  --(4.667,5.554)--(4.670,5.569)--(4.672,5.583)--(4.675,5.597)--(4.678,5.612)--(4.680,5.626)%
  --(4.683,5.640)--(4.685,5.654)--(4.688,5.669)--(4.691,5.683)--(4.693,5.697)--(4.696,5.711)%
  --(4.698,5.725)--(4.701,5.739)--(4.703,5.754)--(4.706,5.768)--(4.709,5.782)--(4.711,5.796)%
  --(4.714,5.810)--(4.716,5.824)--(4.719,5.838)--(4.722,5.851)--(4.724,5.865)--(4.727,5.879)%
  --(4.729,5.893)--(4.732,5.907)--(4.735,5.920)--(4.737,5.934)--(4.740,5.948)--(4.742,5.962)%
  --(4.745,5.975)--(4.747,5.989)--(4.750,6.002)--(4.753,6.016)--(4.755,6.030)--(4.758,6.043)%
  --(4.760,6.057)--(4.763,6.070)--(4.766,6.083)--(4.768,6.097)--(4.771,6.110)--(4.773,6.123)%
  --(4.776,6.137)--(4.779,6.150)--(4.781,6.163)--(4.784,6.176)--(4.786,6.190)--(4.789,6.203)%
  --(4.791,6.216)--(4.794,6.229)--(4.797,6.242)--(4.799,6.255)--(4.802,6.268)--(4.804,6.281)%
  --(4.807,6.294)--(4.810,6.306)--(4.812,6.319)--(4.815,6.332)--(4.817,6.345)--(4.820,6.357)%
  --(4.823,6.370)--(4.825,6.383)--(4.828,6.395)--(4.830,6.408)--(4.833,6.420)--(4.835,6.433)%
  --(4.838,6.445)--(4.841,6.458)--(4.843,6.470)--(4.846,6.482)--(4.848,6.495)--(4.851,6.507)%
  --(4.854,6.519)--(4.856,6.531)--(4.859,6.543)--(4.861,6.556)--(4.864,6.568)--(4.867,6.580)%
  --(4.869,6.592)--(4.872,6.603)--(4.874,6.615)--(4.877,6.627)--(4.879,6.639)--(4.882,6.651)%
  --(4.885,6.662)--(4.887,6.674)--(4.890,6.686)--(4.892,6.697)--(4.895,6.709)--(4.898,6.720)%
  --(4.900,6.732)--(4.903,6.743)--(4.905,6.755)--(4.908,6.766)--(4.911,6.777)--(4.913,6.789)%
  --(4.916,6.800)--(4.918,6.811)--(4.921,6.822)--(4.923,6.833)--(4.926,6.844)--(4.929,6.855)%
  --(4.931,6.866)--(4.934,6.877)--(4.936,6.888)--(4.939,6.898)--(4.942,6.909)--(4.944,6.920)%
  --(4.947,6.931)--(4.949,6.941)--(4.952,6.952)--(4.955,6.962)--(4.957,6.973)--(4.960,6.983)%
  --(4.962,6.993)--(4.965,7.004)--(4.967,7.014)--(4.970,7.024)--(4.973,7.034)--(4.975,7.044)%
  --(4.978,7.055)--(4.980,7.065)--(4.983,7.075)--(4.986,7.084)--(4.988,7.094)--(4.991,7.104)%
  --(4.993,7.114)--(4.996,7.124)--(4.999,7.133)--(5.001,7.143)--(5.004,7.152)--(5.006,7.162)%
  --(5.009,7.171)--(5.012,7.181)--(5.014,7.190)--(5.017,7.199)--(5.019,7.209)--(5.022,7.218)%
  --(5.024,7.227)--(5.027,7.236)--(5.030,7.245)--(5.032,7.254)--(5.035,7.263)--(5.037,7.272)%
  --(5.040,7.281)--(5.043,7.289)--(5.045,7.298)--(5.048,7.307)--(5.050,7.315)--(5.053,7.324)%
  --(5.056,7.332)--(5.058,7.341)--(5.061,7.349)--(5.063,7.358)--(5.066,7.366)--(5.068,7.374)%
  --(5.071,7.382)--(5.074,7.390)--(5.076,7.398)--(5.079,7.406)--(5.081,7.414)--(5.084,7.422)%
  --(5.087,7.430)--(5.089,7.438)--(5.092,7.445)--(5.094,7.453)--(5.097,7.461)--(5.100,7.468)%
  --(5.102,7.476)--(5.105,7.483)--(5.107,7.490)--(5.110,7.498)--(5.112,7.505)--(5.115,7.512)%
  --(5.118,7.519)--(5.120,7.526)--(5.123,7.533)--(5.125,7.540)--(5.128,7.547)--(5.131,7.554)%
  --(5.133,7.561)--(5.136,7.568)--(5.138,7.574)--(5.141,7.581)--(5.144,7.587)--(5.146,7.594)%
  --(5.149,7.600)--(5.151,7.607)--(5.154,7.613)--(5.156,7.619)--(5.159,7.625)--(5.162,7.631)%
  --(5.164,7.637)--(5.167,7.643)--(5.169,7.649)--(5.172,7.655)--(5.175,7.661)--(5.177,7.667)%
  --(5.180,7.672)--(5.182,7.678)--(5.185,7.684)--(5.188,7.689)--(5.190,7.695)--(5.193,7.700)%
  --(5.195,7.705)--(5.198,7.710)--(5.200,7.716)--(5.203,7.721)--(5.206,7.726)--(5.208,7.731)%
  --(5.211,7.736)--(5.213,7.741)--(5.216,7.745)--(5.219,7.750)--(5.221,7.755)--(5.224,7.760)%
  --(5.226,7.764)--(5.229,7.769)--(5.232,7.773)--(5.234,7.777)--(5.237,7.782)--(5.239,7.786)%
  --(5.242,7.790)--(5.244,7.794)--(5.247,7.798)--(5.250,7.802)--(5.252,7.806)--(5.255,7.810)%
  --(5.257,7.814)--(5.260,7.818)--(5.263,7.821)--(5.265,7.825)--(5.268,7.828)--(5.270,7.832)%
  --(5.273,7.835)--(5.276,7.839)--(5.278,7.842)--(5.281,7.845)--(5.283,7.848)--(5.286,7.851)%
  --(5.288,7.854)--(5.291,7.857)--(5.294,7.860)--(5.296,7.863)--(5.299,7.866)--(5.301,7.868)%
  --(5.304,7.871)--(5.307,7.874)--(5.309,7.876)--(5.312,7.879)--(5.314,7.881)--(5.317,7.883)%
  --(5.320,7.885)--(5.322,7.888)--(5.325,7.890)--(5.327,7.892)--(5.330,7.894)--(5.332,7.896)%
  --(5.335,7.897)--(5.338,7.899)--(5.340,7.901)--(5.343,7.903)--(5.345,7.904)--(5.348,7.906)%
  --(5.351,7.907)--(5.353,7.908)--(5.356,7.910)--(5.358,7.911)--(5.361,7.912)--(5.364,7.913)%
  --(5.366,7.914)--(5.369,7.915)--(5.371,7.916)--(5.374,7.917)--(5.376,7.918)--(5.379,7.919)%
  --(5.382,7.919)--(5.384,7.920)--(5.387,7.920)--(5.389,7.921)--(5.392,7.921)--(5.395,7.921)%
  --(5.397,7.922)--(5.400,7.922)--(5.402,7.922)--(5.405,7.922)--(5.408,7.922)--(5.410,7.922)%
  --(5.413,7.922)--(5.415,7.922)--(5.418,7.921)--(5.420,7.921)--(5.423,7.920)--(5.426,7.920)%
  --(5.428,7.919)--(5.431,7.919)--(5.433,7.918)--(5.436,7.917)--(5.439,7.917)--(5.441,7.916)%
  --(5.444,7.915)--(5.446,7.914)--(5.449,7.913)--(5.452,7.912)--(5.454,7.910)--(5.457,7.909)%
  --(5.459,7.908)--(5.462,7.906)--(5.464,7.905)--(5.467,7.903)--(5.470,7.902)--(5.472,7.900)%
  --(5.475,7.898)--(5.477,7.896)--(5.480,7.895)--(5.483,7.893)--(5.485,7.891)--(5.488,7.889)%
  --(5.490,7.886)--(5.493,7.884)--(5.496,7.882)--(5.498,7.880)--(5.501,7.877)--(5.503,7.875)%
  --(5.506,7.872)--(5.508,7.870)--(5.511,7.867)--(5.514,7.864)--(5.516,7.861)--(5.519,7.859)%
  --(5.521,7.856)--(5.524,7.853)--(5.527,7.850)--(5.529,7.847)--(5.532,7.843)--(5.534,7.840)%
  --(5.537,7.837)--(5.540,7.833)--(5.542,7.830)--(5.545,7.826)--(5.547,7.823)--(5.550,7.819)%
  --(5.552,7.816)--(5.555,7.812)--(5.558,7.808)--(5.560,7.804)--(5.563,7.800)--(5.565,7.796)%
  --(5.568,7.792)--(5.571,7.788)--(5.573,7.784)--(5.576,7.779)--(5.578,7.775)--(5.581,7.771)%
  --(5.584,7.766)--(5.586,7.762)--(5.589,7.757)--(5.591,7.752)--(5.594,7.748)--(5.596,7.743)%
  --(5.599,7.738)--(5.602,7.733)--(5.604,7.728)--(5.607,7.723)--(5.609,7.718)--(5.612,7.713)%
  --(5.615,7.708)--(5.617,7.702)--(5.620,7.697)--(5.622,7.692)--(5.625,7.686)--(5.628,7.681)%
  --(5.630,7.675)--(5.633,7.669)--(5.635,7.664)--(5.638,7.658)--(5.640,7.652)--(5.643,7.646)%
  --(5.646,7.640)--(5.648,7.634)--(5.651,7.628)--(5.653,7.622)--(5.656,7.616)--(5.659,7.610)%
  --(5.661,7.603)--(5.664,7.597)--(5.666,7.590)--(5.669,7.584)--(5.672,7.577)--(5.674,7.571)%
  --(5.677,7.564)--(5.679,7.557)--(5.682,7.551)--(5.684,7.544)--(5.687,7.537)--(5.690,7.530)%
  --(5.692,7.523)--(5.695,7.516)--(5.697,7.508)--(5.700,7.501)--(5.703,7.494)--(5.705,7.487)%
  --(5.708,7.479)--(5.710,7.472)--(5.713,7.464)--(5.716,7.457)--(5.718,7.449)--(5.721,7.441)%
  --(5.723,7.434)--(5.726,7.426)--(5.728,7.418)--(5.731,7.410)--(5.734,7.402)--(5.736,7.394)%
  --(5.739,7.386)--(5.741,7.378)--(5.744,7.370)--(5.747,7.362)--(5.749,7.353)--(5.752,7.345)%
  --(5.754,7.336)--(5.757,7.328)--(5.760,7.319)--(5.762,7.311)--(5.765,7.302)--(5.767,7.294)%
  --(5.770,7.285)--(5.772,7.276)--(5.775,7.267)--(5.778,7.258)--(5.780,7.249)--(5.783,7.240)%
  --(5.785,7.231)--(5.788,7.222)--(5.791,7.213)--(5.793,7.204)--(5.796,7.195)--(5.798,7.185)%
  --(5.801,7.176)--(5.804,7.166)--(5.806,7.157)--(5.809,7.147)--(5.811,7.138)--(5.814,7.128)%
  --(5.816,7.118)--(5.819,7.109)--(5.822,7.099)--(5.824,7.089)--(5.827,7.079)--(5.829,7.069)%
  --(5.832,7.059)--(5.835,7.049)--(5.837,7.039)--(5.840,7.029)--(5.842,7.019)--(5.845,7.009)%
  --(5.848,6.998)--(5.850,6.988)--(5.853,6.978)--(5.855,6.967)--(5.858,6.957)--(5.860,6.946)%
  --(5.863,6.936)--(5.866,6.925)--(5.868,6.914)--(5.871,6.904)--(5.873,6.893)--(5.876,6.882)%
  --(5.879,6.871)--(5.881,6.860)--(5.884,6.849)--(5.886,6.838)--(5.889,6.827)--(5.892,6.816)%
  --(5.894,6.805)--(5.897,6.794)--(5.899,6.783)--(5.902,6.771)--(5.904,6.760)--(5.907,6.749)%
  --(5.910,6.737)--(5.912,6.726)--(5.915,6.714)--(5.917,6.703)--(5.920,6.691)--(5.923,6.680)%
  --(5.925,6.668)--(5.928,6.656)--(5.930,6.645)--(5.933,6.633)--(5.936,6.621)--(5.938,6.609)%
  --(5.941,6.597)--(5.943,6.585)--(5.946,6.573)--(5.949,6.561)--(5.951,6.549)--(5.954,6.537)%
  --(5.956,6.525)--(5.959,6.513)--(5.961,6.500)--(5.964,6.488)--(5.967,6.476)--(5.969,6.464)%
  --(5.972,6.451)--(5.974,6.439)--(5.977,6.426)--(5.980,6.414)--(5.982,6.401)--(5.985,6.389)%
  --(5.987,6.376)--(5.990,6.363)--(5.993,6.351)--(5.995,6.338)--(5.998,6.325)--(6.000,6.312)%
  --(6.003,6.300)--(6.005,6.287)--(6.008,6.274)--(6.011,6.261)--(6.013,6.248)--(6.016,6.235)%
  --(6.018,6.222)--(6.021,6.209)--(6.024,6.196)--(6.026,6.183)--(6.029,6.169)--(6.031,6.156)%
  --(6.034,6.143)--(6.037,6.130)--(6.039,6.116)--(6.042,6.103)--(6.044,6.090)--(6.047,6.076)%
  --(6.049,6.063)--(6.052,6.049)--(6.055,6.036)--(6.057,6.022)--(6.060,6.009)--(6.062,5.995)%
  --(6.065,5.982)--(6.068,5.968)--(6.070,5.954)--(6.073,5.941)--(6.075,5.927)--(6.078,5.913)%
  --(6.081,5.899)--(6.083,5.886)--(6.086,5.872)--(6.088,5.858)--(6.091,5.844)--(6.093,5.830)%
  --(6.096,5.816)--(6.099,5.802)--(6.101,5.788)--(6.104,5.774)--(6.106,5.760)--(6.109,5.746)%
  --(6.112,5.732)--(6.114,5.718)--(6.117,5.704)--(6.119,5.690)--(6.122,5.675)--(6.125,5.661)%
  --(6.127,5.647)--(6.130,5.633)--(6.132,5.618)--(6.135,5.604)--(6.137,5.590)--(6.140,5.575)%
  --(6.143,5.561)--(6.145,5.547)--(6.148,5.532)--(6.150,5.518)--(6.153,5.503)--(6.156,5.489)%
  --(6.158,5.474)--(6.161,5.460)--(6.163,5.445)--(6.166,5.431)--(6.169,5.416)--(6.171,5.402)%
  --(6.174,5.387)--(6.176,5.372)--(6.179,5.358)--(6.181,5.343)--(6.184,5.329)--(6.187,5.314)%
  --(6.189,5.299)--(6.192,5.284)--(6.194,5.270)--(6.197,5.255)--(6.200,5.240)--(6.202,5.225)%
  --(6.205,5.211)--(6.207,5.196)--(6.210,5.181)--(6.213,5.166)--(6.215,5.151)--(6.218,5.136)%
  --(6.220,5.121)--(6.223,5.107)--(6.225,5.092)--(6.228,5.077)--(6.231,5.062)--(6.233,5.047)%
  --(6.236,5.032)--(6.238,5.017)--(6.241,5.002)--(6.244,4.987)--(6.246,4.972)--(6.249,4.957)%
  --(6.251,4.942)--(6.254,4.927)--(6.257,4.912)--(6.259,4.897)--(6.262,4.882)--(6.264,4.867)%
  --(6.267,4.852)--(6.269,4.837)--(6.272,4.821)--(6.275,4.806)--(6.277,4.791)--(6.280,4.776)%
  --(6.282,4.761)--(6.285,4.746)--(6.288,4.731)--(6.290,4.716)--(6.293,4.701)--(6.295,4.685)%
  --(6.298,4.670)--(6.301,4.655)--(6.303,4.640)--(6.306,4.625)--(6.308,4.610)--(6.311,4.595)%
  --(6.313,4.579)--(6.316,4.564)--(6.319,4.549)--(6.321,4.534)--(6.324,4.519)--(6.326,4.504)%
  --(6.329,4.488)--(6.332,4.473)--(6.334,4.458)--(6.337,4.443)--(6.339,4.428)--(6.342,4.412)%
  --(6.345,4.397)--(6.347,4.382)--(6.350,4.367)--(6.352,4.352)--(6.355,4.337)--(6.357,4.321)%
  --(6.360,4.306)--(6.363,4.291)--(6.365,4.276)--(6.368,4.261)--(6.370,4.246)--(6.373,4.231)%
  --(6.376,4.215)--(6.378,4.200)--(6.381,4.185)--(6.383,4.170)--(6.386,4.155)--(6.389,4.140)%
  --(6.391,4.125)--(6.394,4.110)--(6.396,4.094)--(6.399,4.079)--(6.401,4.064)--(6.404,4.049)%
  --(6.407,4.034)--(6.409,4.019)--(6.412,4.004)--(6.414,3.989)--(6.417,3.974)--(6.420,3.959)%
  --(6.422,3.944)--(6.425,3.929)--(6.427,3.914)--(6.430,3.899)--(6.433,3.884)--(6.435,3.869)%
  --(6.438,3.854)--(6.440,3.839)--(6.443,3.824)--(6.445,3.809)--(6.448,3.794)--(6.451,3.779)%
  --(6.453,3.765)--(6.456,3.750)--(6.458,3.735)--(6.461,3.720)--(6.464,3.705)--(6.466,3.690)%
  --(6.469,3.676)--(6.471,3.661)--(6.474,3.646)--(6.477,3.631)--(6.479,3.617)--(6.482,3.602)%
  --(6.484,3.587)--(6.487,3.572)--(6.489,3.558)--(6.492,3.543)--(6.495,3.529)--(6.497,3.514)%
  --(6.500,3.499)--(6.502,3.485)--(6.505,3.470)--(6.508,3.456)--(6.510,3.441)--(6.513,3.427)%
  --(6.515,3.412)--(6.518,3.398)--(6.521,3.383)--(6.523,3.369)--(6.526,3.354)--(6.528,3.340)%
  --(6.531,3.326)--(6.533,3.311)--(6.536,3.297)--(6.539,3.283)--(6.541,3.268)--(6.544,3.254)%
  --(6.546,3.240)--(6.549,3.226)--(6.552,3.212)--(6.554,3.197)--(6.557,3.183)--(6.559,3.169)%
  --(6.562,3.155)--(6.565,3.141)--(6.567,3.127)--(6.570,3.113)--(6.572,3.099)--(6.575,3.085)%
  --(6.577,3.071)--(6.580,3.057)--(6.583,3.043)--(6.585,3.030)--(6.588,3.016)--(6.590,3.002)%
  --(6.593,2.988)--(6.596,2.974)--(6.598,2.961)--(6.601,2.947)--(6.603,2.933)--(6.606,2.920)%
  --(6.609,2.906)--(6.611,2.893)--(6.614,2.879)--(6.616,2.866)--(6.619,2.852)--(6.621,2.839)%
  --(6.624,2.825)--(6.627,2.812)--(6.629,2.798)--(6.632,2.785)--(6.634,2.772)--(6.637,2.759)%
  --(6.640,2.745)--(6.642,2.732)--(6.645,2.719)--(6.647,2.706)--(6.650,2.693)--(6.653,2.680)%
  --(6.655,2.667)--(6.658,2.654)--(6.660,2.641)--(6.663,2.628)--(6.665,2.615)--(6.668,2.602)%
  --(6.671,2.589)--(6.673,2.577)--(6.676,2.564)--(6.678,2.551)--(6.681,2.538)--(6.684,2.526)%
  --(6.686,2.513)--(6.689,2.501)--(6.691,2.488)--(6.694,2.476)--(6.697,2.463)--(6.699,2.451)%
  --(6.702,2.438)--(6.704,2.426)--(6.707,2.414)--(6.709,2.402)--(6.712,2.389)--(6.715,2.377)%
  --(6.717,2.365)--(6.720,2.353)--(6.722,2.341)--(6.725,2.329)--(6.728,2.317)--(6.730,2.305)%
  --(6.733,2.293)--(6.735,2.281)--(6.738,2.269)--(6.741,2.258)--(6.743,2.246)--(6.746,2.234)%
  --(6.748,2.223)--(6.751,2.211)--(6.753,2.199)--(6.756,2.188)--(6.759,2.176)--(6.761,2.165)%
  --(6.764,2.154)--(6.766,2.142)--(6.769,2.131)--(6.772,2.120)--(6.774,2.109)--(6.777,2.097)%
  --(6.779,2.086)--(6.782,2.075)--(6.785,2.064)--(6.787,2.053)--(6.790,2.042)--(6.792,2.031)%
  --(6.795,2.021)--(6.797,2.010)--(6.800,1.999)--(6.803,1.988)--(6.805,1.978)--(6.808,1.967)%
  --(6.810,1.957)--(6.813,1.946)--(6.816,1.936)--(6.818,1.925)--(6.821,1.915)--(6.823,1.904)%
  --(6.826,1.894)--(6.829,1.884)--(6.831,1.874)--(6.834,1.864)--(6.836,1.854)--(6.839,1.844)%
  --(6.841,1.834)--(6.844,1.824)--(6.847,1.814)--(6.849,1.804)--(6.852,1.794)--(6.854,1.785)%
  --(6.857,1.775)--(6.860,1.765)--(6.862,1.756)--(6.865,1.746)--(6.867,1.737)--(6.870,1.727)%
  --(6.873,1.718)--(6.875,1.709)--(6.878,1.699)--(6.880,1.690)--(6.883,1.681)--(6.886,1.672)%
  --(6.888,1.663)--(6.891,1.654)--(6.893,1.645)--(6.896,1.636)--(6.898,1.627)--(6.901,1.619)%
  --(6.904,1.610)--(6.906,1.601)--(6.909,1.593)--(6.911,1.584)--(6.914,1.576)--(6.917,1.567)%
  --(6.919,1.559)--(6.922,1.550)--(6.924,1.542)--(6.927,1.534)--(6.930,1.526)--(6.932,1.518)%
  --(6.935,1.510)--(6.937,1.502)--(6.940,1.494)--(6.942,1.486)--(6.945,1.478)--(6.948,1.470)%
  --(6.950,1.462)--(6.953,1.455)--(6.955,1.447)--(6.958,1.440)--(6.961,1.432)--(6.963,1.425)%
  --(6.966,1.417)--(6.968,1.410)--(6.971,1.403)--(6.974,1.396)--(6.976,1.388)--(6.979,1.381)%
  --(6.981,1.374)--(6.984,1.367)--(6.986,1.361)--(6.989,1.354)--(6.992,1.347)--(6.994,1.340)%
  --(6.997,1.334)--(6.999,1.327)--(7.002,1.320)--(7.005,1.314)--(7.007,1.308)--(7.010,1.301)%
  --(7.012,1.295)--(7.015,1.289)--(7.018,1.282)--(7.020,1.276)--(7.023,1.270)--(7.025,1.264)%
  --(7.028,1.258)--(7.030,1.252)--(7.033,1.247)--(7.036,1.241)--(7.038,1.235)--(7.041,1.230)%
  --(7.043,1.224)--(7.046,1.219)--(7.049,1.213)--(7.051,1.208)--(7.054,1.202)--(7.056,1.197)%
  --(7.059,1.192)--(7.062,1.187)--(7.064,1.182)--(7.067,1.177)--(7.069,1.172)--(7.072,1.167)%
  --(7.074,1.162)--(7.077,1.157)--(7.080,1.153)--(7.082,1.148)--(7.085,1.143)--(7.087,1.139)%
  --(7.090,1.134)--(7.093,1.130)--(7.095,1.126)--(7.098,1.122)--(7.100,1.117)--(7.103,1.113)%
  --(7.106,1.109)--(7.108,1.105)--(7.111,1.101)--(7.113,1.097)--(7.116,1.094)--(7.118,1.090)%
  --(7.121,1.086)--(7.124,1.083)--(7.126,1.079)--(7.129,1.076)--(7.131,1.072)--(7.134,1.069)%
  --(7.137,1.066)--(7.139,1.062)--(7.142,1.059)--(7.144,1.056)--(7.147,1.053)--(7.150,1.050)%
  --(7.152,1.047)--(7.155,1.044)--(7.157,1.042)--(7.160,1.039)--(7.162,1.036)--(7.165,1.034)%
  --(7.168,1.031)--(7.170,1.029)--(7.173,1.026)--(7.175,1.024)--(7.178,1.022)--(7.181,1.020)%
  --(7.183,1.018)--(7.186,1.016)--(7.188,1.014)--(7.191,1.012)--(7.194,1.010)--(7.196,1.008)%
  --(7.199,1.006)--(7.201,1.005)--(7.204,1.003)--(7.206,1.002)--(7.209,1.000)--(7.212,0.999)%
  --(7.214,0.997)--(7.217,0.996)--(7.219,0.995)--(7.222,0.994)--(7.225,0.993)--(7.227,0.992)%
  --(7.230,0.991)--(7.232,0.990)--(7.235,0.989)--(7.238,0.989)--(7.240,0.988)--(7.243,0.987)%
  --(7.245,0.987)--(7.248,0.986)--(7.250,0.986)--(7.253,0.986)--(7.256,0.985)--(7.258,0.985)%
  --(7.261,0.985)--(7.263,0.985)--(7.266,0.985)--(7.269,0.985)--(7.271,0.985)--(7.274,0.985)%
  --(7.276,0.986)--(7.279,0.986)--(7.282,0.986)--(7.284,0.987)--(7.287,0.987)--(7.289,0.988)%
  --(7.292,0.989)--(7.294,0.989)--(7.297,0.990)--(7.300,0.991)--(7.302,0.992)--(7.305,0.993)%
  --(7.307,0.994)--(7.310,0.995)--(7.313,0.997)--(7.315,0.998)--(7.318,0.999)--(7.320,1.001)%
  --(7.323,1.002)--(7.326,1.004)--(7.328,1.005)--(7.331,1.007)--(7.333,1.009)--(7.336,1.010)%
  --(7.338,1.012)--(7.341,1.014)--(7.344,1.016)--(7.346,1.018)--(7.349,1.020)--(7.351,1.023)%
  --(7.354,1.025)--(7.357,1.027)--(7.359,1.029)--(7.362,1.032)--(7.364,1.034)--(7.367,1.037)%
  --(7.370,1.040)--(7.372,1.042)--(7.375,1.045)--(7.377,1.048)--(7.380,1.051)--(7.382,1.054)%
  --(7.385,1.057)--(7.388,1.060)--(7.390,1.063)--(7.393,1.066)--(7.395,1.070)--(7.398,1.073)%
  --(7.401,1.077)--(7.403,1.080)--(7.406,1.084)--(7.408,1.087)--(7.411,1.091)--(7.414,1.095)%
  --(7.416,1.099)--(7.419,1.102)--(7.421,1.106)--(7.424,1.110)--(7.426,1.114)--(7.429,1.119)%
  --(7.432,1.123)--(7.434,1.127)--(7.437,1.131)--(7.439,1.136)--(7.442,1.140)--(7.445,1.145)%
  --(7.447,1.149)--(7.450,1.154)--(7.452,1.159)--(7.455,1.163)--(7.458,1.168)--(7.460,1.173)%
  --(7.463,1.178)--(7.465,1.183)--(7.468,1.188)--(7.470,1.193)--(7.473,1.199)--(7.476,1.204)%
  --(7.478,1.209)--(7.481,1.215)--(7.483,1.220)--(7.486,1.226)--(7.489,1.231)--(7.491,1.237)%
  --(7.494,1.243)--(7.496,1.248)--(7.499,1.254)--(7.502,1.260)--(7.504,1.266)--(7.507,1.272)%
  --(7.509,1.278)--(7.512,1.284)--(7.514,1.290)--(7.517,1.297)--(7.520,1.303)--(7.522,1.309)%
  --(7.525,1.316)--(7.527,1.322)--(7.530,1.329)--(7.533,1.335)--(7.535,1.342)--(7.538,1.349)%
  --(7.540,1.356)--(7.543,1.363)--(7.546,1.369)--(7.548,1.376)--(7.551,1.383)--(7.553,1.391)%
  --(7.556,1.398)--(7.558,1.405)--(7.561,1.412)--(7.564,1.419)--(7.566,1.427)--(7.569,1.434)%
  --(7.571,1.442)--(7.574,1.449)--(7.577,1.457)--(7.579,1.465)--(7.582,1.472)--(7.584,1.480)%
  --(7.587,1.488)--(7.590,1.496)--(7.592,1.504)--(7.595,1.512)--(7.597,1.520)--(7.600,1.528)%
  --(7.602,1.536)--(7.605,1.544)--(7.608,1.553)--(7.610,1.561)--(7.613,1.570)--(7.615,1.578)%
  --(7.618,1.586)--(7.621,1.595)--(7.623,1.604)--(7.626,1.612)--(7.628,1.621)--(7.631,1.630)%
  --(7.634,1.639)--(7.636,1.648)--(7.639,1.657)--(7.641,1.666)--(7.644,1.675)--(7.646,1.684)%
  --(7.649,1.693)--(7.652,1.702)--(7.654,1.711)--(7.657,1.721)--(7.659,1.730)--(7.662,1.739)%
  --(7.665,1.749)--(7.667,1.758)--(7.670,1.768)--(7.672,1.778)--(7.675,1.787)--(7.678,1.797)%
  --(7.680,1.807)--(7.683,1.817)--(7.685,1.827)--(7.688,1.837)--(7.690,1.847)--(7.693,1.857)%
  --(7.696,1.867)--(7.698,1.877)--(7.701,1.887)--(7.703,1.897)--(7.706,1.907)--(7.709,1.918)%
  --(7.711,1.928)--(7.714,1.939)--(7.716,1.949)--(7.719,1.960)--(7.722,1.970)--(7.724,1.981)%
  --(7.727,1.991)--(7.729,2.002)--(7.732,2.013)--(7.734,2.024)--(7.737,2.035)--(7.740,2.045)%
  --(7.742,2.056)--(7.745,2.067)--(7.747,2.078)--(7.750,2.089)--(7.753,2.101)--(7.755,2.112)%
  --(7.758,2.123)--(7.760,2.134)--(7.763,2.146)--(7.766,2.157)--(7.768,2.168)--(7.771,2.180)%
  --(7.773,2.191)--(7.776,2.203)--(7.778,2.214)--(7.781,2.226)--(7.784,2.238)--(7.786,2.249)%
  --(7.789,2.261)--(7.791,2.273)--(7.794,2.285)--(7.797,2.296)--(7.799,2.308)--(7.802,2.320)%
  --(7.804,2.332)--(7.807,2.344)--(7.810,2.356)--(7.812,2.369)--(7.815,2.381)--(7.817,2.393)%
  --(7.820,2.405)--(7.823,2.417)--(7.825,2.430)--(7.828,2.442)--(7.830,2.454)--(7.833,2.467)%
  --(7.835,2.479)--(7.838,2.492)--(7.841,2.504)--(7.843,2.517)--(7.846,2.529)--(7.848,2.542)%
  --(7.851,2.555)--(7.854,2.567)--(7.856,2.580)--(7.859,2.593)--(7.861,2.606)--(7.864,2.619)%
  --(7.867,2.632)--(7.869,2.645)--(7.872,2.657)--(7.874,2.670)--(7.877,2.684)--(7.879,2.697)%
  --(7.882,2.710)--(7.885,2.723)--(7.887,2.736)--(7.890,2.749)--(7.892,2.762)--(7.895,2.776)%
  --(7.898,2.789)--(7.900,2.802)--(7.903,2.816)--(7.905,2.829)--(7.908,2.842)--(7.911,2.856)%
  --(7.913,2.869)--(7.916,2.883)--(7.918,2.896)--(7.921,2.910)--(7.923,2.924)--(7.926,2.937)%
  --(7.929,2.951)--(7.931,2.965)--(7.934,2.978)--(7.936,2.992)--(7.939,3.006)--(7.942,3.020)%
  --(7.944,3.034)--(7.947,3.047)--(7.949,3.061)--(7.952,3.075)--(7.955,3.089)--(7.957,3.103)%
  --(7.960,3.117)--(7.962,3.131)--(7.965,3.145)--(7.967,3.159)--(7.970,3.173)--(7.973,3.187)%
  --(7.975,3.202)--(7.978,3.216)--(7.980,3.230)--(7.983,3.244)--(7.986,3.258)--(7.988,3.273)%
  --(7.991,3.287)--(7.993,3.301)--(7.996,3.315)--(7.999,3.330)--(8.001,3.344)--(8.004,3.359)%
  --(8.006,3.373)--(8.009,3.387)--(8.011,3.402)--(8.014,3.416)--(8.017,3.431)--(8.019,3.445)%
  --(8.022,3.460)--(8.024,3.474)--(8.027,3.489)--(8.030,3.504)--(8.032,3.518)--(8.035,3.533)%
  --(8.037,3.547)--(8.040,3.562)--(8.043,3.577)--(8.045,3.591)--(8.048,3.606)--(8.050,3.621)%
  --(8.053,3.636)--(8.055,3.650)--(8.058,3.665)--(8.061,3.680)--(8.063,3.695)--(8.066,3.710)%
  --(8.068,3.724)--(8.071,3.739)--(8.074,3.754)--(8.076,3.769)--(8.079,3.784)--(8.081,3.799)%
  --(8.084,3.814)--(8.087,3.829)--(8.089,3.843)--(8.092,3.858)--(8.094,3.873)--(8.097,3.888)%
  --(8.099,3.903)--(8.102,3.918)--(8.105,3.933)--(8.107,3.948)--(8.110,3.963)--(8.112,3.978)%
  --(8.115,3.993)--(8.118,4.008)--(8.120,4.023)--(8.123,4.039)--(8.125,4.054)--(8.128,4.069)%
  --(8.131,4.084)--(8.133,4.099)--(8.136,4.114)--(8.138,4.129)--(8.141,4.144)--(8.143,4.159)%
  --(8.146,4.174)--(8.149,4.190)--(8.151,4.205)--(8.154,4.220)--(8.156,4.235)--(8.159,4.250)%
  --(8.162,4.265)--(8.164,4.280)--(8.167,4.296)--(8.169,4.311)--(8.172,4.326)--(8.175,4.341)%
  --(8.177,4.356)--(8.180,4.371)--(8.182,4.387)--(8.185,4.402)--(8.187,4.417)--(8.190,4.432)%
  --(8.193,4.447)--(8.195,4.462)--(8.198,4.478)--(8.200,4.493)--(8.203,4.508)--(8.206,4.523)%
  --(8.208,4.538)--(8.211,4.553)--(8.213,4.569)--(8.216,4.584)--(8.219,4.599)--(8.221,4.614)%
  --(8.224,4.629)--(8.226,4.644)--(8.229,4.660)--(8.231,4.675)--(8.234,4.690)--(8.237,4.705)%
  --(8.239,4.720)--(8.242,4.735)--(8.244,4.750)--(8.247,4.765)--(8.250,4.781)--(8.252,4.796)%
  --(8.255,4.811)--(8.257,4.826)--(8.260,4.841)--(8.263,4.856)--(8.265,4.871)--(8.268,4.886)%
  --(8.270,4.901)--(8.273,4.916)--(8.275,4.931)--(8.278,4.946)--(8.281,4.961)--(8.283,4.976)%
  --(8.286,4.991)--(8.288,5.006)--(8.291,5.021)--(8.294,5.036)--(8.296,5.051)--(8.299,5.066)%
  --(8.301,5.081)--(8.304,5.096)--(8.307,5.111)--(8.309,5.126)--(8.312,5.141)--(8.314,5.155)%
  --(8.317,5.170)--(8.319,5.185)--(8.322,5.200)--(8.325,5.215)--(8.327,5.230)--(8.330,5.244)%
  --(8.332,5.259)--(8.335,5.274)--(8.338,5.289)--(8.340,5.303)--(8.343,5.318)--(8.345,5.333)%
  --(8.348,5.347)--(8.351,5.362)--(8.353,5.377)--(8.356,5.391)--(8.358,5.406)--(8.361,5.421)%
  --(8.363,5.435)--(8.366,5.450)--(8.369,5.464)--(8.371,5.479)--(8.374,5.493)--(8.376,5.508)%
  --(8.379,5.522)--(8.382,5.536)--(8.384,5.551)--(8.387,5.565)--(8.389,5.580)--(8.392,5.594)%
  --(8.395,5.608)--(8.397,5.623)--(8.400,5.637)--(8.402,5.651)--(8.405,5.665)--(8.407,5.680)%
  --(8.410,5.694)--(8.413,5.708)--(8.415,5.722)--(8.418,5.736)--(8.420,5.750)--(8.423,5.764)%
  --(8.426,5.778)--(8.428,5.792)--(8.431,5.806)--(8.433,5.820)--(8.436,5.834)--(8.439,5.848)%
  --(8.441,5.862)--(8.444,5.876)--(8.446,5.890)--(8.449,5.903)--(8.451,5.917)--(8.454,5.931)%
  --(8.457,5.945)--(8.459,5.958)--(8.462,5.972)--(8.464,5.986)--(8.467,5.999)--(8.470,6.013)%
  --(8.472,6.026)--(8.475,6.040)--(8.477,6.053)--(8.480,6.067)--(8.483,6.080)--(8.485,6.094)%
  --(8.488,6.107)--(8.490,6.120)--(8.493,6.134)--(8.495,6.147)--(8.498,6.160)--(8.501,6.173)%
  --(8.503,6.186)--(8.506,6.200)--(8.508,6.213)--(8.511,6.226)--(8.514,6.239)--(8.516,6.252)%
  --(8.519,6.265)--(8.521,6.278)--(8.524,6.290)--(8.527,6.303)--(8.529,6.316)--(8.532,6.329)%
  --(8.534,6.342)--(8.537,6.354)--(8.539,6.367)--(8.542,6.380)--(8.545,6.392)--(8.547,6.405)%
  --(8.550,6.417)--(8.552,6.430)--(8.555,6.442)--(8.558,6.455)--(8.560,6.467)--(8.563,6.479)%
  --(8.565,6.492)--(8.568,6.504)--(8.571,6.516)--(8.573,6.528)--(8.576,6.541)--(8.578,6.553)%
  --(8.581,6.565)--(8.583,6.577)--(8.586,6.589)--(8.589,6.601)--(8.591,6.613)--(8.594,6.624)%
  --(8.596,6.636)--(8.599,6.648)--(8.602,6.660)--(8.604,6.671)--(8.607,6.683)--(8.609,6.695)%
  --(8.612,6.706)--(8.615,6.718)--(8.617,6.729)--(8.620,6.741)--(8.622,6.752)--(8.625,6.763)%
  --(8.627,6.775)--(8.630,6.786)--(8.633,6.797)--(8.635,6.808)--(8.638,6.819)--(8.640,6.830)%
  --(8.643,6.841)--(8.646,6.852)--(8.648,6.863)--(8.651,6.874)--(8.653,6.885)--(8.656,6.896)%
  --(8.659,6.907)--(8.661,6.917)--(8.664,6.928)--(8.666,6.939)--(8.669,6.949)--(8.671,6.960)%
  --(8.674,6.970)--(8.677,6.981)--(8.679,6.991)--(8.682,7.001)--(8.684,7.012)--(8.687,7.022)%
  --(8.690,7.032)--(8.692,7.042)--(8.695,7.052)--(8.697,7.062)--(8.700,7.072)--(8.703,7.082)%
  --(8.705,7.092)--(8.708,7.102)--(8.710,7.112)--(8.713,7.121)--(8.715,7.131)--(8.718,7.141)%
  --(8.721,7.150)--(8.723,7.160)--(8.726,7.169)--(8.728,7.179)--(8.731,7.188)--(8.734,7.197)%
  --(8.736,7.206)--(8.739,7.216)--(8.741,7.225)--(8.744,7.234)--(8.747,7.243)--(8.749,7.252)%
  --(8.752,7.261)--(8.754,7.270)--(8.757,7.279)--(8.760,7.287)--(8.762,7.296)--(8.765,7.305)%
  --(8.767,7.313)--(8.770,7.322)--(8.772,7.330)--(8.775,7.339)--(8.778,7.347)--(8.780,7.356)%
  --(8.783,7.364)--(8.785,7.372)--(8.788,7.380)--(8.791,7.388)--(8.793,7.396)--(8.796,7.404)%
  --(8.798,7.412)--(8.801,7.420)--(8.804,7.428)--(8.806,7.436)--(8.809,7.444)--(8.811,7.451)%
  --(8.814,7.459)--(8.816,7.466)--(8.819,7.474)--(8.822,7.481)--(8.824,7.489)--(8.827,7.496)%
  --(8.829,7.503)--(8.832,7.511)--(8.835,7.518)--(8.837,7.525)--(8.840,7.532)--(8.842,7.539)%
  --(8.845,7.546)--(8.848,7.552)--(8.850,7.559)--(8.853,7.566)--(8.855,7.573)--(8.858,7.579)%
  --(8.860,7.586)--(8.863,7.592)--(8.866,7.599)--(8.868,7.605)--(8.871,7.611)--(8.873,7.618)%
  --(8.876,7.624)--(8.879,7.630)--(8.881,7.636)--(8.884,7.642)--(8.886,7.648)--(8.889,7.654)%
  --(8.892,7.660)--(8.894,7.665)--(8.897,7.671)--(8.899,7.677)--(8.902,7.682)--(8.904,7.688)%
  --(8.907,7.693)--(8.910,7.699)--(8.912,7.704)--(8.915,7.709)--(8.917,7.714)--(8.920,7.720)%
  --(8.923,7.725)--(8.925,7.730)--(8.928,7.735)--(8.930,7.740)--(8.933,7.744)--(8.936,7.749)%
  --(8.938,7.754)--(8.941,7.758)--(8.943,7.763)--(8.946,7.768)--(8.948,7.772)--(8.951,7.776)%
  --(8.954,7.781)--(8.956,7.785)--(8.959,7.789)--(8.961,7.793)--(8.964,7.797)--(8.967,7.801)%
  --(8.969,7.805)--(8.972,7.809)--(8.974,7.813)--(8.977,7.817)--(8.980,7.820)--(8.982,7.824)%
  --(8.985,7.828)--(8.987,7.831)--(8.990,7.834)--(8.992,7.838)--(8.995,7.841)--(8.998,7.844)%
  --(9.000,7.847)--(9.003,7.851)--(9.005,7.854)--(9.008,7.857)--(9.011,7.859)--(9.013,7.862)%
  --(9.016,7.865)--(9.018,7.868)--(9.021,7.870)--(9.024,7.873)--(9.026,7.875)--(9.029,7.878)%
  --(9.031,7.880)--(9.034,7.883)--(9.036,7.885)--(9.039,7.887)--(9.042,7.889)--(9.044,7.891)%
  --(9.047,7.893)--(9.049,7.895)--(9.052,7.897)--(9.055,7.899)--(9.057,7.900)--(9.060,7.902)%
  --(9.062,7.904)--(9.065,7.905)--(9.068,7.907)--(9.070,7.908)--(9.073,7.909)--(9.075,7.911)%
  --(9.078,7.912)--(9.080,7.913)--(9.083,7.914)--(9.086,7.915)--(9.088,7.916)--(9.091,7.917)%
  --(9.093,7.918)--(9.096,7.918)--(9.099,7.919)--(9.101,7.920)--(9.104,7.920)--(9.106,7.921)%
  --(9.109,7.921)--(9.112,7.921)--(9.114,7.922)--(9.117,7.922)--(9.119,7.922)--(9.122,7.922)%
  --(9.124,7.922)--(9.127,7.922)--(9.130,7.922)--(9.132,7.922)--(9.135,7.921)--(9.137,7.921)%
  --(9.140,7.921)--(9.143,7.920)--(9.145,7.920)--(9.148,7.919)--(9.150,7.918)--(9.153,7.918)%
  --(9.156,7.917)--(9.158,7.916)--(9.161,7.915)--(9.163,7.914)--(9.166,7.913)--(9.168,7.912)%
  --(9.171,7.911)--(9.174,7.909)--(9.176,7.908)--(9.179,7.907)--(9.181,7.905)--(9.184,7.904)%
  --(9.187,7.902)--(9.189,7.900)--(9.192,7.899)--(9.194,7.897)--(9.197,7.895)--(9.200,7.893)%
  --(9.202,7.891)--(9.205,7.889)--(9.207,7.887)--(9.210,7.885)--(9.212,7.883)--(9.215,7.880)%
  --(9.218,7.878)--(9.220,7.875)--(9.223,7.873)--(9.225,7.870)--(9.228,7.868)--(9.231,7.865)%
  --(9.233,7.862)--(9.236,7.859)--(9.238,7.856)--(9.241,7.853)--(9.244,7.850)--(9.246,7.847)%
  --(9.249,7.844)--(9.251,7.841)--(9.254,7.838)--(9.256,7.834)--(9.259,7.831)--(9.262,7.827)%
  --(9.264,7.824)--(9.267,7.820)--(9.269,7.816)--(9.272,7.813)--(9.275,7.809)--(9.277,7.805)%
  --(9.280,7.801)--(9.282,7.797)--(9.285,7.793)--(9.288,7.789)--(9.290,7.785)--(9.293,7.780)%
  --(9.295,7.776)--(9.298,7.772)--(9.300,7.767)--(9.303,7.763)--(9.306,7.758)--(9.308,7.754)%
  --(9.311,7.749)--(9.313,7.744)--(9.316,7.739)--(9.319,7.734)--(9.321,7.729)--(9.324,7.724)%
  --(9.326,7.719)--(9.329,7.714)--(9.332,7.709)--(9.334,7.704)--(9.337,7.698)--(9.339,7.693)%
  --(9.342,7.688)--(9.344,7.682)--(9.347,7.676)--(9.350,7.671)--(9.352,7.665)--(9.355,7.659)%
  --(9.357,7.654)--(9.360,7.648)--(9.363,7.642)--(9.365,7.636)--(9.368,7.630)--(9.370,7.624)%
  --(9.373,7.617)--(9.376,7.611)--(9.378,7.605)--(9.381,7.598)--(9.383,7.592)--(9.386,7.585)%
  --(9.388,7.579)--(9.391,7.572)--(9.394,7.566)--(9.396,7.559)--(9.399,7.552)--(9.401,7.545)%
  --(9.404,7.538)--(9.407,7.531)--(9.409,7.524)--(9.412,7.517)--(9.414,7.510)--(9.417,7.503)%
  --(9.420,7.496)--(9.422,7.488)--(9.425,7.481)--(9.427,7.474)--(9.430,7.466)--(9.432,7.459)%
  --(9.435,7.451)--(9.438,7.443)--(9.440,7.436)--(9.443,7.428)--(9.445,7.420)--(9.448,7.412)%
  --(9.451,7.404)--(9.453,7.396)--(9.456,7.388)--(9.458,7.380)--(9.461,7.372)--(9.464,7.363)%
  --(9.466,7.355)--(9.469,7.347)--(9.471,7.338)--(9.474,7.330)--(9.476,7.322)--(9.479,7.313)%
  --(9.482,7.304)--(9.484,7.296)--(9.487,7.287)--(9.489,7.278)--(9.492,7.269)--(9.495,7.260)%
  --(9.497,7.252)--(9.500,7.243)--(9.502,7.233)--(9.505,7.224)--(9.508,7.215)--(9.510,7.206)%
  --(9.513,7.197)--(9.515,7.187)--(9.518,7.178)--(9.520,7.169)--(9.523,7.159)--(9.526,7.150)%
  --(9.528,7.140)--(9.531,7.130)--(9.533,7.121)--(9.536,7.111)--(9.539,7.101)--(9.541,7.091)%
  --(9.544,7.082)--(9.546,7.072)--(9.549,7.062)--(9.552,7.052)--(9.554,7.042)--(9.557,7.031)%
  --(9.559,7.021)--(9.562,7.011)--(9.564,7.001)--(9.567,6.990)--(9.570,6.980)--(9.572,6.970)%
  --(9.575,6.959)--(9.577,6.949)--(9.580,6.938)--(9.583,6.927)--(9.585,6.917)--(9.588,6.906)%
  --(9.590,6.895)--(9.593,6.885)--(9.596,6.874)--(9.598,6.863)--(9.601,6.852)--(9.603,6.841)%
  --(9.606,6.830)--(9.608,6.819)--(9.611,6.808)--(9.614,6.797)--(9.616,6.785)--(9.619,6.774)%
  --(9.621,6.763)--(9.624,6.751)--(9.627,6.740)--(9.629,6.729)--(9.632,6.717)--(9.634,6.706)%
  --(9.637,6.694)--(9.640,6.682)--(9.642,6.671)--(9.645,6.659)--(9.647,6.647)--(9.650,6.636)%
  --(9.652,6.624)--(9.655,6.612)--(9.658,6.600)--(9.660,6.588)--(9.663,6.576)--(9.665,6.564)%
  --(9.668,6.552)--(9.671,6.540)--(9.673,6.528)--(9.676,6.516)--(9.678,6.503)--(9.681,6.491)%
  --(9.684,6.479)--(9.686,6.466)--(9.689,6.454)--(9.691,6.442)--(9.694,6.429)--(9.697,6.417)%
  --(9.699,6.404)--(9.702,6.392)--(9.704,6.379)--(9.707,6.366)--(9.709,6.354)--(9.712,6.341)%
  --(9.715,6.328)--(9.717,6.315)--(9.720,6.303)--(9.722,6.290)--(9.725,6.277)--(9.728,6.264)%
  --(9.730,6.251)--(9.733,6.238)--(9.735,6.225)--(9.738,6.212)--(9.741,6.199)--(9.743,6.186)%
  --(9.746,6.173)--(9.748,6.159)--(9.751,6.146)--(9.753,6.133)--(9.756,6.120)--(9.759,6.106)%
  --(9.761,6.093)--(9.764,6.080)--(9.766,6.066)--(9.769,6.053)--(9.772,6.039)--(9.774,6.026)%
  --(9.777,6.012)--(9.779,5.999)--(9.782,5.985)--(9.785,5.971)--(9.787,5.958)--(9.790,5.944)%
  --(9.792,5.930)--(9.795,5.917)--(9.797,5.903)--(9.800,5.889)--(9.803,5.875)--(9.805,5.861)%
  --(9.808,5.847)--(9.810,5.833)--(9.813,5.820)--(9.816,5.806)--(9.818,5.792)--(9.821,5.778)%
  --(9.823,5.764)--(9.826,5.749)--(9.829,5.735)--(9.831,5.721)--(9.834,5.707)--(9.836,5.693)%
  --(9.839,5.679)--(9.841,5.665)--(9.844,5.650)--(9.847,5.636)--(9.849,5.622)--(9.852,5.608)%
  --(9.854,5.593)--(9.857,5.579)--(9.860,5.565)--(9.862,5.550)--(9.865,5.536)--(9.867,5.521)%
  --(9.870,5.507)--(9.873,5.492)--(9.875,5.478)--(9.878,5.463)--(9.880,5.449)--(9.883,5.434)%
  --(9.885,5.420)--(9.888,5.405)--(9.891,5.391)--(9.893,5.376)--(9.896,5.361)--(9.898,5.347)%
  --(9.901,5.332)--(9.904,5.317)--(9.906,5.303)--(9.909,5.288)--(9.911,5.273)--(9.914,5.258)%
  --(9.917,5.244)--(9.919,5.229)--(9.922,5.214)--(9.924,5.199)--(9.927,5.184)--(9.929,5.170)%
  --(9.932,5.155)--(9.935,5.140)--(9.937,5.125)--(9.940,5.110)--(9.942,5.095)--(9.945,5.080)%
  --(9.948,5.065)--(9.950,5.050)--(9.953,5.035)--(9.955,5.020)--(9.958,5.005)--(9.961,4.991)%
  --(9.963,4.976)--(9.966,4.961)--(9.968,4.945)--(9.971,4.930)--(9.973,4.915)--(9.976,4.900)%
  --(9.979,4.885)--(9.981,4.870)--(9.984,4.855)--(9.986,4.840)--(9.989,4.825)--(9.992,4.810)%
  --(9.994,4.795)--(9.997,4.780)--(9.999,4.765)--(10.002,4.750)--(10.005,4.734)--(10.007,4.719)%
  --(10.010,4.704)--(10.012,4.689)--(10.015,4.674)--(10.017,4.659)--(10.020,4.644)--(10.023,4.628)%
  --(10.025,4.613)--(10.028,4.598)--(10.030,4.583)--(10.033,4.568)--(10.036,4.553)--(10.038,4.537)%
  --(10.041,4.522)--(10.043,4.507)--(10.046,4.492)--(10.049,4.477)--(10.051,4.462)--(10.054,4.446)%
  --(10.056,4.431)--(10.059,4.416)--(10.061,4.401)--(10.064,4.386)--(10.067,4.371)--(10.069,4.355)%
  --(10.072,4.340)--(10.074,4.325)--(10.077,4.310)--(10.080,4.295)--(10.082,4.280)--(10.085,4.264)%
  --(10.087,4.249)--(10.090,4.234)--(10.093,4.219)--(10.095,4.204)--(10.098,4.189)--(10.100,4.174)%
  --(10.103,4.158)--(10.105,4.143)--(10.108,4.128)--(10.111,4.113)--(10.113,4.098)--(10.116,4.083)%
  --(10.118,4.068)--(10.121,4.053)--(10.124,4.038)--(10.126,4.023)--(10.129,4.008)--(10.131,3.993)%
  --(10.134,3.978)--(10.137,3.963)--(10.139,3.947)--(10.142,3.932)--(10.144,3.917)--(10.147,3.903)%
  --(10.149,3.888)--(10.152,3.873)--(10.155,3.858)--(10.157,3.843)--(10.160,3.828)--(10.162,3.813)%
  --(10.165,3.798)--(10.168,3.783)--(10.170,3.768)--(10.173,3.753)--(10.175,3.738)--(10.178,3.724)%
  --(10.181,3.709)--(10.183,3.694)--(10.186,3.679)--(10.188,3.664)--(10.191,3.650)--(10.193,3.635)%
  --(10.196,3.620)--(10.199,3.605)--(10.201,3.591)--(10.204,3.576)--(10.206,3.561)--(10.209,3.547)%
  --(10.212,3.532)--(10.214,3.517)--(10.217,3.503)--(10.219,3.488)--(10.222,3.474)--(10.225,3.459)%
  --(10.227,3.445)--(10.230,3.430)--(10.232,3.416)--(10.235,3.401)--(10.237,3.387)--(10.240,3.372)%
  --(10.243,3.358)--(10.245,3.343)--(10.248,3.329)--(10.250,3.315)--(10.253,3.300)--(10.256,3.286)%
  --(10.258,3.272)--(10.261,3.258)--(10.263,3.243)--(10.266,3.229)--(10.269,3.215)--(10.271,3.201)%
  --(10.274,3.187)--(10.276,3.173)--(10.279,3.158)--(10.281,3.144)--(10.284,3.130)--(10.287,3.116)%
  --(10.289,3.102)--(10.292,3.088)--(10.294,3.074)--(10.297,3.061)--(10.300,3.047)--(10.302,3.033)%
  --(10.305,3.019)--(10.307,3.005)--(10.310,2.991)--(10.313,2.978)--(10.315,2.964)--(10.318,2.950)%
  --(10.320,2.937)--(10.323,2.923)--(10.325,2.909)--(10.328,2.896)--(10.331,2.882)--(10.333,2.869)%
  --(10.336,2.855)--(10.338,2.842)--(10.341,2.828)--(10.344,2.815)--(10.346,2.802)--(10.349,2.788)%
  --(10.351,2.775)--(10.354,2.762)--(10.357,2.748)--(10.359,2.735)--(10.362,2.722)--(10.364,2.709)%
  --(10.367,2.696)--(10.369,2.683)--(10.372,2.670)--(10.375,2.657)--(10.377,2.644)--(10.380,2.631)%
  --(10.382,2.618)--(10.385,2.605)--(10.388,2.592)--(10.390,2.580)--(10.393,2.567)--(10.395,2.554)%
  --(10.398,2.541)--(10.401,2.529)--(10.403,2.516)--(10.406,2.504)--(10.408,2.491)--(10.411,2.479)%
  --(10.413,2.466)--(10.416,2.454)--(10.419,2.441)--(10.421,2.429)--(10.424,2.417)--(10.426,2.404)%
  --(10.429,2.392)--(10.432,2.380)--(10.434,2.368)--(10.437,2.356)--(10.439,2.344)--(10.442,2.332)%
  --(10.445,2.320)--(10.447,2.308)--(10.450,2.296)--(10.452,2.284)--(10.455,2.272)--(10.457,2.260)%
  --(10.460,2.249)--(10.463,2.237)--(10.465,2.225)--(10.468,2.214)--(10.470,2.202)--(10.473,2.191)%
  --(10.476,2.179)--(10.478,2.168)--(10.481,2.156)--(10.483,2.145)--(10.486,2.134)--(10.489,2.122)%
  --(10.491,2.111)--(10.494,2.100)--(10.496,2.089)--(10.499,2.078)--(10.501,2.067)--(10.504,2.056)%
  --(10.507,2.045)--(10.509,2.034)--(10.512,2.023)--(10.514,2.012)--(10.517,2.002)--(10.520,1.991)%
  --(10.522,1.980)--(10.525,1.970)--(10.527,1.959)--(10.530,1.949)--(10.533,1.938)--(10.535,1.928)%
  --(10.538,1.917)--(10.540,1.907)--(10.543,1.897)--(10.545,1.886)--(10.548,1.876)--(10.551,1.866)%
  --(10.553,1.856)--(10.556,1.846)--(10.558,1.836)--(10.561,1.826)--(10.564,1.816)--(10.566,1.806)%
  --(10.569,1.797)--(10.571,1.787)--(10.574,1.777)--(10.577,1.768)--(10.579,1.758)--(10.582,1.748)%
  --(10.584,1.739)--(10.587,1.730)--(10.589,1.720)--(10.592,1.711)--(10.595,1.702)--(10.597,1.692)%
  --(10.600,1.683)--(10.602,1.674)--(10.605,1.665)--(10.608,1.656)--(10.610,1.647)--(10.613,1.638)%
  --(10.615,1.629)--(10.618,1.621)--(10.621,1.612)--(10.623,1.603)--(10.626,1.595)--(10.628,1.586)%
  --(10.631,1.578)--(10.634,1.569)--(10.636,1.561)--(10.639,1.552)--(10.641,1.544)--(10.644,1.536)%
  --(10.646,1.528)--(10.649,1.520)--(10.652,1.511)--(10.654,1.503)--(10.657,1.495)--(10.659,1.488)%
  --(10.662,1.480)--(10.665,1.472)--(10.667,1.464)--(10.670,1.457)--(10.672,1.449)--(10.675,1.441)%
  --(10.678,1.434)--(10.680,1.426)--(10.683,1.419)--(10.685,1.412)--(10.688,1.405)--(10.690,1.397)%
  --(10.693,1.390)--(10.696,1.383)--(10.698,1.376)--(10.701,1.369)--(10.703,1.362)--(10.706,1.355)%
  --(10.709,1.349)--(10.711,1.342)--(10.714,1.335)--(10.716,1.329)--(10.719,1.322)--(10.722,1.315)%
  --(10.724,1.309)--(10.727,1.303)--(10.729,1.296)--(10.732,1.290)--(10.734,1.284)--(10.737,1.278)%
  --(10.740,1.272)--(10.742,1.266)--(10.745,1.260)--(10.747,1.254)--(10.750,1.248)--(10.753,1.242)%
  --(10.755,1.237)--(10.758,1.231)--(10.760,1.225)--(10.763,1.220)--(10.766,1.214)--(10.768,1.209)%
  --(10.771,1.204)--(10.773,1.198)--(10.776,1.193)--(10.778,1.188)--(10.781,1.183)--(10.784,1.178)%
  --(10.786,1.173)--(10.789,1.168)--(10.791,1.163)--(10.794,1.158)--(10.797,1.154)--(10.799,1.149)%
  --(10.802,1.145)--(10.804,1.140)--(10.807,1.136)--(10.810,1.131)--(10.812,1.127)--(10.815,1.123)%
  --(10.817,1.118)--(10.820,1.114)--(10.822,1.110)--(10.825,1.106)--(10.828,1.102)--(10.830,1.098)%
  --(10.833,1.095)--(10.835,1.091)--(10.838,1.087)--(10.841,1.083)--(10.843,1.080)--(10.846,1.076)%
  --(10.848,1.073)--(10.851,1.070)--(10.854,1.066)--(10.856,1.063)--(10.859,1.060)--(10.861,1.057)%
  --(10.864,1.054)--(10.866,1.051)--(10.869,1.048)--(10.872,1.045)--(10.874,1.042)--(10.877,1.040)%
  --(10.879,1.037)--(10.882,1.034)--(10.885,1.032)--(10.887,1.029)--(10.890,1.027)--(10.892,1.025)%
  --(10.895,1.022)--(10.898,1.020)--(10.900,1.018)--(10.903,1.016)--(10.905,1.014)--(10.908,1.012)%
  --(10.910,1.010)--(10.913,1.008)--(10.916,1.007)--(10.918,1.005)--(10.921,1.003)--(10.923,1.002)%
  --(10.926,1.000)--(10.929,0.999)--(10.931,0.998)--(10.934,0.996)--(10.936,0.995)--(10.939,0.994)%
  --(10.942,0.993)--(10.944,0.992)--(10.947,0.991)--(10.949,0.990)--(10.952,0.989)--(10.954,0.989)%
  --(10.957,0.988)--(10.960,0.987)--(10.962,0.987)--(10.965,0.986)--(10.967,0.986)--(10.970,0.986)%
  --(10.973,0.985)--(10.975,0.985)--(10.978,0.985)--(10.980,0.985)--(10.983,0.985)--(10.986,0.985)%
  --(10.988,0.985)--(10.991,0.985)--(10.993,0.986)--(10.996,0.986)--(10.998,0.986)--(11.001,0.987)%
  --(11.004,0.987)--(11.006,0.988)--(11.009,0.989)--(11.011,0.989)--(11.014,0.990)--(11.017,0.991)%
  --(11.019,0.992)--(11.022,0.993)--(11.024,0.994)--(11.027,0.995)--(11.030,0.996)--(11.032,0.997)%
  --(11.035,0.999)--(11.037,1.000)--(11.040,1.002)--(11.042,1.003)--(11.045,1.005)--(11.048,1.006)%
  --(11.050,1.008)--(11.053,1.010)--(11.055,1.012)--(11.058,1.014)--(11.061,1.016)--(11.063,1.018)%
  --(11.066,1.020)--(11.068,1.022)--(11.071,1.024)--(11.074,1.027)--(11.076,1.029)--(11.079,1.031)%
  --(11.081,1.034)--(11.084,1.036)--(11.086,1.039)--(11.089,1.042)--(11.092,1.045)--(11.094,1.047)%
  --(11.097,1.050)--(11.099,1.053)--(11.102,1.056)--(11.105,1.059)--(11.107,1.062)--(11.110,1.066)%
  --(11.112,1.069)--(11.115,1.072)--(11.118,1.076)--(11.120,1.079)--(11.123,1.083)--(11.125,1.086)%
  --(11.128,1.090)--(11.130,1.094)--(11.133,1.098)--(11.136,1.101)--(11.138,1.105)--(11.141,1.109)%
  --(11.143,1.113)--(11.146,1.118)--(11.149,1.122)--(11.151,1.126)--(11.154,1.130)--(11.156,1.135)%
  --(11.159,1.139)--(11.162,1.144)--(11.164,1.148)--(11.167,1.153)--(11.169,1.158)--(11.172,1.162)%
  --(11.174,1.167)--(11.177,1.172)--(11.180,1.177)--(11.182,1.182)--(11.185,1.187)--(11.187,1.192)%
  --(11.190,1.197)--(11.193,1.203)--(11.195,1.208)--(11.198,1.213)--(11.200,1.219)--(11.203,1.224)%
  --(11.206,1.230)--(11.208,1.236)--(11.211,1.241)--(11.213,1.247)--(11.216,1.253)--(11.218,1.259)%
  --(11.221,1.265)--(11.224,1.271)--(11.226,1.277)--(11.229,1.283)--(11.231,1.289)--(11.234,1.295)%
  --(11.237,1.301)--(11.239,1.308)--(11.242,1.314)--(11.244,1.321)--(11.247,1.327)--(11.250,1.334)%
  --(11.252,1.341)--(11.255,1.347)--(11.257,1.354)--(11.260,1.361)--(11.262,1.368)--(11.265,1.375)%
  --(11.268,1.382)--(11.270,1.389)--(11.273,1.396)--(11.275,1.403)--(11.278,1.410)--(11.281,1.418)%
  --(11.283,1.425)--(11.286,1.433)--(11.288,1.440)--(11.291,1.448)--(11.294,1.455)--(11.296,1.463)%
  --(11.299,1.471)--(11.301,1.478)--(11.304,1.486)--(11.306,1.494)--(11.309,1.502)--(11.312,1.510)%
  --(11.314,1.518)--(11.317,1.526)--(11.319,1.534)--(11.322,1.543)--(11.325,1.551)--(11.327,1.559)%
  --(11.330,1.568)--(11.332,1.576)--(11.335,1.584)--(11.338,1.593)--(11.340,1.602)--(11.343,1.610)%
  --(11.345,1.619)--(11.348,1.628)--(11.350,1.637)--(11.353,1.645)--(11.356,1.654)--(11.358,1.663)%
  --(11.361,1.672)--(11.363,1.682)--(11.366,1.691)--(11.369,1.700)--(11.371,1.709)--(11.374,1.718)%
  --(11.376,1.728)--(11.379,1.737)--(11.382,1.747)--(11.384,1.756)--(11.387,1.766)--(11.389,1.775)%
  --(11.392,1.785)--(11.394,1.795)--(11.397,1.805)--(11.400,1.814)--(11.402,1.824)--(11.405,1.834)%
  --(11.407,1.844)--(11.410,1.854)--(11.413,1.864)--(11.415,1.874)--(11.418,1.885)--(11.420,1.895)%
  --(11.423,1.905)--(11.426,1.915)--(11.428,1.926)--(11.431,1.936)--(11.433,1.947)--(11.436,1.957)%
  --(11.438,1.968)--(11.441,1.978)--(11.444,1.989)--(11.446,2.000)--(11.449,2.010)--(11.451,2.021)%
  --(11.454,2.032)--(11.457,2.043)--(11.459,2.054)--(11.462,2.065)--(11.464,2.076)--(11.467,2.087)%
  --(11.470,2.098)--(11.472,2.109)--(11.475,2.120)--(11.477,2.132)--(11.480,2.143)--(11.482,2.154)%
  --(11.485,2.166)--(11.488,2.177)--(11.490,2.189)--(11.493,2.200)--(11.495,2.212)--(11.498,2.223)%
  --(11.501,2.235)--(11.503,2.247)--(11.506,2.258)--(11.508,2.270)--(11.511,2.282)--(11.514,2.294)%
  --(11.516,2.306)--(11.519,2.317)--(11.521,2.329)--(11.524,2.341)--(11.526,2.354)--(11.529,2.366)%
  --(11.532,2.378)--(11.534,2.390)--(11.537,2.402)--(11.539,2.414)--(11.542,2.427)--(11.545,2.439)%
  --(11.547,2.451)--(11.550,2.464)--(11.552,2.476)--(11.555,2.489)--(11.558,2.501)--(11.560,2.514)%
  --(11.563,2.526)--(11.565,2.539)--(11.568,2.552)--(11.571,2.564)--(11.573,2.577)--(11.576,2.590)%
  --(11.578,2.603)--(11.581,2.616)--(11.583,2.629)--(11.586,2.641)--(11.589,2.654)--(11.591,2.667)%
  --(11.594,2.680)--(11.596,2.693)--(11.599,2.707)--(11.602,2.720)--(11.604,2.733)--(11.607,2.746)%
  --(11.609,2.759)--(11.612,2.773)--(11.615,2.786)--(11.617,2.799)--(11.620,2.812)--(11.622,2.826)%
  --(11.625,2.839)--(11.627,2.853)--(11.630,2.866)--(11.633,2.880)--(11.635,2.893)--(11.638,2.907)%
  --(11.640,2.920)--(11.643,2.934)--(11.646,2.948)--(11.648,2.961)--(11.651,2.975)--(11.653,2.989)%
  --(11.656,3.003)--(11.659,3.016)--(11.661,3.030)--(11.664,3.044)--(11.666,3.058)--(11.669,3.072)%
  --(11.671,3.086)--(11.674,3.100)--(11.677,3.114)--(11.679,3.128)--(11.682,3.142)--(11.684,3.156)%
  --(11.687,3.170)--(11.690,3.184)--(11.692,3.198)--(11.695,3.212)--(11.697,3.227)--(11.700,3.241)%
  --(11.703,3.255)--(11.705,3.269)--(11.708,3.283)--(11.710,3.298)--(11.713,3.312)--(11.715,3.326)%
  --(11.718,3.341)--(11.721,3.355)--(11.723,3.370)--(11.726,3.384)--(11.728,3.398)--(11.731,3.413)%
  --(11.734,3.427)--(11.736,3.442)--(11.739,3.456)--(11.741,3.471)--(11.744,3.486)--(11.747,3.500)%
  --(11.749,3.515)--(11.752,3.529)--(11.754,3.544)--(11.757,3.559)--(11.759,3.573)--(11.762,3.588)%
  --(11.765,3.603)--(11.767,3.617)--(11.770,3.632)--(11.772,3.647)--(11.775,3.662)--(11.778,3.676)%
  --(11.780,3.691)--(11.783,3.706)--(11.785,3.721)--(11.788,3.736)--(11.791,3.751)--(11.793,3.765)%
  --(11.796,3.780)--(11.798,3.795)--(11.801,3.810)--(11.803,3.825)--(11.806,3.840)--(11.809,3.855)%
  --(11.811,3.870)--(11.814,3.885)--(11.816,3.900)--(11.819,3.915)--(11.822,3.930)--(11.824,3.945)%
  --(11.827,3.960)--(11.829,3.975)--(11.832,3.990)--(11.835,4.005)--(11.837,4.020)--(11.840,4.035)%
  --(11.842,4.050)--(11.845,4.065)--(11.847,4.080)--(11.850,4.095)--(11.853,4.110)--(11.855,4.125)%
  --(11.858,4.141)--(11.860,4.156)--(11.863,4.171)--(11.866,4.186)--(11.868,4.201)--(11.871,4.216)%
  --(11.873,4.231)--(11.876,4.246)--(11.879,4.262)--(11.881,4.277)--(11.884,4.292)--(11.886,4.307)%
  --(11.889,4.322)--(11.891,4.337)--(11.894,4.353)--(11.897,4.368)--(11.899,4.383)--(11.902,4.398)%
  --(11.904,4.413)--(11.907,4.428)--(11.910,4.444)--(11.912,4.459)--(11.915,4.474)--(11.917,4.489)%
  --(11.920,4.504)--(11.923,4.519)--(11.925,4.535)--(11.928,4.550)--(11.930,4.565)--(11.933,4.580)%
  --(11.935,4.595)--(11.938,4.610)--(11.941,4.626)--(11.943,4.641)--(11.946,4.656)--(11.948,4.671)%
  --(11.951,4.686)--(11.954,4.701)--(11.956,4.716)--(11.959,4.732)--(11.961,4.747)--(11.964,4.762)%
  --(11.967,4.777)--(11.969,4.792)--(11.972,4.807)--(11.974,4.822)--(11.977,4.837)--(11.979,4.852)%
  --(11.982,4.867)--(11.985,4.883)--(11.987,4.898)--(11.990,4.913)--(11.992,4.928)--(11.995,4.943)%
  --(11.998,4.958)--(12.000,4.973)--(12.003,4.988)--(12.005,5.003)--(12.008,5.018)--(12.011,5.033)%
  --(12.013,5.048)--(12.016,5.063)--(12.018,5.077)--(12.021,5.092)--(12.023,5.107)--(12.026,5.122)%
  --(12.029,5.137)--(12.031,5.152)--(12.034,5.167)--(12.036,5.182)--(12.039,5.196);
\draw[gp path] (1.688,8.381)--(1.688,0.985)--(12.039,0.985)--(12.039,8.381)--cycle;
%% coordinates of the plot area
\gpdefrectangularnode{gp plot 1}{\pgfpoint{1.688cm}{0.985cm}}{\pgfpoint{12.039cm}{8.381cm}}
\end{tikzpicture}
%% gnuplot variables

}
  \caption{Transmission of the variable attenuator}
  \label{fig:ana_var_att}
\end{figure}
To calculate the extinction ratio we determine the minimal and maximal values of the (attenuated) transmitted power. The values are given in table \ref{tab:ana_extinction}. For the error weighted mean values calculated via:
\begin{align*}
\bar{X} = \frac{\sum_{i=0}^N \frac{X_i}{(\Delta X_i)^2}}{\sum_{i=0}^N \frac{1}{(\Delta X_i)^2}}\hspace{2cm} \Delta \bar{X} = \frac{N}{\sum_{i=0}^N \frac{1}{(\Delta X_i)^2}}
\end{align*}
we get:
\begin{align*}
P_\textrm{max}^\textrm{att}&=\unit[(118 \pm 1)]{\mu W}\\
P_\textrm{min}^\textrm{att}&=\unit[(0.51 \pm 0.01)]{\mu W}\textrm{.}
\end{align*}
So the extinction ratio is given by
\begin{align}
\Gamma=\frac{P_\textrm{max}^\textrm{att}}{P_\textrm{min}^\textrm{att}}=\unit[(229 \pm 5)]{}\textrm{,}
\end{align}
with the error
\begin{align*}
\Delta\Gamma=\sqrt{\left(\frac{\Delta P_\textrm{max}^\textrm{att}}{P_\textrm{min}^\textrm{att}}\right)^2+\left(\frac{ P_\textrm{max}^\textrm{att}}{\left(P_\textrm{min}^\textrm{att}\right)^2}\Delta P_\textrm{min}^\textrm{att}\right)^2}\textrm{.}
\end{align*}
\begin{table}[H]
  \centering
\resizebox{0.35\textwidth}{!}{
  \begin{tabular}{lc}
    \toprule
      $\Phi$ [$\unit[]{�}$] $\pm$ $\unit[1]{�}$ & $P$ [$\unit[]{\mu W}$]\\
    \midrule[0.75pt]
$0$ & $118 \pm 1$\\
$45$ & $0.34 \pm 0.01$\\
$90$ & $117 \pm 1$\\
$135$ & $0.80 \pm 0.01$\\
$180$ & $118 \pm 1$\\
$225$ & $0.40 \pm 0.01$\\
    \bottomrule
  \end{tabular}
}
\caption{Minimal and maximal transmitted power of the $\lambda / 2$ plate (attenuated)}
  \label{tab:ana_extinction}
\end{table}
\section{Abstract}

\begin{appendix}
\section{Figures}

\section{Tables}
\begin{table}[H]
  \centering
\resizebox{0.8\textwidth}{!}{
  \begin{tabular}{lcccc}
    \toprule
      $I$ [$\unit[]{mA}$] $\pm$ $\unit[0.5]{mA}$ & $P$ [$\unit[]{\mu W}$]  &  $P_\textrm{background}$ [$\unit[]{\mu W}$] & $\Delta P$ [$\unit[]{\mu W}$]\\
    \midrule[0.75pt]
$1$ & $0.000$ & $0.001$ & $0.001$\\
$11$ & $0.001$ & $0.001$ & $0.001$\\
$20$ & $0.001$ & $0.001$ & $0.001$\\
$30$ & $0.001$ & $0.001$ & $0.001$\\
$40$ & $0.002$ & $0.001$ & $0.001$\\
$45$ & $0.003$ & $0.001$ & $0.001$\\
$50$ & $0.003$ & $0.001$ & $0.001$\\
$55$ & $0.004$ & $0.001$ & $0.001$\\
$60$ & $0.610$ & $0.001$ & $0.001$\\
$65$ & $3.33$ & $0.01$ & $0.01$\\
$70$ & $6.34$ & $0.01$ & $0.01$\\
$75$ & $9.58$ & $0.01$ & $0.01$\\
$80$ & $12.9$ & $0.1$ & $0.1$\\
$90$ & $16.5$ & $0.1$ & $0.1$\\
$100$ & $24.9$ & $0.1$ & $0.1$\\
$110$ & $28.4$ & $0.1$ & $0.1$\\
$120$ & $35.4$ & $0.1$ & $0.1$\\
$130$ & $41.0$ & $0.1$ & $0.1$\\
$140$ & $46.0$ & $0.1$ & $0.1$\\
$150$ & $52.0$ & $0.1$ & $0.1$\\
$160$ & $58.5$ & $0.1$ & $0.1$\\
$170$ & $64.0$ & $0.1$ & $0.1$\\
$180$ & $69.8$ & $0.1$ & $0.1$\\
$190$ & $75.7$ & $0.1$ & $0.1$\\
$200$ & $79.9$ & $0.1$ & $0.1$\\
$210$ & $86.0$ & $0.1$ & $0.1$\\
$220$ & $90.3$ & $0.1$ & $0.1$\\
$230$ & $98.0$ & $0.1$ & $0.1$\\
$240$ & $104$ & $1$ & $1$\\
$250$ & $112$ & $1$ & $1$\\
$260$ & $114$ & $1$ & $1$\\
$270$ & $124$ & $1$ & $1$\\
$280$ & $126$ & $1$ & $1$\\
    \bottomrule
  \end{tabular}
}
\caption{Measured laser power for different input currents with attenuator}
  \label{tab:ana_laserpower_att}
\end{table}
\begin{table}[H]
  \centering
\resizebox{0.8\textwidth}{!}{
  \begin{tabular}{lcccc}
    \toprule
      $I$ [$\unit[]{mA}$] $\pm$ $\unit[0.5]{mA}$ & $P$ [$\unit[]{mW}$]  &  $P_\textrm{background}$ [$\unit[]{mW}$] & $\Delta P$ [$\unit[]{mW}$]\\
    \midrule[0.75pt]
$60$ & $0.65$ & $0.00$ & $0.01$\\
$65$ & $3.07$ & $0.00$ & $0.01$\\
$70$ & $5.56$ & $0.00$ & $0.01$\\
$75$ & $9.08$ & $0.00$ & $0.01$\\
$80$ & $12.2$ & $0.0$ & $0.1$\\
$85$ & $14.5$ & $0.0$ & $0.1$\\
$90$ & $17.4$ & $0.0$ & $0.1$\\
$95$ & $21.4$ & $0.0$ & $0.1$\\
$100$ & $24.2$ & $0.0$ & $0.1$\\
    \bottomrule
  \end{tabular}
}
\caption{Measured laser power for different input currents without attenuator}
  \label{tab:ana_laserpower_woatt}
\end{table}

\begin{table}[H]
  \centering
\begin{floatrow}
\footnotesize
\resizebox{0.33\textwidth}{!}{
  \begin{tabular}{lcc}
    \toprule
      $\Phi$ [$\unit[]{�}$] $\pm$ $\unit[1]{�}$ & $P_\textrm{attenuated}$ [$\unit[]{mW}$]\\
    \midrule[0.75pt]
$0.0$ & $118.0 \pm 1.0$ \\
$2.5$ & $115.0 \pm 1.0$ \\
$5.0$ & $113.0 \pm 1.0$ \\
$7.5$ & $107.0 \pm 1.0$ \\
$10.0$ & $101.0 \pm 1.0$ \\
$12.5$ & $94.0 \pm 1.0$ \\
$15.0$ & $85.0 \pm 1.0$ \\
$17.5$ & $75.0 \pm 1.0$ \\
$20.0$ & $65.0 \pm 1.0$ \\
$22.5$ & $55.0 \pm 1.0$ \\
$25.0$ & $46.0 \pm 1.0$ \\
$27.5$ & $36.0 \pm 1.0$ \\
$30.0$ & $26.0 \pm 1.0$ \\
$32.5$ & $20.0 \pm 1.0$ \\
$35.0$ & $13.0 \pm 1.0$ \\
$37.5$ & $7.0 \pm 0.1$ \\
$40.0$ & $3.0 \pm 0.1$ \\
$42.5$ & $0.8 \pm 0.1$ \\
$45.0$ & $0.34 \pm 0.01$ \\
$47.5$ & $1.5 \pm 0.1$ \\
$50.0$ & $4.8 \pm 0.1$ \\
$52.5$ & $8.8 \pm 0.1$ \\
$55.0$ & $16.0 \pm 1.0$ \\
$57.5$ & $23.0 \pm 1.0$ \\
$60.0$ & $32.0 \pm 1.0$ \\
$62.5$ & $41.0 \pm 1.0$ \\
$65.0$ & $51.0 \pm 1.0$ \\
$67.5$ & $61.0 \pm 1.0$ \\
$70.0$ & $72.0 \pm 1.0$ \\
$72.5$ & $80.0 \pm 1.0$ \\
$75.0$ & $90.0 \pm 1.0$ \\
    \bottomrule
  \end{tabular}
}
\resizebox{0.33\textwidth}{!}{
  \begin{tabular}{lcc}
    \toprule
      $\Phi$ [$\unit[]{�}$] $\pm$ $\unit[1]{�}$ & $P_\textrm{attenuated}$ [$\unit[]{mW}$]\\
    \midrule[0.75pt]
$77.5$ & $97.0 \pm 1.0$ \\
$80.0$ & $105.0 \pm 1.0$ \\
$82.5$ & $110.0 \pm 1.0$ \\
$85.0$ & $115.0 \pm 1.0$ \\
$87.5$ & $117.0 \pm 1.0$ \\
$90.0$ & $117.0 \pm 1.0$ \\
$92.5$ & $116.0 \pm 1.0$ \\
$95.0$ & $113.0 \pm 1.0$ \\
$97.5$ & $109.0 \pm 1.0$ \\
$100.0$ & $102.0 \pm 1.0$ \\
$102.5$ & $95.0 \pm 1.0$ \\
$105.0$ & $87.0 \pm 1.0$ \\
$107.5$ & $78.0 \pm 1.0$ \\
$110.0$ & $67.0 \pm 1.0$ \\
$112.5$ & $58.0 \pm 1.0$ \\
$115.0$ & $46.0 \pm 1.0$ \\
$117.5$ & $39.0 \pm 1.0$ \\
$120.0$ & $28.0 \pm 1.0$ \\
$122.5$ & $21.0 \pm 1.0$ \\
$125.0$ & $13.0 \pm 1.0$ \\
$127.5$ & $8.0 \pm 0.1$ \\
$130.0$ & $4.0 \pm 0.1$ \\
$132.5$ & $2.0 \pm 0.1$ \\
$135.0$ & $0.80 \pm 0.01$ \\
$137.5$ & $1.9 \pm 0.1$ \\
$140.0$ & $5.0 \pm 0.1$ \\
$142.5$ & $9.0 \pm 0.1$ \\
$145.0$ & $16.0 \pm 1.0$ \\
$147.5$ & $22.0 \pm 1.0$ \\
$150.0$ & $31.0 \pm 1.0$ \\
    \bottomrule
  \end{tabular}
}
\resizebox{0.33\textwidth}{!}{
  \begin{tabular}{lcc}
    \toprule
      $\Phi$ [$\unit[]{�}$] $\pm$ $\unit[1]{�}$ & $P_\textrm{attenuated}$ [$\unit[]{mW}$]\\
    \midrule[0.75pt]
$152.5$ & $40.0 \pm 1.0$ \\
$155.0$ & $50.0 \pm 1.0$ \\
$157.5$ & $60.0 \pm 1.0$ \\
$160.0$ & $71.0 \pm 1.0$ \\
$162.5$ & $80.0 \pm 1.0$ \\
$165.0$ & $90.0 \pm 1.0$ \\
$167.5$ & $97.0 \pm 1.0$ \\
$170.0$ & $106.0 \pm 1.0$ \\
$172.5$ & $111.0 \pm 1.0$ \\
$175.0$ & $115.0 \pm 1.0$ \\
$177.5$ & $117.0 \pm 1.0$ \\
$180.0$ & $118.0 \pm 1.0$ \\
$182.5$ & $117.0 \pm 1.0$ \\
$185.0$ & $113.0 \pm 1.0$ \\
$187.5$ & $110.0 \pm 1.0$ \\
$190.0$ & $102.0 \pm 1.0$ \\
$192.5$ & $96.0 \pm 1.0$ \\
$195.0$ & $86.0 \pm 1.0$ \\
$197.5$ & $78.0 \pm 1.0$ \\
$200.0$ & $67.0 \pm 1.0$ \\
$202.5$ & $58.0 \pm 1.0$ \\
$205.0$ & $46.0 \pm 1.0$ \\
$207.5$ & $39.0 \pm 1.0$ \\
$210.0$ & $28.0 \pm 1.0$ \\
$212.5$ & $20.0 \pm 1.0$ \\
$215.0$ & $13.0 \pm 1.0$ \\
$217.5$ & $8.0 \pm 0.1$ \\
$220.0$ & $4.0 \pm 0.1$ \\
$222.5$ & $1.0 \pm 0.1$ \\
$225.0$ & $0.40 \pm 0.01$ \\
    \bottomrule
  \end{tabular}
}
\end{floatrow}
\caption{Transmission of the variable attenuator depending on the rotation angle of the $\lambda / 2$ plate}
  \label{tab:ana_var_att}
\end{table}
\Literatur{quellen}

\end{appendix}
\end{document}


