\documentclass{protokoll_en}
\newcommand{\assistent}{U. Wiedemann}
\newcommand{\versuch}{Optical Frequency Doubling}
\newcommand{\nummer}{A245}

\begin{document}

\section{Preface}

\section{Theoretical Background}

\section{Experimentation and Analysis}
\subsection{Diode Laser and power measurements}\label{ana_laserpower}
To get familiar with the laser we first examine the laser power depencence on the input current. Therefore we put a gauged photodiode directly in front of the laser and record the laser power for different values of the input current. Do avoid the photodiode to leave its 'linear range' we put an attenuator in front of it. The measured values are displayed in table \ref{tab:ana_laserpower_att} in the appendix. In order to gauge the attenuator we also recorded some values in the range between threshold current and an output of $\unit[50]{\mu W}$ without using the attenuator. These values are displayed in table \ref{tab:ana_laserpower_woatt} in the appendix.

Figure \ref{fig:ana_laser_watt} and \ref{fig:ana_laser_woatt} show plots of laser power with and without the attenuator as well as linear fits to the data taken above the threshold current. The fit results read:
\begin{align}
P_\textrm{att}(I) &= b_\textrm{att} + m_\textrm{att} \,\cdotp\, I = \unit[(-33.8 \pm 0.14)]{\mu W} + \unit[(0.574 \pm 0.002
)]{\frac{\mu W}{mA}}\,\cdotp\, I\\
P_\textrm{real}(I) &= b_\textrm{real} + m_\textrm{real} \,\cdotp\, I =\unit[(-36 \pm 1.5)]{mW} + \unit[(0.60 \pm 0.02
)]{\frac{mW}{mA}}\,\cdotp\, I
\end{align}
The ratio of the gradients yields the attenuation:
\begin{align}
\alpha = \unit[(1049 \pm 36)]{}
\end{align}
where the error is calculated via
\begin{align*}
\Delta \alpha = \sqrt{\left(\frac{\Delta m_\textrm{att}}{m_\textrm{real}}\right)^2 + \left(\frac{m_\textrm{att}}{m^2_\textrm{real}}\Delta m_\textrm{real}\right)^2}
\end{align*}
\begin{figure}[H]
\begin{floatrow}
\ffigbox[0.5\textwidth]{}
{
\resizebox{0.5\textwidth}{!}{
	\begin{tikzpicture}[gnuplot]
%% generated with GNUPLOT 4.4p0 (Lua 5.1.4; terminal rev. 97, script rev. 96a)
%% 29.04.2010 21:37:31
\gpcolor{gp lt color border}
\gpsetlinetype{gp lt border}
\gpsetlinewidth{1.00}
\draw[gp path] (1.688,1.337)--(1.868,1.337);
\draw[gp path] (12.039,1.337)--(11.859,1.337);
\node[gp node right] at (1.504,1.337) { 0};
\draw[gp path] (1.688,3.098)--(1.868,3.098);
\draw[gp path] (12.039,3.098)--(11.859,3.098);
\node[gp node right] at (1.504,3.098) { 50};
\draw[gp path] (1.688,4.859)--(1.868,4.859);
\draw[gp path] (12.039,4.859)--(11.859,4.859);
\node[gp node right] at (1.504,4.859) { 100};
\draw[gp path] (1.688,6.620)--(1.868,6.620);
\draw[gp path] (12.039,6.620)--(11.859,6.620);
\node[gp node right] at (1.504,6.620) { 150};
\draw[gp path] (1.688,8.381)--(1.868,8.381);
\draw[gp path] (12.039,8.381)--(11.859,8.381);
\node[gp node right] at (1.504,8.381) { 200};
\draw[gp path] (1.688,0.985)--(1.688,1.165);
\draw[gp path] (1.688,8.381)--(1.688,8.201);
\node[gp node center] at (1.688,0.677) { 0};
\draw[gp path] (3.413,0.985)--(3.413,1.165);
\draw[gp path] (3.413,8.381)--(3.413,8.201);
\node[gp node center] at (3.413,0.677) { 50};
\draw[gp path] (5.138,0.985)--(5.138,1.165);
\draw[gp path] (5.138,8.381)--(5.138,8.201);
\node[gp node center] at (5.138,0.677) { 100};
\draw[gp path] (6.863,0.985)--(6.863,1.165);
\draw[gp path] (6.863,8.381)--(6.863,8.201);
\node[gp node center] at (6.863,0.677) { 150};
\draw[gp path] (8.589,0.985)--(8.589,1.165);
\draw[gp path] (8.589,8.381)--(8.589,8.201);
\node[gp node center] at (8.589,0.677) { 200};
\draw[gp path] (10.314,0.985)--(10.314,1.165);
\draw[gp path] (10.314,8.381)--(10.314,8.201);
\node[gp node center] at (10.314,0.677) { 250};
\draw[gp path] (12.039,0.985)--(12.039,1.165);
\draw[gp path] (12.039,8.381)--(12.039,8.201);
\node[gp node center] at (12.039,0.677) { 300};
\draw[gp path] (1.688,8.381)--(1.688,0.985)--(12.039,0.985)--(12.039,8.381)--cycle;
\node[gp node center,rotate=-270] at (0.430,4.683) {Laser power [$\mu$W]};
\node[gp node center] at (6.863,0.215) {Input current [mA]};
\node[gp node right] at (10.571,8.047) {Measured Data};
\gpcolor{gp lt color 0}
\gpsetlinetype{gp lt plot 0}
\draw[gp path] (10.755,8.047)--(11.671,8.047);
\draw[gp path] (10.755,8.137)--(10.755,7.957);
\draw[gp path] (11.671,8.137)--(11.671,7.957);
\draw[gp path] (1.633,1.337)--(1.813,1.337);
\draw[gp path] (1.633,1.337)--(1.813,1.337);
\draw[gp path] (1.978,1.337)--(2.158,1.337);
\draw[gp path] (1.978,1.337)--(2.158,1.337);
\draw[gp path] (2.288,1.337)--(2.468,1.337);
\draw[gp path] (2.288,1.337)--(2.468,1.337);
\draw[gp path] (2.633,1.337)--(2.813,1.337);
\draw[gp path] (2.633,1.337)--(2.813,1.337);
\draw[gp path] (2.978,1.337)--(3.158,1.337);
\draw[gp path] (2.978,1.337)--(3.158,1.337);
\draw[gp path] (3.151,1.337)--(3.331,1.337);
\draw[gp path] (3.151,1.337)--(3.331,1.337);
\draw[gp path] (3.323,1.337)--(3.503,1.337);
\draw[gp path] (3.323,1.337)--(3.503,1.337);
\draw[gp path] (3.496,1.337)--(3.676,1.337);
\draw[gp path] (3.496,1.337)--(3.676,1.337);
\draw[gp path] (3.668,1.359)--(3.848,1.359);
\draw[gp path] (3.668,1.359)--(3.848,1.359);
\draw[gp path] (3.931,1.454)--(3.931,1.455);
\draw[gp path] (3.841,1.454)--(4.021,1.454);
\draw[gp path] (3.841,1.455)--(4.021,1.455);
\draw[gp path] (4.103,1.560)--(4.103,1.561);
\draw[gp path] (4.013,1.560)--(4.193,1.560);
\draw[gp path] (4.013,1.561)--(4.193,1.561);
\draw[gp path] (4.276,1.674)--(4.276,1.675);
\draw[gp path] (4.186,1.674)--(4.366,1.674);
\draw[gp path] (4.186,1.675)--(4.366,1.675);
\draw[gp path] (4.448,1.788)--(4.448,1.795);
\draw[gp path] (4.358,1.788)--(4.538,1.788);
\draw[gp path] (4.358,1.795)--(4.538,1.795);
\draw[gp path] (4.793,1.915)--(4.793,1.922);
\draw[gp path] (4.703,1.915)--(4.883,1.915);
\draw[gp path] (4.703,1.922)--(4.883,1.922);
\draw[gp path] (5.138,2.211)--(5.138,2.218);
\draw[gp path] (5.048,2.211)--(5.228,2.211);
\draw[gp path] (5.048,2.218)--(5.228,2.218);
\draw[gp path] (5.483,2.334)--(5.483,2.341);
\draw[gp path] (5.393,2.334)--(5.573,2.334);
\draw[gp path] (5.393,2.341)--(5.573,2.341);
\draw[gp path] (5.828,2.580)--(5.828,2.587);
\draw[gp path] (5.738,2.580)--(5.918,2.580);
\draw[gp path] (5.738,2.587)--(5.918,2.587);
\draw[gp path] (6.173,2.778)--(6.173,2.785);
\draw[gp path] (6.083,2.778)--(6.263,2.778);
\draw[gp path] (6.083,2.785)--(6.263,2.785);
\draw[gp path] (6.518,2.954)--(6.518,2.961);
\draw[gp path] (6.428,2.954)--(6.608,2.954);
\draw[gp path] (6.428,2.961)--(6.608,2.961);
\draw[gp path] (6.863,3.165)--(6.863,3.172);
\draw[gp path] (6.773,3.165)--(6.953,3.165);
\draw[gp path] (6.773,3.172)--(6.953,3.172);
\draw[gp path] (7.209,3.394)--(7.209,3.401);
\draw[gp path] (7.119,3.394)--(7.299,3.394);
\draw[gp path] (7.119,3.401)--(7.299,3.401);
\draw[gp path] (7.554,3.588)--(7.554,3.595);
\draw[gp path] (7.464,3.588)--(7.644,3.588);
\draw[gp path] (7.464,3.595)--(7.644,3.595);
\draw[gp path] (7.899,3.792)--(7.899,3.799);
\draw[gp path] (7.809,3.792)--(7.989,3.792);
\draw[gp path] (7.809,3.799)--(7.989,3.799);
\draw[gp path] (8.244,4.000)--(8.244,4.007);
\draw[gp path] (8.154,4.000)--(8.334,4.000);
\draw[gp path] (8.154,4.007)--(8.334,4.007);
\draw[gp path] (8.589,4.148)--(8.589,4.155);
\draw[gp path] (8.499,4.148)--(8.679,4.148);
\draw[gp path] (8.499,4.155)--(8.679,4.155);
\draw[gp path] (8.934,4.363)--(8.934,4.370);
\draw[gp path] (8.844,4.363)--(9.024,4.363);
\draw[gp path] (8.844,4.370)--(9.024,4.370);
\draw[gp path] (9.279,4.514)--(9.279,4.521);
\draw[gp path] (9.189,4.514)--(9.369,4.514);
\draw[gp path] (9.189,4.521)--(9.369,4.521);
\draw[gp path] (9.624,4.785)--(9.624,4.792);
\draw[gp path] (9.534,4.785)--(9.714,4.785);
\draw[gp path] (9.534,4.792)--(9.714,4.792);
\draw[gp path] (9.969,4.965)--(9.969,5.035);
\draw[gp path] (9.879,4.965)--(10.059,4.965);
\draw[gp path] (9.879,5.035)--(10.059,5.035);
\draw[gp path] (10.314,5.247)--(10.314,5.317);
\draw[gp path] (10.224,5.247)--(10.404,5.247);
\draw[gp path] (10.224,5.317)--(10.404,5.317);
\draw[gp path] (10.659,5.317)--(10.659,5.387);
\draw[gp path] (10.569,5.317)--(10.749,5.317);
\draw[gp path] (10.569,5.387)--(10.749,5.387);
\draw[gp path] (11.004,5.669)--(11.004,5.740);
\draw[gp path] (10.914,5.669)--(11.094,5.669);
\draw[gp path] (10.914,5.740)--(11.094,5.740);
\draw[gp path] (11.349,5.740)--(11.349,5.810);
\draw[gp path] (11.259,5.740)--(11.439,5.740);
\draw[gp path] (11.259,5.810)--(11.439,5.810);
\draw[gp path] (1.705,1.337)--(1.740,1.337);
\draw[gp path] (1.705,1.247)--(1.705,1.427);
\draw[gp path] (1.740,1.247)--(1.740,1.427);
\draw[gp path] (2.050,1.337)--(2.085,1.337);
\draw[gp path] (2.050,1.247)--(2.050,1.427);
\draw[gp path] (2.085,1.247)--(2.085,1.427);
\draw[gp path] (2.361,1.337)--(2.395,1.337);
\draw[gp path] (2.361,1.247)--(2.361,1.427);
\draw[gp path] (2.395,1.247)--(2.395,1.427);
\draw[gp path] (2.706,1.337)--(2.740,1.337);
\draw[gp path] (2.706,1.247)--(2.706,1.427);
\draw[gp path] (2.740,1.247)--(2.740,1.427);
\draw[gp path] (3.051,1.337)--(3.085,1.337);
\draw[gp path] (3.051,1.247)--(3.051,1.427);
\draw[gp path] (3.085,1.247)--(3.085,1.427);
\draw[gp path] (3.223,1.337)--(3.258,1.337);
\draw[gp path] (3.223,1.247)--(3.223,1.427);
\draw[gp path] (3.258,1.247)--(3.258,1.427);
\draw[gp path] (3.396,1.337)--(3.430,1.337);
\draw[gp path] (3.396,1.247)--(3.396,1.427);
\draw[gp path] (3.430,1.247)--(3.430,1.427);
\draw[gp path] (3.568,1.337)--(3.603,1.337);
\draw[gp path] (3.568,1.247)--(3.568,1.427);
\draw[gp path] (3.603,1.247)--(3.603,1.427);
\draw[gp path] (3.741,1.359)--(3.775,1.359);
\draw[gp path] (3.741,1.269)--(3.741,1.449);
\draw[gp path] (3.775,1.269)--(3.775,1.449);
\draw[gp path] (3.913,1.454)--(3.948,1.454);
\draw[gp path] (3.913,1.364)--(3.913,1.544);
\draw[gp path] (3.948,1.364)--(3.948,1.544);
\draw[gp path] (4.086,1.560)--(4.120,1.560);
\draw[gp path] (4.086,1.470)--(4.086,1.650);
\draw[gp path] (4.120,1.470)--(4.120,1.650);
\draw[gp path] (4.258,1.675)--(4.293,1.675);
\draw[gp path] (4.258,1.585)--(4.258,1.765);
\draw[gp path] (4.293,1.585)--(4.293,1.765);
\draw[gp path] (4.431,1.792)--(4.466,1.792);
\draw[gp path] (4.431,1.702)--(4.431,1.882);
\draw[gp path] (4.466,1.702)--(4.466,1.882);
\draw[gp path] (4.776,1.918)--(4.811,1.918);
\draw[gp path] (4.776,1.828)--(4.776,2.008);
\draw[gp path] (4.811,1.828)--(4.811,2.008);
\draw[gp path] (5.121,2.214)--(5.156,2.214);
\draw[gp path] (5.121,2.124)--(5.121,2.304);
\draw[gp path] (5.156,2.124)--(5.156,2.304);
\draw[gp path] (5.466,2.337)--(5.501,2.337);
\draw[gp path] (5.466,2.247)--(5.466,2.427);
\draw[gp path] (5.501,2.247)--(5.501,2.427);
\draw[gp path] (5.811,2.584)--(5.846,2.584);
\draw[gp path] (5.811,2.494)--(5.811,2.674);
\draw[gp path] (5.846,2.494)--(5.846,2.674);
\draw[gp path] (6.156,2.781)--(6.191,2.781);
\draw[gp path] (6.156,2.691)--(6.156,2.871);
\draw[gp path] (6.191,2.691)--(6.191,2.871);
\draw[gp path] (6.501,2.957)--(6.536,2.957);
\draw[gp path] (6.501,2.867)--(6.501,3.047);
\draw[gp path] (6.536,2.867)--(6.536,3.047);
\draw[gp path] (6.846,3.169)--(6.881,3.169);
\draw[gp path] (6.846,3.079)--(6.846,3.259);
\draw[gp path] (6.881,3.079)--(6.881,3.259);
\draw[gp path] (7.191,3.398)--(7.226,3.398);
\draw[gp path] (7.191,3.308)--(7.191,3.488);
\draw[gp path] (7.226,3.308)--(7.226,3.488);
\draw[gp path] (7.536,3.591)--(7.571,3.591);
\draw[gp path] (7.536,3.501)--(7.536,3.681);
\draw[gp path] (7.571,3.501)--(7.571,3.681);
\draw[gp path] (7.881,3.795)--(7.916,3.795);
\draw[gp path] (7.881,3.705)--(7.881,3.885);
\draw[gp path] (7.916,3.705)--(7.916,3.885);
\draw[gp path] (8.226,4.003)--(8.261,4.003);
\draw[gp path] (8.226,3.913)--(8.226,4.093);
\draw[gp path] (8.261,3.913)--(8.261,4.093);
\draw[gp path] (8.571,4.151)--(8.606,4.151);
\draw[gp path] (8.571,4.061)--(8.571,4.241);
\draw[gp path] (8.606,4.061)--(8.606,4.241);
\draw[gp path] (8.916,4.366)--(8.951,4.366);
\draw[gp path] (8.916,4.276)--(8.916,4.456);
\draw[gp path] (8.951,4.276)--(8.951,4.456);
\draw[gp path] (9.261,4.517)--(9.296,4.517);
\draw[gp path] (9.261,4.427)--(9.261,4.607);
\draw[gp path] (9.296,4.427)--(9.296,4.607);
\draw[gp path] (9.607,4.789)--(9.641,4.789);
\draw[gp path] (9.607,4.699)--(9.607,4.879);
\draw[gp path] (9.641,4.699)--(9.641,4.879);
\draw[gp path] (9.952,5.000)--(9.986,5.000);
\draw[gp path] (9.952,4.910)--(9.952,5.090);
\draw[gp path] (9.986,4.910)--(9.986,5.090);
\draw[gp path] (10.297,5.282)--(10.331,5.282);
\draw[gp path] (10.297,5.192)--(10.297,5.372);
\draw[gp path] (10.331,5.192)--(10.331,5.372);
\draw[gp path] (10.642,5.352)--(10.676,5.352);
\draw[gp path] (10.642,5.262)--(10.642,5.442);
\draw[gp path] (10.676,5.262)--(10.676,5.442);
\draw[gp path] (10.987,5.704)--(11.021,5.704);
\draw[gp path] (10.987,5.614)--(10.987,5.794);
\draw[gp path] (11.021,5.614)--(11.021,5.794);
\draw[gp path] (11.332,5.775)--(11.366,5.775);
\draw[gp path] (11.332,5.685)--(11.332,5.865);
\draw[gp path] (11.366,5.685)--(11.366,5.865);
\gpsetpointsize{4.00}
\gppoint{gp mark 1}{(1.723,1.337)}
\gppoint{gp mark 1}{(2.068,1.337)}
\gppoint{gp mark 1}{(2.378,1.337)}
\gppoint{gp mark 1}{(2.723,1.337)}
\gppoint{gp mark 1}{(3.068,1.337)}
\gppoint{gp mark 1}{(3.241,1.337)}
\gppoint{gp mark 1}{(3.413,1.337)}
\gppoint{gp mark 1}{(3.586,1.337)}
\gppoint{gp mark 1}{(3.758,1.359)}
\gppoint{gp mark 1}{(3.931,1.454)}
\gppoint{gp mark 1}{(4.103,1.560)}
\gppoint{gp mark 1}{(4.276,1.675)}
\gppoint{gp mark 1}{(4.448,1.792)}
\gppoint{gp mark 1}{(4.793,1.918)}
\gppoint{gp mark 1}{(5.138,2.214)}
\gppoint{gp mark 1}{(5.483,2.337)}
\gppoint{gp mark 1}{(5.828,2.584)}
\gppoint{gp mark 1}{(6.173,2.781)}
\gppoint{gp mark 1}{(6.518,2.957)}
\gppoint{gp mark 1}{(6.863,3.169)}
\gppoint{gp mark 1}{(7.209,3.398)}
\gppoint{gp mark 1}{(7.554,3.591)}
\gppoint{gp mark 1}{(7.899,3.795)}
\gppoint{gp mark 1}{(8.244,4.003)}
\gppoint{gp mark 1}{(8.589,4.151)}
\gppoint{gp mark 1}{(8.934,4.366)}
\gppoint{gp mark 1}{(9.279,4.517)}
\gppoint{gp mark 1}{(9.624,4.789)}
\gppoint{gp mark 1}{(9.969,5.000)}
\gppoint{gp mark 1}{(10.314,5.282)}
\gppoint{gp mark 1}{(10.659,5.352)}
\gppoint{gp mark 1}{(11.004,5.704)}
\gppoint{gp mark 1}{(11.349,5.775)}
\gppoint{gp mark 1}{(11.213,8.047)}
\gpcolor{gp lt color border}
\node[gp node right] at (10.571,7.739) {Fit};
\gpsetlinetype{gp lt border}
\draw[gp path] (10.755,7.739)--(11.671,7.739);
\draw[gp path] (3.120,0.985)--(3.152,1.004)--(3.256,1.065)--(3.361,1.126)--(3.465,1.187)%
  --(3.570,1.249)--(3.675,1.310)--(3.779,1.371)--(3.884,1.432)--(3.988,1.493)--(4.093,1.555)%
  --(4.197,1.616)--(4.302,1.677)--(4.406,1.738)--(4.511,1.800)--(4.616,1.861)--(4.720,1.922)%
  --(4.825,1.983)--(4.929,2.045)--(5.034,2.106)--(5.138,2.167)--(5.243,2.228)--(5.347,2.289)%
  --(5.452,2.351)--(5.557,2.412)--(5.661,2.473)--(5.766,2.534)--(5.870,2.596)--(5.975,2.657)%
  --(6.079,2.718)--(6.184,2.779)--(6.288,2.841)--(6.393,2.902)--(6.498,2.963)--(6.602,3.024)%
  --(6.707,3.086)--(6.811,3.147)--(6.916,3.208)--(7.020,3.269)--(7.125,3.330)--(7.229,3.392)%
  --(7.334,3.453)--(7.439,3.514)--(7.543,3.575)--(7.648,3.637)--(7.752,3.698)--(7.857,3.759)%
  --(7.961,3.820)--(8.066,3.882)--(8.170,3.943)--(8.275,4.004)--(8.380,4.065)--(8.484,4.127)%
  --(8.589,4.188)--(8.693,4.249)--(8.798,4.310)--(8.902,4.371)--(9.007,4.433)--(9.111,4.494)%
  --(9.216,4.555)--(9.321,4.616)--(9.425,4.678)--(9.530,4.739)--(9.634,4.800)--(9.739,4.861)%
  --(9.843,4.923)--(9.948,4.984)--(10.052,5.045)--(10.157,5.106)--(10.262,5.167)--(10.366,5.229)%
  --(10.471,5.290)--(10.575,5.351)--(10.680,5.412)--(10.784,5.474)--(10.889,5.535)--(10.993,5.596)%
  --(11.098,5.657)--(11.203,5.719)--(11.307,5.780)--(11.412,5.841)--(11.516,5.902)--(11.621,5.964)%
  --(11.725,6.025)--(11.830,6.086)--(11.934,6.147)--(12.039,6.208);
\draw[gp path] (1.688,8.381)--(1.688,0.985)--(12.039,0.985)--(12.039,8.381)--cycle;
%% coordinates of the plot area
\gpdefrectangularnode{gp plot 1}{\pgfpoint{1.688cm}{0.985cm}}{\pgfpoint{12.039cm}{8.381cm}}
\end{tikzpicture}
%% gnuplot variables

}
	\caption{Attenuated laser output power versus input current and corresponding linear fit}
	\label{fig:ana_laser_watt}
}
\ffigbox[0.5\textwidth]{}
{
\resizebox{0.5\textwidth}{!}{
	\begin{tikzpicture}[gnuplot]
%% generated with GNUPLOT 4.4p0 (Lua 5.1.4; terminal rev. 97, script rev. 96a)
%% 29.04.2010 21:37:32
\gpcolor{gp lt color border}
\gpsetlinetype{gp lt border}
\gpsetlinewidth{1.00}
\draw[gp path] (1.504,0.985)--(1.684,0.985);
\draw[gp path] (12.039,0.985)--(11.859,0.985);
\node[gp node right] at (1.320,0.985) { 0};
\draw[gp path] (1.504,2.218)--(1.684,2.218);
\draw[gp path] (12.039,2.218)--(11.859,2.218);
\node[gp node right] at (1.320,2.218) { 5};
\draw[gp path] (1.504,3.450)--(1.684,3.450);
\draw[gp path] (12.039,3.450)--(11.859,3.450);
\node[gp node right] at (1.320,3.450) { 10};
\draw[gp path] (1.504,4.683)--(1.684,4.683);
\draw[gp path] (12.039,4.683)--(11.859,4.683);
\node[gp node right] at (1.320,4.683) { 15};
\draw[gp path] (1.504,5.916)--(1.684,5.916);
\draw[gp path] (12.039,5.916)--(11.859,5.916);
\node[gp node right] at (1.320,5.916) { 20};
\draw[gp path] (1.504,7.148)--(1.684,7.148);
\draw[gp path] (12.039,7.148)--(11.859,7.148);
\node[gp node right] at (1.320,7.148) { 25};
\draw[gp path] (1.504,8.381)--(1.684,8.381);
\draw[gp path] (12.039,8.381)--(11.859,8.381);
\node[gp node right] at (1.320,8.381) { 30};
\draw[gp path] (2.557,0.985)--(2.557,1.165);
\draw[gp path] (2.557,8.381)--(2.557,8.201);
\node[gp node center] at (2.557,0.677) { 60};
\draw[gp path] (4.664,0.985)--(4.664,1.165);
\draw[gp path] (4.664,8.381)--(4.664,8.201);
\node[gp node center] at (4.664,0.677) { 70};
\draw[gp path] (6.771,0.985)--(6.771,1.165);
\draw[gp path] (6.771,8.381)--(6.771,8.201);
\node[gp node center] at (6.771,0.677) { 80};
\draw[gp path] (8.878,0.985)--(8.878,1.165);
\draw[gp path] (8.878,8.381)--(8.878,8.201);
\node[gp node center] at (8.878,0.677) { 90};
\draw[gp path] (10.985,0.985)--(10.985,1.165);
\draw[gp path] (10.985,8.381)--(10.985,8.201);
\node[gp node center] at (10.985,0.677) { 100};
\draw[gp path] (1.504,8.381)--(1.504,0.985)--(12.039,0.985)--(12.039,8.381)--cycle;
\node[gp node center,rotate=-270] at (0.430,4.683) {Laser power [mW]};
\node[gp node center] at (6.771,0.215) {Input current [mA]};
\node[gp node right] at (10.571,8.047) {Measured Data};
\gpcolor{gp lt color 0}
\gpsetlinetype{gp lt plot 0}
\draw[gp path] (10.755,8.047)--(11.671,8.047);
\draw[gp path] (10.755,8.137)--(10.755,7.957);
\draw[gp path] (11.671,8.137)--(11.671,7.957);
\draw[gp path] (2.557,1.142)--(2.557,1.146);
\draw[gp path] (2.467,1.142)--(2.647,1.142);
\draw[gp path] (2.467,1.146)--(2.647,1.146);
\draw[gp path] (3.611,1.739)--(3.611,1.744);
\draw[gp path] (3.521,1.739)--(3.701,1.739);
\draw[gp path] (3.521,1.744)--(3.701,1.744);
\draw[gp path] (4.664,2.353)--(4.664,2.358);
\draw[gp path] (4.574,2.353)--(4.754,2.353);
\draw[gp path] (4.574,2.358)--(4.754,2.358);
\draw[gp path] (5.718,3.221)--(5.718,3.226);
\draw[gp path] (5.628,3.221)--(5.808,3.221);
\draw[gp path] (5.628,3.226)--(5.808,3.226);
\draw[gp path] (6.771,3.968)--(6.771,4.017);
\draw[gp path] (6.681,3.968)--(6.861,3.968);
\draw[gp path] (6.681,4.017)--(6.861,4.017);
\draw[gp path] (7.825,4.535)--(7.825,4.584);
\draw[gp path] (7.735,4.535)--(7.915,4.535);
\draw[gp path] (7.735,4.584)--(7.915,4.584);
\draw[gp path] (8.878,5.250)--(8.878,5.299);
\draw[gp path] (8.788,5.250)--(8.968,5.250);
\draw[gp path] (8.788,5.299)--(8.968,5.299);
\draw[gp path] (9.932,6.236)--(9.932,6.285);
\draw[gp path] (9.842,6.236)--(10.022,6.236);
\draw[gp path] (9.842,6.285)--(10.022,6.285);
\draw[gp path] (10.985,6.926)--(10.985,6.976);
\draw[gp path] (10.895,6.926)--(11.075,6.926);
\draw[gp path] (10.895,6.976)--(11.075,6.976);
\draw[gp path] (2.452,1.144)--(2.663,1.144);
\draw[gp path] (2.452,1.054)--(2.452,1.234);
\draw[gp path] (2.663,1.054)--(2.663,1.234);
\draw[gp path] (3.506,1.742)--(3.716,1.742);
\draw[gp path] (3.506,1.652)--(3.506,1.832);
\draw[gp path] (3.716,1.652)--(3.716,1.832);
\draw[gp path] (4.559,2.356)--(4.770,2.356);
\draw[gp path] (4.559,2.266)--(4.559,2.446);
\draw[gp path] (4.770,2.266)--(4.770,2.446);
\draw[gp path] (5.613,3.224)--(5.823,3.224);
\draw[gp path] (5.613,3.134)--(5.613,3.314);
\draw[gp path] (5.823,3.134)--(5.823,3.314);
\draw[gp path] (6.666,3.993)--(6.877,3.993);
\draw[gp path] (6.666,3.903)--(6.666,4.083);
\draw[gp path] (6.877,3.903)--(6.877,4.083);
\draw[gp path] (7.720,4.560)--(7.930,4.560);
\draw[gp path] (7.720,4.470)--(7.720,4.650);
\draw[gp path] (7.930,4.470)--(7.930,4.650);
\draw[gp path] (8.773,5.275)--(8.984,5.275);
\draw[gp path] (8.773,5.185)--(8.773,5.365);
\draw[gp path] (8.984,5.185)--(8.984,5.365);
\draw[gp path] (9.827,6.261)--(10.037,6.261);
\draw[gp path] (9.827,6.171)--(9.827,6.351);
\draw[gp path] (10.037,6.171)--(10.037,6.351);
\draw[gp path] (10.880,6.951)--(11.091,6.951);
\draw[gp path] (10.880,6.861)--(10.880,7.041);
\draw[gp path] (11.091,6.861)--(11.091,7.041);
\gpsetpointsize{4.00}
\gppoint{gp mark 1}{(2.557,1.144)}
\gppoint{gp mark 1}{(3.611,1.742)}
\gppoint{gp mark 1}{(4.664,2.356)}
\gppoint{gp mark 1}{(5.718,3.224)}
\gppoint{gp mark 1}{(6.771,3.993)}
\gppoint{gp mark 1}{(7.825,4.560)}
\gppoint{gp mark 1}{(8.878,5.275)}
\gppoint{gp mark 1}{(9.932,6.261)}
\gppoint{gp mark 1}{(10.985,6.951)}
\gppoint{gp mark 1}{(11.213,8.047)}
\gpcolor{gp lt color border}
\node[gp node right] at (10.571,7.739) {Fit};
\gpsetlinetype{gp lt border}
\draw[gp path] (10.755,7.739)--(11.671,7.739);
\draw[gp path] (2.598,0.985)--(2.675,1.039)--(2.781,1.114)--(2.887,1.189)--(2.994,1.264)%
  --(3.100,1.339)--(3.207,1.414)--(3.313,1.489)--(3.419,1.564)--(3.526,1.639)--(3.632,1.714)%
  --(3.739,1.788)--(3.845,1.863)--(3.952,1.938)--(4.058,2.013)--(4.164,2.088)--(4.271,2.163)%
  --(4.377,2.238)--(4.484,2.313)--(4.590,2.388)--(4.696,2.463)--(4.803,2.538)--(4.909,2.613)%
  --(5.016,2.688)--(5.122,2.763)--(5.228,2.838)--(5.335,2.912)--(5.441,2.987)--(5.548,3.062)%
  --(5.654,3.137)--(5.761,3.212)--(5.867,3.287)--(5.973,3.362)--(6.080,3.437)--(6.186,3.512)%
  --(6.293,3.587)--(6.399,3.662)--(6.505,3.737)--(6.612,3.812)--(6.718,3.887)--(6.825,3.962)%
  --(6.931,4.036)--(7.038,4.111)--(7.144,4.186)--(7.250,4.261)--(7.357,4.336)--(7.463,4.411)%
  --(7.570,4.486)--(7.676,4.561)--(7.782,4.636)--(7.889,4.711)--(7.995,4.786)--(8.102,4.861)%
  --(8.208,4.936)--(8.315,5.011)--(8.421,5.086)--(8.527,5.160)--(8.634,5.235)--(8.740,5.310)%
  --(8.847,5.385)--(8.953,5.460)--(9.059,5.535)--(9.166,5.610)--(9.272,5.685)--(9.379,5.760)%
  --(9.485,5.835)--(9.591,5.910)--(9.698,5.985)--(9.804,6.060)--(9.911,6.135)--(10.017,6.210)%
  --(10.124,6.284)--(10.230,6.359)--(10.336,6.434)--(10.443,6.509)--(10.549,6.584)--(10.656,6.659)%
  --(10.762,6.734)--(10.868,6.809)--(10.975,6.884)--(11.081,6.959)--(11.188,7.034)--(11.294,7.109)%
  --(11.401,7.184)--(11.507,7.259)--(11.613,7.334)--(11.720,7.408)--(11.826,7.483)--(11.933,7.558)%
  --(12.039,7.633);
\draw[gp path] (1.504,8.381)--(1.504,0.985)--(12.039,0.985)--(12.039,8.381)--cycle;
%% coordinates of the plot area
\gpdefrectangularnode{gp plot 1}{\pgfpoint{1.504cm}{0.985cm}}{\pgfpoint{12.039cm}{8.381cm}}
\end{tikzpicture}
%% gnuplot variables

}
	\caption{Real laser output power versus input current and corresponding linear fit}
	\label{fig:ana_laser_woatt}
}
\end{floatrow}
\end{figure}
With this value we can determine the dependency of the laser power on the input current:
\begin{align}
P(I) &= b + m\,\cdotp\, I=\unit[(-3.6 \pm 0.1)\cdotp 10^{4}]{\mu W} + \unit[(602 \pm 21)]{\frac{\mu W}{mA}}\,\cdotp\, I
\end{align}
and we can deduce the threshold current:
\begin{align}
I_\textrm{thr} = -\frac{b}{m} = \unit[(59 \pm 3)]{mA}
\end{align}
where the error is given by:
\begin{align}
\Delta I_\textrm{thr} = \sqrt{\left(\frac{\Delta b}{m}\right)^2+\left(\frac{b}{m^2}\Delta m\right)^2}
\end{align}
The quantum efficiency $\eta$, which is defined as the number of emitted photons per electron injected in the laser diode, can be deduced using:
\begin{align}
P = \frac{h\nu N_\gamma}{t}\hspace{1cm} I = \frac{eN_\textrm{e}}{t}
\end{align}
where $N_\gamma$ and $N_\textrm{e}$ are the number of photons emitted and the number of electrons injected in the time intervall $t$ respectively. With the wavelength of the laser $\lambda = \unit[987]{nm}$ we get:
\begin{align}
\eta = \frac{N_\gamma}{N_\textrm{e}} = \frac{eP}{Ih\nu} \approx \frac{\partial P}{\partial I}\frac{e\lambda}{hc} = m\frac{e\lambda}{hc} = \unit[(0,48 \pm 0,02)]{}
\end{align}
with the error
\begin{align}
\Delta \eta = \frac{e\lambda}{hc}\Delta m
\end{align}
\section{Abstract}

\begin{appendix}
\section{Figures}

\section{Tables}
\begin{table}[H]
  \centering
\resizebox{0.8\textwidth}{!}{
  \begin{tabular}{lcccc}
    \toprule
      $I$ [$\unit[]{mA}$] $\pm$ $\unit[0.5]{mA}$ & $P$ [$\unit[]{\mu W}$]  &  $P_\textrm{background}$ [$\unit[]{\mu W}$] & $\Delta P$ [$\unit[]{\mu W}$]\\
    \midrule[0.75pt]
$1$ & $0.000$ & $0.001$ & $0.001$\\
$11$ & $0.001$ & $0.001$ & $0.001$\\
$20$ & $0.001$ & $0.001$ & $0.001$\\
$30$ & $0.001$ & $0.001$ & $0.001$\\
$40$ & $0.002$ & $0.001$ & $0.001$\\
$45$ & $0.003$ & $0.001$ & $0.001$\\
$50$ & $0.003$ & $0.001$ & $0.001$\\
$55$ & $0.004$ & $0.001$ & $0.001$\\
$60$ & $0.610$ & $0.001$ & $0.001$\\
$65$ & $3.33$ & $0.01$ & $0.01$\\
$70$ & $6.34$ & $0.01$ & $0.01$\\
$75$ & $9.58$ & $0.01$ & $0.01$\\
$80$ & $12.9$ & $0.1$ & $0.1$\\
$90$ & $16.5$ & $0.1$ & $0.1$\\
$100$ & $24.9$ & $0.1$ & $0.1$\\
$110$ & $28.4$ & $0.1$ & $0.1$\\
$120$ & $35.4$ & $0.1$ & $0.1$\\
$130$ & $41.0$ & $0.1$ & $0.1$\\
$140$ & $46.0$ & $0.1$ & $0.1$\\
$150$ & $52.0$ & $0.1$ & $0.1$\\
$160$ & $58.5$ & $0.1$ & $0.1$\\
$170$ & $64.0$ & $0.1$ & $0.1$\\
$180$ & $69.8$ & $0.1$ & $0.1$\\
$190$ & $75.7$ & $0.1$ & $0.1$\\
$200$ & $79.9$ & $0.1$ & $0.1$\\
$210$ & $86.0$ & $0.1$ & $0.1$\\
$220$ & $90.3$ & $0.1$ & $0.1$\\
$230$ & $98.0$ & $0.1$ & $0.1$\\
$240$ & $104$ & $1$ & $1$\\
$250$ & $112$ & $1$ & $1$\\
$260$ & $114$ & $1$ & $1$\\
$270$ & $124$ & $1$ & $1$\\
$280$ & $126$ & $1$ & $1$\\
    \bottomrule
  \end{tabular}
}
\caption{Measured laser power for different input currents with attenuator}
  \label{tab:ana_laserpower_att}
\end{table}
\begin{table}[H]
  \centering
\resizebox{0.8\textwidth}{!}{
  \begin{tabular}{lcccc}
    \toprule
      $I$ [$\unit[]{mA}$] $\pm$ $\unit[0.5]{mA}$ & $P$ [$\unit[]{mW}$]  &  $P_\textrm{background}$ [$\unit[]{mW}$] & $\Delta P$ [$\unit[]{mW}$]\\
    \midrule[0.75pt]
$60$ & $0.65$ & $0.00$ & $0.01$\\
$65$ & $3.07$ & $0.00$ & $0.01$\\
$70$ & $5.56$ & $0.00$ & $0.01$\\
$75$ & $9.08$ & $0.00$ & $0.01$\\
$80$ & $12.2$ & $0.0$ & $0.1$\\
$85$ & $14.5$ & $0.0$ & $0.1$\\
$90$ & $17.4$ & $0.0$ & $0.1$\\
$95$ & $21.4$ & $0.0$ & $0.1$\\
$100$ & $24.2$ & $0.0$ & $0.1$\\
    \bottomrule
  \end{tabular}
}
\caption{Measured laser power for different input currents without attenuator}
  \label{tab:ana_laserpower_woatt}
\end{table}
\Literatur{quellen}

\end{appendix}
\end{document}


