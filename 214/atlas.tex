\documentclass{protokoll}
\usepackage{booktabs,multirow,xspace,gnuplot-lua-tikz}
\newcommand{\assistent}{M. Schmitz}
\newcommand{\versuch}{ATLAS}
\newcommand{\nummer}{E214}

\begin{document}

\section{Einleitung}
In diesem Praktikumsversuch soll die Physik des ATLAS-Experimentes kennengelernt und untersucht werden. So wird sich einleitend anhand von \emph{event displays} mit den zu Grunde liegenden Prozessen und dem ATLAS vertraut gemacht. Anschlie�end wird die Messung der Elektronenenergie anhand der $Z^0$-Masse kalibriert, woraufhin dann entweder eine Bestimmung der W-Boson-Masse oder aber die Suche nach neuer Physik anhand von simulierten ATLAS Daten erfolgt.

\section{Theoretische Grundlagen}
\subsection{Das Standardmodell}
\subsubsection{Teilchen}

\subsubsection{Wechselwirkungen}

\subsection{Neue Physik}
Die �ber das Standardmodell hinausgehenden theoretischen �berlegungen werden auch als "`neuer Physik"' bezeichnet. Man erhofft sich vom LHC (auch �ber ATLAS) u.\ a.\ hierzu Hinweise zu liefern. 
Die hier relevanten Kandidaten f�r neue Physik sind:
\begin{enumerate}
\item{Das \textbf{\textsc{Higgs}-Boson} ist zwar Teil des Standardmodells und wird dort ben�tigt, um den schweren Eichbosonen sowie den �brigen Teilchen ihre Masse zu geben, wurde bisher aber noch nicht experimentell nachgewiesen, weshalb es hier unter "`neuer Physik"' aufgef�hrt wird. In der einfachsten Theorie ist das \textsc{Higgs}-Boson ein reelles skalares Feld, hat also Spin 0 und w�rde sich �ber einen Peak in der Verteilung der invarianten Masse zeigen.}
\item{In der Theorie der \textbf{Supersymmetrie} werden Partnerteilchen zu den Standardmodell-Teilchen postuliert und zwar zu jedem Fermion ein bosonischer supersymmetrischer Partner und zu jedem Boson ein fermionischer supersymmetrischer Partner. Die Massen der SuSy-Teilchen sind nicht bekannt, sie sollten jedoch deutlich h�her liegen als die ihrer Partnerteilchen, da man sie sonst bereits schon gefunden haben m�sste. Auch hier gibt es mehrere zur Auswahl stehende Theorien, wobei wir uns hier auf eine Theorie mit Neutralino als LSP (lightest supersymmetric particle) beschr�nken. Das Neutralino m�sste sich im Experiment durch fehlenden Impuls kennzeichnen.}
\item{Eine weitere M�glichkeit f�r neue Physik sind \textbf{schwere Quarks}. So ist es durchaus denkbar, dass es noch eine weitere, vierte Leptonen- und Quark-Generation gibt. Aus der Breite des $Z^0$-Peaks erh�lt man jedoch als obere Grenze f�r die Anzahl der leichten Neutrinos drei, g�be es also eine vierte Familie, so m�ssten die Teilchen dieser  deutlich schwerer sein als die bisherigen. Die Quarks dieser Familie w�ren also schwer und w�rden sich vor allem durch eine hohe Anzahl hadronischer Jets auszeichnen.}
\item{In verschiedenen Theorien werden \textbf{weitere schwere Eichbosonen $Z'$} vorhergesagt. Hier sollen die Daten beispielsweise darauf untersucht werden, ob es ein weiteres schweres Eichboson mit �hnlichen Eigenschaften wie das $Z^0$ gibt, das aber deutlich schwerer w�re als dieses. Es soll ebenfalls an Quarks und Leptonen gleicherma�en koppeln und w�rde in ein Leptonenpaar oder ein Quarkpaar zerfallen. Es kann nicht nur einzeln, sondern auch paarweise oder zusammen mit einem $Z^0$ produziert werden und w�rde mit letzteren beiden M�glichkeiten zu einem Vier-Leptonen-Endzustand beitragen, was f�r die Auswertung der Daten interessant ist.}
\end{enumerate}

\subsection{Kinematik}

\subsection{Der LHC}

\subsection{ATLAS}

\subsection{Koordinatensysteme}
Bei der Wahl eines angemessenen Koordinatensystems bieten sich im Wesentlichen drei an. Zum einen besteht die M�glichkeit in einem kartesischen Koordinatensystem die Strahlrichtung als z-Achse zu w�hlen, zum Mittelpunkt des LHC-Ringes zeigt die x-Achse und orthogonal auf beiden steht die y-Achse.
Zum anderen steht die Wahl eines kanonischen Kugelkoordinatensystems zur Verf�gung. Da jedoch die Ereignisse nicht gleichverteilt auftreten, sondern auf Grund der nur geringen Transversalimpulse eine H�ufung bei kleinen Winkeln zu erwarten ist, modifiziert man das Koordinatensystem und f�hrt eine neue Gr��e ein, die Rapidit�t $y \equiv \frac{1}{2} \ln{\frac{E+p_z}{E-p_z}}$. Rapidit�tsdifferenzen sind invariant unter Lorentz-Boosts parallel zu z-Achse. In Systemen mit einer verschwindenden invarianten Masse erh�lt man mit dem Winkel $\theta$ zur Strahlachse stattdessen die sogenannte Pseudo-Rapidit�t
\begin{align}
\eta = - \ln{\left(\tan{\frac{\theta}{2}}\right)}  \ \ldotp
\end{align}
Der ATLAS-Detektor kann Pseudo-Rapidit�ten $|\eta| < 2.5$ f�r Leptonen sowie $|\eta| < 5$ f�r Hadronen aufl�sen.


\section{Aufgaben vor Versuchsbeginn}
\subsection*{Frage A: Zerfall des $Z^0$-Bosons}
\emph{Welchen Wert hat der Impuls eines Elektrons beim Zerfall des $Z^0$-Bosons ($Z^0 \rightarrow e^+ e^-$) im Ruhesystem des $Z^0$-Bosons?}

Das $Z^0$-Boson wird in Ruhe betrachten, hat also einen verschwindenden Dreier-Impuls $\vec{p}_Z$. Da Vierer-Impuls-Erhaltung gilt, verschwindet auch die Summe der Dreier-Impulse von Positron und Elektron ($\vec{p}_{e^+} = - \vec{p}_{e^-}$). Man betrachte also die 0-Komponente der Vierer-Impulse:

%noch sch�ner machen:
\begin{eqnarray*}
               E_{Z^0} &=& E_{e^+} + E_{e^-} \\
\Rightarrow  m^2_{Z^0} &=& 2 E_{e^-} \\
                       &=& 2 \sqrt{m^2_e + \vec{p_e}^2} \\
\Rightarrow  \vec{p_e}^2 &=& \frac{m^2_{Z^0}}{4} - m^2_e \\
\Rightarrow  \vec{p_e}   &=& \sqrt{\frac{m^2_{Z^0}}{4} - m^2_e} \\
                         &=& \unit[45.6]{GeV}
\end{eqnarray*}

\subsection*{Frage B: Elektron-Positron-Streuung}
\emph{Wie gro� ist der Impuls der $\tau$-Leptonen bei der Reaktion $e^+ e^- \rightarrow \tau^+ \tau^-$ im CMS ($\sqrt{s} = \unit[5]{GeV}$)?}

Es gilt Energieerhaltung ($s = s'$), somit folgt:
\begin{eqnarray*}
    \sqrt{s} &=& E_{\tau^+} + E_{\tau^-} \\
             &=& 2 E_{\tau^+} \\
             &=& 2 \sqrt{m^2_{\tau} + p^2_{\tau}} \\
\Rightarrow  p^2_{\tau} &=& \frac{s}{4} - m^2_{\tau} \\
\Rightarrow  p_{\tau}   &=& \sqrt{\frac{s}{4} - m^2_{\tau}} \\
\Rightarrow  p_{\tau}   &=& \unit[1.76]{GeV}
\end{eqnarray*}

\subsection*{Fragen zu Versuchsteil 2}
\emph{Wie kann der Transversalimpuls des W-Bosons bestimmt werden? Wie k�nnen Sie die Variable} \texttt{ptw} \emph{aus den anderen ROOT-Variablen bestimmen?}
%antwort

\emph{Leiten Sie aus Gleichung (4.4) die Gleichung (4.5) her.~\cite{skript}}
%antwort

\subsection*{Fragen zu Versuchsteil 3}
%...

\subsection*{Fragen zu Versuchsteil 4}
\emph{Was ist die minimale invariante Masse der vier Leptonen aus einem $Z^0$-Paar? Weshalb findet man auch unterhalb dieser Schwelle Vier-Lepton-Ereignisse?}
%antwort

\emph{Ein \textsc{Higgs}-Boson zerfalle in zwei $Z^0$. Wie sieht die Verteilung der invarianten Masse der vier Leptonen aus?}
%antwort

\emph{Ein idealer Detektor wird vorrausgesetzt. Welches ist das typische $\slash{E}_T$, wenn ein $Z^0$-Paar in Elektronen oder Myonen-Paare zerf�llt? Wie wird $\slash{E}_T$ bei einem realen Detektor aussehen?}
%antwort

\emph{noch ne frage ...}
%antwort

\emph{un noch eine ...}
%antwort


\section{Versuchsdurchf�hrung und Auswertung}

\section{Zusammenfassung}

\begin{appendix}
  \section{Tabellen}

  \Literatur{quellen}

\end{appendix}
\end{document}
