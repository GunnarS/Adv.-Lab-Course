\documentclass{protokoll_en}
\newcommand{\assistent}{D. Sch�tze}
\newcommand{\versuch}{Holography}
\newcommand{\nummer}{E215}
\newcommand{\linbo}{\ensuremath{{\mathrm{LiNbO}_3}\;}}

\begin{document}

\section{Preface}
The object of this experiment is to get familiar with basic characteristics of holography. First holograms on holographic films and plates are recorded and reconstructed. In a second step we write a elementary hologram in a \linbo crystal and determine some important parameters. For all these purposes we use a $\unit[633]{nm}$ He-Ne laser.

\section{Theoretical Background}
\subsection{first}
\subsection{Coherence and Interference}
The theory of optical interference rests on the principle of linear superposition of electromagnetic fields. This principle, which is a consequence of \textsc{Maxwell}'s equation for the vacuum, states that the electric field produced at one point in space by several sources will be the vector sum of the contributions from each of the sources (linear superposition does not always hold in the presence of matter, however, as is widely exploited in non-linear optics). Consider two electric fields given by:
\begin{align}
\vec{E}_1 &= \vec{E}_0 \exp{i \left( \vec{k}_1 \cdot \vec{r} - \omega t + \phi_1 \right)} \\
\vec{E}_2 &= \vec{E'}_0 \exp{i \left( \vec{k}_2 \cdot \vec{r} - \omega t + \phi_2 \right)}
\end{align}
The two fields (and resulting plane waves) are mutually coherent if the phase difference $\phi_1 - \phi_2$ is constant within a relevant range. When two coherent plane waves are incident on a point, the irradiance function caused by the superposition of the waves at that point becomes
\begin{align}
I = |\vec{E}|^2 = I_1 + I_2 + 2 |\vec{E}_1| |\vec{E}_2| \cos{\theta}
\end{align}
The final term is the interference term, and this gives rise to the spatial variations in intensity we see as the interference pattern between the two waves. If the two waves are mutually incoherent on the other hand, then the phase difference $\phi_1 - \phi_2$ varies randomly which causes the average value of $\cos{\theta}$ to fall to zero, and thus there is no interference produced.

\subsection{Photorefractive Effect}
When a photorefractive material is illuminated by coherent light, the resulting interference pattern that forms within the crystal is able to locally change the refractive index of the crystal. This occurs in various stages. Firstly, within the illuminated regions of the dark and bright fringes, electrons from an impurity level (which is in between the conduction band and the valence band) may be excited into the conduction band, where they are able to diffuse freely. Due to the density of electrons moving into the conduction band from the illuminated-fringe regions, the net diffusion in the conduction band is towards the dark-fringe regions. The electrons move until they encounter a hole and as such return to the impurity level, but in general this will occur in the dark-fringe regions to which the electrons have drifted. The result is a higher density of holes in the illuminated-fringe regions, and a higher density of electrons in the dark-fringe regions. This modulated charge distribution causes a space-charge field to form within the crystal, which persists even when the crystal is no longer illuminated due to the fact that the electrons and holes are fixed in place. Finally the space-charge field, via the electro-optic effect, causes a modulation of the refractive index within the crystal. The pattern produced by the refractive index modulation can now be used as a diffraction grating that follows the spatial pattern of the original incident light.

\section{Experimentation and Analysis}
\subsection{Transmission Hologram}
To take a transmission hologram we install the set-up of figure \ref{fig:aufbau_transmiss}. It is useful to choose highly reflecting objects like silver objects.
\begin{figure}[H]
  \centering
  \includegraphics[width=0.5\textwidth]{graphics/aufbau_transmiss}
  \caption{Set-up for taking a transmission hologram}
  \label{fig:aufbau_transmiss}
\end{figure}
First of all one has to consider that the object and the holographic film are illuminated homogeneously. But due to reasons of coherence it is also important that the difference in pathlength of object and reference beam does not exceed approximately $\unit[30]{cm}$. Furthermore one inserts an OD filter into the reference beam to adjust the intensities of both beams. If everything is aligned properly the light is switched off except for a small LED panel and the exposure of the holographic film is done for $\unit[5]{s}$. Then the film is developed like explained in the description~\cite{skript}. For reconstruction of the object the film is placed at its position of recording, the object is removed and the object beam is blocked. Now we are indeed able to observe a three dimensional transmission hologram at the former position of the object (see figure \ref{fig:transmiss}), but the angle range for observation is rather small.
\begin{figure}[H]
    \begin{minipage}{0.2\textheight}
  \includegraphics[width=1.0\textwidth]{graphics/transmiss1}
    \end{minipage}
    \hspace{1.3cm}
    \begin{minipage}{0.17\textheight}
  \includegraphics[width=1.0\textwidth]{graphics/transmiss2}
    \end{minipage}
  \caption{Photographs of our transmission hologram}
  \label{fig:transmiss}
\end{figure}
If one uses normal light only a spectral pattern due to the mentioned grating-like structure of the developed film is observed.

\subsection{Reflection Hologram}
The next task is to record a reflection hologram. Again a homogeneous illumination of the holographic plate is important (set-up see figure \ref{fig:aufbau_refl}).
\begin{figure}[H]
  \centering
  \includegraphics[width=0.5\textwidth]{graphics/aufbau_refl}
  \caption{Set-up for taking a reflection hologram}
  \label{fig:aufbau_refl}
\end{figure}
The object has to be placed behind the holographic plate, whereas the coating points towards the object. The advantage is that the difference in pathlength is minimal and the illumination of the holographic plate has maximal intensity.

With this set-up we take two holograms. While the first one uses a non-processed holographic plate, the other one first of all is treated with an expanding agent. Actually this is done directly at the beginning of the experimentation, because it takes some time to dry. Both are illuminated for about $\unit[8]{s}$ and are then treated as before with respect to developement.

\subsubsection*{Without Expanding Agent}
Unfortunately we missed to remove the OD filter while exposing the film. Therefore our hologram is weaker than it was supposed to be. Therefore it was difficult to compare it with the transmission hologram. The reflection hologram should have a higher intensity and be more detailed than the transmission hologram. Furthermore the angle range of possible observation is expected to be larger. Indeed we could hardly see anything. Besides removing the filter also a longer illumination time could have improved the recording.

\subsubsection*{With Expanding Agent}
Using an expanding agent indeed promises an even better result than simple reflection holography, because after the expansion the coating contracts again such that the lattice constant gets smaller. Therefore it is possible to see the hologram at shorter wavelength. So we expect that it is not possible to reconstruct it with the red laser light we used. Instead it should be possible to observe the hologram with a bulb of high intensity emitting white light, because the \textsc{Bragg} condition will just "choose" the right wavelength as described in section ...\ref{cha:bragg}.... But in practise a big problem arose because of a very long drying time. In fact the expanded plate was not dry after more than 22 hours. We are of the opinion that this should have happened because of wet desiccant. Anyway it of course was not possible to take a holograph with a wet holographic plate. So unfortunately we could not confirm the above mentioned indications.


\subsection{Holography with Lithium Niobate}
\label{subsec:ana_linbo3}
\subsubsection{Recording the Writing Curve}
\label{subsubsec:ana_writing}
In this part of the experiment we want to store a transmission phase hologram in a photorefractive lithium niobate crystal. In oder to do this we set up the experiment as shown in figure \ref{fig:setup_writing}.
\begin{figure}[H]
	\centering
		\includegraphics[width=0.5\textwidth]{graphics/setup_writing}
	\caption{Setup for recording the writing curve \cite{skript}}
	\label{fig:setup_writing}
\end{figure}
We adjust the angle between the normal to the recording surface and the writing beams to $\gamma = \unit[(10.0\pm 0.1)]{^\circ}$. Therefore the angle between the two beams inside the crystal is given by the \textsc{Snellius'} law:
\begin{align}
\theta = \arcsin\left(\frac{n_\textrm{Air}}{n_\textrm{crystal}}\sin\gamma\right) = \unit[(8.59 \pm 0.08)]{^\circ}
\end{align}
because the refractive index of the \linbo crystal is given by $n_\textrm{crystal} = \unit[2.29]{}$ for light with a wavelength of $\lambda = \unit[633]{nm}$.
After having optimized the setup we illuminate the crystal with a tungsten lamp for around one hour to erase possibly existing holograms. 

Now we block the laser and perform background measurements for both photodiodes. Because they are connected to an amplifier, we do the measurement for each range. The results are displayed in table \ref{tab:ana_background}. Eventually we are ready to record the writing curve. We unblock the laser and measure the diffraction effiency in time intervalls of about one minute. To measure the intensities we block one of the writing beams so one of the photodiodes measures the diffracted and the other one the transmitted intensity.

\begin{table}[H]
  \centering
\resizebox{0.5\textwidth}{!}{
  \begin{tabular}{l|cc}
    \toprule
     Range & $I_\textrm{diff}^\textrm{bgd}$ [$\unit[]{mW}$] & $I_\textrm{trans}^\textrm{bgd}$ [$\unit[]{mW}$]\\
    \midrule[0.75pt]
$3$ & $-4.0 \pm 0.5$ & $0.0 \pm 0.5$\\
$4$ & $-4.0 \pm 0.5$ & $0.0 \pm 0.5$\\
$5$ & $-4.0 \pm 0.5$ & $0.0 \pm 0.5$\\
$6$ & $-4.0 \pm 0.5$ & $0.0 \pm 0.5$\\
$7$ & $-2.0 \pm 0.5$ & $2.0 \pm 0.5$\\
$8$ & $17.0 \pm 0.5$ & $25.0 \pm 0.5$\\
    \bottomrule
  \end{tabular}
}
\caption{Background measurements for the two photodiodes}
  \label{tab:ana_background}
\end{table}

The first curve we recorded this way behaved well for the first 20 minutes, but then began contrary to our expectation to fall again. Indeed a reflection went through the crystal again an therefore it was deleted right after beiing written. So we realigned our setup, dropped the data and restarted the measurement. The data is shown in tables \ref{tab:ana_writing_first} and \ref{tab:ana_writing_second} in the appendix and plotted in figure \ref{fig:ana_writing}.
\begin{figure}[H]
\begin{floatrow}
\resizebox{0.5\textwidth}{!}{
	\begin{tikzpicture}[gnuplot]
%% generated with GNUPLOT 4.4p0 (Lua 5.1.4; terminal rev. 97, script rev. 96a)
%% 09.05.2010 23:59:42
\gpcolor{gp lt color border}
\gpsetlinetype{gp lt border}
\gpsetlinewidth{1.00}
\draw[gp path] (1.504,0.985)--(1.684,0.985);
\draw[gp path] (12.039,0.985)--(11.859,0.985);
\node[gp node right] at (1.320,0.985) { 0};
\draw[gp path] (1.504,2.042)--(1.684,2.042);
\draw[gp path] (12.039,2.042)--(11.859,2.042);
\node[gp node right] at (1.320,2.042) { 2};
\draw[gp path] (1.504,3.098)--(1.684,3.098);
\draw[gp path] (12.039,3.098)--(11.859,3.098);
\node[gp node right] at (1.320,3.098) { 4};
\draw[gp path] (1.504,4.155)--(1.684,4.155);
\draw[gp path] (12.039,4.155)--(11.859,4.155);
\node[gp node right] at (1.320,4.155) { 6};
\draw[gp path] (1.504,5.211)--(1.684,5.211);
\draw[gp path] (12.039,5.211)--(11.859,5.211);
\node[gp node right] at (1.320,5.211) { 8};
\draw[gp path] (1.504,6.268)--(1.684,6.268);
\draw[gp path] (12.039,6.268)--(11.859,6.268);
\node[gp node right] at (1.320,6.268) { 10};
\draw[gp path] (1.504,7.324)--(1.684,7.324);
\draw[gp path] (12.039,7.324)--(11.859,7.324);
\node[gp node right] at (1.320,7.324) { 12};
\draw[gp path] (1.504,8.381)--(1.684,8.381);
\draw[gp path] (12.039,8.381)--(11.859,8.381);
\node[gp node right] at (1.320,8.381) { 14};
\draw[gp path] (1.504,0.985)--(1.504,1.165);
\draw[gp path] (1.504,8.381)--(1.504,8.201);
\node[gp node center] at (1.504,0.677) { 0};
\draw[gp path] (3.203,0.985)--(3.203,1.165);
\draw[gp path] (3.203,8.381)--(3.203,8.201);
\node[gp node center] at (3.203,0.677) { 500};
\draw[gp path] (4.902,0.985)--(4.902,1.165);
\draw[gp path] (4.902,8.381)--(4.902,8.201);
\node[gp node center] at (4.902,0.677) { 1000};
\draw[gp path] (6.602,0.985)--(6.602,1.165);
\draw[gp path] (6.602,8.381)--(6.602,8.201);
\node[gp node center] at (6.602,0.677) { 1500};
\draw[gp path] (8.301,0.985)--(8.301,1.165);
\draw[gp path] (8.301,8.381)--(8.301,8.201);
\node[gp node center] at (8.301,0.677) { 2000};
\draw[gp path] (10.000,0.985)--(10.000,1.165);
\draw[gp path] (10.000,8.381)--(10.000,8.201);
\node[gp node center] at (10.000,0.677) { 2500};
\draw[gp path] (11.699,0.985)--(11.699,1.165);
\draw[gp path] (11.699,8.381)--(11.699,8.201);
\node[gp node center] at (11.699,0.677) { 3000};
\draw[gp path] (1.504,8.381)--(1.504,0.985)--(12.039,0.985)--(12.039,8.381)--cycle;
\node[gp node center,rotate=-270] at (0.430,4.683) {$\eta$ [$\unit[]{10^{-3}}$]};
\node[gp node center] at (6.771,0.215) {t [s]};
\node[gp node right] at (10.571,8.047) {Measured Data};
\gpcolor{gp lt color 0}
\gpsetlinetype{gp lt plot 0}
\draw[gp path] (10.755,8.047)--(11.671,8.047);
\draw[gp path] (10.755,8.137)--(10.755,7.957);
\draw[gp path] (11.671,8.137)--(11.671,7.957);
\draw[gp path] (1.504,1.019)--(1.504,1.020);
\draw[gp path] (1.414,1.019)--(1.594,1.019);
\draw[gp path] (1.414,1.020)--(1.594,1.020);
\draw[gp path] (1.636,1.044)--(1.816,1.044);
\draw[gp path] (1.636,1.044)--(1.816,1.044);
\draw[gp path] (1.948,1.091)--(1.948,1.092);
\draw[gp path] (1.858,1.091)--(2.038,1.091);
\draw[gp path] (1.858,1.092)--(2.038,1.092);
\draw[gp path] (2.181,1.115)--(2.181,1.116);
\draw[gp path] (2.091,1.115)--(2.271,1.115);
\draw[gp path] (2.091,1.116)--(2.271,1.116);
\draw[gp path] (2.368,1.150)--(2.368,1.151);
\draw[gp path] (2.278,1.150)--(2.458,1.150);
\draw[gp path] (2.278,1.151)--(2.458,1.151);
\draw[gp path] (2.542,1.191)--(2.542,1.192);
\draw[gp path] (2.452,1.191)--(2.632,1.191);
\draw[gp path] (2.452,1.192)--(2.632,1.192);
\draw[gp path] (2.737,1.233)--(2.737,1.235);
\draw[gp path] (2.647,1.233)--(2.827,1.233);
\draw[gp path] (2.647,1.235)--(2.827,1.235);
\draw[gp path] (2.964,1.263)--(2.964,1.264);
\draw[gp path] (2.874,1.263)--(3.054,1.263);
\draw[gp path] (2.874,1.264)--(3.054,1.264);
\draw[gp path] (3.136,1.315)--(3.136,1.316);
\draw[gp path] (3.046,1.315)--(3.226,1.315);
\draw[gp path] (3.046,1.316)--(3.226,1.316);
\draw[gp path] (3.361,1.364)--(3.361,1.365);
\draw[gp path] (3.271,1.364)--(3.451,1.364);
\draw[gp path] (3.271,1.365)--(3.451,1.365);
\draw[gp path] (3.574,1.419)--(3.574,1.420);
\draw[gp path] (3.484,1.419)--(3.664,1.419);
\draw[gp path] (3.484,1.420)--(3.664,1.420);
\draw[gp path] (3.796,1.468)--(3.796,1.470);
\draw[gp path] (3.706,1.468)--(3.886,1.468);
\draw[gp path] (3.706,1.470)--(3.886,1.470);
\draw[gp path] (3.995,1.536)--(3.995,1.537);
\draw[gp path] (3.905,1.536)--(4.085,1.536);
\draw[gp path] (3.905,1.537)--(4.085,1.537);
\draw[gp path] (4.218,1.590)--(4.218,1.591);
\draw[gp path] (4.128,1.590)--(4.308,1.590);
\draw[gp path] (4.128,1.591)--(4.308,1.591);
\draw[gp path] (4.456,1.640)--(4.456,1.641);
\draw[gp path] (4.366,1.640)--(4.546,1.640);
\draw[gp path] (4.366,1.641)--(4.546,1.641);
\draw[gp path] (4.627,1.710)--(4.627,1.712);
\draw[gp path] (4.537,1.710)--(4.717,1.710);
\draw[gp path] (4.537,1.712)--(4.717,1.712);
\draw[gp path] (4.852,1.766)--(4.852,1.768);
\draw[gp path] (4.762,1.766)--(4.942,1.766);
\draw[gp path] (4.762,1.768)--(4.942,1.768);
\draw[gp path] (5.078,1.830)--(5.078,1.831);
\draw[gp path] (4.988,1.830)--(5.168,1.830);
\draw[gp path] (4.988,1.831)--(5.168,1.831);
\draw[gp path] (5.294,1.898)--(5.294,1.900);
\draw[gp path] (5.204,1.898)--(5.384,1.898);
\draw[gp path] (5.204,1.900)--(5.384,1.900);
\draw[gp path] (5.486,1.981)--(5.486,1.991);
\draw[gp path] (5.396,1.981)--(5.576,1.981);
\draw[gp path] (5.396,1.991)--(5.576,1.991);
\draw[gp path] (5.717,2.064)--(5.717,2.074);
\draw[gp path] (5.627,2.064)--(5.807,2.064);
\draw[gp path] (5.627,2.074)--(5.807,2.074);
\draw[gp path] (5.918,2.142)--(5.918,2.152);
\draw[gp path] (5.828,2.142)--(6.008,2.142);
\draw[gp path] (5.828,2.152)--(6.008,2.152);
\draw[gp path] (6.137,2.225)--(6.137,2.235);
\draw[gp path] (6.047,2.225)--(6.227,2.225);
\draw[gp path] (6.047,2.235)--(6.227,2.235);
\draw[gp path] (6.308,2.300)--(6.308,2.310);
\draw[gp path] (6.218,2.300)--(6.398,2.300);
\draw[gp path] (6.218,2.310)--(6.398,2.310);
\draw[gp path] (6.544,2.400)--(6.544,2.410);
\draw[gp path] (6.454,2.400)--(6.634,2.400);
\draw[gp path] (6.454,2.410)--(6.634,2.410);
\draw[gp path] (6.772,2.483)--(6.772,2.493);
\draw[gp path] (6.682,2.483)--(6.862,2.483);
\draw[gp path] (6.682,2.493)--(6.862,2.493);
\draw[gp path] (6.978,2.596)--(6.978,2.606);
\draw[gp path] (6.888,2.596)--(7.068,2.596);
\draw[gp path] (6.888,2.606)--(7.068,2.606);
\draw[gp path] (7.179,2.710)--(7.179,2.720);
\draw[gp path] (7.089,2.710)--(7.269,2.710);
\draw[gp path] (7.089,2.720)--(7.269,2.720);
\draw[gp path] (7.375,2.831)--(7.375,2.841);
\draw[gp path] (7.285,2.831)--(7.465,2.831);
\draw[gp path] (7.285,2.841)--(7.465,2.841);
\draw[gp path] (7.604,2.951)--(7.604,2.962);
\draw[gp path] (7.514,2.951)--(7.694,2.951);
\draw[gp path] (7.514,2.962)--(7.694,2.962);
\draw[gp path] (7.780,3.105)--(7.780,3.115);
\draw[gp path] (7.690,3.105)--(7.870,3.105);
\draw[gp path] (7.690,3.115)--(7.870,3.115);
\draw[gp path] (7.999,3.229)--(7.999,3.240);
\draw[gp path] (7.909,3.229)--(8.089,3.229);
\draw[gp path] (7.909,3.240)--(8.089,3.240);
\draw[gp path] (8.197,3.367)--(8.197,3.377);
\draw[gp path] (8.107,3.367)--(8.287,3.367);
\draw[gp path] (8.107,3.377)--(8.287,3.377);
\draw[gp path] (8.432,3.534)--(8.432,3.545);
\draw[gp path] (8.342,3.534)--(8.522,3.534);
\draw[gp path] (8.342,3.545)--(8.522,3.545);
\draw[gp path] (8.652,3.691)--(8.652,3.702);
\draw[gp path] (8.562,3.691)--(8.742,3.691);
\draw[gp path] (8.562,3.702)--(8.742,3.702);
\draw[gp path] (8.863,3.835)--(8.863,3.846);
\draw[gp path] (8.773,3.835)--(8.953,3.835);
\draw[gp path] (8.773,3.846)--(8.953,3.846);
\draw[gp path] (9.072,4.025)--(9.072,4.036);
\draw[gp path] (8.982,4.025)--(9.162,4.025);
\draw[gp path] (8.982,4.036)--(9.162,4.036);
\draw[gp path] (9.295,4.214)--(9.295,4.226);
\draw[gp path] (9.205,4.214)--(9.385,4.214);
\draw[gp path] (9.205,4.226)--(9.385,4.226);
\draw[gp path] (9.492,4.384)--(9.492,4.396);
\draw[gp path] (9.402,4.384)--(9.582,4.384);
\draw[gp path] (9.402,4.396)--(9.582,4.396);
\draw[gp path] (9.666,4.576)--(9.666,4.588);
\draw[gp path] (9.576,4.576)--(9.756,4.576);
\draw[gp path] (9.576,4.588)--(9.756,4.588);
\draw[gp path] (9.896,4.801)--(9.896,4.812);
\draw[gp path] (9.806,4.801)--(9.986,4.801);
\draw[gp path] (9.806,4.812)--(9.986,4.812);
\draw[gp path] (10.068,5.029)--(10.068,5.041);
\draw[gp path] (9.978,5.029)--(10.158,5.029);
\draw[gp path] (9.978,5.041)--(10.158,5.041);
\draw[gp path] (10.295,5.250)--(10.295,5.263);
\draw[gp path] (10.205,5.250)--(10.385,5.250);
\draw[gp path] (10.205,5.263)--(10.385,5.263);
\draw[gp path] (10.491,5.490)--(10.491,5.502);
\draw[gp path] (10.401,5.490)--(10.581,5.490);
\draw[gp path] (10.401,5.502)--(10.581,5.502);
\draw[gp path] (10.666,5.770)--(10.666,5.783);
\draw[gp path] (10.576,5.770)--(10.756,5.770);
\draw[gp path] (10.576,5.783)--(10.756,5.783);
\draw[gp path] (10.895,6.026)--(10.895,6.039);
\draw[gp path] (10.805,6.026)--(10.985,6.026);
\draw[gp path] (10.805,6.039)--(10.985,6.039);
\draw[gp path] (11.107,6.288)--(11.107,6.302);
\draw[gp path] (11.017,6.288)--(11.197,6.288);
\draw[gp path] (11.017,6.302)--(11.197,6.302);
\draw[gp path] (11.332,6.595)--(11.332,6.609);
\draw[gp path] (11.242,6.595)--(11.422,6.595);
\draw[gp path] (11.242,6.609)--(11.422,6.609);
\draw[gp path] (11.546,6.865)--(11.546,6.879);
\draw[gp path] (11.456,6.865)--(11.636,6.865);
\draw[gp path] (11.456,6.879)--(11.636,6.879);
\draw[gp path] (11.727,7.174)--(11.727,7.189);
\draw[gp path] (11.637,7.174)--(11.817,7.174);
\draw[gp path] (11.637,7.189)--(11.817,7.189);
\draw[gp path] (11.907,7.452)--(11.907,7.467);
\draw[gp path] (11.817,7.452)--(11.997,7.452);
\draw[gp path] (11.817,7.467)--(11.997,7.467);
\draw[gp path] (1.504,1.019)--(1.506,1.019);
\draw[gp path] (1.504,0.929)--(1.504,1.109);
\draw[gp path] (1.506,0.929)--(1.506,1.109);
\draw[gp path] (1.724,1.044)--(1.728,1.044);
\draw[gp path] (1.724,0.954)--(1.724,1.134);
\draw[gp path] (1.728,0.954)--(1.728,1.134);
\draw[gp path] (1.946,1.091)--(1.949,1.091);
\draw[gp path] (1.946,1.001)--(1.946,1.181);
\draw[gp path] (1.949,1.001)--(1.949,1.181);
\draw[gp path] (2.179,1.115)--(2.183,1.115);
\draw[gp path] (2.179,1.025)--(2.179,1.205);
\draw[gp path] (2.183,1.025)--(2.183,1.205);
\draw[gp path] (2.366,1.151)--(2.370,1.151);
\draw[gp path] (2.366,1.061)--(2.366,1.241);
\draw[gp path] (2.370,1.061)--(2.370,1.241);
\draw[gp path] (2.540,1.191)--(2.543,1.191);
\draw[gp path] (2.540,1.101)--(2.540,1.281);
\draw[gp path] (2.543,1.101)--(2.543,1.281);
\draw[gp path] (2.735,1.234)--(2.739,1.234);
\draw[gp path] (2.735,1.144)--(2.735,1.324);
\draw[gp path] (2.739,1.144)--(2.739,1.324);
\draw[gp path] (2.962,1.263)--(2.965,1.263);
\draw[gp path] (2.962,1.173)--(2.962,1.353);
\draw[gp path] (2.965,1.173)--(2.965,1.353);
\draw[gp path] (3.135,1.315)--(3.138,1.315);
\draw[gp path] (3.135,1.225)--(3.135,1.405);
\draw[gp path] (3.138,1.225)--(3.138,1.405);
\draw[gp path] (3.359,1.364)--(3.363,1.364);
\draw[gp path] (3.359,1.274)--(3.359,1.454);
\draw[gp path] (3.363,1.274)--(3.363,1.454);
\draw[gp path] (3.572,1.419)--(3.575,1.419);
\draw[gp path] (3.572,1.329)--(3.572,1.509);
\draw[gp path] (3.575,1.329)--(3.575,1.509);
\draw[gp path] (3.794,1.469)--(3.797,1.469);
\draw[gp path] (3.794,1.379)--(3.794,1.559);
\draw[gp path] (3.797,1.379)--(3.797,1.559);
\draw[gp path] (3.993,1.537)--(3.997,1.537);
\draw[gp path] (3.993,1.447)--(3.993,1.627);
\draw[gp path] (3.997,1.447)--(3.997,1.627);
\draw[gp path] (4.217,1.590)--(4.220,1.590);
\draw[gp path] (4.217,1.500)--(4.217,1.680);
\draw[gp path] (4.220,1.500)--(4.220,1.680);
\draw[gp path] (4.454,1.640)--(4.458,1.640);
\draw[gp path] (4.454,1.550)--(4.454,1.730);
\draw[gp path] (4.458,1.550)--(4.458,1.730);
\draw[gp path] (4.625,1.711)--(4.628,1.711);
\draw[gp path] (4.625,1.621)--(4.625,1.801);
\draw[gp path] (4.628,1.621)--(4.628,1.801);
\draw[gp path] (4.850,1.767)--(4.854,1.767);
\draw[gp path] (4.850,1.677)--(4.850,1.857);
\draw[gp path] (4.854,1.677)--(4.854,1.857);
\draw[gp path] (5.077,1.831)--(5.080,1.831);
\draw[gp path] (5.077,1.741)--(5.077,1.921);
\draw[gp path] (5.080,1.741)--(5.080,1.921);
\draw[gp path] (5.292,1.899)--(5.295,1.899);
\draw[gp path] (5.292,1.809)--(5.292,1.989);
\draw[gp path] (5.295,1.809)--(5.295,1.989);
\draw[gp path] (5.484,1.986)--(5.488,1.986);
\draw[gp path] (5.484,1.896)--(5.484,2.076);
\draw[gp path] (5.488,1.896)--(5.488,2.076);
\draw[gp path] (5.715,2.069)--(5.718,2.069);
\draw[gp path] (5.715,1.979)--(5.715,2.159);
\draw[gp path] (5.718,1.979)--(5.718,2.159);
\draw[gp path] (5.916,2.147)--(5.920,2.147);
\draw[gp path] (5.916,2.057)--(5.916,2.237);
\draw[gp path] (5.920,2.057)--(5.920,2.237);
\draw[gp path] (6.135,2.230)--(6.138,2.230);
\draw[gp path] (6.135,2.140)--(6.135,2.320);
\draw[gp path] (6.138,2.140)--(6.138,2.320);
\draw[gp path] (6.306,2.305)--(6.310,2.305);
\draw[gp path] (6.306,2.215)--(6.306,2.395);
\draw[gp path] (6.310,2.215)--(6.310,2.395);
\draw[gp path] (6.543,2.405)--(6.546,2.405);
\draw[gp path] (6.543,2.315)--(6.543,2.495);
\draw[gp path] (6.546,2.315)--(6.546,2.495);
\draw[gp path] (6.770,2.488)--(6.774,2.488);
\draw[gp path] (6.770,2.398)--(6.770,2.578);
\draw[gp path] (6.774,2.398)--(6.774,2.578);
\draw[gp path] (6.976,2.601)--(6.980,2.601);
\draw[gp path] (6.976,2.511)--(6.976,2.691);
\draw[gp path] (6.980,2.511)--(6.980,2.691);
\draw[gp path] (7.177,2.715)--(7.181,2.715);
\draw[gp path] (7.177,2.625)--(7.177,2.805);
\draw[gp path] (7.181,2.625)--(7.181,2.805);
\draw[gp path] (7.373,2.836)--(7.376,2.836);
\draw[gp path] (7.373,2.746)--(7.373,2.926);
\draw[gp path] (7.376,2.746)--(7.376,2.926);
\draw[gp path] (7.603,2.956)--(7.606,2.956);
\draw[gp path] (7.603,2.866)--(7.603,3.046);
\draw[gp path] (7.606,2.866)--(7.606,3.046);
\draw[gp path] (7.778,3.110)--(7.782,3.110);
\draw[gp path] (7.778,3.020)--(7.778,3.200);
\draw[gp path] (7.782,3.020)--(7.782,3.200);
\draw[gp path] (7.998,3.235)--(8.001,3.235);
\draw[gp path] (7.998,3.145)--(7.998,3.325);
\draw[gp path] (8.001,3.145)--(8.001,3.325);
\draw[gp path] (8.195,3.372)--(8.199,3.372);
\draw[gp path] (8.195,3.282)--(8.195,3.462);
\draw[gp path] (8.199,3.282)--(8.199,3.462);
\draw[gp path] (8.430,3.539)--(8.434,3.539);
\draw[gp path] (8.430,3.449)--(8.430,3.629);
\draw[gp path] (8.434,3.449)--(8.434,3.629);
\draw[gp path] (8.650,3.697)--(8.654,3.697);
\draw[gp path] (8.650,3.607)--(8.650,3.787);
\draw[gp path] (8.654,3.607)--(8.654,3.787);
\draw[gp path] (8.861,3.841)--(8.865,3.841);
\draw[gp path] (8.861,3.751)--(8.861,3.931);
\draw[gp path] (8.865,3.751)--(8.865,3.931);
\draw[gp path] (9.070,4.031)--(9.073,4.031);
\draw[gp path] (9.070,3.941)--(9.070,4.121);
\draw[gp path] (9.073,3.941)--(9.073,4.121);
\draw[gp path] (9.293,4.220)--(9.296,4.220);
\draw[gp path] (9.293,4.130)--(9.293,4.310);
\draw[gp path] (9.296,4.130)--(9.296,4.310);
\draw[gp path] (9.490,4.390)--(9.494,4.390);
\draw[gp path] (9.490,4.300)--(9.490,4.480);
\draw[gp path] (9.494,4.300)--(9.494,4.480);
\draw[gp path] (9.664,4.582)--(9.668,4.582);
\draw[gp path] (9.664,4.492)--(9.664,4.672);
\draw[gp path] (9.668,4.492)--(9.668,4.672);
\draw[gp path] (9.894,4.806)--(9.898,4.806);
\draw[gp path] (9.894,4.716)--(9.894,4.896);
\draw[gp path] (9.898,4.716)--(9.898,4.896);
\draw[gp path] (10.067,5.035)--(10.070,5.035);
\draw[gp path] (10.067,4.945)--(10.067,5.125);
\draw[gp path] (10.070,4.945)--(10.070,5.125);
\draw[gp path] (10.293,5.257)--(10.297,5.257);
\draw[gp path] (10.293,5.167)--(10.293,5.347);
\draw[gp path] (10.297,5.167)--(10.297,5.347);
\draw[gp path] (10.489,5.496)--(10.493,5.496);
\draw[gp path] (10.489,5.406)--(10.489,5.586);
\draw[gp path] (10.493,5.406)--(10.493,5.586);
\draw[gp path] (10.664,5.776)--(10.668,5.776);
\draw[gp path] (10.664,5.686)--(10.664,5.866);
\draw[gp path] (10.668,5.686)--(10.668,5.866);
\draw[gp path] (10.894,6.032)--(10.897,6.032);
\draw[gp path] (10.894,5.942)--(10.894,6.122);
\draw[gp path] (10.897,5.942)--(10.897,6.122);
\draw[gp path] (11.105,6.295)--(11.109,6.295);
\draw[gp path] (11.105,6.205)--(11.105,6.385);
\draw[gp path] (11.109,6.205)--(11.109,6.385);
\draw[gp path] (11.331,6.602)--(11.334,6.602);
\draw[gp path] (11.331,6.512)--(11.331,6.692);
\draw[gp path] (11.334,6.512)--(11.334,6.692);
\draw[gp path] (11.544,6.872)--(11.547,6.872);
\draw[gp path] (11.544,6.782)--(11.544,6.962);
\draw[gp path] (11.547,6.782)--(11.547,6.962);
\draw[gp path] (11.725,7.181)--(11.728,7.181);
\draw[gp path] (11.725,7.091)--(11.725,7.271);
\draw[gp path] (11.728,7.091)--(11.728,7.271);
\draw[gp path] (11.905,7.459)--(11.909,7.459);
\draw[gp path] (11.905,7.369)--(11.905,7.549);
\draw[gp path] (11.909,7.369)--(11.909,7.549);
\gpsetpointsize{4.00}
\gppoint{gp mark 1}{(1.504,1.019)}
\gppoint{gp mark 1}{(1.726,1.044)}
\gppoint{gp mark 1}{(1.948,1.091)}
\gppoint{gp mark 1}{(2.181,1.115)}
\gppoint{gp mark 1}{(2.368,1.151)}
\gppoint{gp mark 1}{(2.542,1.191)}
\gppoint{gp mark 1}{(2.737,1.234)}
\gppoint{gp mark 1}{(2.964,1.263)}
\gppoint{gp mark 1}{(3.136,1.315)}
\gppoint{gp mark 1}{(3.361,1.364)}
\gppoint{gp mark 1}{(3.574,1.419)}
\gppoint{gp mark 1}{(3.796,1.469)}
\gppoint{gp mark 1}{(3.995,1.537)}
\gppoint{gp mark 1}{(4.218,1.590)}
\gppoint{gp mark 1}{(4.456,1.640)}
\gppoint{gp mark 1}{(4.627,1.711)}
\gppoint{gp mark 1}{(4.852,1.767)}
\gppoint{gp mark 1}{(5.078,1.831)}
\gppoint{gp mark 1}{(5.294,1.899)}
\gppoint{gp mark 1}{(5.486,1.986)}
\gppoint{gp mark 1}{(5.717,2.069)}
\gppoint{gp mark 1}{(5.918,2.147)}
\gppoint{gp mark 1}{(6.137,2.230)}
\gppoint{gp mark 1}{(6.308,2.305)}
\gppoint{gp mark 1}{(6.544,2.405)}
\gppoint{gp mark 1}{(6.772,2.488)}
\gppoint{gp mark 1}{(6.978,2.601)}
\gppoint{gp mark 1}{(7.179,2.715)}
\gppoint{gp mark 1}{(7.375,2.836)}
\gppoint{gp mark 1}{(7.604,2.956)}
\gppoint{gp mark 1}{(7.780,3.110)}
\gppoint{gp mark 1}{(7.999,3.235)}
\gppoint{gp mark 1}{(8.197,3.372)}
\gppoint{gp mark 1}{(8.432,3.539)}
\gppoint{gp mark 1}{(8.652,3.697)}
\gppoint{gp mark 1}{(8.863,3.841)}
\gppoint{gp mark 1}{(9.072,4.031)}
\gppoint{gp mark 1}{(9.295,4.220)}
\gppoint{gp mark 1}{(9.492,4.390)}
\gppoint{gp mark 1}{(9.666,4.582)}
\gppoint{gp mark 1}{(9.896,4.806)}
\gppoint{gp mark 1}{(10.068,5.035)}
\gppoint{gp mark 1}{(10.295,5.257)}
\gppoint{gp mark 1}{(10.491,5.496)}
\gppoint{gp mark 1}{(10.666,5.776)}
\gppoint{gp mark 1}{(10.895,6.032)}
\gppoint{gp mark 1}{(11.107,6.295)}
\gppoint{gp mark 1}{(11.332,6.602)}
\gppoint{gp mark 1}{(11.546,6.872)}
\gppoint{gp mark 1}{(11.727,7.181)}
\gppoint{gp mark 1}{(11.907,7.459)}
\gppoint{gp mark 1}{(11.213,8.047)}
\gpcolor{gp lt color border}
\gpsetlinetype{gp lt border}
\draw[gp path] (1.504,8.381)--(1.504,0.985)--(12.039,0.985)--(12.039,8.381)--cycle;
%% coordinates of the plot area
\gpdefrectangularnode{gp plot 1}{\pgfpoint{1.504cm}{0.985cm}}{\pgfpoint{12.039cm}{8.381cm}}
\end{tikzpicture}
%% gnuplot variables

}
\resizebox{0.5\textwidth}{!}{
	\begin{tikzpicture}[gnuplot]
%% generated with GNUPLOT 4.4p0 (Lua 5.1.4; terminal rev. 97, script rev. 96a)
%% 10.05.2010 09:43:07
\gpcolor{gp lt color border}
\gpsetlinetype{gp lt border}
\gpsetlinewidth{1.00}
\draw[gp path] (1.688,0.985)--(1.868,0.985);
\draw[gp path] (12.039,0.985)--(11.859,0.985);
\node[gp node right] at (1.504,0.985) { 0};
\draw[gp path] (1.688,2.218)--(1.868,2.218);
\draw[gp path] (12.039,2.218)--(11.859,2.218);
\node[gp node right] at (1.504,2.218) { 20};
\draw[gp path] (1.688,3.450)--(1.868,3.450);
\draw[gp path] (12.039,3.450)--(11.859,3.450);
\node[gp node right] at (1.504,3.450) { 40};
\draw[gp path] (1.688,4.683)--(1.868,4.683);
\draw[gp path] (12.039,4.683)--(11.859,4.683);
\node[gp node right] at (1.504,4.683) { 60};
\draw[gp path] (1.688,5.916)--(1.868,5.916);
\draw[gp path] (12.039,5.916)--(11.859,5.916);
\node[gp node right] at (1.504,5.916) { 80};
\draw[gp path] (1.688,7.148)--(1.868,7.148);
\draw[gp path] (12.039,7.148)--(11.859,7.148);
\node[gp node right] at (1.504,7.148) { 100};
\draw[gp path] (1.688,8.381)--(1.868,8.381);
\draw[gp path] (12.039,8.381)--(11.859,8.381);
\node[gp node right] at (1.504,8.381) { 120};
\draw[gp path] (1.688,0.985)--(1.688,1.165);
\draw[gp path] (1.688,8.381)--(1.688,8.201);
\node[gp node center] at (1.688,0.677) { 0};
\draw[gp path] (3.358,0.985)--(3.358,1.165);
\draw[gp path] (3.358,8.381)--(3.358,8.201);
\node[gp node center] at (3.358,0.677) { 500};
\draw[gp path] (5.027,0.985)--(5.027,1.165);
\draw[gp path] (5.027,8.381)--(5.027,8.201);
\node[gp node center] at (5.027,0.677) { 1000};
\draw[gp path] (6.697,0.985)--(6.697,1.165);
\draw[gp path] (6.697,8.381)--(6.697,8.201);
\node[gp node center] at (6.697,0.677) { 1500};
\draw[gp path] (8.366,0.985)--(8.366,1.165);
\draw[gp path] (8.366,8.381)--(8.366,8.201);
\node[gp node center] at (8.366,0.677) { 2000};
\draw[gp path] (10.036,0.985)--(10.036,1.165);
\draw[gp path] (10.036,8.381)--(10.036,8.201);
\node[gp node center] at (10.036,0.677) { 2500};
\draw[gp path] (11.705,0.985)--(11.705,1.165);
\draw[gp path] (11.705,8.381)--(11.705,8.201);
\node[gp node center] at (11.705,0.677) { 3000};
\draw[gp path] (1.688,8.381)--(1.688,0.985)--(12.039,0.985)--(12.039,8.381)--cycle;
\node[gp node center,rotate=-270] at (0.430,4.683) {$\Delta n$ [$\unit[]{10^{-7}}$]};
\node[gp node center] at (6.863,0.215) {t [s]};
\node[gp node right] at (10.571,8.047) {Measured Data};
\gpcolor{gp lt color 0}
\gpsetlinetype{gp lt plot 0}
\draw[gp path] (10.755,8.047)--(11.671,8.047);
\draw[gp path] (10.755,8.137)--(10.755,7.957);
\draw[gp path] (11.671,8.137)--(11.671,7.957);
\draw[gp path] (1.688,1.480)--(1.688,1.481);
\draw[gp path] (1.598,1.480)--(1.778,1.480);
\draw[gp path] (1.598,1.481)--(1.778,1.481);
\draw[gp path] (1.906,1.633)--(1.906,1.634);
\draw[gp path] (1.816,1.633)--(1.996,1.633);
\draw[gp path] (1.816,1.634)--(1.996,1.634);
\draw[gp path] (2.124,1.855)--(2.124,1.859);
\draw[gp path] (2.034,1.855)--(2.214,1.855);
\draw[gp path] (2.034,1.859)--(2.214,1.859);
\draw[gp path] (2.353,1.946)--(2.353,1.950);
\draw[gp path] (2.263,1.946)--(2.443,1.946);
\draw[gp path] (2.263,1.950)--(2.443,1.950);
\draw[gp path] (2.537,2.071)--(2.537,2.074);
\draw[gp path] (2.447,2.071)--(2.627,2.071);
\draw[gp path] (2.447,2.074)--(2.627,2.074);
\draw[gp path] (2.708,2.196)--(2.708,2.199);
\draw[gp path] (2.618,2.196)--(2.798,2.196);
\draw[gp path] (2.618,2.199)--(2.798,2.199);
\draw[gp path] (2.899,2.317)--(2.899,2.319);
\draw[gp path] (2.809,2.317)--(2.989,2.317);
\draw[gp path] (2.809,2.319)--(2.989,2.319);
\draw[gp path] (3.122,2.393)--(3.122,2.396);
\draw[gp path] (3.032,2.393)--(3.212,2.393);
\draw[gp path] (3.032,2.396)--(3.212,2.396);
\draw[gp path] (3.292,2.519)--(3.292,2.522);
\draw[gp path] (3.202,2.519)--(3.382,2.519);
\draw[gp path] (3.202,2.522)--(3.382,2.522);
\draw[gp path] (3.513,2.629)--(3.513,2.632);
\draw[gp path] (3.423,2.629)--(3.603,2.629);
\draw[gp path] (3.423,2.632)--(3.603,2.632);
\draw[gp path] (3.721,2.745)--(3.721,2.747);
\draw[gp path] (3.631,2.745)--(3.811,2.745);
\draw[gp path] (3.631,2.747)--(3.811,2.747);
\draw[gp path] (3.940,2.842)--(3.940,2.845);
\draw[gp path] (3.850,2.842)--(4.030,2.842);
\draw[gp path] (3.850,2.845)--(4.030,2.845);
\draw[gp path] (4.135,2.968)--(4.135,2.970);
\draw[gp path] (4.045,2.968)--(4.225,2.968);
\draw[gp path] (4.045,2.970)--(4.225,2.970);
\draw[gp path] (4.355,3.063)--(4.355,3.065);
\draw[gp path] (4.265,3.063)--(4.445,3.063);
\draw[gp path] (4.265,3.065)--(4.445,3.065);
\draw[gp path] (4.588,3.147)--(4.588,3.149);
\draw[gp path] (4.498,3.147)--(4.678,3.147);
\draw[gp path] (4.498,3.149)--(4.678,3.149);
\draw[gp path] (4.756,3.260)--(4.756,3.262);
\draw[gp path] (4.666,3.260)--(4.846,3.260);
\draw[gp path] (4.666,3.262)--(4.846,3.262);
\draw[gp path] (4.978,3.346)--(4.978,3.349);
\draw[gp path] (4.888,3.346)--(5.068,3.346);
\draw[gp path] (4.888,3.349)--(5.068,3.349);
\draw[gp path] (5.200,3.440)--(5.200,3.443);
\draw[gp path] (5.110,3.440)--(5.290,3.440);
\draw[gp path] (5.110,3.443)--(5.290,3.443);
\draw[gp path] (5.412,3.538)--(5.412,3.541);
\draw[gp path] (5.322,3.538)--(5.502,3.538);
\draw[gp path] (5.322,3.541)--(5.502,3.541);
\draw[gp path] (5.601,3.651)--(5.601,3.665);
\draw[gp path] (5.511,3.651)--(5.691,3.651);
\draw[gp path] (5.511,3.665)--(5.691,3.665);
\draw[gp path] (5.827,3.761)--(5.827,3.773);
\draw[gp path] (5.737,3.761)--(5.917,3.761);
\draw[gp path] (5.737,3.773)--(5.917,3.773);
\draw[gp path] (6.025,3.859)--(6.025,3.871);
\draw[gp path] (5.935,3.859)--(6.115,3.859);
\draw[gp path] (5.935,3.871)--(6.115,3.871);
\draw[gp path] (6.240,3.960)--(6.240,3.972);
\draw[gp path] (6.150,3.960)--(6.330,3.960);
\draw[gp path] (6.150,3.972)--(6.330,3.972);
\draw[gp path] (6.408,4.049)--(6.408,4.060);
\draw[gp path] (6.318,4.049)--(6.498,4.049);
\draw[gp path] (6.318,4.060)--(6.498,4.060);
\draw[gp path] (6.640,4.164)--(6.640,4.175);
\draw[gp path] (6.550,4.164)--(6.730,4.164);
\draw[gp path] (6.550,4.175)--(6.730,4.175);
\draw[gp path] (6.864,4.255)--(6.864,4.266);
\draw[gp path] (6.774,4.255)--(6.954,4.255);
\draw[gp path] (6.774,4.266)--(6.954,4.266);
\draw[gp path] (7.066,4.377)--(7.066,4.388);
\draw[gp path] (6.976,4.377)--(7.156,4.377);
\draw[gp path] (6.976,4.388)--(7.156,4.388);
\draw[gp path] (7.264,4.495)--(7.264,4.506);
\draw[gp path] (7.174,4.495)--(7.354,4.495);
\draw[gp path] (7.174,4.506)--(7.354,4.506);
\draw[gp path] (7.456,4.616)--(7.456,4.626);
\draw[gp path] (7.366,4.616)--(7.546,4.616);
\draw[gp path] (7.366,4.626)--(7.546,4.626);
\draw[gp path] (7.682,4.733)--(7.682,4.743);
\draw[gp path] (7.592,4.733)--(7.772,4.733);
\draw[gp path] (7.592,4.743)--(7.772,4.743);
\draw[gp path] (7.854,4.877)--(7.854,4.886);
\draw[gp path] (7.764,4.877)--(7.944,4.877);
\draw[gp path] (7.764,4.886)--(7.944,4.886);
\draw[gp path] (8.070,4.989)--(8.070,4.999);
\draw[gp path] (7.980,4.989)--(8.160,4.989);
\draw[gp path] (7.980,4.999)--(8.160,4.999);
\draw[gp path] (8.264,5.110)--(8.264,5.119);
\draw[gp path] (8.174,5.110)--(8.354,5.110);
\draw[gp path] (8.174,5.119)--(8.354,5.119);
\draw[gp path] (8.495,5.253)--(8.495,5.262);
\draw[gp path] (8.405,5.253)--(8.585,5.253);
\draw[gp path] (8.405,5.262)--(8.585,5.262);
\draw[gp path] (8.711,5.383)--(8.711,5.392);
\draw[gp path] (8.621,5.383)--(8.801,5.383);
\draw[gp path] (8.621,5.392)--(8.801,5.392);
\draw[gp path] (8.919,5.499)--(8.919,5.507);
\draw[gp path] (8.829,5.499)--(9.009,5.499);
\draw[gp path] (8.829,5.507)--(9.009,5.507);
\draw[gp path] (9.124,5.647)--(9.124,5.655);
\draw[gp path] (9.034,5.647)--(9.214,5.647);
\draw[gp path] (9.034,5.655)--(9.214,5.655);
\draw[gp path] (9.343,5.790)--(9.343,5.798);
\draw[gp path] (9.253,5.790)--(9.433,5.790);
\draw[gp path] (9.253,5.798)--(9.433,5.798);
\draw[gp path] (9.536,5.915)--(9.536,5.923);
\draw[gp path] (9.446,5.915)--(9.626,5.915);
\draw[gp path] (9.446,5.923)--(9.626,5.923);
\draw[gp path] (9.707,6.053)--(9.707,6.061);
\draw[gp path] (9.617,6.053)--(9.797,6.053);
\draw[gp path] (9.617,6.061)--(9.797,6.061);
\draw[gp path] (9.934,6.209)--(9.934,6.217);
\draw[gp path] (9.844,6.209)--(10.024,6.209);
\draw[gp path] (9.844,6.217)--(10.024,6.217);
\draw[gp path] (10.103,6.364)--(10.103,6.372);
\draw[gp path] (10.013,6.364)--(10.193,6.364);
\draw[gp path] (10.013,6.372)--(10.193,6.372);
\draw[gp path] (10.326,6.509)--(10.326,6.517);
\draw[gp path] (10.236,6.509)--(10.416,6.509);
\draw[gp path] (10.236,6.517)--(10.416,6.517);
\draw[gp path] (10.518,6.662)--(10.518,6.670);
\draw[gp path] (10.428,6.662)--(10.608,6.662);
\draw[gp path] (10.428,6.670)--(10.608,6.670);
\draw[gp path] (10.690,6.837)--(10.690,6.845);
\draw[gp path] (10.600,6.837)--(10.780,6.837);
\draw[gp path] (10.600,6.845)--(10.780,6.845);
\draw[gp path] (10.915,6.992)--(10.915,7.000);
\draw[gp path] (10.825,6.992)--(11.005,6.992);
\draw[gp path] (10.825,7.000)--(11.005,7.000);
\draw[gp path] (11.123,7.146)--(11.123,7.154);
\draw[gp path] (11.033,7.146)--(11.213,7.146);
\draw[gp path] (11.033,7.154)--(11.213,7.154);
\draw[gp path] (11.345,7.323)--(11.345,7.331);
\draw[gp path] (11.255,7.323)--(11.435,7.323);
\draw[gp path] (11.255,7.331)--(11.435,7.331);
\draw[gp path] (11.554,7.474)--(11.554,7.482);
\draw[gp path] (11.464,7.474)--(11.644,7.474);
\draw[gp path] (11.464,7.482)--(11.644,7.482);
\draw[gp path] (11.732,7.643)--(11.732,7.651);
\draw[gp path] (11.642,7.643)--(11.822,7.643);
\draw[gp path] (11.642,7.651)--(11.822,7.651);
\draw[gp path] (11.909,7.792)--(11.909,7.800);
\draw[gp path] (11.819,7.792)--(11.999,7.792);
\draw[gp path] (11.819,7.800)--(11.999,7.800);
\draw[gp path] (1.688,1.481)--(1.690,1.481);
\draw[gp path] (1.688,1.391)--(1.688,1.571);
\draw[gp path] (1.690,1.391)--(1.690,1.571);
\draw[gp path] (1.904,1.633)--(1.908,1.633);
\draw[gp path] (1.904,1.543)--(1.904,1.723);
\draw[gp path] (1.908,1.543)--(1.908,1.723);
\draw[gp path] (2.122,1.857)--(2.126,1.857);
\draw[gp path] (2.122,1.767)--(2.122,1.947);
\draw[gp path] (2.126,1.767)--(2.126,1.947);
\draw[gp path] (2.352,1.948)--(2.355,1.948);
\draw[gp path] (2.352,1.858)--(2.352,2.038);
\draw[gp path] (2.355,1.858)--(2.355,2.038);
\draw[gp path] (2.535,2.073)--(2.538,2.073);
\draw[gp path] (2.535,1.983)--(2.535,2.163);
\draw[gp path] (2.538,1.983)--(2.538,2.163);
\draw[gp path] (2.706,2.198)--(2.709,2.198);
\draw[gp path] (2.706,2.108)--(2.706,2.288);
\draw[gp path] (2.709,2.108)--(2.709,2.288);
\draw[gp path] (2.898,2.318)--(2.901,2.318);
\draw[gp path] (2.898,2.228)--(2.898,2.408);
\draw[gp path] (2.901,2.228)--(2.901,2.408);
\draw[gp path] (3.121,2.394)--(3.124,2.394);
\draw[gp path] (3.121,2.304)--(3.121,2.484);
\draw[gp path] (3.124,2.304)--(3.124,2.484);
\draw[gp path] (3.290,2.520)--(3.293,2.520);
\draw[gp path] (3.290,2.430)--(3.290,2.610);
\draw[gp path] (3.293,2.430)--(3.293,2.610);
\draw[gp path] (3.511,2.630)--(3.514,2.630);
\draw[gp path] (3.511,2.540)--(3.511,2.720);
\draw[gp path] (3.514,2.540)--(3.514,2.720);
\draw[gp path] (3.720,2.746)--(3.723,2.746);
\draw[gp path] (3.720,2.656)--(3.720,2.836);
\draw[gp path] (3.723,2.656)--(3.723,2.836);
\draw[gp path] (3.938,2.843)--(3.941,2.843);
\draw[gp path] (3.938,2.753)--(3.938,2.933);
\draw[gp path] (3.941,2.753)--(3.941,2.933);
\draw[gp path] (4.134,2.969)--(4.137,2.969);
\draw[gp path] (4.134,2.879)--(4.134,3.059);
\draw[gp path] (4.137,2.879)--(4.137,3.059);
\draw[gp path] (4.353,3.064)--(4.357,3.064);
\draw[gp path] (4.353,2.974)--(4.353,3.154);
\draw[gp path] (4.357,2.974)--(4.357,3.154);
\draw[gp path] (4.587,3.148)--(4.590,3.148);
\draw[gp path] (4.587,3.058)--(4.587,3.238);
\draw[gp path] (4.590,3.058)--(4.590,3.238);
\draw[gp path] (4.754,3.261)--(4.758,3.261);
\draw[gp path] (4.754,3.171)--(4.754,3.351);
\draw[gp path] (4.758,3.171)--(4.758,3.351);
\draw[gp path] (4.976,3.348)--(4.979,3.348);
\draw[gp path] (4.976,3.258)--(4.976,3.438);
\draw[gp path] (4.979,3.258)--(4.979,3.438);
\draw[gp path] (5.198,3.442)--(5.202,3.442);
\draw[gp path] (5.198,3.352)--(5.198,3.532);
\draw[gp path] (5.202,3.352)--(5.202,3.532);
\draw[gp path] (5.410,3.539)--(5.413,3.539);
\draw[gp path] (5.410,3.449)--(5.410,3.629);
\draw[gp path] (5.413,3.449)--(5.413,3.629);
\draw[gp path] (5.599,3.658)--(5.602,3.658);
\draw[gp path] (5.599,3.568)--(5.599,3.748);
\draw[gp path] (5.602,3.568)--(5.602,3.748);
\draw[gp path] (5.825,3.767)--(5.829,3.767);
\draw[gp path] (5.825,3.677)--(5.825,3.857);
\draw[gp path] (5.829,3.677)--(5.829,3.857);
\draw[gp path] (6.023,3.865)--(6.027,3.865);
\draw[gp path] (6.023,3.775)--(6.023,3.955);
\draw[gp path] (6.027,3.775)--(6.027,3.955);
\draw[gp path] (6.238,3.966)--(6.241,3.966);
\draw[gp path] (6.238,3.876)--(6.238,4.056);
\draw[gp path] (6.241,3.876)--(6.241,4.056);
\draw[gp path] (6.406,4.055)--(6.410,4.055);
\draw[gp path] (6.406,3.965)--(6.406,4.145);
\draw[gp path] (6.410,3.965)--(6.410,4.145);
\draw[gp path] (6.639,4.170)--(6.642,4.170);
\draw[gp path] (6.639,4.080)--(6.639,4.260);
\draw[gp path] (6.642,4.080)--(6.642,4.260);
\draw[gp path] (6.862,4.261)--(6.866,4.261);
\draw[gp path] (6.862,4.171)--(6.862,4.351);
\draw[gp path] (6.866,4.171)--(6.866,4.351);
\draw[gp path] (7.065,4.383)--(7.068,4.383);
\draw[gp path] (7.065,4.293)--(7.065,4.473);
\draw[gp path] (7.068,4.293)--(7.068,4.473);
\draw[gp path] (7.262,4.500)--(7.266,4.500);
\draw[gp path] (7.262,4.410)--(7.262,4.590);
\draw[gp path] (7.266,4.410)--(7.266,4.590);
\draw[gp path] (7.454,4.621)--(7.458,4.621);
\draw[gp path] (7.454,4.531)--(7.454,4.711);
\draw[gp path] (7.458,4.531)--(7.458,4.711);
\draw[gp path] (7.680,4.738)--(7.683,4.738);
\draw[gp path] (7.680,4.648)--(7.680,4.828);
\draw[gp path] (7.683,4.648)--(7.683,4.828);
\draw[gp path] (7.853,4.881)--(7.856,4.881);
\draw[gp path] (7.853,4.791)--(7.853,4.971);
\draw[gp path] (7.856,4.791)--(7.856,4.971);
\draw[gp path] (8.068,4.994)--(8.071,4.994);
\draw[gp path] (8.068,4.904)--(8.068,5.084);
\draw[gp path] (8.071,4.904)--(8.071,5.084);
\draw[gp path] (8.262,5.115)--(8.266,5.115);
\draw[gp path] (8.262,5.025)--(8.262,5.205);
\draw[gp path] (8.266,5.025)--(8.266,5.205);
\draw[gp path] (8.493,5.257)--(8.497,5.257);
\draw[gp path] (8.493,5.167)--(8.493,5.347);
\draw[gp path] (8.497,5.167)--(8.497,5.347);
\draw[gp path] (8.710,5.387)--(8.713,5.387);
\draw[gp path] (8.710,5.297)--(8.710,5.477);
\draw[gp path] (8.713,5.297)--(8.713,5.477);
\draw[gp path] (8.917,5.503)--(8.920,5.503);
\draw[gp path] (8.917,5.413)--(8.917,5.593);
\draw[gp path] (8.920,5.413)--(8.920,5.593);
\draw[gp path] (9.122,5.651)--(9.125,5.651);
\draw[gp path] (9.122,5.561)--(9.122,5.741);
\draw[gp path] (9.125,5.561)--(9.125,5.741);
\draw[gp path] (9.341,5.794)--(9.344,5.794);
\draw[gp path] (9.341,5.704)--(9.341,5.884);
\draw[gp path] (9.344,5.704)--(9.344,5.884);
\draw[gp path] (9.535,5.919)--(9.538,5.919);
\draw[gp path] (9.535,5.829)--(9.535,6.009);
\draw[gp path] (9.538,5.829)--(9.538,6.009);
\draw[gp path] (9.706,6.057)--(9.709,6.057);
\draw[gp path] (9.706,5.967)--(9.706,6.147);
\draw[gp path] (9.709,5.967)--(9.709,6.147);
\draw[gp path] (9.932,6.213)--(9.935,6.213);
\draw[gp path] (9.932,6.123)--(9.932,6.303);
\draw[gp path] (9.935,6.123)--(9.935,6.303);
\draw[gp path] (10.101,6.368)--(10.104,6.368);
\draw[gp path] (10.101,6.278)--(10.101,6.458);
\draw[gp path] (10.104,6.278)--(10.104,6.458);
\draw[gp path] (10.324,6.513)--(10.327,6.513);
\draw[gp path] (10.324,6.423)--(10.324,6.603);
\draw[gp path] (10.327,6.423)--(10.327,6.603);
\draw[gp path] (10.516,6.666)--(10.520,6.666);
\draw[gp path] (10.516,6.576)--(10.516,6.756);
\draw[gp path] (10.520,6.576)--(10.520,6.756);
\draw[gp path] (10.688,6.841)--(10.692,6.841);
\draw[gp path] (10.688,6.751)--(10.688,6.931);
\draw[gp path] (10.692,6.751)--(10.692,6.931);
\draw[gp path] (10.914,6.996)--(10.917,6.996);
\draw[gp path] (10.914,6.906)--(10.914,7.086);
\draw[gp path] (10.917,6.906)--(10.917,7.086);
\draw[gp path] (11.122,7.150)--(11.125,7.150);
\draw[gp path] (11.122,7.060)--(11.122,7.240);
\draw[gp path] (11.125,7.060)--(11.125,7.240);
\draw[gp path] (11.343,7.327)--(11.346,7.327);
\draw[gp path] (11.343,7.237)--(11.343,7.417);
\draw[gp path] (11.346,7.237)--(11.346,7.417);
\draw[gp path] (11.553,7.478)--(11.556,7.478);
\draw[gp path] (11.553,7.388)--(11.553,7.568);
\draw[gp path] (11.556,7.388)--(11.556,7.568);
\draw[gp path] (11.730,7.647)--(11.734,7.647);
\draw[gp path] (11.730,7.557)--(11.730,7.737);
\draw[gp path] (11.734,7.557)--(11.734,7.737);
\draw[gp path] (11.907,7.796)--(11.911,7.796);
\draw[gp path] (11.907,7.706)--(11.907,7.886);
\draw[gp path] (11.911,7.706)--(11.911,7.886);
\gpsetpointsize{4.00}
\gppoint{gp mark 1}{(1.688,1.481)}
\gppoint{gp mark 1}{(1.906,1.633)}
\gppoint{gp mark 1}{(2.124,1.857)}
\gppoint{gp mark 1}{(2.353,1.948)}
\gppoint{gp mark 1}{(2.537,2.073)}
\gppoint{gp mark 1}{(2.708,2.198)}
\gppoint{gp mark 1}{(2.899,2.318)}
\gppoint{gp mark 1}{(3.122,2.394)}
\gppoint{gp mark 1}{(3.292,2.520)}
\gppoint{gp mark 1}{(3.513,2.630)}
\gppoint{gp mark 1}{(3.721,2.746)}
\gppoint{gp mark 1}{(3.940,2.843)}
\gppoint{gp mark 1}{(4.135,2.969)}
\gppoint{gp mark 1}{(4.355,3.064)}
\gppoint{gp mark 1}{(4.588,3.148)}
\gppoint{gp mark 1}{(4.756,3.261)}
\gppoint{gp mark 1}{(4.978,3.348)}
\gppoint{gp mark 1}{(5.200,3.442)}
\gppoint{gp mark 1}{(5.412,3.539)}
\gppoint{gp mark 1}{(5.601,3.658)}
\gppoint{gp mark 1}{(5.827,3.767)}
\gppoint{gp mark 1}{(6.025,3.865)}
\gppoint{gp mark 1}{(6.240,3.966)}
\gppoint{gp mark 1}{(6.408,4.055)}
\gppoint{gp mark 1}{(6.640,4.170)}
\gppoint{gp mark 1}{(6.864,4.261)}
\gppoint{gp mark 1}{(7.066,4.383)}
\gppoint{gp mark 1}{(7.264,4.500)}
\gppoint{gp mark 1}{(7.456,4.621)}
\gppoint{gp mark 1}{(7.682,4.738)}
\gppoint{gp mark 1}{(7.854,4.881)}
\gppoint{gp mark 1}{(8.070,4.994)}
\gppoint{gp mark 1}{(8.264,5.115)}
\gppoint{gp mark 1}{(8.495,5.257)}
\gppoint{gp mark 1}{(8.711,5.387)}
\gppoint{gp mark 1}{(8.919,5.503)}
\gppoint{gp mark 1}{(9.124,5.651)}
\gppoint{gp mark 1}{(9.343,5.794)}
\gppoint{gp mark 1}{(9.536,5.919)}
\gppoint{gp mark 1}{(9.707,6.057)}
\gppoint{gp mark 1}{(9.934,6.213)}
\gppoint{gp mark 1}{(10.103,6.368)}
\gppoint{gp mark 1}{(10.326,6.513)}
\gppoint{gp mark 1}{(10.518,6.666)}
\gppoint{gp mark 1}{(10.690,6.841)}
\gppoint{gp mark 1}{(10.915,6.996)}
\gppoint{gp mark 1}{(11.123,7.150)}
\gppoint{gp mark 1}{(11.345,7.327)}
\gppoint{gp mark 1}{(11.554,7.478)}
\gppoint{gp mark 1}{(11.732,7.647)}
\gppoint{gp mark 1}{(11.909,7.796)}
\gppoint{gp mark 1}{(11.213,8.047)}
\gpcolor{gp lt color border}
\gpsetlinetype{gp lt border}
\draw[gp path] (1.688,8.381)--(1.688,0.985)--(12.039,0.985)--(12.039,8.381)--cycle;
%% coordinates of the plot area
\gpdefrectangularnode{gp plot 1}{\pgfpoint{1.688cm}{0.985cm}}{\pgfpoint{12.039cm}{8.381cm}}
\end{tikzpicture}
%% gnuplot variables

}
	\caption{Diffraction intensity $\eta$ (left) and change in refractive index of the crystal (right) versus writing time}
	\label{fig:ana_writing}
\end{floatrow}
\end{figure}
The change of the refractive index should be described by an exponential increase. Therefore we would like to fit a function of the form $\Delta n(t) = \Delta n(t=\infty) \left(1-e^{-\frac{t}{\tau_\textrm{w}}}\right)$ to the obtained graph. As one can clearly see in figure \ref{fig:ana_writing} our data is not described by such a function. No fit with reasonable parameters can be performed, therefore we reject it.

\subsubsection{Recording the Erasing Curve}
\label{subsubsec:ana_erasing}
With the same experimental setup as before we now measure the erasing curve. In order to do that we block one of the writing beams and measure the transmitted and the diffracted intensity of the crystal in time intervalls of about one minute. The results are displayed in table \ref{tab:ana_erasing} in the appendix and plotted in figure \ref{fig:ana_erasing}.
\begin{figure}[H]
\begin{floatrow}
\resizebox{0.5\textwidth}{!}{
	\begin{tikzpicture}[gnuplot]
%% generated with GNUPLOT 4.4p0 (Lua 5.1.4; terminal rev. 97, script rev. 96a)
%% 10.05.2010 00:53:42
\gpcolor{gp lt color border}
\gpsetlinetype{gp lt border}
\gpsetlinewidth{1.00}
\draw[gp path] (1.872,0.985)--(2.052,0.985);
\draw[gp path] (12.039,0.985)--(11.859,0.985);
\node[gp node right] at (1.688,0.985) { 9.8};
\draw[gp path] (1.872,2.042)--(2.052,2.042);
\draw[gp path] (12.039,2.042)--(11.859,2.042);
\node[gp node right] at (1.688,2.042) { 10};
\draw[gp path] (1.872,3.098)--(2.052,3.098);
\draw[gp path] (12.039,3.098)--(11.859,3.098);
\node[gp node right] at (1.688,3.098) { 10.2};
\draw[gp path] (1.872,4.155)--(2.052,4.155);
\draw[gp path] (12.039,4.155)--(11.859,4.155);
\node[gp node right] at (1.688,4.155) { 10.4};
\draw[gp path] (1.872,5.211)--(2.052,5.211);
\draw[gp path] (12.039,5.211)--(11.859,5.211);
\node[gp node right] at (1.688,5.211) { 10.6};
\draw[gp path] (1.872,6.268)--(2.052,6.268);
\draw[gp path] (12.039,6.268)--(11.859,6.268);
\node[gp node right] at (1.688,6.268) { 10.8};
\draw[gp path] (1.872,7.324)--(2.052,7.324);
\draw[gp path] (12.039,7.324)--(11.859,7.324);
\node[gp node right] at (1.688,7.324) { 11};
\draw[gp path] (1.872,8.381)--(2.052,8.381);
\draw[gp path] (12.039,8.381)--(11.859,8.381);
\node[gp node right] at (1.688,8.381) { 11.2};
\draw[gp path] (1.913,0.985)--(1.913,1.165);
\draw[gp path] (1.913,8.381)--(1.913,8.201);
\node[gp node center] at (1.913,0.677) { 0};
\draw[gp path] (3.938,0.985)--(3.938,1.165);
\draw[gp path] (3.938,8.381)--(3.938,8.201);
\node[gp node center] at (3.938,0.677) { 500};
\draw[gp path] (5.963,0.985)--(5.963,1.165);
\draw[gp path] (5.963,8.381)--(5.963,8.201);
\node[gp node center] at (5.963,0.677) { 1000};
\draw[gp path] (7.988,0.985)--(7.988,1.165);
\draw[gp path] (7.988,8.381)--(7.988,8.201);
\node[gp node center] at (7.988,0.677) { 1500};
\draw[gp path] (10.014,0.985)--(10.014,1.165);
\draw[gp path] (10.014,8.381)--(10.014,8.201);
\node[gp node center] at (10.014,0.677) { 2000};
\draw[gp path] (12.039,0.985)--(12.039,1.165);
\draw[gp path] (12.039,8.381)--(12.039,8.201);
\node[gp node center] at (12.039,0.677) { 2500};
\draw[gp path] (1.872,8.381)--(1.872,0.985)--(12.039,0.985)--(12.039,8.381)--cycle;
\node[gp node center,rotate=-270] at (0.430,4.683) {$\eta$ [$\unit[]{10^{-3}}$]};
\node[gp node center] at (6.955,0.215) {t [s]};
\node[gp node right] at (10.571,8.047) {Measured Data};
\gpcolor{gp lt color 0}
\gpsetlinetype{gp lt plot 0}
\draw[gp path] (10.755,8.047)--(11.671,8.047);
\draw[gp path] (10.755,8.137)--(10.755,7.957);
\draw[gp path] (11.671,8.137)--(11.671,7.957);
\draw[gp path] (1.913,8.112)--(1.913,8.255);
\draw[gp path] (1.823,8.112)--(2.003,8.112);
\draw[gp path] (1.823,8.255)--(2.003,8.255);
\draw[gp path] (2.172,7.767)--(2.172,7.909);
\draw[gp path] (2.082,7.767)--(2.262,7.767);
\draw[gp path] (2.082,7.909)--(2.262,7.909);
\draw[gp path] (2.455,7.496)--(2.455,7.638);
\draw[gp path] (2.365,7.496)--(2.545,7.496);
\draw[gp path] (2.365,7.638)--(2.545,7.638);
\draw[gp path] (2.662,7.266)--(2.662,7.408);
\draw[gp path] (2.572,7.266)--(2.752,7.266);
\draw[gp path] (2.572,7.408)--(2.752,7.408);
\draw[gp path] (2.881,6.941)--(2.881,7.082);
\draw[gp path] (2.791,6.941)--(2.971,6.941);
\draw[gp path] (2.791,7.082)--(2.971,7.082);
\draw[gp path] (3.164,6.730)--(3.164,6.871);
\draw[gp path] (3.074,6.730)--(3.254,6.730);
\draw[gp path] (3.074,6.871)--(3.254,6.871);
\draw[gp path] (3.415,6.416)--(3.415,6.557);
\draw[gp path] (3.325,6.416)--(3.505,6.416);
\draw[gp path] (3.325,6.557)--(3.505,6.557);
\draw[gp path] (3.670,6.189)--(3.670,6.329);
\draw[gp path] (3.580,6.189)--(3.760,6.189);
\draw[gp path] (3.580,6.329)--(3.760,6.329);
\draw[gp path] (3.914,5.927)--(3.914,6.067);
\draw[gp path] (3.824,5.927)--(4.004,5.927);
\draw[gp path] (3.824,6.067)--(4.004,6.067);
\draw[gp path] (4.132,5.682)--(4.132,5.822);
\draw[gp path] (4.042,5.682)--(4.222,5.682);
\draw[gp path] (4.042,5.822)--(4.222,5.822);
\draw[gp path] (4.400,5.506)--(4.400,5.645);
\draw[gp path] (4.310,5.506)--(4.490,5.506);
\draw[gp path] (4.310,5.645)--(4.490,5.645);
\draw[gp path] (4.643,5.272)--(4.643,5.411);
\draw[gp path] (4.553,5.272)--(4.733,5.272);
\draw[gp path] (4.553,5.411)--(4.733,5.411);
\draw[gp path] (4.886,5.110)--(4.886,5.248);
\draw[gp path] (4.796,5.110)--(4.976,5.110);
\draw[gp path] (4.796,5.248)--(4.976,5.248);
\draw[gp path] (5.121,4.892)--(5.121,5.031);
\draw[gp path] (5.031,4.892)--(5.211,4.892);
\draw[gp path] (5.031,5.031)--(5.211,5.031);
\draw[gp path] (5.351,4.722)--(5.351,4.861);
\draw[gp path] (5.261,4.722)--(5.441,4.722);
\draw[gp path] (5.261,4.861)--(5.441,4.861);
\draw[gp path] (5.611,4.491)--(5.611,4.629);
\draw[gp path] (5.521,4.491)--(5.701,4.491);
\draw[gp path] (5.521,4.629)--(5.701,4.629);
\draw[gp path] (5.878,4.295)--(5.878,4.433);
\draw[gp path] (5.788,4.295)--(5.968,4.295);
\draw[gp path] (5.788,4.433)--(5.968,4.433);
\draw[gp path] (6.129,4.179)--(6.129,4.317);
\draw[gp path] (6.039,4.179)--(6.219,4.179);
\draw[gp path] (6.039,4.317)--(6.219,4.317);
\draw[gp path] (6.372,3.965)--(6.372,4.102);
\draw[gp path] (6.282,3.965)--(6.462,3.965);
\draw[gp path] (6.282,4.102)--(6.462,4.102);
\draw[gp path] (6.599,3.833)--(6.599,3.971);
\draw[gp path] (6.509,3.833)--(6.689,3.833);
\draw[gp path] (6.509,3.971)--(6.689,3.971);
\draw[gp path] (6.838,3.674)--(6.838,3.811);
\draw[gp path] (6.748,3.674)--(6.928,3.674);
\draw[gp path] (6.748,3.811)--(6.928,3.811);
\draw[gp path] (7.073,3.509)--(7.073,3.646);
\draw[gp path] (6.983,3.509)--(7.163,3.509);
\draw[gp path] (6.983,3.646)--(7.163,3.646);
\draw[gp path] (7.304,3.375)--(7.304,3.511);
\draw[gp path] (7.214,3.375)--(7.394,3.375);
\draw[gp path] (7.214,3.511)--(7.394,3.511);
\draw[gp path] (7.551,3.269)--(7.551,3.405);
\draw[gp path] (7.461,3.269)--(7.641,3.269);
\draw[gp path] (7.461,3.405)--(7.641,3.405);
\draw[gp path] (7.782,3.090)--(7.782,3.226);
\draw[gp path] (7.692,3.090)--(7.872,3.090);
\draw[gp path] (7.692,3.226)--(7.872,3.226);
\draw[gp path] (8.061,3.012)--(8.061,3.149);
\draw[gp path] (7.971,3.012)--(8.151,3.012);
\draw[gp path] (7.971,3.149)--(8.151,3.149);
\draw[gp path] (8.321,2.835)--(8.321,2.971);
\draw[gp path] (8.231,2.835)--(8.411,2.835);
\draw[gp path] (8.231,2.971)--(8.411,2.971);
\draw[gp path] (8.527,2.757)--(8.527,2.893);
\draw[gp path] (8.437,2.757)--(8.617,2.757);
\draw[gp path] (8.437,2.893)--(8.617,2.893);
\draw[gp path] (8.807,2.645)--(8.807,2.781);
\draw[gp path] (8.717,2.645)--(8.897,2.645);
\draw[gp path] (8.717,2.781)--(8.897,2.781);
\draw[gp path] (9.074,2.532)--(9.074,2.667);
\draw[gp path] (8.984,2.532)--(9.164,2.532);
\draw[gp path] (8.984,2.667)--(9.164,2.667);
\draw[gp path] (9.341,2.397)--(9.341,2.533);
\draw[gp path] (9.251,2.397)--(9.431,2.397);
\draw[gp path] (9.251,2.533)--(9.431,2.533);
\draw[gp path] (9.556,2.352)--(9.556,2.487);
\draw[gp path] (9.466,2.352)--(9.646,2.352);
\draw[gp path] (9.466,2.487)--(9.646,2.487);
\draw[gp path] (9.815,2.196)--(9.815,2.331);
\draw[gp path] (9.725,2.196)--(9.905,2.196);
\draw[gp path] (9.725,2.331)--(9.905,2.331);
\draw[gp path] (10.026,2.139)--(10.026,2.274);
\draw[gp path] (9.936,2.139)--(10.116,2.139);
\draw[gp path] (9.936,2.274)--(10.116,2.274);
\draw[gp path] (10.281,1.981)--(10.281,2.116);
\draw[gp path] (10.191,1.981)--(10.371,1.981);
\draw[gp path] (10.191,2.116)--(10.371,2.116);
\draw[gp path] (10.561,1.923)--(10.561,2.058);
\draw[gp path] (10.471,1.923)--(10.651,1.923);
\draw[gp path] (10.471,2.058)--(10.651,2.058);
\draw[gp path] (10.832,1.863)--(10.832,1.998);
\draw[gp path] (10.742,1.863)--(10.922,1.863);
\draw[gp path] (10.742,1.998)--(10.922,1.998);
\draw[gp path] (11.103,1.739)--(11.103,1.874);
\draw[gp path] (11.013,1.739)--(11.193,1.739);
\draw[gp path] (11.013,1.874)--(11.193,1.874);
\draw[gp path] (11.318,1.664)--(11.318,1.799);
\draw[gp path] (11.228,1.664)--(11.408,1.664);
\draw[gp path] (11.228,1.799)--(11.408,1.799);
\draw[gp path] (11.577,1.618)--(11.577,1.753);
\draw[gp path] (11.487,1.618)--(11.667,1.618);
\draw[gp path] (11.487,1.753)--(11.667,1.753);
\draw[gp path] (11.836,1.556)--(11.836,1.690);
\draw[gp path] (11.746,1.556)--(11.926,1.556);
\draw[gp path] (11.746,1.690)--(11.926,1.690);
\draw[gp path] (1.910,8.183)--(1.915,8.183);
\draw[gp path] (1.910,8.093)--(1.910,8.273);
\draw[gp path] (1.915,8.093)--(1.915,8.273);
\draw[gp path] (2.170,7.838)--(2.174,7.838);
\draw[gp path] (2.170,7.748)--(2.170,7.928);
\draw[gp path] (2.174,7.748)--(2.174,7.928);
\draw[gp path] (2.453,7.567)--(2.457,7.567);
\draw[gp path] (2.453,7.477)--(2.453,7.657);
\draw[gp path] (2.457,7.477)--(2.457,7.657);
\draw[gp path] (2.660,7.337)--(2.664,7.337);
\draw[gp path] (2.660,7.247)--(2.660,7.427);
\draw[gp path] (2.664,7.247)--(2.664,7.427);
\draw[gp path] (2.879,7.011)--(2.883,7.011);
\draw[gp path] (2.879,6.921)--(2.879,7.101);
\draw[gp path] (2.883,6.921)--(2.883,7.101);
\draw[gp path] (3.162,6.800)--(3.166,6.800);
\draw[gp path] (3.162,6.710)--(3.162,6.890);
\draw[gp path] (3.166,6.710)--(3.166,6.890);
\draw[gp path] (3.413,6.487)--(3.417,6.487);
\draw[gp path] (3.413,6.397)--(3.413,6.577);
\draw[gp path] (3.417,6.397)--(3.417,6.577);
\draw[gp path] (3.668,6.259)--(3.672,6.259);
\draw[gp path] (3.668,6.169)--(3.668,6.349);
\draw[gp path] (3.672,6.169)--(3.672,6.349);
\draw[gp path] (3.911,5.997)--(3.916,5.997);
\draw[gp path] (3.911,5.907)--(3.911,6.087);
\draw[gp path] (3.916,5.907)--(3.916,6.087);
\draw[gp path] (4.130,5.752)--(4.134,5.752);
\draw[gp path] (4.130,5.662)--(4.130,5.842);
\draw[gp path] (4.134,5.662)--(4.134,5.842);
\draw[gp path] (4.398,5.575)--(4.402,5.575);
\draw[gp path] (4.398,5.485)--(4.398,5.665);
\draw[gp path] (4.402,5.485)--(4.402,5.665);
\draw[gp path] (4.641,5.341)--(4.645,5.341);
\draw[gp path] (4.641,5.251)--(4.641,5.431);
\draw[gp path] (4.645,5.251)--(4.645,5.431);
\draw[gp path] (4.884,5.179)--(4.888,5.179);
\draw[gp path] (4.884,5.089)--(4.884,5.269);
\draw[gp path] (4.888,5.089)--(4.888,5.269);
\draw[gp path] (5.119,4.962)--(5.123,4.962);
\draw[gp path] (5.119,4.872)--(5.119,5.052);
\draw[gp path] (5.123,4.872)--(5.123,5.052);
\draw[gp path] (5.349,4.791)--(5.353,4.791);
\draw[gp path] (5.349,4.701)--(5.349,4.881);
\draw[gp path] (5.353,4.701)--(5.353,4.881);
\draw[gp path] (5.609,4.560)--(5.613,4.560);
\draw[gp path] (5.609,4.470)--(5.609,4.650);
\draw[gp path] (5.613,4.470)--(5.613,4.650);
\draw[gp path] (5.876,4.364)--(5.880,4.364);
\draw[gp path] (5.876,4.274)--(5.876,4.454);
\draw[gp path] (5.880,4.274)--(5.880,4.454);
\draw[gp path] (6.127,4.248)--(6.131,4.248);
\draw[gp path] (6.127,4.158)--(6.127,4.338);
\draw[gp path] (6.131,4.158)--(6.131,4.338);
\draw[gp path] (6.370,4.033)--(6.374,4.033);
\draw[gp path] (6.370,3.943)--(6.370,4.123);
\draw[gp path] (6.374,3.943)--(6.374,4.123);
\draw[gp path] (6.597,3.902)--(6.601,3.902);
\draw[gp path] (6.597,3.812)--(6.597,3.992);
\draw[gp path] (6.601,3.812)--(6.601,3.992);
\draw[gp path] (6.836,3.742)--(6.840,3.742);
\draw[gp path] (6.836,3.652)--(6.836,3.832);
\draw[gp path] (6.840,3.652)--(6.840,3.832);
\draw[gp path] (7.071,3.577)--(7.075,3.577);
\draw[gp path] (7.071,3.487)--(7.071,3.667);
\draw[gp path] (7.075,3.487)--(7.075,3.667);
\draw[gp path] (7.302,3.443)--(7.306,3.443);
\draw[gp path] (7.302,3.353)--(7.302,3.533);
\draw[gp path] (7.306,3.353)--(7.306,3.533);
\draw[gp path] (7.549,3.337)--(7.553,3.337);
\draw[gp path] (7.549,3.247)--(7.549,3.427);
\draw[gp path] (7.553,3.247)--(7.553,3.427);
\draw[gp path] (7.780,3.158)--(7.784,3.158);
\draw[gp path] (7.780,3.068)--(7.780,3.248);
\draw[gp path] (7.784,3.068)--(7.784,3.248);
\draw[gp path] (8.059,3.081)--(8.063,3.081);
\draw[gp path] (8.059,2.991)--(8.059,3.171);
\draw[gp path] (8.063,2.991)--(8.063,3.171);
\draw[gp path] (8.319,2.903)--(8.323,2.903);
\draw[gp path] (8.319,2.813)--(8.319,2.993);
\draw[gp path] (8.323,2.813)--(8.323,2.993);
\draw[gp path] (8.525,2.825)--(8.529,2.825);
\draw[gp path] (8.525,2.735)--(8.525,2.915);
\draw[gp path] (8.529,2.735)--(8.529,2.915);
\draw[gp path] (8.805,2.713)--(8.809,2.713);
\draw[gp path] (8.805,2.623)--(8.805,2.803);
\draw[gp path] (8.809,2.623)--(8.809,2.803);
\draw[gp path] (9.072,2.600)--(9.076,2.600);
\draw[gp path] (9.072,2.510)--(9.072,2.690);
\draw[gp path] (9.076,2.510)--(9.076,2.690);
\draw[gp path] (9.339,2.465)--(9.343,2.465);
\draw[gp path] (9.339,2.375)--(9.339,2.555);
\draw[gp path] (9.343,2.375)--(9.343,2.555);
\draw[gp path] (9.554,2.419)--(9.558,2.419);
\draw[gp path] (9.554,2.329)--(9.554,2.509);
\draw[gp path] (9.558,2.329)--(9.558,2.509);
\draw[gp path] (9.813,2.263)--(9.817,2.263);
\draw[gp path] (9.813,2.173)--(9.813,2.353);
\draw[gp path] (9.817,2.173)--(9.817,2.353);
\draw[gp path] (10.024,2.206)--(10.028,2.206);
\draw[gp path] (10.024,2.116)--(10.024,2.296);
\draw[gp path] (10.028,2.116)--(10.028,2.296);
\draw[gp path] (10.279,2.049)--(10.283,2.049);
\draw[gp path] (10.279,1.959)--(10.279,2.139);
\draw[gp path] (10.283,1.959)--(10.283,2.139);
\draw[gp path] (10.559,1.991)--(10.563,1.991);
\draw[gp path] (10.559,1.901)--(10.559,2.081);
\draw[gp path] (10.563,1.901)--(10.563,2.081);
\draw[gp path] (10.830,1.931)--(10.834,1.931);
\draw[gp path] (10.830,1.841)--(10.830,2.021);
\draw[gp path] (10.834,1.841)--(10.834,2.021);
\draw[gp path] (11.101,1.807)--(11.105,1.807);
\draw[gp path] (11.101,1.717)--(11.101,1.897);
\draw[gp path] (11.105,1.717)--(11.105,1.897);
\draw[gp path] (11.316,1.731)--(11.320,1.731);
\draw[gp path] (11.316,1.641)--(11.316,1.821);
\draw[gp path] (11.320,1.641)--(11.320,1.821);
\draw[gp path] (11.575,1.686)--(11.579,1.686);
\draw[gp path] (11.575,1.596)--(11.575,1.776);
\draw[gp path] (11.579,1.596)--(11.579,1.776);
\draw[gp path] (11.834,1.623)--(11.838,1.623);
\draw[gp path] (11.834,1.533)--(11.834,1.713);
\draw[gp path] (11.838,1.533)--(11.838,1.713);
\gpsetpointsize{4.00}
\gppoint{gp mark 1}{(1.913,8.183)}
\gppoint{gp mark 1}{(2.172,7.838)}
\gppoint{gp mark 1}{(2.455,7.567)}
\gppoint{gp mark 1}{(2.662,7.337)}
\gppoint{gp mark 1}{(2.881,7.011)}
\gppoint{gp mark 1}{(3.164,6.800)}
\gppoint{gp mark 1}{(3.415,6.487)}
\gppoint{gp mark 1}{(3.670,6.259)}
\gppoint{gp mark 1}{(3.914,5.997)}
\gppoint{gp mark 1}{(4.132,5.752)}
\gppoint{gp mark 1}{(4.400,5.575)}
\gppoint{gp mark 1}{(4.643,5.341)}
\gppoint{gp mark 1}{(4.886,5.179)}
\gppoint{gp mark 1}{(5.121,4.962)}
\gppoint{gp mark 1}{(5.351,4.791)}
\gppoint{gp mark 1}{(5.611,4.560)}
\gppoint{gp mark 1}{(5.878,4.364)}
\gppoint{gp mark 1}{(6.129,4.248)}
\gppoint{gp mark 1}{(6.372,4.033)}
\gppoint{gp mark 1}{(6.599,3.902)}
\gppoint{gp mark 1}{(6.838,3.742)}
\gppoint{gp mark 1}{(7.073,3.577)}
\gppoint{gp mark 1}{(7.304,3.443)}
\gppoint{gp mark 1}{(7.551,3.337)}
\gppoint{gp mark 1}{(7.782,3.158)}
\gppoint{gp mark 1}{(8.061,3.081)}
\gppoint{gp mark 1}{(8.321,2.903)}
\gppoint{gp mark 1}{(8.527,2.825)}
\gppoint{gp mark 1}{(8.807,2.713)}
\gppoint{gp mark 1}{(9.074,2.600)}
\gppoint{gp mark 1}{(9.341,2.465)}
\gppoint{gp mark 1}{(9.556,2.419)}
\gppoint{gp mark 1}{(9.815,2.263)}
\gppoint{gp mark 1}{(10.026,2.206)}
\gppoint{gp mark 1}{(10.281,2.049)}
\gppoint{gp mark 1}{(10.561,1.991)}
\gppoint{gp mark 1}{(10.832,1.931)}
\gppoint{gp mark 1}{(11.103,1.807)}
\gppoint{gp mark 1}{(11.318,1.731)}
\gppoint{gp mark 1}{(11.577,1.686)}
\gppoint{gp mark 1}{(11.836,1.623)}
\gppoint{gp mark 1}{(11.213,8.047)}
\gpcolor{gp lt color border}
\gpsetlinetype{gp lt border}
\draw[gp path] (1.872,8.381)--(1.872,0.985)--(12.039,0.985)--(12.039,8.381)--cycle;
%% coordinates of the plot area
\gpdefrectangularnode{gp plot 1}{\pgfpoint{1.872cm}{0.985cm}}{\pgfpoint{12.039cm}{8.381cm}}
\end{tikzpicture}
%% gnuplot variables

}
\resizebox{0.5\textwidth}{!}{
	\begin{tikzpicture}[gnuplot]
%% generated with GNUPLOT 4.4p0 (Lua 5.1.4; terminal rev. 97, script rev. 96a)
%% 10.05.2010 00:53:42
\gpcolor{gp lt color border}
\gpsetlinetype{gp lt border}
\gpsetlinewidth{1.00}
\draw[gp path] (1.688,0.985)--(1.868,0.985);
\draw[gp path] (12.039,0.985)--(11.859,0.985);
\node[gp node right] at (1.504,0.985) { 99};
\draw[gp path] (1.688,2.042)--(1.868,2.042);
\draw[gp path] (12.039,2.042)--(11.859,2.042);
\node[gp node right] at (1.504,2.042) { 100};
\draw[gp path] (1.688,3.098)--(1.868,3.098);
\draw[gp path] (12.039,3.098)--(11.859,3.098);
\node[gp node right] at (1.504,3.098) { 101};
\draw[gp path] (1.688,4.155)--(1.868,4.155);
\draw[gp path] (12.039,4.155)--(11.859,4.155);
\node[gp node right] at (1.504,4.155) { 102};
\draw[gp path] (1.688,5.211)--(1.868,5.211);
\draw[gp path] (12.039,5.211)--(11.859,5.211);
\node[gp node right] at (1.504,5.211) { 103};
\draw[gp path] (1.688,6.268)--(1.868,6.268);
\draw[gp path] (12.039,6.268)--(11.859,6.268);
\node[gp node right] at (1.504,6.268) { 104};
\draw[gp path] (1.688,7.324)--(1.868,7.324);
\draw[gp path] (12.039,7.324)--(11.859,7.324);
\node[gp node right] at (1.504,7.324) { 105};
\draw[gp path] (1.688,8.381)--(1.868,8.381);
\draw[gp path] (12.039,8.381)--(11.859,8.381);
\node[gp node right] at (1.504,8.381) { 106};
\draw[gp path] (1.688,0.985)--(1.688,1.165);
\draw[gp path] (1.688,8.381)--(1.688,8.201);
\node[gp node center] at (1.688,0.677) {-500};
\draw[gp path] (3.413,0.985)--(3.413,1.165);
\draw[gp path] (3.413,8.381)--(3.413,8.201);
\node[gp node center] at (3.413,0.677) { 0};
\draw[gp path] (5.138,0.985)--(5.138,1.165);
\draw[gp path] (5.138,8.381)--(5.138,8.201);
\node[gp node center] at (5.138,0.677) { 500};
\draw[gp path] (6.864,0.985)--(6.864,1.165);
\draw[gp path] (6.864,8.381)--(6.864,8.201);
\node[gp node center] at (6.864,0.677) { 1000};
\draw[gp path] (8.589,0.985)--(8.589,1.165);
\draw[gp path] (8.589,8.381)--(8.589,8.201);
\node[gp node center] at (8.589,0.677) { 1500};
\draw[gp path] (10.314,0.985)--(10.314,1.165);
\draw[gp path] (10.314,8.381)--(10.314,8.201);
\node[gp node center] at (10.314,0.677) { 2000};
\draw[gp path] (12.039,0.985)--(12.039,1.165);
\draw[gp path] (12.039,8.381)--(12.039,8.201);
\node[gp node center] at (12.039,0.677) { 2500};
\draw[gp path] (1.688,8.381)--(1.688,0.985)--(12.039,0.985)--(12.039,8.381)--cycle;
\node[gp node center,rotate=-270] at (0.430,4.683) {$\Delta n$ [$\unit[]{10^{-7}}$]};
\node[gp node center] at (6.863,0.215) {t [s]};
\node[gp node right] at (10.571,8.047) {Measured Data};
\gpcolor{gp lt color 0}
\gpsetlinetype{gp lt plot 0}
\draw[gp path] (10.755,8.047)--(11.671,8.047);
\draw[gp path] (10.755,8.137)--(10.755,7.957);
\draw[gp path] (11.671,8.137)--(11.671,7.957);
\draw[gp path] (3.413,7.722)--(3.413,7.857);
\draw[gp path] (3.323,7.722)--(3.503,7.722);
\draw[gp path] (3.323,7.857)--(3.503,7.857);
\draw[gp path] (3.634,7.394)--(3.634,7.529);
\draw[gp path] (3.544,7.394)--(3.724,7.394);
\draw[gp path] (3.544,7.529)--(3.724,7.529);
\draw[gp path] (3.876,7.136)--(3.876,7.272);
\draw[gp path] (3.786,7.136)--(3.966,7.136);
\draw[gp path] (3.786,7.272)--(3.966,7.272);
\draw[gp path] (4.051,6.917)--(4.051,7.052);
\draw[gp path] (3.961,6.917)--(4.141,6.917);
\draw[gp path] (3.961,7.052)--(4.141,7.052);
\draw[gp path] (4.238,6.605)--(4.238,6.741);
\draw[gp path] (4.148,6.605)--(4.328,6.605);
\draw[gp path] (4.148,6.741)--(4.328,6.741);
\draw[gp path] (4.479,6.403)--(4.479,6.538);
\draw[gp path] (4.389,6.403)--(4.569,6.403);
\draw[gp path] (4.389,6.538)--(4.569,6.538);
\draw[gp path] (4.693,6.102)--(4.693,6.237);
\draw[gp path] (4.603,6.102)--(4.783,6.102);
\draw[gp path] (4.603,6.237)--(4.783,6.237);
\draw[gp path] (4.911,5.882)--(4.911,6.018);
\draw[gp path] (4.821,5.882)--(5.001,5.882);
\draw[gp path] (4.821,6.018)--(5.001,6.018);
\draw[gp path] (5.118,5.630)--(5.118,5.765);
\draw[gp path] (5.028,5.630)--(5.208,5.630);
\draw[gp path] (5.028,5.765)--(5.208,5.765);
\draw[gp path] (5.304,5.393)--(5.304,5.528);
\draw[gp path] (5.214,5.393)--(5.394,5.393);
\draw[gp path] (5.214,5.528)--(5.394,5.528);
\draw[gp path] (5.532,5.222)--(5.532,5.357);
\draw[gp path] (5.442,5.222)--(5.622,5.222);
\draw[gp path] (5.442,5.357)--(5.622,5.357);
\draw[gp path] (5.739,4.995)--(5.739,5.130);
\draw[gp path] (5.649,4.995)--(5.829,4.995);
\draw[gp path] (5.649,5.130)--(5.829,5.130);
\draw[gp path] (5.946,4.837)--(5.946,4.972);
\draw[gp path] (5.856,4.837)--(6.036,4.837);
\draw[gp path] (5.856,4.972)--(6.036,4.972);
\draw[gp path] (6.146,4.625)--(6.146,4.760);
\draw[gp path] (6.056,4.625)--(6.236,4.625);
\draw[gp path] (6.056,4.760)--(6.236,4.760);
\draw[gp path] (6.342,4.459)--(6.342,4.594);
\draw[gp path] (6.252,4.459)--(6.432,4.459);
\draw[gp path] (6.252,4.594)--(6.432,4.594);
\draw[gp path] (6.563,4.233)--(6.563,4.368);
\draw[gp path] (6.473,4.233)--(6.653,4.233);
\draw[gp path] (6.473,4.368)--(6.653,4.368);
\draw[gp path] (6.791,4.041)--(6.791,4.176);
\draw[gp path] (6.701,4.041)--(6.881,4.041);
\draw[gp path] (6.701,4.176)--(6.881,4.176);
\draw[gp path] (7.005,3.927)--(7.005,4.062);
\draw[gp path] (6.915,3.927)--(7.095,3.927);
\draw[gp path] (6.915,4.062)--(7.095,4.062);
\draw[gp path] (7.212,3.716)--(7.212,3.851);
\draw[gp path] (7.122,3.716)--(7.302,3.716);
\draw[gp path] (7.122,3.851)--(7.302,3.851);
\draw[gp path] (7.405,3.587)--(7.405,3.722);
\draw[gp path] (7.315,3.587)--(7.495,3.587);
\draw[gp path] (7.315,3.722)--(7.495,3.722);
\draw[gp path] (7.609,3.430)--(7.609,3.565);
\draw[gp path] (7.519,3.430)--(7.699,3.430);
\draw[gp path] (7.519,3.565)--(7.699,3.565);
\draw[gp path] (7.809,3.267)--(7.809,3.402);
\draw[gp path] (7.719,3.267)--(7.899,3.267);
\draw[gp path] (7.719,3.402)--(7.899,3.402);
\draw[gp path] (8.006,3.135)--(8.006,3.269);
\draw[gp path] (7.916,3.135)--(8.096,3.135);
\draw[gp path] (7.916,3.269)--(8.096,3.269);
\draw[gp path] (8.216,3.030)--(8.216,3.165);
\draw[gp path] (8.126,3.030)--(8.306,3.030);
\draw[gp path] (8.126,3.165)--(8.306,3.165);
\draw[gp path] (8.413,2.853)--(8.413,2.988);
\draw[gp path] (8.323,2.853)--(8.503,2.853);
\draw[gp path] (8.323,2.988)--(8.503,2.988);
\draw[gp path] (8.651,2.776)--(8.651,2.911);
\draw[gp path] (8.561,2.776)--(8.741,2.776);
\draw[gp path] (8.561,2.911)--(8.741,2.911);
\draw[gp path] (8.872,2.600)--(8.872,2.735);
\draw[gp path] (8.782,2.600)--(8.962,2.600);
\draw[gp path] (8.782,2.735)--(8.962,2.735);
\draw[gp path] (9.048,2.522)--(9.048,2.657);
\draw[gp path] (8.958,2.522)--(9.138,2.522);
\draw[gp path] (8.958,2.657)--(9.138,2.657);
\draw[gp path] (9.286,2.411)--(9.286,2.546);
\draw[gp path] (9.196,2.411)--(9.376,2.411);
\draw[gp path] (9.196,2.546)--(9.376,2.546);
\draw[gp path] (9.513,2.298)--(9.513,2.433);
\draw[gp path] (9.423,2.298)--(9.603,2.298);
\draw[gp path] (9.423,2.433)--(9.603,2.433);
\draw[gp path] (9.741,2.164)--(9.741,2.299);
\draw[gp path] (9.651,2.164)--(9.831,2.164);
\draw[gp path] (9.651,2.299)--(9.831,2.299);
\draw[gp path] (9.924,2.118)--(9.924,2.253);
\draw[gp path] (9.834,2.118)--(10.014,2.118);
\draw[gp path] (9.834,2.253)--(10.014,2.253);
\draw[gp path] (10.145,1.962)--(10.145,2.097);
\draw[gp path] (10.055,1.962)--(10.235,1.962);
\draw[gp path] (10.055,2.097)--(10.235,2.097);
\draw[gp path] (10.324,1.906)--(10.324,2.040);
\draw[gp path] (10.234,1.906)--(10.414,1.906);
\draw[gp path] (10.234,2.040)--(10.414,2.040);
\draw[gp path] (10.542,1.748)--(10.542,1.883);
\draw[gp path] (10.452,1.748)--(10.632,1.748);
\draw[gp path] (10.452,1.883)--(10.632,1.883);
\draw[gp path] (10.780,1.690)--(10.780,1.825);
\draw[gp path] (10.690,1.690)--(10.870,1.690);
\draw[gp path] (10.690,1.825)--(10.870,1.825);
\draw[gp path] (11.011,1.630)--(11.011,1.765);
\draw[gp path] (10.921,1.630)--(11.101,1.630);
\draw[gp path] (10.921,1.765)--(11.101,1.765);
\draw[gp path] (11.242,1.505)--(11.242,1.640);
\draw[gp path] (11.152,1.505)--(11.332,1.505);
\draw[gp path] (11.152,1.640)--(11.332,1.640);
\draw[gp path] (11.425,1.430)--(11.425,1.565);
\draw[gp path] (11.335,1.430)--(11.515,1.430);
\draw[gp path] (11.335,1.565)--(11.515,1.565);
\draw[gp path] (11.646,1.384)--(11.646,1.519);
\draw[gp path] (11.556,1.384)--(11.736,1.384);
\draw[gp path] (11.556,1.519)--(11.736,1.519);
\draw[gp path] (11.866,1.321)--(11.866,1.456);
\draw[gp path] (11.776,1.321)--(11.956,1.321);
\draw[gp path] (11.776,1.456)--(11.956,1.456);
\draw[gp path] (3.411,7.790)--(3.415,7.790);
\draw[gp path] (3.411,7.700)--(3.411,7.880);
\draw[gp path] (3.415,7.700)--(3.415,7.880);
\draw[gp path] (3.632,7.462)--(3.636,7.462);
\draw[gp path] (3.632,7.372)--(3.632,7.552);
\draw[gp path] (3.636,7.372)--(3.636,7.552);
\draw[gp path] (3.874,7.204)--(3.877,7.204);
\draw[gp path] (3.874,7.114)--(3.874,7.294);
\draw[gp path] (3.877,7.114)--(3.877,7.294);
\draw[gp path] (4.050,6.985)--(4.053,6.985);
\draw[gp path] (4.050,6.895)--(4.050,7.075);
\draw[gp path] (4.053,6.895)--(4.053,7.075);
\draw[gp path] (4.236,6.673)--(4.240,6.673);
\draw[gp path] (4.236,6.583)--(4.236,6.763);
\draw[gp path] (4.240,6.583)--(4.240,6.763);
\draw[gp path] (4.478,6.471)--(4.481,6.471);
\draw[gp path] (4.478,6.381)--(4.478,6.561);
\draw[gp path] (4.481,6.381)--(4.481,6.561);
\draw[gp path] (4.692,6.169)--(4.695,6.169);
\draw[gp path] (4.692,6.079)--(4.692,6.259);
\draw[gp path] (4.695,6.079)--(4.695,6.259);
\draw[gp path] (4.909,5.950)--(4.912,5.950);
\draw[gp path] (4.909,5.860)--(4.909,6.040);
\draw[gp path] (4.912,5.860)--(4.912,6.040);
\draw[gp path] (5.116,5.697)--(5.119,5.697);
\draw[gp path] (5.116,5.607)--(5.116,5.787);
\draw[gp path] (5.119,5.607)--(5.119,5.787);
\draw[gp path] (5.302,5.460)--(5.306,5.460);
\draw[gp path] (5.302,5.370)--(5.302,5.550);
\draw[gp path] (5.306,5.370)--(5.306,5.550);
\draw[gp path] (5.530,5.289)--(5.533,5.289);
\draw[gp path] (5.530,5.199)--(5.530,5.379);
\draw[gp path] (5.533,5.199)--(5.533,5.379);
\draw[gp path] (5.737,5.062)--(5.740,5.062);
\draw[gp path] (5.737,4.972)--(5.737,5.152);
\draw[gp path] (5.740,4.972)--(5.740,5.152);
\draw[gp path] (5.944,4.904)--(5.947,4.904);
\draw[gp path] (5.944,4.814)--(5.944,4.994);
\draw[gp path] (5.947,4.814)--(5.947,4.994);
\draw[gp path] (6.144,4.693)--(6.148,4.693);
\draw[gp path] (6.144,4.603)--(6.144,4.783);
\draw[gp path] (6.148,4.603)--(6.148,4.783);
\draw[gp path] (6.341,4.527)--(6.344,4.527);
\draw[gp path] (6.341,4.437)--(6.341,4.617);
\draw[gp path] (6.344,4.437)--(6.344,4.617);
\draw[gp path] (6.562,4.301)--(6.565,4.301);
\draw[gp path] (6.562,4.211)--(6.562,4.391);
\draw[gp path] (6.565,4.211)--(6.565,4.391);
\draw[gp path] (6.789,4.109)--(6.793,4.109);
\draw[gp path] (6.789,4.019)--(6.789,4.199);
\draw[gp path] (6.793,4.019)--(6.793,4.199);
\draw[gp path] (7.003,3.994)--(7.007,3.994);
\draw[gp path] (7.003,3.904)--(7.003,4.084);
\draw[gp path] (7.007,3.904)--(7.007,4.084);
\draw[gp path] (7.210,3.784)--(7.214,3.784);
\draw[gp path] (7.210,3.694)--(7.210,3.874);
\draw[gp path] (7.214,3.694)--(7.214,3.874);
\draw[gp path] (7.403,3.655)--(7.407,3.655);
\draw[gp path] (7.403,3.565)--(7.403,3.745);
\draw[gp path] (7.407,3.565)--(7.407,3.745);
\draw[gp path] (7.607,3.497)--(7.610,3.497);
\draw[gp path] (7.607,3.407)--(7.607,3.587);
\draw[gp path] (7.610,3.407)--(7.610,3.587);
\draw[gp path] (7.807,3.335)--(7.811,3.335);
\draw[gp path] (7.807,3.245)--(7.807,3.425);
\draw[gp path] (7.811,3.245)--(7.811,3.425);
\draw[gp path] (8.004,3.202)--(8.007,3.202);
\draw[gp path] (8.004,3.112)--(8.004,3.292);
\draw[gp path] (8.007,3.112)--(8.007,3.292);
\draw[gp path] (8.214,3.097)--(8.218,3.097);
\draw[gp path] (8.214,3.007)--(8.214,3.187);
\draw[gp path] (8.218,3.007)--(8.218,3.187);
\draw[gp path] (8.411,2.920)--(8.414,2.920);
\draw[gp path] (8.411,2.830)--(8.411,3.010);
\draw[gp path] (8.414,2.830)--(8.414,3.010);
\draw[gp path] (8.649,2.843)--(8.652,2.843);
\draw[gp path] (8.649,2.753)--(8.649,2.933);
\draw[gp path] (8.652,2.753)--(8.652,2.933);
\draw[gp path] (8.870,2.667)--(8.873,2.667);
\draw[gp path] (8.870,2.577)--(8.870,2.757);
\draw[gp path] (8.873,2.577)--(8.873,2.757);
\draw[gp path] (9.046,2.590)--(9.049,2.590);
\draw[gp path] (9.046,2.500)--(9.046,2.680);
\draw[gp path] (9.049,2.500)--(9.049,2.680);
\draw[gp path] (9.284,2.478)--(9.287,2.478);
\draw[gp path] (9.284,2.388)--(9.284,2.568);
\draw[gp path] (9.287,2.388)--(9.287,2.568);
\draw[gp path] (9.512,2.365)--(9.515,2.365);
\draw[gp path] (9.512,2.275)--(9.512,2.455);
\draw[gp path] (9.515,2.275)--(9.515,2.455);
\draw[gp path] (9.739,2.231)--(9.743,2.231);
\draw[gp path] (9.739,2.141)--(9.739,2.321);
\draw[gp path] (9.743,2.141)--(9.743,2.321);
\draw[gp path] (9.922,2.186)--(9.926,2.186);
\draw[gp path] (9.922,2.096)--(9.922,2.276);
\draw[gp path] (9.926,2.096)--(9.926,2.276);
\draw[gp path] (10.143,2.030)--(10.146,2.030);
\draw[gp path] (10.143,1.940)--(10.143,2.120);
\draw[gp path] (10.146,1.940)--(10.146,2.120);
\draw[gp path] (10.322,1.973)--(10.326,1.973);
\draw[gp path] (10.322,1.883)--(10.322,2.063);
\draw[gp path] (10.326,1.883)--(10.326,2.063);
\draw[gp path] (10.540,1.815)--(10.543,1.815);
\draw[gp path] (10.540,1.725)--(10.540,1.905);
\draw[gp path] (10.543,1.725)--(10.543,1.905);
\draw[gp path] (10.778,1.757)--(10.781,1.757);
\draw[gp path] (10.778,1.667)--(10.778,1.847);
\draw[gp path] (10.781,1.667)--(10.781,1.847);
\draw[gp path] (11.009,1.697)--(11.013,1.697);
\draw[gp path] (11.009,1.607)--(11.009,1.787);
\draw[gp path] (11.013,1.607)--(11.013,1.787);
\draw[gp path] (11.240,1.573)--(11.244,1.573);
\draw[gp path] (11.240,1.483)--(11.240,1.663);
\draw[gp path] (11.244,1.483)--(11.244,1.663);
\draw[gp path] (11.423,1.497)--(11.427,1.497);
\draw[gp path] (11.423,1.407)--(11.423,1.587);
\draw[gp path] (11.427,1.407)--(11.427,1.587);
\draw[gp path] (11.644,1.451)--(11.647,1.451);
\draw[gp path] (11.644,1.361)--(11.644,1.541);
\draw[gp path] (11.647,1.361)--(11.647,1.541);
\draw[gp path] (11.865,1.389)--(11.868,1.389);
\draw[gp path] (11.865,1.299)--(11.865,1.479);
\draw[gp path] (11.868,1.299)--(11.868,1.479);
\gpsetpointsize{4.00}
\gppoint{gp mark 1}{(3.413,7.790)}
\gppoint{gp mark 1}{(3.634,7.462)}
\gppoint{gp mark 1}{(3.876,7.204)}
\gppoint{gp mark 1}{(4.051,6.985)}
\gppoint{gp mark 1}{(4.238,6.673)}
\gppoint{gp mark 1}{(4.479,6.471)}
\gppoint{gp mark 1}{(4.693,6.169)}
\gppoint{gp mark 1}{(4.911,5.950)}
\gppoint{gp mark 1}{(5.118,5.697)}
\gppoint{gp mark 1}{(5.304,5.460)}
\gppoint{gp mark 1}{(5.532,5.289)}
\gppoint{gp mark 1}{(5.739,5.062)}
\gppoint{gp mark 1}{(5.946,4.904)}
\gppoint{gp mark 1}{(6.146,4.693)}
\gppoint{gp mark 1}{(6.342,4.527)}
\gppoint{gp mark 1}{(6.563,4.301)}
\gppoint{gp mark 1}{(6.791,4.109)}
\gppoint{gp mark 1}{(7.005,3.994)}
\gppoint{gp mark 1}{(7.212,3.784)}
\gppoint{gp mark 1}{(7.405,3.655)}
\gppoint{gp mark 1}{(7.609,3.497)}
\gppoint{gp mark 1}{(7.809,3.335)}
\gppoint{gp mark 1}{(8.006,3.202)}
\gppoint{gp mark 1}{(8.216,3.097)}
\gppoint{gp mark 1}{(8.413,2.920)}
\gppoint{gp mark 1}{(8.651,2.843)}
\gppoint{gp mark 1}{(8.872,2.667)}
\gppoint{gp mark 1}{(9.048,2.590)}
\gppoint{gp mark 1}{(9.286,2.478)}
\gppoint{gp mark 1}{(9.513,2.365)}
\gppoint{gp mark 1}{(9.741,2.231)}
\gppoint{gp mark 1}{(9.924,2.186)}
\gppoint{gp mark 1}{(10.145,2.030)}
\gppoint{gp mark 1}{(10.324,1.973)}
\gppoint{gp mark 1}{(10.542,1.815)}
\gppoint{gp mark 1}{(10.780,1.757)}
\gppoint{gp mark 1}{(11.011,1.697)}
\gppoint{gp mark 1}{(11.242,1.573)}
\gppoint{gp mark 1}{(11.425,1.497)}
\gppoint{gp mark 1}{(11.646,1.451)}
\gppoint{gp mark 1}{(11.866,1.389)}
\gppoint{gp mark 1}{(11.213,8.047)}
\gpcolor{gp lt color border}
\node[gp node right] at (10.571,7.739) {fit};
\gpsetlinetype{gp lt border}
\draw[gp path] (10.755,7.739)--(11.671,7.739);
\draw[gp path] (3.411,7.820)--(3.497,7.699)--(3.582,7.579)--(3.668,7.461)--(3.753,7.345)%
  --(3.839,7.230)--(3.924,7.117)--(4.009,7.005)--(4.095,6.895)--(4.180,6.787)--(4.266,6.680)%
  --(4.351,6.575)--(4.437,6.471)--(4.522,6.369)--(4.607,6.268)--(4.693,6.169)--(4.778,6.071)%
  --(4.864,5.974)--(4.949,5.879)--(5.034,5.785)--(5.120,5.693)--(5.205,5.601)--(5.291,5.512)%
  --(5.376,5.423)--(5.462,5.336)--(5.547,5.249)--(5.632,5.165)--(5.718,5.081)--(5.803,4.998)%
  --(5.889,4.917)--(5.974,4.837)--(6.060,4.758)--(6.145,4.680)--(6.230,4.603)--(6.316,4.528)%
  --(6.401,4.453)--(6.487,4.379)--(6.572,4.307)--(6.657,4.235)--(6.743,4.165)--(6.828,4.096)%
  --(6.914,4.027)--(6.999,3.960)--(7.085,3.893)--(7.170,3.828)--(7.255,3.763)--(7.341,3.699)%
  --(7.426,3.636)--(7.512,3.574)--(7.597,3.513)--(7.683,3.453)--(7.768,3.394)--(7.853,3.336)%
  --(7.939,3.278)--(8.024,3.221)--(8.110,3.165)--(8.195,3.110)--(8.280,3.055)--(8.366,3.002)%
  --(8.451,2.949)--(8.537,2.897)--(8.622,2.845)--(8.708,2.795)--(8.793,2.745)--(8.878,2.696)%
  --(8.964,2.647)--(9.049,2.599)--(9.135,2.552)--(9.220,2.506)--(9.306,2.460)--(9.391,2.415)%
  --(9.476,2.370)--(9.562,2.326)--(9.647,2.283)--(9.733,2.240)--(9.818,2.198)--(9.904,2.157)%
  --(9.989,2.116)--(10.074,2.076)--(10.160,2.036)--(10.245,1.997)--(10.331,1.958)--(10.416,1.920)%
  --(10.501,1.883)--(10.587,1.846)--(10.672,1.810)--(10.758,1.774)--(10.843,1.738)--(10.929,1.703)%
  --(11.014,1.669)--(11.099,1.635)--(11.185,1.602)--(11.270,1.569)--(11.356,1.536)--(11.441,1.504)%
  --(11.527,1.473)--(11.612,1.442)--(11.697,1.411)--(11.783,1.381)--(11.868,1.351);
\draw[gp path] (1.688,8.381)--(1.688,0.985)--(12.039,0.985)--(12.039,8.381)--cycle;
%% coordinates of the plot area
\gpdefrectangularnode{gp plot 1}{\pgfpoint{1.688cm}{0.985cm}}{\pgfpoint{12.039cm}{8.381cm}}
\end{tikzpicture}
%% gnuplot variables

}
	\caption{Diffraction intensity $\eta$ (left) and change in refractive index of the crystal (right) versus erasing time}
	\label{fig:ana_erasing}
\end{floatrow}
\end{figure}
This time we fit a function of the form  $\Delta n(t) = \Delta n(t=\infty) \cdotp e^{-\frac{t}{\tau_\textrm{e}}} + \textrm{const.}$ to the graph. We get:
\begin{align}
\Delta n(t=\infty) &= \unit[(8.07 \pm 0.05)]{}\\
\notag \tau_\textrm{e} &= \unit[(1726 \pm 23)]{s}\\
\notag \textrm{const.} &= \unit[(97.40 \pm 0.06)]{}\\
\end{align}

\subsubsection{Measuring a rocking curve}
\label{subsubsec:ana_rocking}
In the last part of the experiment we want to measure the dependency of the diffraction effiency $\eta$ on the angle of incidence $\theta$ of the reading beam. This angle can be changed by rotating the \linbo crystal by using a micrometer screw at the rotation stage on which the crystal is mounted. In order to gauge the micrometer screw we take a gauge curve by measuring the rotation angle $\alpha$ of the crystal when rotating the micrometer screw. The origin of the scale of the screw is defined arbritary. The data is shown in table \ref{tab:ana_gauge} and plotted in figure \ref{fig:ana_gauge}, which also displays a linear fit to the graph. The fit result reads:
\begin{align}
\alpha(x) = \unit[(178.0 \pm 0.3)]{^\circ} + \unit[(1.9 \pm 0.2)]{\frac{^\circ}{mm}}\cdotp x
\end{align}
As one can see in figure \ref{fig:ana_gauge} the measured values differ strongly from the linear fit. This may be caused by the idling of the screw.
\begin{figure}[H]
\begin{floatrow}
\floatbox[\nocapbeside]{table}[\FBwidth]{}
{
  \centering
\resizebox{0.5\textwidth}{!}{
  \begin{tabular}{lc}
    \toprule
	relative screw position [$\unit[]{mm}$] & Angle $\alpha$ [$\unit[]{^\circ}$]\\
    \midrule[0.75pt]
$0.0 \pm 0.01$	&$178.3 \pm 0.05$\\
$0.5 \pm 0.01$	&$178.7 \pm 0.05$\\
$1.0 \pm 0.01$	&$179.6 \pm 0.05$\\
$1.5 \pm 0.01$	&$181.0 \pm 0.05$\\
$2.0 \pm 0.01$	&$182.0 \pm 0.05$\\
    \bottomrule
  \end{tabular}
}
\caption{Gauge of the micrometer screw}
  \label{tab:ana_gauge}
}
\ffigbox{}
{
\resizebox{0.5\textwidth}{!}{
	\begin{tikzpicture}[gnuplot]
%% generated with GNUPLOT 4.4p0 (Lua 5.1.4; terminal rev. 97, script rev. 96a)
%% 10.05.2010 01:43:53
\gpcolor{gp lt color border}
\gpsetlinetype{gp lt border}
\gpsetlinewidth{1.00}
\draw[gp path] (1.688,0.985)--(1.868,0.985);
\draw[gp path] (12.039,0.985)--(11.859,0.985);
\node[gp node right] at (1.504,0.985) { 177};
\draw[gp path] (1.688,2.218)--(1.868,2.218);
\draw[gp path] (12.039,2.218)--(11.859,2.218);
\node[gp node right] at (1.504,2.218) { 178};
\draw[gp path] (1.688,3.450)--(1.868,3.450);
\draw[gp path] (12.039,3.450)--(11.859,3.450);
\node[gp node right] at (1.504,3.450) { 179};
\draw[gp path] (1.688,4.683)--(1.868,4.683);
\draw[gp path] (12.039,4.683)--(11.859,4.683);
\node[gp node right] at (1.504,4.683) { 180};
\draw[gp path] (1.688,5.916)--(1.868,5.916);
\draw[gp path] (12.039,5.916)--(11.859,5.916);
\node[gp node right] at (1.504,5.916) { 181};
\draw[gp path] (1.688,7.148)--(1.868,7.148);
\draw[gp path] (12.039,7.148)--(11.859,7.148);
\node[gp node right] at (1.504,7.148) { 182};
\draw[gp path] (1.688,8.381)--(1.868,8.381);
\draw[gp path] (12.039,8.381)--(11.859,8.381);
\node[gp node right] at (1.504,8.381) { 183};
\draw[gp path] (1.688,0.985)--(1.688,1.165);
\draw[gp path] (1.688,8.381)--(1.688,8.201);
\node[gp node center] at (1.688,0.677) {-0.5};
\draw[gp path] (3.413,0.985)--(3.413,1.165);
\draw[gp path] (3.413,8.381)--(3.413,8.201);
\node[gp node center] at (3.413,0.677) { 0};
\draw[gp path] (5.138,0.985)--(5.138,1.165);
\draw[gp path] (5.138,8.381)--(5.138,8.201);
\node[gp node center] at (5.138,0.677) { 0.5};
\draw[gp path] (6.864,0.985)--(6.864,1.165);
\draw[gp path] (6.864,8.381)--(6.864,8.201);
\node[gp node center] at (6.864,0.677) { 1};
\draw[gp path] (8.589,0.985)--(8.589,1.165);
\draw[gp path] (8.589,8.381)--(8.589,8.201);
\node[gp node center] at (8.589,0.677) { 1.5};
\draw[gp path] (10.314,0.985)--(10.314,1.165);
\draw[gp path] (10.314,8.381)--(10.314,8.201);
\node[gp node center] at (10.314,0.677) { 2};
\draw[gp path] (12.039,0.985)--(12.039,1.165);
\draw[gp path] (12.039,8.381)--(12.039,8.201);
\node[gp node center] at (12.039,0.677) { 2.5};
\draw[gp path] (1.688,8.381)--(1.688,0.985)--(12.039,0.985)--(12.039,8.381)--cycle;
\node[gp node center,rotate=-270] at (0.430,4.683) {Angle [�]};
\node[gp node center] at (6.863,0.215) {relative micrometer screw position [mm]};
\node[gp node right] at (10.571,8.047) {Measured Data};
\gpcolor{gp lt color 0}
\gpsetlinetype{gp lt plot 0}
\draw[gp path] (10.755,8.047)--(11.671,8.047);
\draw[gp path] (10.755,8.137)--(10.755,7.957);
\draw[gp path] (11.671,8.137)--(11.671,7.957);
\draw[gp path] (3.413,2.540)--(3.413,2.663);
\draw[gp path] (3.323,2.540)--(3.503,2.540);
\draw[gp path] (3.323,2.663)--(3.503,2.663);
\draw[gp path] (5.138,2.954)--(5.138,3.077);
\draw[gp path] (5.048,2.954)--(5.228,2.954);
\draw[gp path] (5.048,3.077)--(5.228,3.077);
\draw[gp path] (6.864,4.128)--(6.864,4.251);
\draw[gp path] (6.774,4.128)--(6.954,4.128);
\draw[gp path] (6.774,4.251)--(6.954,4.251);
\draw[gp path] (8.589,5.873)--(8.589,5.996);
\draw[gp path] (8.499,5.873)--(8.679,5.873);
\draw[gp path] (8.499,5.996)--(8.679,5.996);
\draw[gp path] (10.314,7.056)--(10.314,7.180);
\draw[gp path] (10.224,7.056)--(10.404,7.056);
\draw[gp path] (10.224,7.180)--(10.404,7.180);
\draw[gp path] (3.379,2.601)--(3.448,2.601);
\draw[gp path] (3.379,2.511)--(3.379,2.691);
\draw[gp path] (3.448,2.511)--(3.448,2.691);
\draw[gp path] (5.104,3.016)--(5.173,3.016);
\draw[gp path] (5.104,2.926)--(5.104,3.106);
\draw[gp path] (5.173,2.926)--(5.173,3.106);
\draw[gp path] (6.829,4.189)--(6.898,4.189);
\draw[gp path] (6.829,4.099)--(6.829,4.279);
\draw[gp path] (6.898,4.099)--(6.898,4.279);
\draw[gp path] (8.554,5.935)--(8.623,5.935);
\draw[gp path] (8.554,5.845)--(8.554,6.025);
\draw[gp path] (8.623,5.845)--(8.623,6.025);
\draw[gp path] (10.279,7.118)--(10.348,7.118);
\draw[gp path] (10.279,7.028)--(10.279,7.208);
\draw[gp path] (10.348,7.028)--(10.348,7.208);
\gpsetpointsize{4.00}
\gppoint{gp mark 1}{(3.413,2.601)}
\gppoint{gp mark 1}{(5.138,3.016)}
\gppoint{gp mark 1}{(6.864,4.189)}
\gppoint{gp mark 1}{(8.589,5.935)}
\gppoint{gp mark 1}{(10.314,7.118)}
\gppoint{gp mark 1}{(11.213,8.047)}
\gpcolor{gp lt color border}
\node[gp node right] at (10.571,7.739) {fit};
\gpsetlinetype{gp lt border}
\draw[gp path] (10.755,7.739)--(11.671,7.739);
\draw[gp path] (1.688,0.986)--(1.793,1.058)--(1.897,1.131)--(2.002,1.203)--(2.106,1.276)%
  --(2.211,1.348)--(2.315,1.421)--(2.420,1.493)--(2.524,1.566)--(2.629,1.638)--(2.734,1.710)%
  --(2.838,1.783)--(2.943,1.855)--(3.047,1.928)--(3.152,2.000)--(3.256,2.073)--(3.361,2.145)%
  --(3.465,2.217)--(3.570,2.290)--(3.675,2.362)--(3.779,2.435)--(3.884,2.507)--(3.988,2.580)%
  --(4.093,2.652)--(4.197,2.725)--(4.302,2.797)--(4.406,2.869)--(4.511,2.942)--(4.616,3.014)%
  --(4.720,3.087)--(4.825,3.159)--(4.929,3.232)--(5.034,3.304)--(5.138,3.376)--(5.243,3.449)%
  --(5.347,3.521)--(5.452,3.594)--(5.557,3.666)--(5.661,3.739)--(5.766,3.811)--(5.870,3.884)%
  --(5.975,3.956)--(6.079,4.028)--(6.184,4.101)--(6.288,4.173)--(6.393,4.246)--(6.498,4.318)%
  --(6.602,4.391)--(6.707,4.463)--(6.811,4.535)--(6.916,4.608)--(7.020,4.680)--(7.125,4.753)%
  --(7.229,4.825)--(7.334,4.898)--(7.439,4.970)--(7.543,5.043)--(7.648,5.115)--(7.752,5.187)%
  --(7.857,5.260)--(7.961,5.332)--(8.066,5.405)--(8.170,5.477)--(8.275,5.550)--(8.380,5.622)%
  --(8.484,5.695)--(8.589,5.767)--(8.693,5.839)--(8.798,5.912)--(8.902,5.984)--(9.007,6.057)%
  --(9.111,6.129)--(9.216,6.202)--(9.321,6.274)--(9.425,6.346)--(9.530,6.419)--(9.634,6.491)%
  --(9.739,6.564)--(9.843,6.636)--(9.948,6.709)--(10.052,6.781)--(10.157,6.854)--(10.262,6.926)%
  --(10.366,6.998)--(10.471,7.071)--(10.575,7.143)--(10.680,7.216)--(10.784,7.288)--(10.889,7.361)%
  --(10.993,7.433)--(11.098,7.505)--(11.203,7.578)--(11.307,7.650)--(11.412,7.723)--(11.516,7.795)%
  --(11.621,7.868)--(11.725,7.940)--(11.830,8.013)--(11.934,8.085)--(12.039,8.157);
\draw[gp path] (1.688,8.381)--(1.688,0.985)--(12.039,0.985)--(12.039,8.381)--cycle;
%% coordinates of the plot area
\gpdefrectangularnode{gp plot 1}{\pgfpoint{1.688cm}{0.985cm}}{\pgfpoint{12.039cm}{8.381cm}}
\end{tikzpicture}
%% gnuplot variables

}
	\caption{Gauge of the micrometer screw}
	\label{fig:ana_gauge}
}
\end{floatrow}
\end{figure}
With this gauge we are able to measure the rocking curve. Therefore we erase the crystal by illuminating it with the tungsten lamp for about 15 minutes and then rewrite a hologram as described in section \ref{subsubsec:ana_writing}. Due to the lack of time we just write the hologram for 30 instead of 40 minutes. After the recording is finished we attenuate the reading beam in order to minimize the erasing of the hologram and then measure the diffracted and the transmitted intensity for different rotation angles $\alpha$ of the crystal and therefore different angles of incidence $\theta$ of the reading beam. For $\alpha = \unit[180]{^\circ}$ the value of $\theta$ given in section \ref{subsubsec:ana_writing} holds.
Our data is presented in table \ref{tab:ana_rocking} in the appendix. In figure \ref{fig:ana_rocking} the diffraction effiency $\eta$ is plotted versus $\theta$. We fit a function of the form XXX (reference to equation in theory part) to the data. The fit yields:
\begin{align}
\theta_\textrm{Bragg} &= \unit[(8.5505 \pm 0.0007)]{^\circ}\\
\notag \Lambda &= \unit[(1.03 \pm 0.03)]{\mu m}\\
\notag \Delta n_0 &= \unit[(1.23 \pm  0.02)\cdotp 10^{-5}]{} 
\end{align}
\begin{figure}
\resizebox{0.8\textwidth}{!}{
	\begin{tikzpicture}[gnuplot]
%% generated with GNUPLOT 4.4p0 (Lua 5.1.4; terminal rev. 97, script rev. 96a)
%% 10.05.2010 12:18:17
\gpcolor{gp lt color border}
\gpsetlinetype{gp lt border}
\gpsetlinewidth{1.00}
\draw[gp path] (1.688,0.985)--(1.868,0.985);
\draw[gp path] (12.039,0.985)--(11.859,0.985);
\node[gp node right] at (1.504,0.985) { 0};
\draw[gp path] (1.688,1.807)--(1.868,1.807);
\draw[gp path] (12.039,1.807)--(11.859,1.807);
\node[gp node right] at (1.504,1.807) { 0.5};
\draw[gp path] (1.688,2.629)--(1.868,2.629);
\draw[gp path] (12.039,2.629)--(11.859,2.629);
\node[gp node right] at (1.504,2.629) { 1};
\draw[gp path] (1.688,3.450)--(1.868,3.450);
\draw[gp path] (12.039,3.450)--(11.859,3.450);
\node[gp node right] at (1.504,3.450) { 1.5};
\draw[gp path] (1.688,4.272)--(1.868,4.272);
\draw[gp path] (12.039,4.272)--(11.859,4.272);
\node[gp node right] at (1.504,4.272) { 2};
\draw[gp path] (1.688,5.094)--(1.868,5.094);
\draw[gp path] (12.039,5.094)--(11.859,5.094);
\node[gp node right] at (1.504,5.094) { 2.5};
\draw[gp path] (1.688,5.916)--(1.868,5.916);
\draw[gp path] (12.039,5.916)--(11.859,5.916);
\node[gp node right] at (1.504,5.916) { 3};
\draw[gp path] (1.688,6.737)--(1.868,6.737);
\draw[gp path] (12.039,6.737)--(11.859,6.737);
\node[gp node right] at (1.504,6.737) { 3.5};
\draw[gp path] (1.688,7.559)--(1.868,7.559);
\draw[gp path] (12.039,7.559)--(11.859,7.559);
\node[gp node right] at (1.504,7.559) { 4};
\draw[gp path] (1.688,8.381)--(1.868,8.381);
\draw[gp path] (12.039,8.381)--(11.859,8.381);
\node[gp node right] at (1.504,8.381) { 4.5};
\draw[gp path] (1.688,0.985)--(1.688,1.165);
\draw[gp path] (1.688,8.381)--(1.688,8.201);
\node[gp node center] at (1.688,0.677) { 8.35};
\draw[gp path] (2.982,0.985)--(2.982,1.165);
\draw[gp path] (2.982,8.381)--(2.982,8.201);
\node[gp node center] at (2.982,0.677) { 8.4};
\draw[gp path] (4.276,0.985)--(4.276,1.165);
\draw[gp path] (4.276,8.381)--(4.276,8.201);
\node[gp node center] at (4.276,0.677) { 8.45};
\draw[gp path] (5.570,0.985)--(5.570,1.165);
\draw[gp path] (5.570,8.381)--(5.570,8.201);
\node[gp node center] at (5.570,0.677) { 8.5};
\draw[gp path] (6.864,0.985)--(6.864,1.165);
\draw[gp path] (6.864,8.381)--(6.864,8.201);
\node[gp node center] at (6.864,0.677) { 8.55};
\draw[gp path] (8.157,0.985)--(8.157,1.165);
\draw[gp path] (8.157,8.381)--(8.157,8.201);
\node[gp node center] at (8.157,0.677) { 8.6};
\draw[gp path] (9.451,0.985)--(9.451,1.165);
\draw[gp path] (9.451,8.381)--(9.451,8.201);
\node[gp node center] at (9.451,0.677) { 8.65};
\draw[gp path] (10.745,0.985)--(10.745,1.165);
\draw[gp path] (10.745,8.381)--(10.745,8.201);
\node[gp node center] at (10.745,0.677) { 8.7};
\draw[gp path] (12.039,0.985)--(12.039,1.165);
\draw[gp path] (12.039,8.381)--(12.039,8.201);
\node[gp node center] at (12.039,0.677) { 8.75};
\draw[gp path] (1.688,8.381)--(1.688,0.985)--(12.039,0.985)--(12.039,8.381)--cycle;
\node[gp node center,rotate=-270] at (0.430,4.683) {$\eta$ [$\unit[]{10^{-3}}$]};
\node[gp node center] at (6.863,0.215) {$\theta$ [$\unit[]{^\circ}$]};
\node[gp node right] at (10.571,8.047) {Measured Data};
\gpcolor{gp lt color 0}
\gpsetpointsize{4.00}
\gppoint{gp mark 1}{(1.695,0.994)}
\gppoint{gp mark 1}{(1.904,0.985)}
\gppoint{gp mark 1}{(2.113,0.994)}
\gppoint{gp mark 1}{(2.322,1.041)}
\gppoint{gp mark 1}{(2.530,1.008)}
\gppoint{gp mark 1}{(2.739,0.998)}
\gppoint{gp mark 1}{(2.948,0.985)}
\gppoint{gp mark 1}{(3.157,1.020)}
\gppoint{gp mark 1}{(3.365,1.081)}
\gppoint{gp mark 1}{(3.574,1.155)}
\gppoint{gp mark 1}{(3.783,1.056)}
\gppoint{gp mark 1}{(3.991,1.000)}
\gppoint{gp mark 1}{(4.200,0.988)}
\gppoint{gp mark 1}{(4.409,1.095)}
\gppoint{gp mark 1}{(4.617,1.150)}
\gppoint{gp mark 1}{(4.826,1.365)}
\gppoint{gp mark 1}{(5.035,1.190)}
\gppoint{gp mark 1}{(5.243,1.085)}
\gppoint{gp mark 1}{(5.452,0.987)}
\gppoint{gp mark 1}{(5.660,1.183)}
\gppoint{gp mark 1}{(5.869,1.523)}
\gppoint{gp mark 1}{(6.077,3.474)}
\gppoint{gp mark 1}{(6.285,5.388)}
\gppoint{gp mark 1}{(6.494,5.539)}
\gppoint{gp mark 1}{(6.702,7.753)}
\gppoint{gp mark 1}{(6.911,6.840)}
\gppoint{gp mark 1}{(7.119,5.552)}
\gppoint{gp mark 1}{(7.327,6.184)}
\gppoint{gp mark 1}{(7.536,5.080)}
\gppoint{gp mark 1}{(7.744,2.596)}
\gppoint{gp mark 1}{(7.952,1.546)}
\gppoint{gp mark 1}{(8.160,1.166)}
\gppoint{gp mark 1}{(8.369,0.987)}
\gppoint{gp mark 1}{(8.577,1.096)}
\gppoint{gp mark 1}{(8.785,1.197)}
\gppoint{gp mark 1}{(8.993,1.186)}
\gppoint{gp mark 1}{(9.201,1.231)}
\gppoint{gp mark 1}{(9.409,1.073)}
\gppoint{gp mark 1}{(9.618,0.990)}
\gppoint{gp mark 1}{(9.826,1.007)}
\gppoint{gp mark 1}{(10.034,1.031)}
\gppoint{gp mark 1}{(10.242,1.076)}
\gppoint{gp mark 1}{(10.450,1.047)}
\gppoint{gp mark 1}{(10.658,1.023)}
\gppoint{gp mark 1}{(10.866,0.985)}
\gppoint{gp mark 1}{(11.074,1.008)}
\gppoint{gp mark 1}{(11.282,1.046)}
\gppoint{gp mark 1}{(11.489,1.048)}
\gppoint{gp mark 1}{(11.697,0.994)}
\gppoint{gp mark 1}{(11.905,0.985)}
\gppoint{gp mark 1}{(11.213,8.047)}
\gpcolor{gp lt color border}
\node[gp node right] at (10.571,7.739) {fit};
\draw[gp path] (10.755,7.739)--(11.671,7.739);
\draw[gp path] (1.688,1.033)--(1.690,1.033)--(1.692,1.032)--(1.694,1.032)--(1.696,1.032)%
  --(1.698,1.032)--(1.700,1.032)--(1.702,1.032)--(1.705,1.032)--(1.707,1.031)--(1.709,1.031)%
  --(1.711,1.031)--(1.713,1.031)--(1.715,1.031)--(1.717,1.031)--(1.719,1.031)--(1.721,1.030)%
  --(1.723,1.030)--(1.725,1.030)--(1.727,1.030)--(1.729,1.030)--(1.731,1.030)--(1.734,1.030)%
  --(1.736,1.029)--(1.738,1.029)--(1.740,1.029)--(1.742,1.029)--(1.744,1.029)--(1.746,1.029)%
  --(1.748,1.028)--(1.750,1.028)--(1.752,1.028)--(1.754,1.028)--(1.756,1.028)--(1.758,1.028)%
  --(1.760,1.027)--(1.763,1.027)--(1.765,1.027)--(1.767,1.027)--(1.769,1.027)--(1.771,1.027)%
  --(1.773,1.026)--(1.775,1.026)--(1.777,1.026)--(1.779,1.026)--(1.781,1.026)--(1.783,1.025)%
  --(1.785,1.025)--(1.787,1.025)--(1.789,1.025)--(1.792,1.025)--(1.794,1.024)--(1.796,1.024)%
  --(1.798,1.024)--(1.800,1.024)--(1.802,1.024)--(1.804,1.023)--(1.806,1.023)--(1.808,1.023)%
  --(1.810,1.023)--(1.812,1.023)--(1.814,1.022)--(1.816,1.022)--(1.818,1.022)--(1.821,1.022)%
  --(1.823,1.022)--(1.825,1.021)--(1.827,1.021)--(1.829,1.021)--(1.831,1.021)--(1.833,1.021)%
  --(1.835,1.020)--(1.837,1.020)--(1.839,1.020)--(1.841,1.020)--(1.843,1.019)--(1.845,1.019)%
  --(1.847,1.019)--(1.850,1.019)--(1.852,1.019)--(1.854,1.018)--(1.856,1.018)--(1.858,1.018)%
  --(1.860,1.018)--(1.862,1.018)--(1.864,1.017)--(1.866,1.017)--(1.868,1.017)--(1.870,1.017)%
  --(1.872,1.016)--(1.874,1.016)--(1.876,1.016)--(1.878,1.016)--(1.881,1.015)--(1.883,1.015)%
  --(1.885,1.015)--(1.887,1.015)--(1.889,1.015)--(1.891,1.014)--(1.893,1.014)--(1.895,1.014)%
  --(1.897,1.014)--(1.899,1.013)--(1.901,1.013)--(1.903,1.013)--(1.905,1.013)--(1.907,1.013)%
  --(1.910,1.012)--(1.912,1.012)--(1.914,1.012)--(1.916,1.012)--(1.918,1.011)--(1.920,1.011)%
  --(1.922,1.011)--(1.924,1.011)--(1.926,1.010)--(1.928,1.010)--(1.930,1.010)--(1.932,1.010)%
  --(1.934,1.009)--(1.936,1.009)--(1.939,1.009)--(1.941,1.009)--(1.943,1.009)--(1.945,1.008)%
  --(1.947,1.008)--(1.949,1.008)--(1.951,1.008)--(1.953,1.007)--(1.955,1.007)--(1.957,1.007)%
  --(1.959,1.007)--(1.961,1.006)--(1.963,1.006)--(1.965,1.006)--(1.968,1.006)--(1.970,1.006)%
  --(1.972,1.005)--(1.974,1.005)--(1.976,1.005)--(1.978,1.005)--(1.980,1.004)--(1.982,1.004)%
  --(1.984,1.004)--(1.986,1.004)--(1.988,1.004)--(1.990,1.003)--(1.992,1.003)--(1.994,1.003)%
  --(1.997,1.003)--(1.999,1.002)--(2.001,1.002)--(2.003,1.002)--(2.005,1.002)--(2.007,1.001)%
  --(2.009,1.001)--(2.011,1.001)--(2.013,1.001)--(2.015,1.001)--(2.017,1.000)--(2.019,1.000)%
  --(2.021,1.000)--(2.023,1.000)--(2.026,1.000)--(2.028,0.999)--(2.030,0.999)--(2.032,0.999)%
  --(2.034,0.999)--(2.036,0.998)--(2.038,0.998)--(2.040,0.998)--(2.042,0.998)--(2.044,0.998)%
  --(2.046,0.997)--(2.048,0.997)--(2.050,0.997)--(2.052,0.997)--(2.054,0.997)--(2.057,0.996)%
  --(2.059,0.996)--(2.061,0.996)--(2.063,0.996)--(2.065,0.996)--(2.067,0.995)--(2.069,0.995)%
  --(2.071,0.995)--(2.073,0.995)--(2.075,0.995)--(2.077,0.995)--(2.079,0.994)--(2.081,0.994)%
  --(2.083,0.994)--(2.086,0.994)--(2.088,0.994)--(2.090,0.993)--(2.092,0.993)--(2.094,0.993)%
  --(2.096,0.993)--(2.098,0.993)--(2.100,0.993)--(2.102,0.992)--(2.104,0.992)--(2.106,0.992)%
  --(2.108,0.992)--(2.110,0.992)--(2.112,0.992)--(2.115,0.991)--(2.117,0.991)--(2.119,0.991)%
  --(2.121,0.991)--(2.123,0.991)--(2.125,0.991)--(2.127,0.990)--(2.129,0.990)--(2.131,0.990)%
  --(2.133,0.990)--(2.135,0.990)--(2.137,0.990)--(2.139,0.990)--(2.141,0.989)--(2.144,0.989)%
  --(2.146,0.989)--(2.148,0.989)--(2.150,0.989)--(2.152,0.989)--(2.154,0.989)--(2.156,0.989)%
  --(2.158,0.988)--(2.160,0.988)--(2.162,0.988)--(2.164,0.988)--(2.166,0.988)--(2.168,0.988)%
  --(2.170,0.988)--(2.173,0.988)--(2.175,0.988)--(2.177,0.987)--(2.179,0.987)--(2.181,0.987)%
  --(2.183,0.987)--(2.185,0.987)--(2.187,0.987)--(2.189,0.987)--(2.191,0.987)--(2.193,0.987)%
  --(2.195,0.987)--(2.197,0.986)--(2.199,0.986)--(2.202,0.986)--(2.204,0.986)--(2.206,0.986)%
  --(2.208,0.986)--(2.210,0.986)--(2.212,0.986)--(2.214,0.986)--(2.216,0.986)--(2.218,0.986)%
  --(2.220,0.986)--(2.222,0.986)--(2.224,0.986)--(2.226,0.986)--(2.228,0.985)--(2.231,0.985)%
  --(2.233,0.985)--(2.235,0.985)--(2.237,0.985)--(2.239,0.985)--(2.241,0.985)--(2.243,0.985)%
  --(2.245,0.985)--(2.247,0.985)--(2.249,0.985)--(2.251,0.985)--(2.253,0.985)--(2.255,0.985)%
  --(2.257,0.985)--(2.259,0.985)--(2.262,0.985)--(2.264,0.985)--(2.266,0.985)--(2.268,0.985)%
  --(2.270,0.985)--(2.272,0.985)--(2.274,0.985)--(2.276,0.985)--(2.278,0.985)--(2.280,0.985)%
  --(2.282,0.985)--(2.284,0.985)--(2.286,0.985)--(2.288,0.985)--(2.291,0.985)--(2.293,0.985)%
  --(2.295,0.985)--(2.297,0.985)--(2.299,0.985)--(2.301,0.985)--(2.303,0.985)--(2.305,0.985)%
  --(2.307,0.985)--(2.309,0.985)--(2.311,0.986)--(2.313,0.986)--(2.315,0.986)--(2.317,0.986)%
  --(2.320,0.986)--(2.322,0.986)--(2.324,0.986)--(2.326,0.986)--(2.328,0.986)--(2.330,0.986)%
  --(2.332,0.986)--(2.334,0.986)--(2.336,0.986)--(2.338,0.986)--(2.340,0.986)--(2.342,0.987)%
  --(2.344,0.987)--(2.346,0.987)--(2.349,0.987)--(2.351,0.987)--(2.353,0.987)--(2.355,0.987)%
  --(2.357,0.987)--(2.359,0.987)--(2.361,0.987)--(2.363,0.988)--(2.365,0.988)--(2.367,0.988)%
  --(2.369,0.988)--(2.371,0.988)--(2.373,0.988)--(2.375,0.988)--(2.378,0.988)--(2.380,0.989)%
  --(2.382,0.989)--(2.384,0.989)--(2.386,0.989)--(2.388,0.989)--(2.390,0.989)--(2.392,0.990)%
  --(2.394,0.990)--(2.396,0.990)--(2.398,0.990)--(2.400,0.990)--(2.402,0.990)--(2.404,0.990)%
  --(2.407,0.991)--(2.409,0.991)--(2.411,0.991)--(2.413,0.991)--(2.415,0.991)--(2.417,0.992)%
  --(2.419,0.992)--(2.421,0.992)--(2.423,0.992)--(2.425,0.992)--(2.427,0.993)--(2.429,0.993)%
  --(2.431,0.993)--(2.433,0.993)--(2.435,0.993)--(2.438,0.994)--(2.440,0.994)--(2.442,0.994)%
  --(2.444,0.994)--(2.446,0.994)--(2.448,0.995)--(2.450,0.995)--(2.452,0.995)--(2.454,0.995)%
  --(2.456,0.996)--(2.458,0.996)--(2.460,0.996)--(2.462,0.996)--(2.464,0.996)--(2.467,0.997)%
  --(2.469,0.997)--(2.471,0.997)--(2.473,0.997)--(2.475,0.998)--(2.477,0.998)--(2.479,0.998)%
  --(2.481,0.998)--(2.483,0.999)--(2.485,0.999)--(2.487,0.999)--(2.489,1.000)--(2.491,1.000)%
  --(2.493,1.000)--(2.496,1.000)--(2.498,1.001)--(2.500,1.001)--(2.502,1.001)--(2.504,1.001)%
  --(2.506,1.002)--(2.508,1.002)--(2.510,1.002)--(2.512,1.003)--(2.514,1.003)--(2.516,1.003)%
  --(2.518,1.004)--(2.520,1.004)--(2.522,1.004)--(2.525,1.004)--(2.527,1.005)--(2.529,1.005)%
  --(2.531,1.005)--(2.533,1.006)--(2.535,1.006)--(2.537,1.006)--(2.539,1.007)--(2.541,1.007)%
  --(2.543,1.007)--(2.545,1.008)--(2.547,1.008)--(2.549,1.008)--(2.551,1.009)--(2.554,1.009)%
  --(2.556,1.009)--(2.558,1.010)--(2.560,1.010)--(2.562,1.010)--(2.564,1.011)--(2.566,1.011)%
  --(2.568,1.011)--(2.570,1.012)--(2.572,1.012)--(2.574,1.012)--(2.576,1.013)--(2.578,1.013)%
  --(2.580,1.013)--(2.583,1.014)--(2.585,1.014)--(2.587,1.014)--(2.589,1.015)--(2.591,1.015)%
  --(2.593,1.016)--(2.595,1.016)--(2.597,1.016)--(2.599,1.017)--(2.601,1.017)--(2.603,1.017)%
  --(2.605,1.018)--(2.607,1.018)--(2.609,1.018)--(2.611,1.019)--(2.614,1.019)--(2.616,1.020)%
  --(2.618,1.020)--(2.620,1.020)--(2.622,1.021)--(2.624,1.021)--(2.626,1.021)--(2.628,1.022)%
  --(2.630,1.022)--(2.632,1.023)--(2.634,1.023)--(2.636,1.023)--(2.638,1.024)--(2.640,1.024)%
  --(2.643,1.025)--(2.645,1.025)--(2.647,1.025)--(2.649,1.026)--(2.651,1.026)--(2.653,1.027)%
  --(2.655,1.027)--(2.657,1.027)--(2.659,1.028)--(2.661,1.028)--(2.663,1.029)--(2.665,1.029)%
  --(2.667,1.029)--(2.669,1.030)--(2.672,1.030)--(2.674,1.031)--(2.676,1.031)--(2.678,1.031)%
  --(2.680,1.032)--(2.682,1.032)--(2.684,1.033)--(2.686,1.033)--(2.688,1.033)--(2.690,1.034)%
  --(2.692,1.034)--(2.694,1.035)--(2.696,1.035)--(2.698,1.035)--(2.701,1.036)--(2.703,1.036)%
  --(2.705,1.037)--(2.707,1.037)--(2.709,1.037)--(2.711,1.038)--(2.713,1.038)--(2.715,1.039)%
  --(2.717,1.039)--(2.719,1.039)--(2.721,1.040)--(2.723,1.040)--(2.725,1.041)--(2.727,1.041)%
  --(2.730,1.041)--(2.732,1.042)--(2.734,1.042)--(2.736,1.043)--(2.738,1.043)--(2.740,1.043)%
  --(2.742,1.044)--(2.744,1.044)--(2.746,1.045)--(2.748,1.045)--(2.750,1.045)--(2.752,1.046)%
  --(2.754,1.046)--(2.756,1.047)--(2.759,1.047)--(2.761,1.047)--(2.763,1.048)--(2.765,1.048)%
  --(2.767,1.049)--(2.769,1.049)--(2.771,1.049)--(2.773,1.050)--(2.775,1.050)--(2.777,1.051)%
  --(2.779,1.051)--(2.781,1.051)--(2.783,1.052)--(2.785,1.052)--(2.787,1.053)--(2.790,1.053)%
  --(2.792,1.053)--(2.794,1.054)--(2.796,1.054)--(2.798,1.055)--(2.800,1.055)--(2.802,1.055)%
  --(2.804,1.056)--(2.806,1.056)--(2.808,1.056)--(2.810,1.057)--(2.812,1.057)--(2.814,1.058)%
  --(2.816,1.058)--(2.819,1.058)--(2.821,1.059)--(2.823,1.059)--(2.825,1.059)--(2.827,1.060)%
  --(2.829,1.060)--(2.831,1.061)--(2.833,1.061)--(2.835,1.061)--(2.837,1.062)--(2.839,1.062)%
  --(2.841,1.062)--(2.843,1.063)--(2.845,1.063)--(2.848,1.063)--(2.850,1.064)--(2.852,1.064)%
  --(2.854,1.065)--(2.856,1.065)--(2.858,1.065)--(2.860,1.066)--(2.862,1.066)--(2.864,1.066)%
  --(2.866,1.067)--(2.868,1.067)--(2.870,1.067)--(2.872,1.068)--(2.874,1.068)--(2.877,1.068)%
  --(2.879,1.069)--(2.881,1.069)--(2.883,1.069)--(2.885,1.070)--(2.887,1.070)--(2.889,1.070)%
  --(2.891,1.070)--(2.893,1.071)--(2.895,1.071)--(2.897,1.071)--(2.899,1.072)--(2.901,1.072)%
  --(2.903,1.072)--(2.906,1.073)--(2.908,1.073)--(2.910,1.073)--(2.912,1.074)--(2.914,1.074)%
  --(2.916,1.074)--(2.918,1.074)--(2.920,1.075)--(2.922,1.075)--(2.924,1.075)--(2.926,1.075)%
  --(2.928,1.076)--(2.930,1.076)--(2.932,1.076)--(2.935,1.077)--(2.937,1.077)--(2.939,1.077)%
  --(2.941,1.077)--(2.943,1.078)--(2.945,1.078)--(2.947,1.078)--(2.949,1.078)--(2.951,1.079)%
  --(2.953,1.079)--(2.955,1.079)--(2.957,1.079)--(2.959,1.079)--(2.961,1.080)--(2.963,1.080)%
  --(2.966,1.080)--(2.968,1.080)--(2.970,1.081)--(2.972,1.081)--(2.974,1.081)--(2.976,1.081)%
  --(2.978,1.081)--(2.980,1.082)--(2.982,1.082)--(2.984,1.082)--(2.986,1.082)--(2.988,1.082)%
  --(2.990,1.083)--(2.992,1.083)--(2.995,1.083)--(2.997,1.083)--(2.999,1.083)--(3.001,1.083)%
  --(3.003,1.084)--(3.005,1.084)--(3.007,1.084)--(3.009,1.084)--(3.011,1.084)--(3.013,1.084)%
  --(3.015,1.085)--(3.017,1.085)--(3.019,1.085)--(3.021,1.085)--(3.024,1.085)--(3.026,1.085)%
  --(3.028,1.085)--(3.030,1.086)--(3.032,1.086)--(3.034,1.086)--(3.036,1.086)--(3.038,1.086)%
  --(3.040,1.086)--(3.042,1.086)--(3.044,1.086)--(3.046,1.086)--(3.048,1.086)--(3.050,1.087)%
  --(3.053,1.087)--(3.055,1.087)--(3.057,1.087)--(3.059,1.087)--(3.061,1.087)--(3.063,1.087)%
  --(3.065,1.087)--(3.067,1.087)--(3.069,1.087)--(3.071,1.087)--(3.073,1.087)--(3.075,1.087)%
  --(3.077,1.087)--(3.079,1.087)--(3.082,1.087)--(3.084,1.088)--(3.086,1.088)--(3.088,1.088)%
  --(3.090,1.088)--(3.092,1.088)--(3.094,1.088)--(3.096,1.088)--(3.098,1.088)--(3.100,1.088)%
  --(3.102,1.088)--(3.104,1.088)--(3.106,1.088)--(3.108,1.088)--(3.111,1.088)--(3.113,1.088)%
  --(3.115,1.088)--(3.117,1.088)--(3.119,1.087)--(3.121,1.087)--(3.123,1.087)--(3.125,1.087)%
  --(3.127,1.087)--(3.129,1.087)--(3.131,1.087)--(3.133,1.087)--(3.135,1.087)--(3.137,1.087)%
  --(3.140,1.087)--(3.142,1.087)--(3.144,1.087)--(3.146,1.087)--(3.148,1.087)--(3.150,1.087)%
  --(3.152,1.086)--(3.154,1.086)--(3.156,1.086)--(3.158,1.086)--(3.160,1.086)--(3.162,1.086)%
  --(3.164,1.086)--(3.166,1.086)--(3.168,1.086)--(3.171,1.086)--(3.173,1.085)--(3.175,1.085)%
  --(3.177,1.085)--(3.179,1.085)--(3.181,1.085)--(3.183,1.085)--(3.185,1.085)--(3.187,1.084)%
  --(3.189,1.084)--(3.191,1.084)--(3.193,1.084)--(3.195,1.084)--(3.197,1.084)--(3.200,1.083)%
  --(3.202,1.083)--(3.204,1.083)--(3.206,1.083)--(3.208,1.083)--(3.210,1.082)--(3.212,1.082)%
  --(3.214,1.082)--(3.216,1.082)--(3.218,1.082)--(3.220,1.081)--(3.222,1.081)--(3.224,1.081)%
  --(3.226,1.081)--(3.229,1.081)--(3.231,1.080)--(3.233,1.080)--(3.235,1.080)--(3.237,1.080)%
  --(3.239,1.079)--(3.241,1.079)--(3.243,1.079)--(3.245,1.079)--(3.247,1.078)--(3.249,1.078)%
  --(3.251,1.078)--(3.253,1.078)--(3.255,1.077)--(3.258,1.077)--(3.260,1.077)--(3.262,1.077)%
  --(3.264,1.076)--(3.266,1.076)--(3.268,1.076)--(3.270,1.075)--(3.272,1.075)--(3.274,1.075)%
  --(3.276,1.075)--(3.278,1.074)--(3.280,1.074)--(3.282,1.074)--(3.284,1.073)--(3.287,1.073)%
  --(3.289,1.073)--(3.291,1.072)--(3.293,1.072)--(3.295,1.072)--(3.297,1.071)--(3.299,1.071)%
  --(3.301,1.071)--(3.303,1.070)--(3.305,1.070)--(3.307,1.070)--(3.309,1.069)--(3.311,1.069)%
  --(3.313,1.069)--(3.316,1.068)--(3.318,1.068)--(3.320,1.068)--(3.322,1.067)--(3.324,1.067)%
  --(3.326,1.067)--(3.328,1.066)--(3.330,1.066)--(3.332,1.065)--(3.334,1.065)--(3.336,1.065)%
  --(3.338,1.064)--(3.340,1.064)--(3.342,1.063)--(3.344,1.063)--(3.347,1.063)--(3.349,1.062)%
  --(3.351,1.062)--(3.353,1.062)--(3.355,1.061)--(3.357,1.061)--(3.359,1.060)--(3.361,1.060)%
  --(3.363,1.060)--(3.365,1.059)--(3.367,1.059)--(3.369,1.058)--(3.371,1.058)--(3.373,1.057)%
  --(3.376,1.057)--(3.378,1.057)--(3.380,1.056)--(3.382,1.056)--(3.384,1.055)--(3.386,1.055)%
  --(3.388,1.054)--(3.390,1.054)--(3.392,1.054)--(3.394,1.053)--(3.396,1.053)--(3.398,1.052)%
  --(3.400,1.052)--(3.402,1.051)--(3.405,1.051)--(3.407,1.051)--(3.409,1.050)--(3.411,1.050)%
  --(3.413,1.049)--(3.415,1.049)--(3.417,1.048)--(3.419,1.048)--(3.421,1.047)--(3.423,1.047)%
  --(3.425,1.046)--(3.427,1.046)--(3.429,1.046)--(3.431,1.045)--(3.434,1.045)--(3.436,1.044)%
  --(3.438,1.044)--(3.440,1.043)--(3.442,1.043)--(3.444,1.042)--(3.446,1.042)--(3.448,1.041)%
  --(3.450,1.041)--(3.452,1.040)--(3.454,1.040)--(3.456,1.040)--(3.458,1.039)--(3.460,1.039)%
  --(3.463,1.038)--(3.465,1.038)--(3.467,1.037)--(3.469,1.037)--(3.471,1.036)--(3.473,1.036)%
  --(3.475,1.035)--(3.477,1.035)--(3.479,1.034)--(3.481,1.034)--(3.483,1.033)--(3.485,1.033)%
  --(3.487,1.032)--(3.489,1.032)--(3.492,1.032)--(3.494,1.031)--(3.496,1.031)--(3.498,1.030)%
  --(3.500,1.030)--(3.502,1.029)--(3.504,1.029)--(3.506,1.028)--(3.508,1.028)--(3.510,1.027)%
  --(3.512,1.027)--(3.514,1.026)--(3.516,1.026)--(3.518,1.025)--(3.520,1.025)--(3.523,1.025)%
  --(3.525,1.024)--(3.527,1.024)--(3.529,1.023)--(3.531,1.023)--(3.533,1.022)--(3.535,1.022)%
  --(3.537,1.021)--(3.539,1.021)--(3.541,1.020)--(3.543,1.020)--(3.545,1.019)--(3.547,1.019)%
  --(3.549,1.019)--(3.552,1.018)--(3.554,1.018)--(3.556,1.017)--(3.558,1.017)--(3.560,1.016)%
  --(3.562,1.016)--(3.564,1.015)--(3.566,1.015)--(3.568,1.015)--(3.570,1.014)--(3.572,1.014)%
  --(3.574,1.013)--(3.576,1.013)--(3.578,1.012)--(3.581,1.012)--(3.583,1.012)--(3.585,1.011)%
  --(3.587,1.011)--(3.589,1.010)--(3.591,1.010)--(3.593,1.009)--(3.595,1.009)--(3.597,1.009)%
  --(3.599,1.008)--(3.601,1.008)--(3.603,1.007)--(3.605,1.007)--(3.607,1.007)--(3.610,1.006)%
  --(3.612,1.006)--(3.614,1.005)--(3.616,1.005)--(3.618,1.005)--(3.620,1.004)--(3.622,1.004)%
  --(3.624,1.003)--(3.626,1.003)--(3.628,1.003)--(3.630,1.002)--(3.632,1.002)--(3.634,1.002)%
  --(3.636,1.001)--(3.639,1.001)--(3.641,1.001)--(3.643,1.000)--(3.645,1.000)--(3.647,0.999)%
  --(3.649,0.999)--(3.651,0.999)--(3.653,0.998)--(3.655,0.998)--(3.657,0.998)--(3.659,0.997)%
  --(3.661,0.997)--(3.663,0.997)--(3.665,0.996)--(3.668,0.996)--(3.670,0.996)--(3.672,0.996)%
  --(3.674,0.995)--(3.676,0.995)--(3.678,0.995)--(3.680,0.994)--(3.682,0.994)--(3.684,0.994)%
  --(3.686,0.994)--(3.688,0.993)--(3.690,0.993)--(3.692,0.993)--(3.694,0.992)--(3.696,0.992)%
  --(3.699,0.992)--(3.701,0.992)--(3.703,0.991)--(3.705,0.991)--(3.707,0.991)--(3.709,0.991)%
  --(3.711,0.990)--(3.713,0.990)--(3.715,0.990)--(3.717,0.990)--(3.719,0.990)--(3.721,0.989)%
  --(3.723,0.989)--(3.725,0.989)--(3.728,0.989)--(3.730,0.989)--(3.732,0.988)--(3.734,0.988)%
  --(3.736,0.988)--(3.738,0.988)--(3.740,0.988)--(3.742,0.988)--(3.744,0.987)--(3.746,0.987)%
  --(3.748,0.987)--(3.750,0.987)--(3.752,0.987)--(3.754,0.987)--(3.757,0.987)--(3.759,0.986)%
  --(3.761,0.986)--(3.763,0.986)--(3.765,0.986)--(3.767,0.986)--(3.769,0.986)--(3.771,0.986)%
  --(3.773,0.986)--(3.775,0.986)--(3.777,0.986)--(3.779,0.985)--(3.781,0.985)--(3.783,0.985)%
  --(3.786,0.985)--(3.788,0.985)--(3.790,0.985)--(3.792,0.985)--(3.794,0.985)--(3.796,0.985)%
  --(3.798,0.985)--(3.800,0.985)--(3.802,0.985)--(3.804,0.985)--(3.806,0.985)--(3.808,0.985)%
  --(3.810,0.985)--(3.812,0.985)--(3.815,0.985)--(3.817,0.985)--(3.819,0.985)--(3.821,0.985)%
  --(3.823,0.985)--(3.825,0.985)--(3.827,0.985)--(3.829,0.985)--(3.831,0.985)--(3.833,0.986)%
  --(3.835,0.986)--(3.837,0.986)--(3.839,0.986)--(3.841,0.986)--(3.844,0.986)--(3.846,0.986)%
  --(3.848,0.986)--(3.850,0.986)--(3.852,0.986)--(3.854,0.987)--(3.856,0.987)--(3.858,0.987)%
  --(3.860,0.987)--(3.862,0.987)--(3.864,0.987)--(3.866,0.988)--(3.868,0.988)--(3.870,0.988)%
  --(3.872,0.988)--(3.875,0.988)--(3.877,0.989)--(3.879,0.989)--(3.881,0.989)--(3.883,0.989)%
  --(3.885,0.989)--(3.887,0.990)--(3.889,0.990)--(3.891,0.990)--(3.893,0.990)--(3.895,0.991)%
  --(3.897,0.991)--(3.899,0.991)--(3.901,0.991)--(3.904,0.992)--(3.906,0.992)--(3.908,0.992)%
  --(3.910,0.993)--(3.912,0.993)--(3.914,0.993)--(3.916,0.994)--(3.918,0.994)--(3.920,0.994)%
  --(3.922,0.995)--(3.924,0.995)--(3.926,0.995)--(3.928,0.996)--(3.930,0.996)--(3.933,0.996)%
  --(3.935,0.997)--(3.937,0.997)--(3.939,0.998)--(3.941,0.998)--(3.943,0.998)--(3.945,0.999)%
  --(3.947,0.999)--(3.949,1.000)--(3.951,1.000)--(3.953,1.001)--(3.955,1.001)--(3.957,1.001)%
  --(3.959,1.002)--(3.962,1.002)--(3.964,1.003)--(3.966,1.003)--(3.968,1.004)--(3.970,1.004)%
  --(3.972,1.005)--(3.974,1.005)--(3.976,1.006)--(3.978,1.006)--(3.980,1.007)--(3.982,1.007)%
  --(3.984,1.008)--(3.986,1.008)--(3.988,1.009)--(3.991,1.009)--(3.993,1.010)--(3.995,1.011)%
  --(3.997,1.011)--(3.999,1.012)--(4.001,1.012)--(4.003,1.013)--(4.005,1.014)--(4.007,1.014)%
  --(4.009,1.015)--(4.011,1.015)--(4.013,1.016)--(4.015,1.017)--(4.017,1.017)--(4.020,1.018)%
  --(4.022,1.019)--(4.024,1.019)--(4.026,1.020)--(4.028,1.020)--(4.030,1.021)--(4.032,1.022)%
  --(4.034,1.023)--(4.036,1.023)--(4.038,1.024)--(4.040,1.025)--(4.042,1.025)--(4.044,1.026)%
  --(4.046,1.027)--(4.049,1.027)--(4.051,1.028)--(4.053,1.029)--(4.055,1.030)--(4.057,1.030)%
  --(4.059,1.031)--(4.061,1.032)--(4.063,1.033)--(4.065,1.033)--(4.067,1.034)--(4.069,1.035)%
  --(4.071,1.036)--(4.073,1.037)--(4.075,1.037)--(4.077,1.038)--(4.080,1.039)--(4.082,1.040)%
  --(4.084,1.041)--(4.086,1.041)--(4.088,1.042)--(4.090,1.043)--(4.092,1.044)--(4.094,1.045)%
  --(4.096,1.046)--(4.098,1.046)--(4.100,1.047)--(4.102,1.048)--(4.104,1.049)--(4.106,1.050)%
  --(4.109,1.051)--(4.111,1.052)--(4.113,1.052)--(4.115,1.053)--(4.117,1.054)--(4.119,1.055)%
  --(4.121,1.056)--(4.123,1.057)--(4.125,1.058)--(4.127,1.059)--(4.129,1.060)--(4.131,1.061)%
  --(4.133,1.062)--(4.135,1.063)--(4.138,1.063)--(4.140,1.064)--(4.142,1.065)--(4.144,1.066)%
  --(4.146,1.067)--(4.148,1.068)--(4.150,1.069)--(4.152,1.070)--(4.154,1.071)--(4.156,1.072)%
  --(4.158,1.073)--(4.160,1.074)--(4.162,1.075)--(4.164,1.076)--(4.167,1.077)--(4.169,1.078)%
  --(4.171,1.079)--(4.173,1.080)--(4.175,1.081)--(4.177,1.082)--(4.179,1.083)--(4.181,1.084)%
  --(4.183,1.085)--(4.185,1.086)--(4.187,1.087)--(4.189,1.088)--(4.191,1.089)--(4.193,1.090)%
  --(4.196,1.091)--(4.198,1.092)--(4.200,1.093)--(4.202,1.094)--(4.204,1.095)--(4.206,1.097)%
  --(4.208,1.098)--(4.210,1.099)--(4.212,1.100)--(4.214,1.101)--(4.216,1.102)--(4.218,1.103)%
  --(4.220,1.104)--(4.222,1.105)--(4.225,1.106)--(4.227,1.107)--(4.229,1.108)--(4.231,1.109)%
  --(4.233,1.110)--(4.235,1.112)--(4.237,1.113)--(4.239,1.114)--(4.241,1.115)--(4.243,1.116)%
  --(4.245,1.117)--(4.247,1.118)--(4.249,1.119)--(4.251,1.120)--(4.253,1.121)--(4.256,1.123)%
  --(4.258,1.124)--(4.260,1.125)--(4.262,1.126)--(4.264,1.127)--(4.266,1.128)--(4.268,1.129)%
  --(4.270,1.130)--(4.272,1.131)--(4.274,1.132)--(4.276,1.134)--(4.278,1.135)--(4.280,1.136)%
  --(4.282,1.137)--(4.285,1.138)--(4.287,1.139)--(4.289,1.140)--(4.291,1.141)--(4.293,1.142)%
  --(4.295,1.144)--(4.297,1.145)--(4.299,1.146)--(4.301,1.147)--(4.303,1.148)--(4.305,1.149)%
  --(4.307,1.150)--(4.309,1.151)--(4.311,1.152)--(4.314,1.154)--(4.316,1.155)--(4.318,1.156)%
  --(4.320,1.157)--(4.322,1.158)--(4.324,1.159)--(4.326,1.160)--(4.328,1.161)--(4.330,1.162)%
  --(4.332,1.164)--(4.334,1.165)--(4.336,1.166)--(4.338,1.167)--(4.340,1.168)--(4.343,1.169)%
  --(4.345,1.170)--(4.347,1.171)--(4.349,1.172)--(4.351,1.173)--(4.353,1.175)--(4.355,1.176)%
  --(4.357,1.177)--(4.359,1.178)--(4.361,1.179)--(4.363,1.180)--(4.365,1.181)--(4.367,1.182)%
  --(4.369,1.183)--(4.372,1.184)--(4.374,1.185)--(4.376,1.186)--(4.378,1.187)--(4.380,1.188)%
  --(4.382,1.190)--(4.384,1.191)--(4.386,1.192)--(4.388,1.193)--(4.390,1.194)--(4.392,1.195)%
  --(4.394,1.196)--(4.396,1.197)--(4.398,1.198)--(4.401,1.199)--(4.403,1.200)--(4.405,1.201)%
  --(4.407,1.202)--(4.409,1.203)--(4.411,1.204)--(4.413,1.205)--(4.415,1.206)--(4.417,1.207)%
  --(4.419,1.208)--(4.421,1.209)--(4.423,1.210)--(4.425,1.211)--(4.427,1.212)--(4.429,1.213)%
  --(4.432,1.214)--(4.434,1.215)--(4.436,1.216)--(4.438,1.217)--(4.440,1.218)--(4.442,1.219)%
  --(4.444,1.220)--(4.446,1.220)--(4.448,1.221)--(4.450,1.222)--(4.452,1.223)--(4.454,1.224)%
  --(4.456,1.225)--(4.458,1.226)--(4.461,1.227)--(4.463,1.228)--(4.465,1.229)--(4.467,1.229)%
  --(4.469,1.230)--(4.471,1.231)--(4.473,1.232)--(4.475,1.233)--(4.477,1.234)--(4.479,1.235)%
  --(4.481,1.235)--(4.483,1.236)--(4.485,1.237)--(4.487,1.238)--(4.490,1.239)--(4.492,1.240)%
  --(4.494,1.240)--(4.496,1.241)--(4.498,1.242)--(4.500,1.243)--(4.502,1.244)--(4.504,1.244)%
  --(4.506,1.245)--(4.508,1.246)--(4.510,1.247)--(4.512,1.247)--(4.514,1.248)--(4.516,1.249)%
  --(4.519,1.249)--(4.521,1.250)--(4.523,1.251)--(4.525,1.252)--(4.527,1.252)--(4.529,1.253)%
  --(4.531,1.254)--(4.533,1.254)--(4.535,1.255)--(4.537,1.256)--(4.539,1.256)--(4.541,1.257)%
  --(4.543,1.257)--(4.545,1.258)--(4.548,1.259)--(4.550,1.259)--(4.552,1.260)--(4.554,1.261)%
  --(4.556,1.261)--(4.558,1.262)--(4.560,1.262)--(4.562,1.263)--(4.564,1.263)--(4.566,1.264)%
  --(4.568,1.264)--(4.570,1.265)--(4.572,1.265)--(4.574,1.266)--(4.577,1.266)--(4.579,1.267)%
  --(4.581,1.267)--(4.583,1.268)--(4.585,1.268)--(4.587,1.269)--(4.589,1.269)--(4.591,1.270)%
  --(4.593,1.270)--(4.595,1.270)--(4.597,1.271)--(4.599,1.271)--(4.601,1.272)--(4.603,1.272)%
  --(4.605,1.272)--(4.608,1.273)--(4.610,1.273)--(4.612,1.273)--(4.614,1.274)--(4.616,1.274)%
  --(4.618,1.274)--(4.620,1.275)--(4.622,1.275)--(4.624,1.275)--(4.626,1.275)--(4.628,1.276)%
  --(4.630,1.276)--(4.632,1.276)--(4.634,1.276)--(4.637,1.277)--(4.639,1.277)--(4.641,1.277)%
  --(4.643,1.277)--(4.645,1.277)--(4.647,1.278)--(4.649,1.278)--(4.651,1.278)--(4.653,1.278)%
  --(4.655,1.278)--(4.657,1.278)--(4.659,1.278)--(4.661,1.279)--(4.663,1.279)--(4.666,1.279)%
  --(4.668,1.279)--(4.670,1.279)--(4.672,1.279)--(4.674,1.279)--(4.676,1.279)--(4.678,1.279)%
  --(4.680,1.279)--(4.682,1.279)--(4.684,1.279)--(4.686,1.279)--(4.688,1.279)--(4.690,1.279)%
  --(4.692,1.279)--(4.695,1.279)--(4.697,1.279)--(4.699,1.279)--(4.701,1.278)--(4.703,1.278)%
  --(4.705,1.278)--(4.707,1.278)--(4.709,1.278)--(4.711,1.278)--(4.713,1.278)--(4.715,1.277)%
  --(4.717,1.277)--(4.719,1.277)--(4.721,1.277)--(4.724,1.277)--(4.726,1.276)--(4.728,1.276)%
  --(4.730,1.276)--(4.732,1.276)--(4.734,1.275)--(4.736,1.275)--(4.738,1.275)--(4.740,1.274)%
  --(4.742,1.274)--(4.744,1.274)--(4.746,1.273)--(4.748,1.273)--(4.750,1.273)--(4.753,1.272)%
  --(4.755,1.272)--(4.757,1.272)--(4.759,1.271)--(4.761,1.271)--(4.763,1.270)--(4.765,1.270)%
  --(4.767,1.269)--(4.769,1.269)--(4.771,1.269)--(4.773,1.268)--(4.775,1.268)--(4.777,1.267)%
  --(4.779,1.267)--(4.781,1.266)--(4.784,1.265)--(4.786,1.265)--(4.788,1.264)--(4.790,1.264)%
  --(4.792,1.263)--(4.794,1.263)--(4.796,1.262)--(4.798,1.261)--(4.800,1.261)--(4.802,1.260)%
  --(4.804,1.260)--(4.806,1.259)--(4.808,1.258)--(4.810,1.258)--(4.813,1.257)--(4.815,1.256)%
  --(4.817,1.256)--(4.819,1.255)--(4.821,1.254)--(4.823,1.253)--(4.825,1.253)--(4.827,1.252)%
  --(4.829,1.251)--(4.831,1.250)--(4.833,1.250)--(4.835,1.249)--(4.837,1.248)--(4.839,1.247)%
  --(4.842,1.246)--(4.844,1.246)--(4.846,1.245)--(4.848,1.244)--(4.850,1.243)--(4.852,1.242)%
  --(4.854,1.241)--(4.856,1.240)--(4.858,1.239)--(4.860,1.239)--(4.862,1.238)--(4.864,1.237)%
  --(4.866,1.236)--(4.868,1.235)--(4.871,1.234)--(4.873,1.233)--(4.875,1.232)--(4.877,1.231)%
  --(4.879,1.230)--(4.881,1.229)--(4.883,1.228)--(4.885,1.227)--(4.887,1.226)--(4.889,1.225)%
  --(4.891,1.224)--(4.893,1.223)--(4.895,1.222)--(4.897,1.221)--(4.900,1.220)--(4.902,1.218)%
  --(4.904,1.217)--(4.906,1.216)--(4.908,1.215)--(4.910,1.214)--(4.912,1.213)--(4.914,1.212)%
  --(4.916,1.211)--(4.918,1.210)--(4.920,1.208)--(4.922,1.207)--(4.924,1.206)--(4.926,1.205)%
  --(4.929,1.204)--(4.931,1.203)--(4.933,1.201)--(4.935,1.200)--(4.937,1.199)--(4.939,1.198)%
  --(4.941,1.196)--(4.943,1.195)--(4.945,1.194)--(4.947,1.193)--(4.949,1.191)--(4.951,1.190)%
  --(4.953,1.189)--(4.955,1.188)--(4.957,1.186)--(4.960,1.185)--(4.962,1.184)--(4.964,1.183)%
  --(4.966,1.181)--(4.968,1.180)--(4.970,1.179)--(4.972,1.177)--(4.974,1.176)--(4.976,1.175)%
  --(4.978,1.173)--(4.980,1.172)--(4.982,1.171)--(4.984,1.169)--(4.986,1.168)--(4.989,1.167)%
  --(4.991,1.165)--(4.993,1.164)--(4.995,1.163)--(4.997,1.161)--(4.999,1.160)--(5.001,1.158)%
  --(5.003,1.157)--(5.005,1.156)--(5.007,1.154)--(5.009,1.153)--(5.011,1.151)--(5.013,1.150)%
  --(5.015,1.149)--(5.018,1.147)--(5.020,1.146)--(5.022,1.144)--(5.024,1.143)--(5.026,1.142)%
  --(5.028,1.140)--(5.030,1.139)--(5.032,1.137)--(5.034,1.136)--(5.036,1.135)--(5.038,1.133)%
  --(5.040,1.132)--(5.042,1.130)--(5.044,1.129)--(5.047,1.127)--(5.049,1.126)--(5.051,1.124)%
  --(5.053,1.123)--(5.055,1.122)--(5.057,1.120)--(5.059,1.119)--(5.061,1.117)--(5.063,1.116)%
  --(5.065,1.114)--(5.067,1.113)--(5.069,1.112)--(5.071,1.110)--(5.073,1.109)--(5.076,1.107)%
  --(5.078,1.106)--(5.080,1.104)--(5.082,1.103)--(5.084,1.101)--(5.086,1.100)--(5.088,1.099)%
  --(5.090,1.097)--(5.092,1.096)--(5.094,1.094)--(5.096,1.093)--(5.098,1.091)--(5.100,1.090)%
  --(5.102,1.089)--(5.105,1.087)--(5.107,1.086)--(5.109,1.084)--(5.111,1.083)--(5.113,1.082)%
  --(5.115,1.080)--(5.117,1.079)--(5.119,1.077)--(5.121,1.076)--(5.123,1.075)--(5.125,1.073)%
  --(5.127,1.072)--(5.129,1.070)--(5.131,1.069)--(5.134,1.068)--(5.136,1.066)--(5.138,1.065)%
  --(5.140,1.064)--(5.142,1.062)--(5.144,1.061)--(5.146,1.060)--(5.148,1.058)--(5.150,1.057)%
  --(5.152,1.056)--(5.154,1.054)--(5.156,1.053)--(5.158,1.052)--(5.160,1.051)--(5.162,1.049)%
  --(5.165,1.048)--(5.167,1.047)--(5.169,1.045)--(5.171,1.044)--(5.173,1.043)--(5.175,1.042)%
  --(5.177,1.041)--(5.179,1.039)--(5.181,1.038)--(5.183,1.037)--(5.185,1.036)--(5.187,1.035)%
  --(5.189,1.033)--(5.191,1.032)--(5.194,1.031)--(5.196,1.030)--(5.198,1.029)--(5.200,1.028)%
  --(5.202,1.027)--(5.204,1.025)--(5.206,1.024)--(5.208,1.023)--(5.210,1.022)--(5.212,1.021)%
  --(5.214,1.020)--(5.216,1.019)--(5.218,1.018)--(5.220,1.017)--(5.223,1.016)--(5.225,1.015)%
  --(5.227,1.014)--(5.229,1.013)--(5.231,1.012)--(5.233,1.011)--(5.235,1.010)--(5.237,1.009)%
  --(5.239,1.009)--(5.241,1.008)--(5.243,1.007)--(5.245,1.006)--(5.247,1.005)--(5.249,1.004)%
  --(5.252,1.003)--(5.254,1.003)--(5.256,1.002)--(5.258,1.001)--(5.260,1.000)--(5.262,1.000)%
  --(5.264,0.999)--(5.266,0.998)--(5.268,0.998)--(5.270,0.997)--(5.272,0.996)--(5.274,0.996)%
  --(5.276,0.995)--(5.278,0.994)--(5.281,0.994)--(5.283,0.993)--(5.285,0.993)--(5.287,0.992)%
  --(5.289,0.992)--(5.291,0.991)--(5.293,0.991)--(5.295,0.990)--(5.297,0.990)--(5.299,0.989)%
  --(5.301,0.989)--(5.303,0.988)--(5.305,0.988)--(5.307,0.988)--(5.310,0.987)--(5.312,0.987)%
  --(5.314,0.987)--(5.316,0.987)--(5.318,0.986)--(5.320,0.986)--(5.322,0.986)--(5.324,0.986)%
  --(5.326,0.986)--(5.328,0.985)--(5.330,0.985)--(5.332,0.985)--(5.334,0.985)--(5.336,0.985)%
  --(5.338,0.985)--(5.341,0.985)--(5.343,0.985)--(5.345,0.985)--(5.347,0.985)--(5.349,0.985)%
  --(5.351,0.985)--(5.353,0.985)--(5.355,0.986)--(5.357,0.986)--(5.359,0.986)--(5.361,0.986)%
  --(5.363,0.986)--(5.365,0.987)--(5.367,0.987)--(5.370,0.987)--(5.372,0.988)--(5.374,0.988)%
  --(5.376,0.988)--(5.378,0.989)--(5.380,0.989)--(5.382,0.990)--(5.384,0.990)--(5.386,0.991)%
  --(5.388,0.991)--(5.390,0.992)--(5.392,0.993)--(5.394,0.993)--(5.396,0.994)--(5.399,0.995)%
  --(5.401,0.995)--(5.403,0.996)--(5.405,0.997)--(5.407,0.998)--(5.409,0.998)--(5.411,0.999)%
  --(5.413,1.000)--(5.415,1.001)--(5.417,1.002)--(5.419,1.003)--(5.421,1.004)--(5.423,1.005)%
  --(5.425,1.006)--(5.428,1.007)--(5.430,1.008)--(5.432,1.010)--(5.434,1.011)--(5.436,1.012)%
  --(5.438,1.013)--(5.440,1.015)--(5.442,1.016)--(5.444,1.017)--(5.446,1.019)--(5.448,1.020)%
  --(5.450,1.021)--(5.452,1.023)--(5.454,1.024)--(5.457,1.026)--(5.459,1.027)--(5.461,1.029)%
  --(5.463,1.031)--(5.465,1.032)--(5.467,1.034)--(5.469,1.036)--(5.471,1.038)--(5.473,1.039)%
  --(5.475,1.041)--(5.477,1.043)--(5.479,1.045)--(5.481,1.047)--(5.483,1.049)--(5.486,1.051)%
  --(5.488,1.053)--(5.490,1.055)--(5.492,1.057)--(5.494,1.059)--(5.496,1.061)--(5.498,1.064)%
  --(5.500,1.066)--(5.502,1.068)--(5.504,1.070)--(5.506,1.073)--(5.508,1.075)--(5.510,1.078)%
  --(5.512,1.080)--(5.514,1.083)--(5.517,1.085)--(5.519,1.088)--(5.521,1.090)--(5.523,1.093)%
  --(5.525,1.096)--(5.527,1.098)--(5.529,1.101)--(5.531,1.104)--(5.533,1.107)--(5.535,1.110)%
  --(5.537,1.113)--(5.539,1.116)--(5.541,1.119)--(5.543,1.122)--(5.546,1.125)--(5.548,1.128)%
  --(5.550,1.131)--(5.552,1.134)--(5.554,1.137)--(5.556,1.141)--(5.558,1.144)--(5.560,1.147)%
  --(5.562,1.151)--(5.564,1.154)--(5.566,1.157)--(5.568,1.161)--(5.570,1.164)--(5.572,1.168)%
  --(5.575,1.172)--(5.577,1.175)--(5.579,1.179)--(5.581,1.183)--(5.583,1.187)--(5.585,1.190)%
  --(5.587,1.194)--(5.589,1.198)--(5.591,1.202)--(5.593,1.206)--(5.595,1.210)--(5.597,1.214)%
  --(5.599,1.218)--(5.601,1.222)--(5.604,1.227)--(5.606,1.231)--(5.608,1.235)--(5.610,1.239)%
  --(5.612,1.244)--(5.614,1.248)--(5.616,1.253)--(5.618,1.257)--(5.620,1.262)--(5.622,1.266)%
  --(5.624,1.271)--(5.626,1.276)--(5.628,1.280)--(5.630,1.285)--(5.633,1.290)--(5.635,1.295)%
  --(5.637,1.299)--(5.639,1.304)--(5.641,1.309)--(5.643,1.314)--(5.645,1.319)--(5.647,1.325)%
  --(5.649,1.330)--(5.651,1.335)--(5.653,1.340)--(5.655,1.345)--(5.657,1.351)--(5.659,1.356)%
  --(5.662,1.361)--(5.664,1.367)--(5.666,1.372)--(5.668,1.378)--(5.670,1.383)--(5.672,1.389)%
  --(5.674,1.395)--(5.676,1.400)--(5.678,1.406)--(5.680,1.412)--(5.682,1.418)--(5.684,1.424)%
  --(5.686,1.430)--(5.688,1.436)--(5.690,1.442)--(5.693,1.448)--(5.695,1.454)--(5.697,1.460)%
  --(5.699,1.466)--(5.701,1.472)--(5.703,1.479)--(5.705,1.485)--(5.707,1.491)--(5.709,1.498)%
  --(5.711,1.504)--(5.713,1.511)--(5.715,1.517)--(5.717,1.524)--(5.719,1.530)--(5.722,1.537)%
  --(5.724,1.544)--(5.726,1.551)--(5.728,1.558)--(5.730,1.564)--(5.732,1.571)--(5.734,1.578)%
  --(5.736,1.585)--(5.738,1.592)--(5.740,1.599)--(5.742,1.607)--(5.744,1.614)--(5.746,1.621)%
  --(5.748,1.628)--(5.751,1.636)--(5.753,1.643)--(5.755,1.650)--(5.757,1.658)--(5.759,1.665)%
  --(5.761,1.673)--(5.763,1.680)--(5.765,1.688)--(5.767,1.696)--(5.769,1.703)--(5.771,1.711)%
  --(5.773,1.719)--(5.775,1.727)--(5.777,1.735)--(5.780,1.743)--(5.782,1.751)--(5.784,1.759)%
  --(5.786,1.767)--(5.788,1.775)--(5.790,1.783)--(5.792,1.791)--(5.794,1.800)--(5.796,1.808)%
  --(5.798,1.816)--(5.800,1.825)--(5.802,1.833)--(5.804,1.842)--(5.806,1.850)--(5.809,1.859)%
  --(5.811,1.867)--(5.813,1.876)--(5.815,1.885)--(5.817,1.894)--(5.819,1.902)--(5.821,1.911)%
  --(5.823,1.920)--(5.825,1.929)--(5.827,1.938)--(5.829,1.947)--(5.831,1.956)--(5.833,1.965)%
  --(5.835,1.974)--(5.838,1.984)--(5.840,1.993)--(5.842,2.002)--(5.844,2.011)--(5.846,2.021)%
  --(5.848,2.030)--(5.850,2.040)--(5.852,2.049)--(5.854,2.059)--(5.856,2.068)--(5.858,2.078)%
  --(5.860,2.087)--(5.862,2.097)--(5.864,2.107)--(5.866,2.117)--(5.869,2.127)--(5.871,2.136)%
  --(5.873,2.146)--(5.875,2.156)--(5.877,2.166)--(5.879,2.176)--(5.881,2.186)--(5.883,2.197)%
  --(5.885,2.207)--(5.887,2.217)--(5.889,2.227)--(5.891,2.237)--(5.893,2.248)--(5.895,2.258)%
  --(5.898,2.269)--(5.900,2.279)--(5.902,2.289)--(5.904,2.300)--(5.906,2.311)--(5.908,2.321)%
  --(5.910,2.332)--(5.912,2.343)--(5.914,2.353)--(5.916,2.364)--(5.918,2.375)--(5.920,2.386)%
  --(5.922,2.397)--(5.924,2.408)--(5.927,2.419)--(5.929,2.430)--(5.931,2.441)--(5.933,2.452)%
  --(5.935,2.463)--(5.937,2.474)--(5.939,2.485)--(5.941,2.496)--(5.943,2.508)--(5.945,2.519)%
  --(5.947,2.530)--(5.949,2.542)--(5.951,2.553)--(5.953,2.565)--(5.956,2.576)--(5.958,2.588)%
  --(5.960,2.599)--(5.962,2.611)--(5.964,2.622)--(5.966,2.634)--(5.968,2.646)--(5.970,2.657)%
  --(5.972,2.669)--(5.974,2.681)--(5.976,2.693)--(5.978,2.705)--(5.980,2.717)--(5.982,2.729)%
  --(5.985,2.741)--(5.987,2.753)--(5.989,2.765)--(5.991,2.777)--(5.993,2.789)--(5.995,2.801)%
  --(5.997,2.813)--(5.999,2.825)--(6.001,2.838)--(6.003,2.850)--(6.005,2.862)--(6.007,2.875)%
  --(6.009,2.887)--(6.011,2.899)--(6.014,2.912)--(6.016,2.924)--(6.018,2.937)--(6.020,2.949)%
  --(6.022,2.962)--(6.024,2.974)--(6.026,2.987)--(6.028,3.000)--(6.030,3.012)--(6.032,3.025)%
  --(6.034,3.038)--(6.036,3.051)--(6.038,3.063)--(6.040,3.076)--(6.043,3.089)--(6.045,3.102)%
  --(6.047,3.115)--(6.049,3.128)--(6.051,3.141)--(6.053,3.154)--(6.055,3.167)--(6.057,3.180)%
  --(6.059,3.193)--(6.061,3.206)--(6.063,3.219)--(6.065,3.232)--(6.067,3.245)--(6.069,3.259)%
  --(6.071,3.272)--(6.074,3.285)--(6.076,3.298)--(6.078,3.312)--(6.080,3.325)--(6.082,3.338)%
  --(6.084,3.352)--(6.086,3.365)--(6.088,3.378)--(6.090,3.392)--(6.092,3.405)--(6.094,3.419)%
  --(6.096,3.432)--(6.098,3.446)--(6.100,3.459)--(6.103,3.473)--(6.105,3.486)--(6.107,3.500)%
  --(6.109,3.514)--(6.111,3.527)--(6.113,3.541)--(6.115,3.554)--(6.117,3.568)--(6.119,3.582)%
  --(6.121,3.596)--(6.123,3.609)--(6.125,3.623)--(6.127,3.637)--(6.129,3.651)--(6.132,3.664)%
  --(6.134,3.678)--(6.136,3.692)--(6.138,3.706)--(6.140,3.720)--(6.142,3.734)--(6.144,3.748)%
  --(6.146,3.762)--(6.148,3.775)--(6.150,3.789)--(6.152,3.803)--(6.154,3.817)--(6.156,3.831)%
  --(6.158,3.845)--(6.161,3.859)--(6.163,3.873)--(6.165,3.887)--(6.167,3.901)--(6.169,3.916)%
  --(6.171,3.930)--(6.173,3.944)--(6.175,3.958)--(6.177,3.972)--(6.179,3.986)--(6.181,4.000)%
  --(6.183,4.014)--(6.185,4.028)--(6.187,4.042)--(6.190,4.057)--(6.192,4.071)--(6.194,4.085)%
  --(6.196,4.099)--(6.198,4.113)--(6.200,4.128)--(6.202,4.142)--(6.204,4.156)--(6.206,4.170)%
  --(6.208,4.184)--(6.210,4.199)--(6.212,4.213)--(6.214,4.227)--(6.216,4.241)--(6.219,4.255)%
  --(6.221,4.270)--(6.223,4.284)--(6.225,4.298)--(6.227,4.312)--(6.229,4.327)--(6.231,4.341)%
  --(6.233,4.355)--(6.235,4.369)--(6.237,4.384)--(6.239,4.398)--(6.241,4.412)--(6.243,4.426)%
  --(6.245,4.441)--(6.247,4.455)--(6.250,4.469)--(6.252,4.483)--(6.254,4.498)--(6.256,4.512)%
  --(6.258,4.526)--(6.260,4.540)--(6.262,4.555)--(6.264,4.569)--(6.266,4.583)--(6.268,4.597)%
  --(6.270,4.611)--(6.272,4.626)--(6.274,4.640)--(6.276,4.654)--(6.279,4.668)--(6.281,4.682)%
  --(6.283,4.697)--(6.285,4.711)--(6.287,4.725)--(6.289,4.739)--(6.291,4.753)--(6.293,4.767)%
  --(6.295,4.781)--(6.297,4.796)--(6.299,4.810)--(6.301,4.824)--(6.303,4.838)--(6.305,4.852)%
  --(6.308,4.866)--(6.310,4.880)--(6.312,4.894)--(6.314,4.908)--(6.316,4.922)--(6.318,4.936)%
  --(6.320,4.950)--(6.322,4.964)--(6.324,4.978)--(6.326,4.992)--(6.328,5.006)--(6.330,5.020)%
  --(6.332,5.034)--(6.334,5.048)--(6.337,5.062)--(6.339,5.075)--(6.341,5.089)--(6.343,5.103)%
  --(6.345,5.117)--(6.347,5.131)--(6.349,5.144)--(6.351,5.158)--(6.353,5.172)--(6.355,5.186)%
  --(6.357,5.199)--(6.359,5.213)--(6.361,5.227)--(6.363,5.240)--(6.366,5.254)--(6.368,5.268)%
  --(6.370,5.281)--(6.372,5.295)--(6.374,5.308)--(6.376,5.322)--(6.378,5.335)--(6.380,5.349)%
  --(6.382,5.362)--(6.384,5.376)--(6.386,5.389)--(6.388,5.402)--(6.390,5.416)--(6.392,5.429)%
  --(6.395,5.442)--(6.397,5.455)--(6.399,5.469)--(6.401,5.482)--(6.403,5.495)--(6.405,5.508)%
  --(6.407,5.521)--(6.409,5.534)--(6.411,5.548)--(6.413,5.561)--(6.415,5.574)--(6.417,5.587)%
  --(6.419,5.600)--(6.421,5.612)--(6.423,5.625)--(6.426,5.638)--(6.428,5.651)--(6.430,5.664)%
  --(6.432,5.677)--(6.434,5.689)--(6.436,5.702)--(6.438,5.715)--(6.440,5.727)--(6.442,5.740)%
  --(6.444,5.753)--(6.446,5.765)--(6.448,5.778)--(6.450,5.790)--(6.452,5.802)--(6.455,5.815)%
  --(6.457,5.827)--(6.459,5.840)--(6.461,5.852)--(6.463,5.864)--(6.465,5.876)--(6.467,5.888)%
  --(6.469,5.901)--(6.471,5.913)--(6.473,5.925)--(6.475,5.937)--(6.477,5.949)--(6.479,5.961)%
  --(6.481,5.973)--(6.484,5.984)--(6.486,5.996)--(6.488,6.008)--(6.490,6.020)--(6.492,6.032)%
  --(6.494,6.043)--(6.496,6.055)--(6.498,6.066)--(6.500,6.078)--(6.502,6.089)--(6.504,6.101)%
  --(6.506,6.112)--(6.508,6.124)--(6.510,6.135)--(6.513,6.146)--(6.515,6.157)--(6.517,6.169)%
  --(6.519,6.180)--(6.521,6.191)--(6.523,6.202)--(6.525,6.213)--(6.527,6.224)--(6.529,6.235)%
  --(6.531,6.246)--(6.533,6.256)--(6.535,6.267)--(6.537,6.278)--(6.539,6.289)--(6.542,6.299)%
  --(6.544,6.310)--(6.546,6.320)--(6.548,6.331)--(6.550,6.341)--(6.552,6.351)--(6.554,6.362)%
  --(6.556,6.372)--(6.558,6.382)--(6.560,6.392)--(6.562,6.403)--(6.564,6.413)--(6.566,6.423)%
  --(6.568,6.433)--(6.571,6.442)--(6.573,6.452)--(6.575,6.462)--(6.577,6.472)--(6.579,6.482)%
  --(6.581,6.491)--(6.583,6.501)--(6.585,6.510)--(6.587,6.520)--(6.589,6.529)--(6.591,6.539)%
  --(6.593,6.548)--(6.595,6.557)--(6.597,6.566)--(6.599,6.575)--(6.602,6.585)--(6.604,6.594)%
  --(6.606,6.603)--(6.608,6.611)--(6.610,6.620)--(6.612,6.629)--(6.614,6.638)--(6.616,6.647)%
  --(6.618,6.655)--(6.620,6.664)--(6.622,6.672)--(6.624,6.681)--(6.626,6.689)--(6.628,6.697)%
  --(6.631,6.706)--(6.633,6.714)--(6.635,6.722)--(6.637,6.730)--(6.639,6.738)--(6.641,6.746)%
  --(6.643,6.754)--(6.645,6.762)--(6.647,6.770)--(6.649,6.777)--(6.651,6.785)--(6.653,6.793)%
  --(6.655,6.800)--(6.657,6.808)--(6.660,6.815)--(6.662,6.822)--(6.664,6.830)--(6.666,6.837)%
  --(6.668,6.844)--(6.670,6.851)--(6.672,6.858)--(6.674,6.865)--(6.676,6.872)--(6.678,6.879)%
  --(6.680,6.885)--(6.682,6.892)--(6.684,6.899)--(6.686,6.905)--(6.689,6.912)--(6.691,6.918)%
  --(6.693,6.925)--(6.695,6.931)--(6.697,6.937)--(6.699,6.943)--(6.701,6.949)--(6.703,6.955)%
  --(6.705,6.961)--(6.707,6.967)--(6.709,6.973)--(6.711,6.979)--(6.713,6.985)--(6.715,6.990)%
  --(6.718,6.996)--(6.720,7.001)--(6.722,7.007)--(6.724,7.012)--(6.726,7.017)--(6.728,7.023)%
  --(6.730,7.028)--(6.732,7.033)--(6.734,7.038)--(6.736,7.043)--(6.738,7.048)--(6.740,7.052)%
  --(6.742,7.057)--(6.744,7.062)--(6.747,7.066)--(6.749,7.071)--(6.751,7.075)--(6.753,7.080)%
  --(6.755,7.084)--(6.757,7.088)--(6.759,7.092)--(6.761,7.096)--(6.763,7.101)--(6.765,7.104)%
  --(6.767,7.108)--(6.769,7.112)--(6.771,7.116)--(6.773,7.120)--(6.775,7.123)--(6.778,7.127)%
  --(6.780,7.130)--(6.782,7.133)--(6.784,7.137)--(6.786,7.140)--(6.788,7.143)--(6.790,7.146)%
  --(6.792,7.149)--(6.794,7.152)--(6.796,7.155)--(6.798,7.158)--(6.800,7.161)--(6.802,7.163)%
  --(6.804,7.166)--(6.807,7.168)--(6.809,7.171)--(6.811,7.173)--(6.813,7.175)--(6.815,7.178)%
  --(6.817,7.180)--(6.819,7.182)--(6.821,7.184)--(6.823,7.186)--(6.825,7.187)--(6.827,7.189)%
  --(6.829,7.191)--(6.831,7.193)--(6.833,7.194)--(6.836,7.196)--(6.838,7.197)--(6.840,7.198)%
  --(6.842,7.199)--(6.844,7.201)--(6.846,7.202)--(6.848,7.203)--(6.850,7.204)--(6.852,7.205)%
  --(6.854,7.205)--(6.856,7.206)--(6.858,7.207)--(6.860,7.207)--(6.862,7.208)--(6.865,7.208)%
  --(6.867,7.209)--(6.869,7.209)--(6.871,7.209)--(6.873,7.209)--(6.875,7.209)--(6.877,7.209)%
  --(6.879,7.209)--(6.881,7.209)--(6.883,7.209)--(6.885,7.208)--(6.887,7.208)--(6.889,7.208)%
  --(6.891,7.207)--(6.894,7.206)--(6.896,7.206)--(6.898,7.205)--(6.900,7.204)--(6.902,7.203)%
  --(6.904,7.202)--(6.906,7.201)--(6.908,7.200)--(6.910,7.199)--(6.912,7.197)--(6.914,7.196)%
  --(6.916,7.195)--(6.918,7.193)--(6.920,7.192)--(6.923,7.190)--(6.925,7.188)--(6.927,7.186)%
  --(6.929,7.185)--(6.931,7.183)--(6.933,7.181)--(6.935,7.178)--(6.937,7.176)--(6.939,7.174)%
  --(6.941,7.172)--(6.943,7.169)--(6.945,7.167)--(6.947,7.164)--(6.949,7.162)--(6.952,7.159)%
  --(6.954,7.156)--(6.956,7.153)--(6.958,7.150)--(6.960,7.148)--(6.962,7.144)--(6.964,7.141)%
  --(6.966,7.138)--(6.968,7.135)--(6.970,7.132)--(6.972,7.128)--(6.974,7.125)--(6.976,7.121)%
  --(6.978,7.117)--(6.980,7.114)--(6.983,7.110)--(6.985,7.106)--(6.987,7.102)--(6.989,7.098)%
  --(6.991,7.094)--(6.993,7.090)--(6.995,7.086)--(6.997,7.082)--(6.999,7.077)--(7.001,7.073)%
  --(7.003,7.068)--(7.005,7.064)--(7.007,7.059)--(7.009,7.054)--(7.012,7.050)--(7.014,7.045)%
  --(7.016,7.040)--(7.018,7.035)--(7.020,7.030)--(7.022,7.025)--(7.024,7.020)--(7.026,7.014)%
  --(7.028,7.009)--(7.030,7.004)--(7.032,6.998)--(7.034,6.993)--(7.036,6.987)--(7.038,6.981)%
  --(7.041,6.976)--(7.043,6.970)--(7.045,6.964)--(7.047,6.958)--(7.049,6.952)--(7.051,6.946)%
  --(7.053,6.940)--(7.055,6.934)--(7.057,6.927)--(7.059,6.921)--(7.061,6.915)--(7.063,6.908)%
  --(7.065,6.902)--(7.067,6.895)--(7.070,6.888)--(7.072,6.882)--(7.074,6.875)--(7.076,6.868)%
  --(7.078,6.861)--(7.080,6.854)--(7.082,6.847)--(7.084,6.840)--(7.086,6.833)--(7.088,6.825)%
  --(7.090,6.818)--(7.092,6.811)--(7.094,6.803)--(7.096,6.796)--(7.099,6.788)--(7.101,6.781)%
  --(7.103,6.773)--(7.105,6.765)--(7.107,6.757)--(7.109,6.749)--(7.111,6.741)--(7.113,6.733)%
  --(7.115,6.725)--(7.117,6.717)--(7.119,6.709)--(7.121,6.701)--(7.123,6.693)--(7.125,6.684)%
  --(7.128,6.676)--(7.130,6.667)--(7.132,6.659)--(7.134,6.650)--(7.136,6.641)--(7.138,6.633)%
  --(7.140,6.624)--(7.142,6.615)--(7.144,6.606)--(7.146,6.597)--(7.148,6.588)--(7.150,6.579)%
  --(7.152,6.570)--(7.154,6.561)--(7.156,6.552)--(7.159,6.542)--(7.161,6.533)--(7.163,6.524)%
  --(7.165,6.514)--(7.167,6.505)--(7.169,6.495)--(7.171,6.486)--(7.173,6.476)--(7.175,6.466)%
  --(7.177,6.456)--(7.179,6.447)--(7.181,6.437)--(7.183,6.427)--(7.185,6.417)--(7.188,6.407)%
  --(7.190,6.397)--(7.192,6.387)--(7.194,6.376)--(7.196,6.366)--(7.198,6.356)--(7.200,6.345)%
  --(7.202,6.335)--(7.204,6.325)--(7.206,6.314)--(7.208,6.304)--(7.210,6.293)--(7.212,6.282)%
  --(7.214,6.272)--(7.217,6.261)--(7.219,6.250)--(7.221,6.239)--(7.223,6.228)--(7.225,6.217)%
  --(7.227,6.206)--(7.229,6.195)--(7.231,6.184)--(7.233,6.173)--(7.235,6.162)--(7.237,6.151)%
  --(7.239,6.140)--(7.241,6.128)--(7.243,6.117)--(7.246,6.106)--(7.248,6.094)--(7.250,6.083)%
  --(7.252,6.071)--(7.254,6.060)--(7.256,6.048)--(7.258,6.036)--(7.260,6.025)--(7.262,6.013)%
  --(7.264,6.001)--(7.266,5.989)--(7.268,5.978)--(7.270,5.966)--(7.272,5.954)--(7.275,5.942)%
  --(7.277,5.930)--(7.279,5.918)--(7.281,5.906)--(7.283,5.893)--(7.285,5.881)--(7.287,5.869)%
  --(7.289,5.857)--(7.291,5.845)--(7.293,5.832)--(7.295,5.820)--(7.297,5.808)--(7.299,5.795)%
  --(7.301,5.783)--(7.304,5.770)--(7.306,5.758)--(7.308,5.745)--(7.310,5.733)--(7.312,5.720)%
  --(7.314,5.707)--(7.316,5.695)--(7.318,5.682)--(7.320,5.669)--(7.322,5.656)--(7.324,5.644)%
  --(7.326,5.631)--(7.328,5.618)--(7.330,5.605)--(7.332,5.592)--(7.335,5.579)--(7.337,5.566)%
  --(7.339,5.553)--(7.341,5.540)--(7.343,5.527)--(7.345,5.514)--(7.347,5.501)--(7.349,5.487)%
  --(7.351,5.474)--(7.353,5.461)--(7.355,5.448)--(7.357,5.434)--(7.359,5.421)--(7.361,5.408)%
  --(7.364,5.394)--(7.366,5.381)--(7.368,5.368)--(7.370,5.354)--(7.372,5.341)--(7.374,5.327)%
  --(7.376,5.314)--(7.378,5.300)--(7.380,5.287)--(7.382,5.273)--(7.384,5.260)--(7.386,5.246)%
  --(7.388,5.232)--(7.390,5.219)--(7.393,5.205)--(7.395,5.191)--(7.397,5.178)--(7.399,5.164)%
  --(7.401,5.150)--(7.403,5.136)--(7.405,5.123)--(7.407,5.109)--(7.409,5.095)--(7.411,5.081)%
  --(7.413,5.067)--(7.415,5.053)--(7.417,5.040)--(7.419,5.026)--(7.422,5.012)--(7.424,4.998)%
  --(7.426,4.984)--(7.428,4.970)--(7.430,4.956)--(7.432,4.942)--(7.434,4.928)--(7.436,4.914)%
  --(7.438,4.900)--(7.440,4.886)--(7.442,4.872)--(7.444,4.858)--(7.446,4.844)--(7.448,4.830)%
  --(7.451,4.815)--(7.453,4.801)--(7.455,4.787)--(7.457,4.773)--(7.459,4.759)--(7.461,4.745)%
  --(7.463,4.731)--(7.465,4.717)--(7.467,4.702)--(7.469,4.688)--(7.471,4.674)--(7.473,4.660)%
  --(7.475,4.646)--(7.477,4.631)--(7.480,4.617)--(7.482,4.603)--(7.484,4.589)--(7.486,4.575)%
  --(7.488,4.560)--(7.490,4.546)--(7.492,4.532)--(7.494,4.518)--(7.496,4.503)--(7.498,4.489)%
  --(7.500,4.475)--(7.502,4.461)--(7.504,4.446)--(7.506,4.432)--(7.508,4.418)--(7.511,4.404)%
  --(7.513,4.389)--(7.515,4.375)--(7.517,4.361)--(7.519,4.347)--(7.521,4.332)--(7.523,4.318)%
  --(7.525,4.304)--(7.527,4.290)--(7.529,4.276)--(7.531,4.261)--(7.533,4.247)--(7.535,4.233)%
  --(7.537,4.219)--(7.540,4.204)--(7.542,4.190)--(7.544,4.176)--(7.546,4.162)--(7.548,4.148)%
  --(7.550,4.133)--(7.552,4.119)--(7.554,4.105)--(7.556,4.091)--(7.558,4.077)--(7.560,4.062)%
  --(7.562,4.048)--(7.564,4.034)--(7.566,4.020)--(7.569,4.006)--(7.571,3.992)--(7.573,3.978)%
  --(7.575,3.964)--(7.577,3.949)--(7.579,3.935)--(7.581,3.921)--(7.583,3.907)--(7.585,3.893)%
  --(7.587,3.879)--(7.589,3.865)--(7.591,3.851)--(7.593,3.837)--(7.595,3.823)--(7.598,3.809)%
  --(7.600,3.795)--(7.602,3.781)--(7.604,3.767)--(7.606,3.753)--(7.608,3.739)--(7.610,3.726)%
  --(7.612,3.712)--(7.614,3.698)--(7.616,3.684)--(7.618,3.670)--(7.620,3.656)--(7.622,3.642)%
  --(7.624,3.629)--(7.627,3.615)--(7.629,3.601)--(7.631,3.587)--(7.633,3.574)--(7.635,3.560)%
  --(7.637,3.546)--(7.639,3.533)--(7.641,3.519)--(7.643,3.505)--(7.645,3.492)--(7.647,3.478)%
  --(7.649,3.465)--(7.651,3.451)--(7.653,3.438)--(7.656,3.424)--(7.658,3.411)--(7.660,3.397)%
  --(7.662,3.384)--(7.664,3.370)--(7.666,3.357)--(7.668,3.344)--(7.670,3.330)--(7.672,3.317)%
  --(7.674,3.304)--(7.676,3.290)--(7.678,3.277)--(7.680,3.264)--(7.682,3.251)--(7.684,3.238)%
  --(7.687,3.224)--(7.689,3.211)--(7.691,3.198)--(7.693,3.185)--(7.695,3.172)--(7.697,3.159)%
  --(7.699,3.146)--(7.701,3.133)--(7.703,3.120)--(7.705,3.107)--(7.707,3.094)--(7.709,3.081)%
  --(7.711,3.069)--(7.713,3.056)--(7.716,3.043)--(7.718,3.030)--(7.720,3.018)--(7.722,3.005)%
  --(7.724,2.992)--(7.726,2.980)--(7.728,2.967)--(7.730,2.954)--(7.732,2.942)--(7.734,2.929)%
  --(7.736,2.917)--(7.738,2.904)--(7.740,2.892)--(7.742,2.880)--(7.745,2.867)--(7.747,2.855)%
  --(7.749,2.843)--(7.751,2.830)--(7.753,2.818)--(7.755,2.806)--(7.757,2.794)--(7.759,2.782)%
  --(7.761,2.770)--(7.763,2.757)--(7.765,2.745)--(7.767,2.733)--(7.769,2.721)--(7.771,2.710)%
  --(7.774,2.698)--(7.776,2.686)--(7.778,2.674)--(7.780,2.662)--(7.782,2.650)--(7.784,2.639)%
  --(7.786,2.627)--(7.788,2.615)--(7.790,2.604)--(7.792,2.592)--(7.794,2.581)--(7.796,2.569)%
  --(7.798,2.558)--(7.800,2.546)--(7.803,2.535)--(7.805,2.523)--(7.807,2.512)--(7.809,2.501)%
  --(7.811,2.490)--(7.813,2.478)--(7.815,2.467)--(7.817,2.456)--(7.819,2.445)--(7.821,2.434)%
  --(7.823,2.423)--(7.825,2.412)--(7.827,2.401)--(7.829,2.390)--(7.832,2.379)--(7.834,2.368)%
  --(7.836,2.358)--(7.838,2.347)--(7.840,2.336)--(7.842,2.325)--(7.844,2.315)--(7.846,2.304)%
  --(7.848,2.294)--(7.850,2.283)--(7.852,2.273)--(7.854,2.262)--(7.856,2.252)--(7.858,2.242)%
  --(7.861,2.231)--(7.863,2.221)--(7.865,2.211)--(7.867,2.201)--(7.869,2.190)--(7.871,2.180)%
  --(7.873,2.170)--(7.875,2.160)--(7.877,2.150)--(7.879,2.140)--(7.881,2.130)--(7.883,2.121)%
  --(7.885,2.111)--(7.887,2.101)--(7.889,2.091)--(7.892,2.082)--(7.894,2.072)--(7.896,2.062)%
  --(7.898,2.053)--(7.900,2.043)--(7.902,2.034)--(7.904,2.024)--(7.906,2.015)--(7.908,2.006)%
  --(7.910,1.996)--(7.912,1.987)--(7.914,1.978)--(7.916,1.969)--(7.918,1.960)--(7.921,1.951)%
  --(7.923,1.942)--(7.925,1.933)--(7.927,1.924)--(7.929,1.915)--(7.931,1.906)--(7.933,1.897)%
  --(7.935,1.888)--(7.937,1.880)--(7.939,1.871)--(7.941,1.862)--(7.943,1.854)--(7.945,1.845)%
  --(7.947,1.837)--(7.950,1.828)--(7.952,1.820)--(7.954,1.811)--(7.956,1.803)--(7.958,1.795)%
  --(7.960,1.786)--(7.962,1.778)--(7.964,1.770)--(7.966,1.762)--(7.968,1.754)--(7.970,1.746)%
  --(7.972,1.738)--(7.974,1.730)--(7.976,1.722)--(7.979,1.714)--(7.981,1.707)--(7.983,1.699)%
  --(7.985,1.691)--(7.987,1.683)--(7.989,1.676)--(7.991,1.668)--(7.993,1.661)--(7.995,1.653)%
  --(7.997,1.646)--(7.999,1.638)--(8.001,1.631)--(8.003,1.624)--(8.005,1.617)--(8.008,1.609)%
  --(8.010,1.602)--(8.012,1.595)--(8.014,1.588)--(8.016,1.581)--(8.018,1.574)--(8.020,1.567)%
  --(8.022,1.560)--(8.024,1.553)--(8.026,1.547)--(8.028,1.540)--(8.030,1.533)--(8.032,1.526)%
  --(8.034,1.520)--(8.037,1.513)--(8.039,1.507)--(8.041,1.500)--(8.043,1.494)--(8.045,1.487)%
  --(8.047,1.481)--(8.049,1.475)--(8.051,1.468)--(8.053,1.462)--(8.055,1.456)--(8.057,1.450)%
  --(8.059,1.444)--(8.061,1.438)--(8.063,1.432)--(8.065,1.426)--(8.068,1.420)--(8.070,1.414)%
  --(8.072,1.408)--(8.074,1.403)--(8.076,1.397)--(8.078,1.391)--(8.080,1.386)--(8.082,1.380)%
  --(8.084,1.374)--(8.086,1.369)--(8.088,1.363)--(8.090,1.358)--(8.092,1.353)--(8.094,1.347)%
  --(8.097,1.342)--(8.099,1.337)--(8.101,1.332)--(8.103,1.326)--(8.105,1.321)--(8.107,1.316)%
  --(8.109,1.311)--(8.111,1.306)--(8.113,1.301)--(8.115,1.296)--(8.117,1.292)--(8.119,1.287)%
  --(8.121,1.282)--(8.123,1.277)--(8.126,1.273)--(8.128,1.268)--(8.130,1.263)--(8.132,1.259)%
  --(8.134,1.254)--(8.136,1.250)--(8.138,1.245)--(8.140,1.241)--(8.142,1.237)--(8.144,1.232)%
  --(8.146,1.228)--(8.148,1.224)--(8.150,1.220)--(8.152,1.216)--(8.155,1.212)--(8.157,1.208)%
  --(8.159,1.204)--(8.161,1.200)--(8.163,1.196)--(8.165,1.192)--(8.167,1.188)--(8.169,1.184)%
  --(8.171,1.180)--(8.173,1.177)--(8.175,1.173)--(8.177,1.169)--(8.179,1.166)--(8.181,1.162)%
  --(8.184,1.159)--(8.186,1.155)--(8.188,1.152)--(8.190,1.148)--(8.192,1.145)--(8.194,1.142)%
  --(8.196,1.138)--(8.198,1.135)--(8.200,1.132)--(8.202,1.129)--(8.204,1.126)--(8.206,1.123)%
  --(8.208,1.120)--(8.210,1.117)--(8.213,1.114)--(8.215,1.111)--(8.217,1.108)--(8.219,1.105)%
  --(8.221,1.102)--(8.223,1.099)--(8.225,1.097)--(8.227,1.094)--(8.229,1.091)--(8.231,1.089)%
  --(8.233,1.086)--(8.235,1.084)--(8.237,1.081)--(8.239,1.079)--(8.241,1.076)--(8.244,1.074)%
  --(8.246,1.071)--(8.248,1.069)--(8.250,1.067)--(8.252,1.064)--(8.254,1.062)--(8.256,1.060)%
  --(8.258,1.058)--(8.260,1.056)--(8.262,1.054)--(8.264,1.052)--(8.266,1.050)--(8.268,1.048)%
  --(8.270,1.046)--(8.273,1.044)--(8.275,1.042)--(8.277,1.040)--(8.279,1.038)--(8.281,1.036)%
  --(8.283,1.035)--(8.285,1.033)--(8.287,1.031)--(8.289,1.030)--(8.291,1.028)--(8.293,1.026)%
  --(8.295,1.025)--(8.297,1.023)--(8.299,1.022)--(8.302,1.020)--(8.304,1.019)--(8.306,1.018)%
  --(8.308,1.016)--(8.310,1.015)--(8.312,1.014)--(8.314,1.012)--(8.316,1.011)--(8.318,1.010)%
  --(8.320,1.009)--(8.322,1.008)--(8.324,1.007)--(8.326,1.006)--(8.328,1.004)--(8.331,1.003)%
  --(8.333,1.002)--(8.335,1.001)--(8.337,1.001)--(8.339,1.000)--(8.341,0.999)--(8.343,0.998)%
  --(8.345,0.997)--(8.347,0.996)--(8.349,0.996)--(8.351,0.995)--(8.353,0.994)--(8.355,0.993)%
  --(8.357,0.993)--(8.360,0.992)--(8.362,0.992)--(8.364,0.991)--(8.366,0.990)--(8.368,0.990)%
  --(8.370,0.989)--(8.372,0.989)--(8.374,0.989)--(8.376,0.988)--(8.378,0.988)--(8.380,0.987)%
  --(8.382,0.987)--(8.384,0.987)--(8.386,0.987)--(8.389,0.986)--(8.391,0.986)--(8.393,0.986)%
  --(8.395,0.986)--(8.397,0.985)--(8.399,0.985)--(8.401,0.985)--(8.403,0.985)--(8.405,0.985)%
  --(8.407,0.985)--(8.409,0.985)--(8.411,0.985)--(8.413,0.985)--(8.415,0.985)--(8.417,0.985)%
  --(8.420,0.985)--(8.422,0.985)--(8.424,0.985)--(8.426,0.986)--(8.428,0.986)--(8.430,0.986)%
  --(8.432,0.986)--(8.434,0.986)--(8.436,0.987)--(8.438,0.987)--(8.440,0.987)--(8.442,0.988)%
  --(8.444,0.988)--(8.446,0.988)--(8.449,0.989)--(8.451,0.989)--(8.453,0.990)--(8.455,0.990)%
  --(8.457,0.990)--(8.459,0.991)--(8.461,0.991)--(8.463,0.992)--(8.465,0.992)--(8.467,0.993)%
  --(8.469,0.994)--(8.471,0.994)--(8.473,0.995)--(8.475,0.995)--(8.478,0.996)--(8.480,0.997)%
  --(8.482,0.997)--(8.484,0.998)--(8.486,0.999)--(8.488,0.999)--(8.490,1.000)--(8.492,1.001)%
  --(8.494,1.002)--(8.496,1.002)--(8.498,1.003)--(8.500,1.004)--(8.502,1.005)--(8.504,1.006)%
  --(8.507,1.007)--(8.509,1.007)--(8.511,1.008)--(8.513,1.009)--(8.515,1.010)--(8.517,1.011)%
  --(8.519,1.012)--(8.521,1.013)--(8.523,1.014)--(8.525,1.015)--(8.527,1.016)--(8.529,1.017)%
  --(8.531,1.018)--(8.533,1.019)--(8.536,1.020)--(8.538,1.021)--(8.540,1.022)--(8.542,1.023)%
  --(8.544,1.024)--(8.546,1.025)--(8.548,1.026)--(8.550,1.027)--(8.552,1.028)--(8.554,1.030)%
  --(8.556,1.031)--(8.558,1.032)--(8.560,1.033)--(8.562,1.034)--(8.565,1.035)--(8.567,1.037)%
  --(8.569,1.038)--(8.571,1.039)--(8.573,1.040)--(8.575,1.041)--(8.577,1.043)--(8.579,1.044)%
  --(8.581,1.045)--(8.583,1.046)--(8.585,1.048)--(8.587,1.049)--(8.589,1.050)--(8.591,1.051)%
  --(8.593,1.053)--(8.596,1.054)--(8.598,1.055)--(8.600,1.057)--(8.602,1.058)--(8.604,1.059)%
  --(8.606,1.061)--(8.608,1.062)--(8.610,1.063)--(8.612,1.065)--(8.614,1.066)--(8.616,1.067)%
  --(8.618,1.069)--(8.620,1.070)--(8.622,1.071)--(8.625,1.073)--(8.627,1.074)--(8.629,1.076)%
  --(8.631,1.077)--(8.633,1.078)--(8.635,1.080)--(8.637,1.081)--(8.639,1.083)--(8.641,1.084)%
  --(8.643,1.085)--(8.645,1.087)--(8.647,1.088)--(8.649,1.090)--(8.651,1.091)--(8.654,1.092)%
  --(8.656,1.094)--(8.658,1.095)--(8.660,1.097)--(8.662,1.098)--(8.664,1.100)--(8.666,1.101)%
  --(8.668,1.102)--(8.670,1.104)--(8.672,1.105)--(8.674,1.107)--(8.676,1.108)--(8.678,1.110)%
  --(8.680,1.111)--(8.683,1.113)--(8.685,1.114)--(8.687,1.115)--(8.689,1.117)--(8.691,1.118)%
  --(8.693,1.120)--(8.695,1.121)--(8.697,1.123)--(8.699,1.124)--(8.701,1.126)--(8.703,1.127)%
  --(8.705,1.128)--(8.707,1.130)--(8.709,1.131)--(8.712,1.133)--(8.714,1.134)--(8.716,1.136)%
  --(8.718,1.137)--(8.720,1.138)--(8.722,1.140)--(8.724,1.141)--(8.726,1.143)--(8.728,1.144)%
  --(8.730,1.145)--(8.732,1.147)--(8.734,1.148)--(8.736,1.150)--(8.738,1.151)--(8.741,1.153)%
  --(8.743,1.154)--(8.745,1.155)--(8.747,1.157)--(8.749,1.158)--(8.751,1.159)--(8.753,1.161)%
  --(8.755,1.162)--(8.757,1.164)--(8.759,1.165)--(8.761,1.166)--(8.763,1.168)--(8.765,1.169)%
  --(8.767,1.170)--(8.770,1.172)--(8.772,1.173)--(8.774,1.174)--(8.776,1.176)--(8.778,1.177)%
  --(8.780,1.178)--(8.782,1.180)--(8.784,1.181)--(8.786,1.182)--(8.788,1.184)--(8.790,1.185)%
  --(8.792,1.186)--(8.794,1.187)--(8.796,1.189)--(8.798,1.190)--(8.801,1.191)--(8.803,1.192)%
  --(8.805,1.194)--(8.807,1.195)--(8.809,1.196)--(8.811,1.197)--(8.813,1.199)--(8.815,1.200)%
  --(8.817,1.201)--(8.819,1.202)--(8.821,1.203)--(8.823,1.205)--(8.825,1.206)--(8.827,1.207)%
  --(8.830,1.208)--(8.832,1.209)--(8.834,1.210)--(8.836,1.212)--(8.838,1.213)--(8.840,1.214)%
  --(8.842,1.215)--(8.844,1.216)--(8.846,1.217)--(8.848,1.218)--(8.850,1.219)--(8.852,1.220)%
  --(8.854,1.222)--(8.856,1.223)--(8.859,1.224)--(8.861,1.225)--(8.863,1.226)--(8.865,1.227)%
  --(8.867,1.228)--(8.869,1.229)--(8.871,1.230)--(8.873,1.231)--(8.875,1.232)--(8.877,1.233)%
  --(8.879,1.234)--(8.881,1.235)--(8.883,1.236)--(8.885,1.237)--(8.888,1.237)--(8.890,1.238)%
  --(8.892,1.239)--(8.894,1.240)--(8.896,1.241)--(8.898,1.242)--(8.900,1.243)--(8.902,1.244)%
  --(8.904,1.245)--(8.906,1.245)--(8.908,1.246)--(8.910,1.247)--(8.912,1.248)--(8.914,1.249)%
  --(8.917,1.250)--(8.919,1.250)--(8.921,1.251)--(8.923,1.252)--(8.925,1.253)--(8.927,1.253)%
  --(8.929,1.254)--(8.931,1.255)--(8.933,1.256)--(8.935,1.256)--(8.937,1.257)--(8.939,1.258)%
  --(8.941,1.258)--(8.943,1.259)--(8.946,1.260)--(8.948,1.260)--(8.950,1.261)--(8.952,1.261)%
  --(8.954,1.262)--(8.956,1.263)--(8.958,1.263)--(8.960,1.264)--(8.962,1.264)--(8.964,1.265)%
  --(8.966,1.266)--(8.968,1.266)--(8.970,1.267)--(8.972,1.267)--(8.974,1.268)--(8.977,1.268)%
  --(8.979,1.269)--(8.981,1.269)--(8.983,1.270)--(8.985,1.270)--(8.987,1.270)--(8.989,1.271)%
  --(8.991,1.271)--(8.993,1.272)--(8.995,1.272)--(8.997,1.272)--(8.999,1.273)--(9.001,1.273)%
  --(9.003,1.274)--(9.006,1.274)--(9.008,1.274)--(9.010,1.275)--(9.012,1.275)--(9.014,1.275)%
  --(9.016,1.275)--(9.018,1.276)--(9.020,1.276)--(9.022,1.276)--(9.024,1.277)--(9.026,1.277)%
  --(9.028,1.277)--(9.030,1.277)--(9.032,1.277)--(9.035,1.278)--(9.037,1.278)--(9.039,1.278)%
  --(9.041,1.278)--(9.043,1.278)--(9.045,1.278)--(9.047,1.279)--(9.049,1.279)--(9.051,1.279)%
  --(9.053,1.279)--(9.055,1.279)--(9.057,1.279)--(9.059,1.279)--(9.061,1.279)--(9.064,1.279)%
  --(9.066,1.279)--(9.068,1.279)--(9.070,1.279)--(9.072,1.279)--(9.074,1.279)--(9.076,1.279)%
  --(9.078,1.279)--(9.080,1.279)--(9.082,1.279)--(9.084,1.279)--(9.086,1.279)--(9.088,1.279)%
  --(9.090,1.279)--(9.093,1.279)--(9.095,1.279)--(9.097,1.278)--(9.099,1.278)--(9.101,1.278)%
  --(9.103,1.278)--(9.105,1.278)--(9.107,1.278)--(9.109,1.277)--(9.111,1.277)--(9.113,1.277)%
  --(9.115,1.277)--(9.117,1.277)--(9.119,1.276)--(9.122,1.276)--(9.124,1.276)--(9.126,1.276)%
  --(9.128,1.275)--(9.130,1.275)--(9.132,1.275)--(9.134,1.274)--(9.136,1.274)--(9.138,1.274)%
  --(9.140,1.273)--(9.142,1.273)--(9.144,1.273)--(9.146,1.272)--(9.148,1.272)--(9.150,1.272)%
  --(9.153,1.271)--(9.155,1.271)--(9.157,1.270)--(9.159,1.270)--(9.161,1.270)--(9.163,1.269)%
  --(9.165,1.269)--(9.167,1.268)--(9.169,1.268)--(9.171,1.267)--(9.173,1.267)--(9.175,1.266)%
  --(9.177,1.266)--(9.179,1.265)--(9.182,1.265)--(9.184,1.264)--(9.186,1.264)--(9.188,1.263)%
  --(9.190,1.263)--(9.192,1.262)--(9.194,1.262)--(9.196,1.261)--(9.198,1.260)--(9.200,1.260)%
  --(9.202,1.259)--(9.204,1.259)--(9.206,1.258)--(9.208,1.257)--(9.211,1.257)--(9.213,1.256)%
  --(9.215,1.255)--(9.217,1.255)--(9.219,1.254)--(9.221,1.253)--(9.223,1.253)--(9.225,1.252)%
  --(9.227,1.251)--(9.229,1.251)--(9.231,1.250)--(9.233,1.249)--(9.235,1.249)--(9.237,1.248)%
  --(9.240,1.247)--(9.242,1.246)--(9.244,1.246)--(9.246,1.245)--(9.248,1.244)--(9.250,1.243)%
  --(9.252,1.242)--(9.254,1.242)--(9.256,1.241)--(9.258,1.240)--(9.260,1.239)--(9.262,1.238)%
  --(9.264,1.238)--(9.266,1.237)--(9.269,1.236)--(9.271,1.235)--(9.273,1.234)--(9.275,1.233)%
  --(9.277,1.233)--(9.279,1.232)--(9.281,1.231)--(9.283,1.230)--(9.285,1.229)--(9.287,1.228)%
  --(9.289,1.227)--(9.291,1.226)--(9.293,1.226)--(9.295,1.225)--(9.298,1.224)--(9.300,1.223)%
  --(9.302,1.222)--(9.304,1.221)--(9.306,1.220)--(9.308,1.219)--(9.310,1.218)--(9.312,1.217)%
  --(9.314,1.216)--(9.316,1.215)--(9.318,1.214)--(9.320,1.213)--(9.322,1.212)--(9.324,1.211)%
  --(9.326,1.210)--(9.329,1.209)--(9.331,1.209)--(9.333,1.208)--(9.335,1.207)--(9.337,1.206)%
  --(9.339,1.205)--(9.341,1.204)--(9.343,1.202)--(9.345,1.201)--(9.347,1.200)--(9.349,1.199)%
  --(9.351,1.198)--(9.353,1.197)--(9.355,1.196)--(9.358,1.195)--(9.360,1.194)--(9.362,1.193)%
  --(9.364,1.192)--(9.366,1.191)--(9.368,1.190)--(9.370,1.189)--(9.372,1.188)--(9.374,1.187)%
  --(9.376,1.186)--(9.378,1.185)--(9.380,1.184)--(9.382,1.183)--(9.384,1.182)--(9.387,1.180)%
  --(9.389,1.179)--(9.391,1.178)--(9.393,1.177)--(9.395,1.176)--(9.397,1.175)--(9.399,1.174)%
  --(9.401,1.173)--(9.403,1.172)--(9.405,1.171)--(9.407,1.170)--(9.409,1.169)--(9.411,1.167)%
  --(9.413,1.166)--(9.416,1.165)--(9.418,1.164)--(9.420,1.163)--(9.422,1.162)--(9.424,1.161)%
  --(9.426,1.160)--(9.428,1.159)--(9.430,1.157)--(9.432,1.156)--(9.434,1.155)--(9.436,1.154)%
  --(9.438,1.153)--(9.440,1.152)--(9.442,1.151)--(9.445,1.150)--(9.447,1.149)--(9.449,1.147)%
  --(9.451,1.146)--(9.453,1.145)--(9.455,1.144)--(9.457,1.143)--(9.459,1.142)--(9.461,1.141)%
  --(9.463,1.140)--(9.465,1.139)--(9.467,1.137)--(9.469,1.136)--(9.471,1.135)--(9.474,1.134)%
  --(9.476,1.133)--(9.478,1.132)--(9.480,1.131)--(9.482,1.130)--(9.484,1.129)--(9.486,1.127)%
  --(9.488,1.126)--(9.490,1.125)--(9.492,1.124)--(9.494,1.123)--(9.496,1.122)--(9.498,1.121)%
  --(9.500,1.120)--(9.502,1.119)--(9.505,1.118)--(9.507,1.116)--(9.509,1.115)--(9.511,1.114)%
  --(9.513,1.113)--(9.515,1.112)--(9.517,1.111)--(9.519,1.110)--(9.521,1.109)--(9.523,1.108)%
  --(9.525,1.107)--(9.527,1.106)--(9.529,1.105)--(9.531,1.103)--(9.534,1.102)--(9.536,1.101)%
  --(9.538,1.100)--(9.540,1.099)--(9.542,1.098)--(9.544,1.097)--(9.546,1.096)--(9.548,1.095)%
  --(9.550,1.094)--(9.552,1.093)--(9.554,1.092)--(9.556,1.091)--(9.558,1.090)--(9.560,1.089)%
  --(9.563,1.088)--(9.565,1.087)--(9.567,1.086)--(9.569,1.085)--(9.571,1.084)--(9.573,1.083)%
  --(9.575,1.082)--(9.577,1.080)--(9.579,1.079)--(9.581,1.078)--(9.583,1.077)--(9.585,1.076)%
  --(9.587,1.075)--(9.589,1.075)--(9.592,1.074)--(9.594,1.073)--(9.596,1.072)--(9.598,1.071)%
  --(9.600,1.070)--(9.602,1.069)--(9.604,1.068)--(9.606,1.067)--(9.608,1.066)--(9.610,1.065)%
  --(9.612,1.064)--(9.614,1.063)--(9.616,1.062)--(9.618,1.061)--(9.621,1.060)--(9.623,1.059)%
  --(9.625,1.058)--(9.627,1.057)--(9.629,1.056)--(9.631,1.056)--(9.633,1.055)--(9.635,1.054)%
  --(9.637,1.053)--(9.639,1.052)--(9.641,1.051)--(9.643,1.050)--(9.645,1.049)--(9.647,1.048)%
  --(9.650,1.048)--(9.652,1.047)--(9.654,1.046)--(9.656,1.045)--(9.658,1.044)--(9.660,1.043)%
  --(9.662,1.043)--(9.664,1.042)--(9.666,1.041)--(9.668,1.040)--(9.670,1.039)--(9.672,1.038)%
  --(9.674,1.038)--(9.676,1.037)--(9.678,1.036)--(9.681,1.035)--(9.683,1.035)--(9.685,1.034)%
  --(9.687,1.033)--(9.689,1.032)--(9.691,1.031)--(9.693,1.031)--(9.695,1.030)--(9.697,1.029)%
  --(9.699,1.028)--(9.701,1.028)--(9.703,1.027)--(9.705,1.026)--(9.707,1.026)--(9.710,1.025)%
  --(9.712,1.024)--(9.714,1.023)--(9.716,1.023)--(9.718,1.022)--(9.720,1.021)--(9.722,1.021)%
  --(9.724,1.020)--(9.726,1.019)--(9.728,1.019)--(9.730,1.018)--(9.732,1.018)--(9.734,1.017)%
  --(9.736,1.016)--(9.739,1.016)--(9.741,1.015)--(9.743,1.014)--(9.745,1.014)--(9.747,1.013)%
  --(9.749,1.013)--(9.751,1.012)--(9.753,1.011)--(9.755,1.011)--(9.757,1.010)--(9.759,1.010)%
  --(9.761,1.009)--(9.763,1.009)--(9.765,1.008)--(9.768,1.008)--(9.770,1.007)--(9.772,1.006)%
  --(9.774,1.006)--(9.776,1.005)--(9.778,1.005)--(9.780,1.004)--(9.782,1.004)--(9.784,1.003)%
  --(9.786,1.003)--(9.788,1.003)--(9.790,1.002)--(9.792,1.002)--(9.794,1.001)--(9.797,1.001)%
  --(9.799,1.000)--(9.801,1.000)--(9.803,0.999)--(9.805,0.999)--(9.807,0.999)--(9.809,0.998)%
  --(9.811,0.998)--(9.813,0.997)--(9.815,0.997)--(9.817,0.997)--(9.819,0.996)--(9.821,0.996)%
  --(9.823,0.995)--(9.826,0.995)--(9.828,0.995)--(9.830,0.994)--(9.832,0.994)--(9.834,0.994)%
  --(9.836,0.993)--(9.838,0.993)--(9.840,0.993)--(9.842,0.992)--(9.844,0.992)--(9.846,0.992)%
  --(9.848,0.992)--(9.850,0.991)--(9.852,0.991)--(9.855,0.991)--(9.857,0.990)--(9.859,0.990)%
  --(9.861,0.990)--(9.863,0.990)--(9.865,0.989)--(9.867,0.989)--(9.869,0.989)--(9.871,0.989)%
  --(9.873,0.989)--(9.875,0.988)--(9.877,0.988)--(9.879,0.988)--(9.881,0.988)--(9.883,0.988)%
  --(9.886,0.987)--(9.888,0.987)--(9.890,0.987)--(9.892,0.987)--(9.894,0.987)--(9.896,0.987)%
  --(9.898,0.987)--(9.900,0.986)--(9.902,0.986)--(9.904,0.986)--(9.906,0.986)--(9.908,0.986)%
  --(9.910,0.986)--(9.912,0.986)--(9.915,0.986)--(9.917,0.986)--(9.919,0.985)--(9.921,0.985)%
  --(9.923,0.985)--(9.925,0.985)--(9.927,0.985)--(9.929,0.985)--(9.931,0.985)--(9.933,0.985)%
  --(9.935,0.985)--(9.937,0.985)--(9.939,0.985)--(9.941,0.985)--(9.944,0.985)--(9.946,0.985)%
  --(9.948,0.985)--(9.950,0.985)--(9.952,0.985)--(9.954,0.985)--(9.956,0.985)--(9.958,0.985)%
  --(9.960,0.985)--(9.962,0.985)--(9.964,0.985)--(9.966,0.985)--(9.968,0.985)--(9.970,0.985)%
  --(9.973,0.985)--(9.975,0.986)--(9.977,0.986)--(9.979,0.986)--(9.981,0.986)--(9.983,0.986)%
  --(9.985,0.986)--(9.987,0.986)--(9.989,0.986)--(9.991,0.986)--(9.993,0.986)--(9.995,0.987)%
  --(9.997,0.987)--(9.999,0.987)--(10.002,0.987)--(10.004,0.987)--(10.006,0.987)--(10.008,0.987)%
  --(10.010,0.988)--(10.012,0.988)--(10.014,0.988)--(10.016,0.988)--(10.018,0.988)--(10.020,0.989)%
  --(10.022,0.989)--(10.024,0.989)--(10.026,0.989)--(10.028,0.989)--(10.031,0.989)--(10.033,0.990)%
  --(10.035,0.990)--(10.037,0.990)--(10.039,0.990)--(10.041,0.991)--(10.043,0.991)--(10.045,0.991)%
  --(10.047,0.991)--(10.049,0.992)--(10.051,0.992)--(10.053,0.992)--(10.055,0.992)--(10.057,0.993)%
  --(10.059,0.993)--(10.062,0.993)--(10.064,0.993)--(10.066,0.994)--(10.068,0.994)--(10.070,0.994)%
  --(10.072,0.995)--(10.074,0.995)--(10.076,0.995)--(10.078,0.995)--(10.080,0.996)--(10.082,0.996)%
  --(10.084,0.996)--(10.086,0.997)--(10.088,0.997)--(10.091,0.997)--(10.093,0.998)--(10.095,0.998)%
  --(10.097,0.998)--(10.099,0.999)--(10.101,0.999)--(10.103,0.999)--(10.105,1.000)--(10.107,1.000)%
  --(10.109,1.000)--(10.111,1.001)--(10.113,1.001)--(10.115,1.001)--(10.117,1.002)--(10.120,1.002)%
  --(10.122,1.003)--(10.124,1.003)--(10.126,1.003)--(10.128,1.004)--(10.130,1.004)--(10.132,1.005)%
  --(10.134,1.005)--(10.136,1.005)--(10.138,1.006)--(10.140,1.006)--(10.142,1.006)--(10.144,1.007)%
  --(10.146,1.007)--(10.149,1.008)--(10.151,1.008)--(10.153,1.008)--(10.155,1.009)--(10.157,1.009)%
  --(10.159,1.010)--(10.161,1.010)--(10.163,1.011)--(10.165,1.011)--(10.167,1.011)--(10.169,1.012)%
  --(10.171,1.012)--(10.173,1.013)--(10.175,1.013)--(10.178,1.014)--(10.180,1.014)--(10.182,1.014)%
  --(10.184,1.015)--(10.186,1.015)--(10.188,1.016)--(10.190,1.016)--(10.192,1.017)--(10.194,1.017)%
  --(10.196,1.018)--(10.198,1.018)--(10.200,1.018)--(10.202,1.019)--(10.204,1.019)--(10.207,1.020)%
  --(10.209,1.020)--(10.211,1.021)--(10.213,1.021)--(10.215,1.022)--(10.217,1.022)--(10.219,1.023)%
  --(10.221,1.023)--(10.223,1.023)--(10.225,1.024)--(10.227,1.024)--(10.229,1.025)--(10.231,1.025)%
  --(10.233,1.026)--(10.235,1.026)--(10.238,1.027)--(10.240,1.027)--(10.242,1.028)--(10.244,1.028)%
  --(10.246,1.029)--(10.248,1.029)--(10.250,1.030)--(10.252,1.030)--(10.254,1.031)--(10.256,1.031)%
  --(10.258,1.031)--(10.260,1.032)--(10.262,1.032)--(10.264,1.033)--(10.267,1.033)--(10.269,1.034)%
  --(10.271,1.034)--(10.273,1.035)--(10.275,1.035)--(10.277,1.036)--(10.279,1.036)--(10.281,1.037)%
  --(10.283,1.037)--(10.285,1.038)--(10.287,1.038)--(10.289,1.039)--(10.291,1.039)--(10.293,1.039)%
  --(10.296,1.040)--(10.298,1.040)--(10.300,1.041)--(10.302,1.041)--(10.304,1.042)--(10.306,1.042)%
  --(10.308,1.043)--(10.310,1.043)--(10.312,1.044)--(10.314,1.044)--(10.316,1.045)--(10.318,1.045)%
  --(10.320,1.045)--(10.322,1.046)--(10.325,1.046)--(10.327,1.047)--(10.329,1.047)--(10.331,1.048)%
  --(10.333,1.048)--(10.335,1.049)--(10.337,1.049)--(10.339,1.050)--(10.341,1.050)--(10.343,1.050)%
  --(10.345,1.051)--(10.347,1.051)--(10.349,1.052)--(10.351,1.052)--(10.354,1.053)--(10.356,1.053)%
  --(10.358,1.054)--(10.360,1.054)--(10.362,1.054)--(10.364,1.055)--(10.366,1.055)--(10.368,1.056)%
  --(10.370,1.056)--(10.372,1.057)--(10.374,1.057)--(10.376,1.057)--(10.378,1.058)--(10.380,1.058)%
  --(10.383,1.059)--(10.385,1.059)--(10.387,1.059)--(10.389,1.060)--(10.391,1.060)--(10.393,1.061)%
  --(10.395,1.061)--(10.397,1.061)--(10.399,1.062)--(10.401,1.062)--(10.403,1.063)--(10.405,1.063)%
  --(10.407,1.063)--(10.409,1.064)--(10.411,1.064)--(10.414,1.065)--(10.416,1.065)--(10.418,1.065)%
  --(10.420,1.066)--(10.422,1.066)--(10.424,1.066)--(10.426,1.067)--(10.428,1.067)--(10.430,1.068)%
  --(10.432,1.068)--(10.434,1.068)--(10.436,1.069)--(10.438,1.069)--(10.440,1.069)--(10.443,1.070)%
  --(10.445,1.070)--(10.447,1.070)--(10.449,1.071)--(10.451,1.071)--(10.453,1.071)--(10.455,1.072)%
  --(10.457,1.072)--(10.459,1.072)--(10.461,1.073)--(10.463,1.073)--(10.465,1.073)--(10.467,1.074)%
  --(10.469,1.074)--(10.472,1.074)--(10.474,1.075)--(10.476,1.075)--(10.478,1.075)--(10.480,1.075)%
  --(10.482,1.076)--(10.484,1.076)--(10.486,1.076)--(10.488,1.077)--(10.490,1.077)--(10.492,1.077)%
  --(10.494,1.077)--(10.496,1.078)--(10.498,1.078)--(10.501,1.078)--(10.503,1.078)--(10.505,1.079)%
  --(10.507,1.079)--(10.509,1.079)--(10.511,1.079)--(10.513,1.080)--(10.515,1.080)--(10.517,1.080)%
  --(10.519,1.080)--(10.521,1.081)--(10.523,1.081)--(10.525,1.081)--(10.527,1.081)--(10.530,1.082)%
  --(10.532,1.082)--(10.534,1.082)--(10.536,1.082)--(10.538,1.082)--(10.540,1.083)--(10.542,1.083)%
  --(10.544,1.083)--(10.546,1.083)--(10.548,1.083)--(10.550,1.083)--(10.552,1.084)--(10.554,1.084)%
  --(10.556,1.084)--(10.559,1.084)--(10.561,1.084)--(10.563,1.084)--(10.565,1.085)--(10.567,1.085)%
  --(10.569,1.085)--(10.571,1.085)--(10.573,1.085)--(10.575,1.085)--(10.577,1.085)--(10.579,1.086)%
  --(10.581,1.086)--(10.583,1.086)--(10.585,1.086)--(10.587,1.086)--(10.590,1.086)--(10.592,1.086)%
  --(10.594,1.086)--(10.596,1.087)--(10.598,1.087)--(10.600,1.087)--(10.602,1.087)--(10.604,1.087)%
  --(10.606,1.087)--(10.608,1.087)--(10.610,1.087)--(10.612,1.087)--(10.614,1.087)--(10.616,1.087)%
  --(10.619,1.087)--(10.621,1.087)--(10.623,1.087)--(10.625,1.088)--(10.627,1.088)--(10.629,1.088)%
  --(10.631,1.088)--(10.633,1.088)--(10.635,1.088)--(10.637,1.088)--(10.639,1.088)--(10.641,1.088)%
  --(10.643,1.088)--(10.645,1.088)--(10.648,1.088)--(10.650,1.088)--(10.652,1.088)--(10.654,1.088)%
  --(10.656,1.088)--(10.658,1.088)--(10.660,1.088)--(10.662,1.088)--(10.664,1.088)--(10.666,1.088)%
  --(10.668,1.088)--(10.670,1.088)--(10.672,1.088)--(10.674,1.088)--(10.677,1.087)--(10.679,1.087)%
  --(10.681,1.087)--(10.683,1.087)--(10.685,1.087)--(10.687,1.087)--(10.689,1.087)--(10.691,1.087)%
  --(10.693,1.087)--(10.695,1.087)--(10.697,1.087)--(10.699,1.087)--(10.701,1.087)--(10.703,1.087)%
  --(10.706,1.086)--(10.708,1.086)--(10.710,1.086)--(10.712,1.086)--(10.714,1.086)--(10.716,1.086)%
  --(10.718,1.086)--(10.720,1.086)--(10.722,1.086)--(10.724,1.085)--(10.726,1.085)--(10.728,1.085)%
  --(10.730,1.085)--(10.732,1.085)--(10.735,1.085)--(10.737,1.085)--(10.739,1.084)--(10.741,1.084)%
  --(10.743,1.084)--(10.745,1.084)--(10.747,1.084)--(10.749,1.084)--(10.751,1.084)--(10.753,1.083)%
  --(10.755,1.083)--(10.757,1.083)--(10.759,1.083)--(10.761,1.083)--(10.764,1.082)--(10.766,1.082)%
  --(10.768,1.082)--(10.770,1.082)--(10.772,1.082)--(10.774,1.081)--(10.776,1.081)--(10.778,1.081)%
  --(10.780,1.081)--(10.782,1.081)--(10.784,1.080)--(10.786,1.080)--(10.788,1.080)--(10.790,1.080)%
  --(10.792,1.079)--(10.795,1.079)--(10.797,1.079)--(10.799,1.079)--(10.801,1.079)--(10.803,1.078)%
  --(10.805,1.078)--(10.807,1.078)--(10.809,1.078)--(10.811,1.077)--(10.813,1.077)--(10.815,1.077)%
  --(10.817,1.077)--(10.819,1.076)--(10.821,1.076)--(10.824,1.076)--(10.826,1.075)--(10.828,1.075)%
  --(10.830,1.075)--(10.832,1.075)--(10.834,1.074)--(10.836,1.074)--(10.838,1.074)--(10.840,1.073)%
  --(10.842,1.073)--(10.844,1.073)--(10.846,1.073)--(10.848,1.072)--(10.850,1.072)--(10.853,1.072)%
  --(10.855,1.071)--(10.857,1.071)--(10.859,1.071)--(10.861,1.070)--(10.863,1.070)--(10.865,1.070)%
  --(10.867,1.069)--(10.869,1.069)--(10.871,1.069)--(10.873,1.068)--(10.875,1.068)--(10.877,1.068)%
  --(10.879,1.067)--(10.882,1.067)--(10.884,1.067)--(10.886,1.066)--(10.888,1.066)--(10.890,1.066)%
  --(10.892,1.065)--(10.894,1.065)--(10.896,1.065)--(10.898,1.064)--(10.900,1.064)--(10.902,1.064)%
  --(10.904,1.063)--(10.906,1.063)--(10.908,1.063)--(10.911,1.062)--(10.913,1.062)--(10.915,1.062)%
  --(10.917,1.061)--(10.919,1.061)--(10.921,1.060)--(10.923,1.060)--(10.925,1.060)--(10.927,1.059)%
  --(10.929,1.059)--(10.931,1.059)--(10.933,1.058)--(10.935,1.058)--(10.937,1.057)--(10.940,1.057)%
  --(10.942,1.057)--(10.944,1.056)--(10.946,1.056)--(10.948,1.056)--(10.950,1.055)--(10.952,1.055)%
  --(10.954,1.054)--(10.956,1.054)--(10.958,1.054)--(10.960,1.053)--(10.962,1.053)--(10.964,1.052)%
  --(10.966,1.052)--(10.968,1.052)--(10.971,1.051)--(10.973,1.051)--(10.975,1.051)--(10.977,1.050)%
  --(10.979,1.050)--(10.981,1.049)--(10.983,1.049)--(10.985,1.049)--(10.987,1.048)--(10.989,1.048)%
  --(10.991,1.047)--(10.993,1.047)--(10.995,1.047)--(10.997,1.046)--(11.000,1.046)--(11.002,1.045)%
  --(11.004,1.045)--(11.006,1.045)--(11.008,1.044)--(11.010,1.044)--(11.012,1.043)--(11.014,1.043)%
  --(11.016,1.042)--(11.018,1.042)--(11.020,1.042)--(11.022,1.041)--(11.024,1.041)--(11.026,1.040)%
  --(11.029,1.040)--(11.031,1.040)--(11.033,1.039)--(11.035,1.039)--(11.037,1.038)--(11.039,1.038)%
  --(11.041,1.038)--(11.043,1.037)--(11.045,1.037)--(11.047,1.036)--(11.049,1.036)--(11.051,1.036)%
  --(11.053,1.035)--(11.055,1.035)--(11.058,1.034)--(11.060,1.034)--(11.062,1.034)--(11.064,1.033)%
  --(11.066,1.033)--(11.068,1.032)--(11.070,1.032)--(11.072,1.032)--(11.074,1.031)--(11.076,1.031)%
  --(11.078,1.030)--(11.080,1.030)--(11.082,1.030)--(11.084,1.029)--(11.087,1.029)--(11.089,1.028)%
  --(11.091,1.028)--(11.093,1.028)--(11.095,1.027)--(11.097,1.027)--(11.099,1.026)--(11.101,1.026)%
  --(11.103,1.026)--(11.105,1.025)--(11.107,1.025)--(11.109,1.024)--(11.111,1.024)--(11.113,1.024)%
  --(11.116,1.023)--(11.118,1.023)--(11.120,1.022)--(11.122,1.022)--(11.124,1.022)--(11.126,1.021)%
  --(11.128,1.021)--(11.130,1.021)--(11.132,1.020)--(11.134,1.020)--(11.136,1.019)--(11.138,1.019)%
  --(11.140,1.019)--(11.142,1.018)--(11.144,1.018)--(11.147,1.018)--(11.149,1.017)--(11.151,1.017)%
  --(11.153,1.016)--(11.155,1.016)--(11.157,1.016)--(11.159,1.015)--(11.161,1.015)--(11.163,1.015)%
  --(11.165,1.014)--(11.167,1.014)--(11.169,1.014)--(11.171,1.013)--(11.173,1.013)--(11.176,1.012)%
  --(11.178,1.012)--(11.180,1.012)--(11.182,1.011)--(11.184,1.011)--(11.186,1.011)--(11.188,1.010)%
  --(11.190,1.010)--(11.192,1.010)--(11.194,1.009)--(11.196,1.009)--(11.198,1.009)--(11.200,1.008)%
  --(11.202,1.008)--(11.205,1.008)--(11.207,1.007)--(11.209,1.007)--(11.211,1.007)--(11.213,1.006)%
  --(11.215,1.006)--(11.217,1.006)--(11.219,1.006)--(11.221,1.005)--(11.223,1.005)--(11.225,1.005)%
  --(11.227,1.004)--(11.229,1.004)--(11.231,1.004)--(11.234,1.003)--(11.236,1.003)--(11.238,1.003)%
  --(11.240,1.002)--(11.242,1.002)--(11.244,1.002)--(11.246,1.002)--(11.248,1.001)--(11.250,1.001)%
  --(11.252,1.001)--(11.254,1.000)--(11.256,1.000)--(11.258,1.000)--(11.260,1.000)--(11.263,0.999)%
  --(11.265,0.999)--(11.267,0.999)--(11.269,0.999)--(11.271,0.998)--(11.273,0.998)--(11.275,0.998)%
  --(11.277,0.998)--(11.279,0.997)--(11.281,0.997)--(11.283,0.997)--(11.285,0.997)--(11.287,0.996)%
  --(11.289,0.996)--(11.292,0.996)--(11.294,0.996)--(11.296,0.995)--(11.298,0.995)--(11.300,0.995)%
  --(11.302,0.995)--(11.304,0.994)--(11.306,0.994)--(11.308,0.994)--(11.310,0.994)--(11.312,0.994)%
  --(11.314,0.993)--(11.316,0.993)--(11.318,0.993)--(11.320,0.993)--(11.323,0.993)--(11.325,0.992)%
  --(11.327,0.992)--(11.329,0.992)--(11.331,0.992)--(11.333,0.992)--(11.335,0.991)--(11.337,0.991)%
  --(11.339,0.991)--(11.341,0.991)--(11.343,0.991)--(11.345,0.991)--(11.347,0.990)--(11.349,0.990)%
  --(11.352,0.990)--(11.354,0.990)--(11.356,0.990)--(11.358,0.990)--(11.360,0.989)--(11.362,0.989)%
  --(11.364,0.989)--(11.366,0.989)--(11.368,0.989)--(11.370,0.989)--(11.372,0.989)--(11.374,0.988)%
  --(11.376,0.988)--(11.378,0.988)--(11.381,0.988)--(11.383,0.988)--(11.385,0.988)--(11.387,0.988)%
  --(11.389,0.988)--(11.391,0.987)--(11.393,0.987)--(11.395,0.987)--(11.397,0.987)--(11.399,0.987)%
  --(11.401,0.987)--(11.403,0.987)--(11.405,0.987)--(11.407,0.987)--(11.410,0.987)--(11.412,0.986)%
  --(11.414,0.986)--(11.416,0.986)--(11.418,0.986)--(11.420,0.986)--(11.422,0.986)--(11.424,0.986)%
  --(11.426,0.986)--(11.428,0.986)--(11.430,0.986)--(11.432,0.986)--(11.434,0.986)--(11.436,0.986)%
  --(11.439,0.986)--(11.441,0.985)--(11.443,0.985)--(11.445,0.985)--(11.447,0.985)--(11.449,0.985)%
  --(11.451,0.985)--(11.453,0.985)--(11.455,0.985)--(11.457,0.985)--(11.459,0.985)--(11.461,0.985)%
  --(11.463,0.985)--(11.465,0.985)--(11.468,0.985)--(11.470,0.985)--(11.472,0.985)--(11.474,0.985)%
  --(11.476,0.985)--(11.478,0.985)--(11.480,0.985)--(11.482,0.985)--(11.484,0.985)--(11.486,0.985)%
  --(11.488,0.985)--(11.490,0.985)--(11.492,0.985)--(11.494,0.985)--(11.496,0.985)--(11.499,0.985)%
  --(11.501,0.985)--(11.503,0.985)--(11.505,0.985)--(11.507,0.985)--(11.509,0.985)--(11.511,0.985)%
  --(11.513,0.985)--(11.515,0.985)--(11.517,0.985)--(11.519,0.985)--(11.521,0.985)--(11.523,0.986)%
  --(11.525,0.986)--(11.528,0.986)--(11.530,0.986)--(11.532,0.986)--(11.534,0.986)--(11.536,0.986)%
  --(11.538,0.986)--(11.540,0.986)--(11.542,0.986)--(11.544,0.986)--(11.546,0.986)--(11.548,0.986)%
  --(11.550,0.986)--(11.552,0.986)--(11.554,0.987)--(11.557,0.987)--(11.559,0.987)--(11.561,0.987)%
  --(11.563,0.987)--(11.565,0.987)--(11.567,0.987)--(11.569,0.987)--(11.571,0.987)--(11.573,0.987)%
  --(11.575,0.988)--(11.577,0.988)--(11.579,0.988)--(11.581,0.988)--(11.583,0.988)--(11.586,0.988)%
  --(11.588,0.988)--(11.590,0.988)--(11.592,0.988)--(11.594,0.989)--(11.596,0.989)--(11.598,0.989)%
  --(11.600,0.989)--(11.602,0.989)--(11.604,0.989)--(11.606,0.989)--(11.608,0.989)--(11.610,0.990)%
  --(11.612,0.990)--(11.615,0.990)--(11.617,0.990)--(11.619,0.990)--(11.621,0.990)--(11.623,0.990)%
  --(11.625,0.991)--(11.627,0.991)--(11.629,0.991)--(11.631,0.991)--(11.633,0.991)--(11.635,0.991)%
  --(11.637,0.992)--(11.639,0.992)--(11.641,0.992)--(11.644,0.992)--(11.646,0.992)--(11.648,0.992)%
  --(11.650,0.993)--(11.652,0.993)--(11.654,0.993)--(11.656,0.993)--(11.658,0.993)--(11.660,0.993)%
  --(11.662,0.994)--(11.664,0.994)--(11.666,0.994)--(11.668,0.994)--(11.670,0.994)--(11.673,0.994)%
  --(11.675,0.995)--(11.677,0.995)--(11.679,0.995)--(11.681,0.995)--(11.683,0.995)--(11.685,0.996)%
  --(11.687,0.996)--(11.689,0.996)--(11.691,0.996)--(11.693,0.996)--(11.695,0.997)--(11.697,0.997)%
  --(11.699,0.997)--(11.701,0.997)--(11.704,0.997)--(11.706,0.998)--(11.708,0.998)--(11.710,0.998)%
  --(11.712,0.998)--(11.714,0.998)--(11.716,0.999)--(11.718,0.999)--(11.720,0.999)--(11.722,0.999)%
  --(11.724,1.000)--(11.726,1.000)--(11.728,1.000)--(11.730,1.000)--(11.733,1.000)--(11.735,1.001)%
  --(11.737,1.001)--(11.739,1.001)--(11.741,1.001)--(11.743,1.001)--(11.745,1.002)--(11.747,1.002)%
  --(11.749,1.002)--(11.751,1.002)--(11.753,1.003)--(11.755,1.003)--(11.757,1.003)--(11.759,1.003)%
  --(11.762,1.003)--(11.764,1.004)--(11.766,1.004)--(11.768,1.004)--(11.770,1.004)--(11.772,1.005)%
  --(11.774,1.005)--(11.776,1.005)--(11.778,1.005)--(11.780,1.006)--(11.782,1.006)--(11.784,1.006)%
  --(11.786,1.006)--(11.788,1.006)--(11.791,1.007)--(11.793,1.007)--(11.795,1.007)--(11.797,1.007)%
  --(11.799,1.008)--(11.801,1.008)--(11.803,1.008)--(11.805,1.008)--(11.807,1.009)--(11.809,1.009)%
  --(11.811,1.009)--(11.813,1.009)--(11.815,1.009)--(11.817,1.010)--(11.820,1.010)--(11.822,1.010)%
  --(11.824,1.010)--(11.826,1.011)--(11.828,1.011)--(11.830,1.011)--(11.832,1.011)--(11.834,1.012)%
  --(11.836,1.012)--(11.838,1.012)--(11.840,1.012)--(11.842,1.012)--(11.844,1.013)--(11.846,1.013)%
  --(11.849,1.013)--(11.851,1.013)--(11.853,1.014)--(11.855,1.014)--(11.857,1.014)--(11.859,1.014)%
  --(11.861,1.015)--(11.863,1.015)--(11.865,1.015)--(11.867,1.015)--(11.869,1.015)--(11.871,1.016)%
  --(11.873,1.016)--(11.875,1.016)--(11.877,1.016)--(11.880,1.017)--(11.882,1.017)--(11.884,1.017)%
  --(11.886,1.017)--(11.888,1.017)--(11.890,1.018)--(11.892,1.018)--(11.894,1.018)--(11.896,1.018)%
  --(11.898,1.019)--(11.900,1.019)--(11.902,1.019)--(11.904,1.019)--(11.906,1.019)--(11.909,1.020)%
  --(11.911,1.020)--(11.913,1.020)--(11.915,1.020)--(11.917,1.021)--(11.919,1.021)--(11.921,1.021)%
  --(11.923,1.021)--(11.925,1.021)--(11.927,1.022)--(11.929,1.022)--(11.931,1.022)--(11.933,1.022)%
  --(11.935,1.022)--(11.938,1.023)--(11.940,1.023)--(11.942,1.023)--(11.944,1.023)--(11.946,1.023)%
  --(11.948,1.024)--(11.950,1.024)--(11.952,1.024)--(11.954,1.024)--(11.956,1.024)--(11.958,1.025)%
  --(11.960,1.025)--(11.962,1.025)--(11.964,1.025)--(11.967,1.025)--(11.969,1.026)--(11.971,1.026)%
  --(11.973,1.026)--(11.975,1.026)--(11.977,1.026)--(11.979,1.027)--(11.981,1.027)--(11.983,1.027)%
  --(11.985,1.027)--(11.987,1.027)--(11.989,1.027)--(11.991,1.028)--(11.993,1.028)--(11.996,1.028)%
  --(11.998,1.028)--(12.000,1.028)--(12.002,1.028)--(12.004,1.029)--(12.006,1.029)--(12.008,1.029)%
  --(12.010,1.029)--(12.012,1.029)--(12.014,1.029)--(12.016,1.030)--(12.018,1.030)--(12.020,1.030)%
  --(12.022,1.030)--(12.025,1.030)--(12.027,1.030)--(12.029,1.031)--(12.031,1.031)--(12.033,1.031)%
  --(12.035,1.031)--(12.037,1.031)--(12.039,1.031);
\draw[gp path] (1.688,8.381)--(1.688,0.985)--(12.039,0.985)--(12.039,8.381)--cycle;
%% coordinates of the plot area
\gpdefrectangularnode{gp plot 1}{\pgfpoint{1.688cm}{0.985cm}}{\pgfpoint{12.039cm}{8.381cm}}
\end{tikzpicture}
%% gnuplot variables

}
	\caption{Rocking curve with fit}
	\label{fig:ana_rocking}
\end{figure}
The function seems to describe the data quite well and the fit results seem to be reasonable. The value of $\Delta n_0$ is somehow twice as high as we would have expected from our writing curve. The bragg angle however is in accordance with the one we used in the previous sections.
\section{Abstract}

\begin{appendix}
\section{Tables}
\begin{table}[H]
  \centering
\footnotesize
\resizebox{1.0\textwidth}{!}{
  \begin{tabular}{l|cccc|cc}
    \toprule
t [$\unit[]{s}$] & $I_\textrm{trans,corr}$ [$\unit[]{mV}$] & Range & $I_\textrm{diffr,corr}$ [$\unit[]{mV}$] & Range & $\eta$ [$\unit[]{10^{-3}}$] & $\Delta n$ [$\unit[]{10^{-3}}$]\\
    \midrule[0.75pt]
$0$ & $545.5$ & $4$ & $355.8$ & $8$ & $0.0652 \pm 0.0001$ & $8.045 \pm 0.007$\\
$65$ & $545.3$ & $4$ & $608.2$ & $8$ & $0.1115 \pm 0.0001$ & $10.520 \pm 0.006$\\
$131$ & $544.3$ & $4$ & $109.7$ & $7$ & $0.202 \pm 0.001$ & $14.14 \pm 0.03$\\
$199$ & $544.5$ & $4$ & $134.1$ & $7$ & $0.246 \pm 0.001$ & $15.63 \pm 0.03$\\
$254$ & $543.9$ & $4$ & $170.8$ & $7$ & $0.314 \pm 0.001$ & $17.65 \pm 0.03$\\
$305$ & $544.7$ & $4$ & $212.6$ & $7$ & $0.390 \pm 0.001$ & $19.68 \pm 0.02$\\
$363$ & $544.0$ & $4$ & $256.5$ & $7$ & $0.471 \pm 0.001$ & $21.63 \pm 0.02$\\
$430$ & $544.5$ & $4$ & $287.0$ & $7$ & $0.527 \pm 0.001$ & $22.87 \pm 0.02$\\
$480$ & $544.5$ & $4$ & $340.6$ & $7$ & $0.625 \pm 0.001$ & $24.91 \pm 0.02$\\
$546$ & $543.4$ & $4$ & $390.5$ & $7$ & $0.718 \pm 0.001$ & $26.69 \pm 0.02$\\
$609$ & $544.1$ & $4$ & $447.8$ & $7$ & $0.822 \pm 0.001$ & $28.57 \pm 0.02$\\
$674$ & $543.8$ & $4$ & $498.6$ & $7$ & $0.916 \pm 0.001$ & $30.15 \pm 0.02$\\
$733$ & $543.7$ & $4$ & $568.3$ & $7$ & $1.044 \pm 0.001$ & $32.19 \pm 0.02$\\
$799$ & $543.5$ & $4$ & $623.6$ & $7$ & $1.146 \pm 0.001$ & $33.73 \pm 0.02$\\
$869$ & $543.2$ & $4$ & $674.7$ & $7$ & $1.241 \pm 0.001$ & $35.09 \pm 0.02$\\
$919$ & $543.3$ & $4$ & $747.5$ & $7$ & $1.374 \pm 0.002$ & $36.93 \pm 0.02$\\
$985$ & $543.1$ & $4$ & $805.0$ & $7$ & $1.480 \pm 0.002$ & $38.33 \pm 0.02$\\
$1052$ & $543.5$ & $4$ & $871.2$ & $7$ & $1.600 \pm 0.002$ & $39.86 \pm 0.02$\\
$1115$ & $542.0$ & $4$ & $939.2$ & $7$ & $1.730 \pm 0.002$ & $41.44 \pm 0.02$\\
$1172$ & $542.7$ & $4$ & $103.0$ & $6$ & $1.894 \pm 0.009$ & $43.37 \pm 0.11$\\
$1240$ & $542.3$ & $4$ & $111.5$ & $6$ & $2.052 \pm 0.009$ & $45.14 \pm 0.10$\\
$1299$ & $541.8$ & $4$ & $119.4$ & $6$ & $2.199 \pm 0.009$ & $46.73 \pm 0.10$\\
$1363$ & $541.8$ & $4$ & $128.0$ & $6$ & $2.356 \pm 0.009$ & $48.37 \pm 0.10$\\
$1414$ & $542.1$ & $4$ & $135.8$ & $6$ & $2.498 \pm 0.009$ & $49.80 \pm 0.09$\\
$1483$ & $542.1$ & $4$ & $146.1$ & $6$ & $2.69 \pm 0.01$ & $51.67 \pm 0.09$\\
$1550$ & $542.0$ & $4$ & $154.6$ & $6$ & $2.84 \pm 0.01$ & $53.15 \pm 0.09$\\
$1611$ & $541.1$ & $4$ & $166.1$ & $6$ & $3.06 \pm 0.01$ & $55.13 \pm 0.09$\\
$1670$ & $540.8$ & $4$ & $177.7$ & $6$ & $3.28 \pm 0.01$ & $57.04 \pm 0.08$\\
$1727$ & $541.4$ & $4$ & $190.4$ & $6$ & $3.50 \pm 0.01$ & $59.00 \pm 0.08$\\
$1795$ & $540.8$ & $4$ & $202.5$ & $6$ & $3.73 \pm 0.01$ & $60.89 \pm 0.08$\\
$1847$ & $540.5$ & $4$ & $218.3$ & $6$ & $4.02 \pm 0.01$ & $63.22 \pm 0.08$\\
$1911$ & $540.5$ & $4$ & $231.1$ & $6$ & $4.26 \pm 0.01$ & $65.05 \pm 0.08$\\
$1969$ & $540.2$ & $4$ & $245.2$ & $6$ & $4.52 \pm 0.01$ & $67.01 \pm 0.07$\\
$2039$ & $540.7$ & $4$ & $262.7$ & $6$ & $4.84 \pm 0.01$ & $69.32 \pm 0.07$\\
$2103$ & $540.2$ & $4$ & $278.7$ & $6$ & $5.13 \pm 0.01$ & $71.43 \pm 0.07$\\
$2165$ & $539.5$ & $4$ & $293.2$ & $6$ & $5.41 \pm 0.01$ & $73.30 \pm 0.07$\\
$2227$ & $539.7$ & $4$ & $313.0$ & $6$ & $5.77 \pm 0.01$ & $75.71 \pm 0.07$\\
$2292$ & $539.1$ & $4$ & $332.2$ & $6$ & $6.12 \pm 0.01$ & $78.03 \pm 0.07$\\
$2350$ & $539.1$ & $4$ & $349.7$ & $6$ & $6.45 \pm 0.01$ & $80.06 \pm 0.07$\\
$2402$ & $539.6$ & $4$ & $369.9$ & $6$ & $6.81 \pm 0.01$ & $82.29 \pm 0.07$\\
$2469$ & $539.1$ & $4$ & $392.8$ & $6$ & $7.23 \pm 0.01$ & $84.82 \pm 0.07$\\
$2520$ & $538.5$ & $4$ & $416.1$ & $6$ & $7.67 \pm 0.01$ & $87.33 \pm 0.07$\\
    \bottomrule
  \end{tabular}
}
\caption{Data for the writing curve part 1. 'Range' determines the amplification in $\unit[10^{-x}]{\frac{W}{mV}}$. $\Delta t = \unit[0.5]{s}$, $\Delta I = \unit[0.14]{mV}$}
  \label{tab:ana_writing_first}
\end{table}
\begin{table}[H]
  \centering
\footnotesize
\resizebox{1.0\textwidth}{!}{
  \begin{tabular}{l|cccc|cc}
    \toprule
t [$\unit[]{s}$] & $I_\textrm{trans,corr}$ [$\unit[]{mV}$] & Range & $I_\textrm{diffr,corr}$ [$\unit[]{mV}$] & Range & $\eta$ [$\unit[]{10^{-3}}$] & $\Delta n$ [$\unit[]{10^{-3}}$]\\
    \midrule[0.75pt]
$2587$ & $539.4$ & $4$ & $439.7$ & $6$ & $8.09 \pm 0.01$ & $89.69 \pm 0.07$\\
$2644$ & $539.3$ & $4$ & $464.5$ & $6$ & $8.54 \pm 0.01$ & $92.18 \pm 0.07$\\
$2696$ & $539.3$ & $4$ & $493.6$ & $6$ & $9.07 \pm 0.01$ & $95.01 \pm 0.06$\\
$2763$ & $538.3$ & $4$ & $519.3$ & $6$ & $9.55 \pm 0.01$ & $97.52 \pm 0.06$\\
$2826$ & $537.9$ & $4$ & $546.1$ & $6$ & $10.05 \pm 0.01$ & $100.03 \pm 0.06$\\
$2892$ & $538.7$ & $4$ & $578.9$ & $6$ & $10.63 \pm 0.01$ & $102.90 \pm 0.06$\\
$2955$ & $538.2$ & $4$ & $606.5$ & $6$ & $11.14 \pm 0.01$ & $105.35 \pm 0.06$\\
$3008$ & $538.3$ & $4$ & $638.8$ & $6$ & $11.73 \pm 0.01$ & $108.09 \pm 0.07$\\
$3061$ & $537.9$ & $4$ & $667.4$ & $6$ & $12.26 \pm 0.01$ & $110.50 \pm 0.07$\\
    \bottomrule
  \end{tabular}
}
\caption{Data for the writing curve part 2. 'Range' determines the amplification in $\unit[10^{-x}]{\frac{W}{mV}}$. $\Delta t = \unit[0.5]{s}$, $\Delta I = \unit[0.14]{mV}$}
  \label{tab:ana_writing_second}
\end{table}
\begin{table}[H]
  \centering
\footnotesize
\resizebox{1.0\textwidth}{!}{
  \begin{tabular}{l|cccc|cc}
    \toprule
t [$\unit[]{s}$] & $I_\textrm{trans,corr}$ [$\unit[]{mV}$] & Range & $I_\textrm{diffr,corr}$ [$\unit[]{mV}$] & Range & $\eta$ [$\unit[]{10^{-3}}$] & $\Delta n$ [$\unit[]{10^{-3}}$]\\
    \midrule[0.75pt]
$0$ & $545.4$ & $4$ & $615.7$ & $6$ & $11.16 \pm 0.01$ & $105.44 \pm 0.06$\\
$64$ & $545.4$ & $4$ & $612.0$ & $6$ & $11.10 \pm 0.01$ & $105.13 \pm 0.06$\\
$134$ & $544.8$ & $4$ & $608.5$ & $6$ & $11.05 \pm 0.01$ & $104.89 \pm 0.06$\\
$185$ & $545.7$ & $4$ & $607.1$ & $6$ & $11.00 \pm 0.01$ & $104.68 \pm 0.06$\\
$239$ & $544.6$ & $4$ & $602.4$ & $6$ & $10.94 \pm 0.01$ & $104.38 \pm 0.06$\\
$309$ & $545.4$ & $4$ & $601.1$ & $6$ & $10.90 \pm 0.01$ & $104.19 \pm 0.06$\\
$371$ & $545.4$ & $4$ & $597.7$ & $6$ & $10.84 \pm 0.01$ & $103.91 \pm 0.06$\\
$434$ & $545.8$ & $4$ & $595.8$ & $6$ & $10.80 \pm 0.01$ & $103.70 \pm 0.06$\\
$494$ & $545.8$ & $4$ & $593.1$ & $6$ & $10.75 \pm 0.01$ & $103.46 \pm 0.06$\\
$548$ & $545.2$ & $4$ & $589.8$ & $6$ & $10.70 \pm 0.01$ & $103.24 \pm 0.06$\\
$614$ & $545.8$ & $4$ & $588.6$ & $6$ & $10.67 \pm 0.01$ & $103.07 \pm 0.06$\\
$674$ & $544.8$ & $4$ & $585.1$ & $6$ & $10.62 \pm 0.01$ & $102.86 \pm 0.06$\\
$734$ & $545.6$ & $4$ & $584.2$ & $6$ & $10.59 \pm 0.01$ & $102.71 \pm 0.06$\\
$792$ & $545.4$ & $4$ & $581.6$ & $6$ & $10.55 \pm 0.01$ & $102.51 \pm 0.06$\\
$849$ & $545.4$ & $4$ & $579.9$ & $6$ & $10.52 \pm 0.01$ & $102.35 \pm 0.06$\\
$913$ & $546.1$ & $4$ & $578.2$ & $6$ & $10.48 \pm 0.01$ & $102.14 \pm 0.06$\\
$979$ & $546.1$ & $4$ & $576.1$ & $6$ & $10.44 \pm 0.01$ & $101.96 \pm 0.06$\\
$1041$ & $545.3$ & $4$ & $574.1$ & $6$ & $10.42 \pm 0.01$ & $101.85 \pm 0.06$\\
$1101$ & $545.9$ & $4$ & $572.4$ & $6$ & $10.38 \pm 0.01$ & $101.65 \pm 0.06$\\
$1157$ & $545.5$ & $4$ & $570.6$ & $6$ & $10.35 \pm 0.01$ & $101.53 \pm 0.06$\\
$1216$ & $545.6$ & $4$ & $569.0$ & $6$ & $10.32 \pm 0.01$ & $101.38 \pm 0.06$\\
$1274$ & $546.0$ & $4$ & $567.7$ & $6$ & $10.29 \pm 0.01$ & $101.22 \pm 0.06$\\
$1331$ & $546.0$ & $4$ & $566.3$ & $6$ & $10.27 \pm 0.01$ & $101.10 \pm 0.06$\\
$1392$ & $545.5$ & $4$ & $564.6$ & $6$ & $10.25 \pm 0.01$ & $101.00 \pm 0.06$\\
$1449$ & $545.6$ & $4$ & $562.9$ & $6$ & $10.21 \pm 0.01$ & $100.83 \pm 0.06$\\
$1518$ & $545.5$ & $4$ & $562.0$ & $6$ & $10.20 \pm 0.01$ & $100.76 \pm 0.06$\\
$1582$ & $546.3$ & $4$ & $561.0$ & $6$ & $10.16 \pm 0.01$ & $100.59 \pm 0.06$\\
$1633$ & $546.6$ & $4$ & $560.4$ & $6$ & $10.15 \pm 0.01$ & $100.52 \pm 0.06$\\
$1702$ & $545.5$ & $4$ & $558.1$ & $6$ & $10.13 \pm 0.01$ & $100.41 \pm 0.06$\\
$1768$ & $545.7$ & $4$ & $557.1$ & $6$ & $10.11 \pm 0.01$ & $100.31 \pm 0.06$\\
$1834$ & $545.9$ & $4$ & $555.9$ & $6$ & $10.08 \pm 0.01$ & $100.18 \pm 0.06$\\
$1887$ & $545.8$ & $4$ & $555.3$ & $6$ & $10.07 \pm 0.01$ & $100.14 \pm 0.06$\\
$1951$ & $546.7$ & $4$ & $554.6$ & $6$ & $10.04 \pm 0.01$ & $99.99 \pm 0.06$\\
$2003$ & $546.5$ & $4$ & $553.8$ & $6$ & $10.03 \pm 0.01$ & $99.94 \pm 0.06$\\
$2066$ & $546.0$ & $4$ & $551.5$ & $6$ & $10.00 \pm 0.01$ & $99.79 \pm 0.06$\\
$2135$ & $546.9$ & $4$ & $551.9$ & $6$ & $9.99 \pm 0.01$ & $99.73 \pm 0.06$\\
$2202$ & $546.6$ & $4$ & $551.0$ & $6$ & $9.98 \pm 0.01$ & $99.67 \pm 0.06$\\
$2269$ & $546.1$ & $4$ & $549.2$ & $6$ & $9.96 \pm 0.01$ & $99.56 \pm 0.06$\\
$2322$ & $546.2$ & $4$ & $548.4$ & $6$ & $9.94 \pm 0.01$ & $99.48 \pm 0.06$\\
$2386$ & $546.4$ & $4$ & $548.1$ & $6$ & $9.93 \pm 0.01$ & $99.44 \pm 0.06$\\
$2450$ & $546.7$ & $4$ & $547.8$ & $6$ & $9.92 \pm 0.01$ & $99.38 \pm 0.06$\\
    \bottomrule
  \end{tabular}
}
\caption{Data for the erasing curve. 'Range' determines the amplification in $\unit[10^{-x}]{\frac{W}{mV}}$. $\Delta t = \unit[0.5]{s}$, $\Delta I = \unit[0.14]{mV}$}
  \label{tab:ana_erasing}
\end{table}
\begin{table}[H]
  \centering
\footnotesize
\resizebox{1.0\textwidth}{!}{
  \begin{tabular}{l|ccc|cccc|cc}
    \toprule
$x$ [$\unit[]{mm}$] & $\alpha$ [$\unit[]{^\circ}$] & $\gamma$ [$\unit[]{^\circ}$] & $\theta$ [$\unit[]{^\circ}$] &   $I_\textrm{t,c}$ [$\unit[]{mV}$] & R & $I_\textrm{d,c}$ [$\unit[]{mV}$] & R & $\eta$ [$\unit[]{10^{-3}}$]\\
    \midrule[0.75pt]
$0.75$ & $179.4 \pm 0.3$ & $9.4 \pm 0.3$ & $8.35 \pm 0.13$ & $540.4$ & $4$ & $29.9$ & $8$ & $0.0055 \pm 0.0002$\\
$0.76$ & $179.4 \pm 0.3$ & $9.4 \pm 0.3$ & $8.36 \pm 0.13$ & $540.7$ & $4$ & $1.1$ & $8$ & $0.0002 \pm 0.0002$\\
$0.77$ & $179.5 \pm 0.3$ & $9.5 \pm 0.3$ & $8.37 \pm 0.13$ & $541.4$ & $4$ & $29.7$ & $8$ & $0.0055 \pm 0.0002$\\
$0.78$ & $179.5 \pm 0.3$ & $9.5 \pm 0.3$ & $8.37 \pm 0.13$ & $540.5$ & $4$ & $184.5$ & $8$ & $0.0341 \pm 0.0002$\\
$0.79$ & $179.5 \pm 0.3$ & $9.5 \pm 0.3$ & $8.38 \pm 0.13$ & $541.2$ & $4$ & $77.1$ & $8$ & $0.0142 \pm 0.0002$\\
$0.80$ & $179.5 \pm 0.3$ & $9.5 \pm 0.3$ & $8.39 \pm 0.13$ & $547.0$ & $4$ & $43.0$ & $8$ & $0.0079 \pm 0.0002$\\
$0.81$ & $179.5 \pm 0.3$ & $9.5 \pm 0.3$ & $8.40 \pm 0.13$ & $542.8$ & $4$ & $0.3$ & $8$ & $0.0001 \pm 0.0002$\\
$0.82$ & $179.6 \pm 0.3$ & $9.6 \pm 0.3$ & $8.41 \pm 0.13$ & $542.4$ & $4$ & $115.2$ & $8$ & $0.0212 \pm 0.0002$\\
$0.83$ & $179.6 \pm 0.3$ & $9.6 \pm 0.3$ & $8.41 \pm 0.13$ & $541.1$ & $4$ & $317.1$ & $8$ & $0.0586 \pm 0.0002$\\
$0.84$ & $179.6 \pm 0.3$ & $9.6 \pm 0.3$ & $8.42 \pm 0.13$ & $547.2$ & $4$ & $566.8$ & $8$ & $0.1036 \pm 0.0003$\\
$0.85$ & $179.6 \pm 0.3$ & $9.6 \pm 0.3$ & $8.43 \pm 0.13$ & $547.2$ & $4$ & $237.4$ & $8$ & $0.0434 \pm 0.0002$\\
$0.86$ & $179.6 \pm 0.3$ & $9.6 \pm 0.3$ & $8.44 \pm 0.13$ & $548.8$ & $4$ & $51.5$ & $8$ & $0.0094 \pm 0.0002$\\
$0.87$ & $179.7 \pm 0.3$ & $9.7 \pm 0.3$ & $8.45 \pm 0.13$ & $548.4$ & $4$ & $10.9$ & $8$ & $0.0020 \pm 0.0002$\\
$0.88$ & $179.7 \pm 0.3$ & $9.7 \pm 0.3$ & $8.46 \pm 0.13$ & $541.4$ & $4$ & $363.0$ & $8$ & $0.0670 \pm 0.0002$\\
$0.89$ & $179.7 \pm 0.3$ & $9.7 \pm 0.3$ & $8.46 \pm 0.13$ & $544.7$ & $4$ & $548.0$ & $8$ & $0.1006 \pm 0.0003$\\
$0.90$ & $179.7 \pm 0.3$ & $9.7 \pm 0.3$ & $8.47 \pm 0.13$ & $542.3$ & $4$ & $125.6$ & $7$ & $0.232 \pm 0.002$\\
$0.91$ & $179.7 \pm 0.3$ & $9.7 \pm 0.3$ & $8.48 \pm 0.13$ & $548.4$ & $4$ & $683.8$ & $8$ & $0.1247 \pm 0.0003$\\
$0.92$ & $179.8 \pm 0.3$ & $9.8 \pm 0.3$ & $8.49 \pm 0.13$ & $548.2$ & $4$ & $335.0$ & $8$ & $0.0611 \pm 0.0002$\\
$0.93$ & $179.8 \pm 0.3$ & $9.8 \pm 0.3$ & $8.50 \pm 0.13$ & $548.1$ & $4$ & $5.3$ & $8$ & $0.0010 \pm 0.0002$\\
$0.94$ & $179.8 \pm 0.3$ & $9.8 \pm 0.3$ & $8.50 \pm 0.14$ & $549.0$ & $4$ & $662.2$ & $8$ & $0.1206 \pm 0.0003$\\
$0.95$ & $179.8 \pm 0.3$ & $9.8 \pm 0.3$ & $8.51 \pm 0.14$ & $547.7$ & $4$ & $179.3$ & $7$ & $0.327 \pm 0.002$\\
$0.96$ & $179.8 \pm 0.3$ & $9.8 \pm 0.3$ & $8.52 \pm 0.14$ & $548.3$ & $4$ & $831.6$ & $7$ & $1.514 \pm 0.003$\\
$0.97$ & $179.9 \pm 0.3$ & $9.9 \pm 0.3$ & $8.53 \pm 0.14$ & $545.9$ & $4$ & $14.7$ & $5$ & $2.7 \pm 0.2$\\
$0.98$ & $179.9 \pm 0.3$ & $9.9 \pm 0.3$ & $8.54 \pm 0.14$ & $544.9$ & $4$ & $15.1$ & $5$ & $2.8 \pm 0.2$\\
$0.99$ & $179.9 \pm 0.3$ & $9.9 \pm 0.3$ & $8.54 \pm 0.14$ & $540.6$ & $4$ & $22.4$ & $5$ & $4.1 \pm 0.2$\\
$1.00$ & $179.9 \pm 0.3$ & $9.9 \pm 0.3$ & $8.55 \pm 0.14$ & $547.5$ & $4$ & $19.6$ & $5$ & $3.6 \pm 0.2$\\
$1.01$ & $179.9 \pm 0.3$ & $9.9 \pm 0.3$ & $8.56 \pm 0.14$ & $548.7$ & $4$ & $15.3$ & $5$ & $2.8 \pm 0.2$\\
$1.02$ & $179.9 \pm 0.3$ & $9.9 \pm 0.3$ & $8.57 \pm 0.14$ & $547.3$ & $4$ & $17.4$ & $5$ & $3.2 \pm 0.2$\\
$1.03$ & $180.0 \pm 0.3$ & $10.0 \pm 0.3$ & $8.58 \pm 0.14$ & $548.9$ & $4$ & $13.7$ & $5$ & $2.5 \pm 0.2$\\
$1.04$ & $180.0 \pm 0.3$ & $10.0 \pm 0.3$ & $8.58 \pm 0.14$ & $549.4$ & $4$ & $539.0$ & $7$ & $0.980 \pm 0.003$\\
$1.05$ & $180.0 \pm 0.3$ & $10.0 \pm 0.3$ & $8.59 \pm 0.14$ & $544.1$ & $4$ & $185.7$ & $7$ & $0.341 \pm 0.002$\\
$1.06$ & $180.0 \pm 0.3$ & $10.0 \pm 0.3$ & $8.60 \pm 0.14$ & $540.7$ & $4$ & $594.5$ & $8$ & $0.1099 \pm 0.0003$\\
$1.07$ & $180.0 \pm 0.3$ & $10.0 \pm 0.3$ & $8.61 \pm 0.14$ & $549.0$ & $4$ & $5.4$ & $8$ & $0.0010 \pm 0.0002$\\
$1.08$ & $180.1 \pm 0.3$ & $10.1 \pm 0.3$ & $8.62 \pm 0.14$ & $546.1$ & $4$ & $369.0$ & $8$ & $0.0676 \pm 0.0002$\\
$1.09$ & $180.1 \pm 0.3$ & $10.1 \pm 0.3$ & $8.62 \pm 0.14$ & $548.6$ & $4$ & $708.6$ & $8$ & $0.1291 \pm 0.0003$\\
$1.10$ & $180.1 \pm 0.3$ & $10.1 \pm 0.3$ & $8.63 \pm 0.14$ & $541.0$ & $4$ & $661.7$ & $8$ & $0.1223 \pm 0.0003$\\
$1.11$ & $180.1 \pm 0.3$ & $10.1 \pm 0.3$ & $8.64 \pm 0.14$ & $547.7$ & $4$ & $820.3$ & $8$ & $0.1497 \pm 0.0003$\\
$1.12$ & $180.1 \pm 0.3$ & $10.1 \pm 0.3$ & $8.65 \pm 0.14$ & $548.1$ & $4$ & $293.8$ & $8$ & $0.0536 \pm 0.0002$\\
$1.13$ & $180.2 \pm 0.4$ & $10.2 \pm 0.4$ & $8.66 \pm 0.15$ & $545.8$ & $4$ & $15.1$ & $8$ & $0.0028 \pm 0.0002$\\
$1.14$ & $180.2 \pm 0.4$ & $10.2 \pm 0.4$ & $8.66 \pm 0.15$ & $541.3$ & $4$ & $70.7$ & $8$ & $0.0131 \pm 0.0002$\\
$1.15$ & $180.2 \pm 0.4$ & $10.2 \pm 0.4$ & $8.67 \pm 0.15$ & $546.9$ & $4$ & $152.6$ & $8$ & $0.0279 \pm 0.0002$\\
$1.16$ & $180.2 \pm 0.4$ & $10.2 \pm 0.4$ & $8.68 \pm 0.15$ & $545.3$ & $4$ & $301.2$ & $8$ & $0.0552 \pm 0.0002$\\
$1.17$ & $180.2 \pm 0.4$ & $10.2 \pm 0.4$ & $8.69 \pm 0.15$ & $548.9$ & $4$ & $206.4$ & $8$ & $0.0376 \pm 0.0002$\\
$1.18$ & $180.3 \pm 0.4$ & $10.3 \pm 0.4$ & $8.70 \pm 0.15$ & $546.4$ & $4$ & $125.8$ & $8$ & $0.0230 \pm 0.0002$\\
$1.19$ & $180.3 \pm 0.4$ & $10.3 \pm 0.4$ & $8.70 \pm 0.15$ & $548.7$ & $4$ & $0.2$ & $8$ & $0.0000 \pm 0.0002$\\
$1.20$ & $180.3 \pm 0.4$ & $10.3 \pm 0.4$ & $8.71 \pm 0.15$ & $540.6$ & $4$ & $77.0$ & $8$ & $0.0142 \pm 0.0002$\\
$1.21$ & $180.3 \pm 0.4$ & $10.3 \pm 0.4$ & $8.72 \pm 0.15$ & $540.7$ & $4$ & $200.3$ & $8$ & $0.0370 \pm 0.0002$\\
$1.22$ & $180.3 \pm 0.4$ & $10.3 \pm 0.4$ & $8.73 \pm 0.15$ & $548.7$ & $4$ & $208.7$ & $8$ & $0.0380 \pm 0.0002$\\
$1.23$ & $180.4 \pm 0.4$ & $10.4 \pm 0.4$ & $8.74 \pm 0.15$ & $541.8$ & $4$ & $28.4$ & $8$ & $0.0052 \pm 0.0002$\\
$1.24$ & $180.4 \pm 0.4$ & $10.4 \pm 0.4$ & $8.74 \pm 0.15$ & $540.6$ & $4$ & $1.0$ & $8$ & $0.0002 \pm 0.0002$\\
$1.25$ & $180.4 \pm 0.4$ & $10.4 \pm 0.4$ & $8.75 \pm 0.15$ & $543.2$ & $4$ & $18.3$ & $8$ & $0.0034 \pm 0.0002$\\

    \bottomrule
  \end{tabular}
}
\caption{Data for the rocking curve. 'R' determines the amplification in $\unit[10^{-x}]{\frac{W}{mV}}$. $\Delta x = \unit[0.01]{mm}$, $\Delta I = \unit[0.14]{mV}$}
  \label{tab:ana_rocking}
\end{table}
\Literatur{quellen}

\end{appendix}
\end{document}



